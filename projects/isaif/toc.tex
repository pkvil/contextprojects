\page[right]
\chapter{Contents}

%\def\contentitemspacing{15pt}
\start
%\setupbodyfont[13pt]
\setupalign[right,nothyphenated]
\setupwhitespace[line]
\setupnarrower[left=27pt]%
\startnarrower[left]%
\setupindenting[-\leftskip,yes,first]%
%\define[1]\myparagraphref{\dontleavehmode\hbox to 3em{\hfil%
%{\it\color[vermillion]{¶\feature[+][pnum]\feature[+][lnum]#1}}}
%}

\define[1]\myparagraphref{\dontleavehmode{\feature[+][pnum]\feature[+][lnum]\inframed[align=left,width=27pt,frame=off]{\tfx¶\color[myred]{#1}\enspace}}}


\tf
\myparagraphref{1}Introduction

\myparagraphref{6}The psychology of modern leftism

\myparagraphref{10}Feelings of inferiority

\myparagraphref{24}Oversocialization

\myparagraphref{33}The power process

\myparagraphref{38}Surrogate activities

\myparagraphref{42}Autonomy

\myparagraphref{45}Sources of social problems

\myparagraphref{59}Disruption of the power process in modern society

\myparagraphref{77}How some people adjust

\myparagraphref{87}The motives of scientists

\myparagraphref{93}The nature of freedom

\myparagraphref{99}Some principles of history

\myparagraphref{111}Industrial-technological society cannot be reformed

\myparagraphref{114}Restriction of freedom is unavoidable in\break industrial society

\myparagraphref{121}The ‘bad’ parts of technology cannot be separated\break from the ‘good’ parts

\myparagraphref{125}Technology is a more powerful social force than\break the aspiration for freedom

\myparagraphref{136}Simpler social problems have proved intractable

\myparagraphref{140}Revolution is easier than reform

\myparagraphref{143}Control of human behavior

\myparagraphref{161}Human race at a crossroads

\myparagraphref{167}Human suffering

\myparagraphref{171}The future

\myparagraphref{180}Strategy

\myparagraphref{207}Two kinds of technology

\myparagraphref{213}The danger of leftism

\myparagraphref{231}Final note

\dontleavehmode\hbox to 27pt{}Endnotes
\stopnarrower
\stop
