\page[right]
\chapter{Contents}
\def\contentitemspacing{15pt}
Introduction\myparagraphref{1}
\blank[\contentitemspacing]
The psychology of modern leftism\myparagraphref{5}
\blank[\contentitemspacing]
Feelings of inferiority\myparagraphref{14}
\blank[\contentitemspacing]
Oversocialization
\blank[\contentitemspacing]
The power process
\blank[\contentitemspacing]
Surrogate activities
\blank[\contentitemspacing]
Autonomy
\blank[\contentitemspacing]
Sources of social problems
\blank[\contentitemspacing]
Disruption of the power process\par in modern society\myparagraphref{17}
\blank[\contentitemspacing]
How some people adjust
\blank[\contentitemspacing]
The motives of scientists
\blank[\contentitemspacing]
The nature of freedom
\blank[\contentitemspacing]
Some principles of history
\blank[\contentitemspacing]
Industrial-technological society\par cannot be reformed
\blank[\contentitemspacing]
Restriction of freedom is unavoidable\par in industrial society
\blank[\contentitemspacing]
The ‘bad’ parts of technology cannot\par be separated from the ‘good’ parts
\blank[\contentitemspacing]
Technology is a more powerful social\par force than the aspiration for freedom
\blank[\contentitemspacing]
Simpler social problems have proved\par intractable
\blank[\contentitemspacing]
Revolution is easier than reform
\blank[\contentitemspacing]
Control of human behavior
\blank[\contentitemspacing]
Human race at a crossroads
\blank[\contentitemspacing]
Human suffering
\blank[\contentitemspacing]
The future
\blank[\contentitemspacing]
Strategy
\blank[\contentitemspacing]
Two kinds of technology
\blank[\contentitemspacing]
The danger of leftism
\blank[\contentitemspacing]
Final note
\blank[\contentitemspacing]
Endnotes
