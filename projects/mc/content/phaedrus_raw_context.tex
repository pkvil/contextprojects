% Enable hyperlinks
\setupinteraction
  [state=start,
  title={Phaedrus},
  author={Plato},
  style=,
  color=,
  contrastcolor=]

% make chapter, section bookmarks visible when opening document
\placebookmarks[chapter, section, subsection, subsubsection, subsubsubsection, subsubsubsubsection][chapter, section]
\setupinteractionscreen[option=bookmark]
\setuptagging[state=start]


% use microtypography
\definefontfeature[default][default][script=latn, protrusion=quality, expansion=quality, itlc=yes, textitalics=yes, onum=yes, pnum=yes]
\definefontfeature[smallcaps][script=latn, protrusion=quality, expansion=quality, smcp=yes, onum=yes, pnum=yes]
\setupalign[hz,hanging]
\setupitaliccorrection[global, always]

\setupbodyfontenvironment[default][em=italic] % use italic as em, not slanted

\definefallbackfamily[mainface][rm][CMU Serif][preset=range:greek, force=yes]
\definefontfamily[mainface][rm][Latin Modern Roman]
\definefontfamily[mainface][mm][Latin Modern Math]
\definefontfamily[mainface][ss][Latin Modern Sans]
\definefontfamily[mainface][tt][Latin Modern Typewriter][features=none]
\setupbodyfont[mainface]

\setupwhitespace[medium]

\setuphead[chapter]            [style=\tfd,header=empty]
\setuphead[section]            [style=\tfc]
\setuphead[subsection]         [style=\tfb]
\setuphead[subsubsection]      [style=\bf]
\setuphead[subsubsubsection]   [style=\sc]
\setuphead[subsubsubsubsection][style=\it]

\setuphead[chapter, section, subsection, subsubsection, subsubsubsection, subsubsubsubsection][number=no]

\definedescription
  [description]
  [headstyle=bold, style=normal, location=hanging, width=broad, margin=1cm, alternative=hanging]

\setupitemize[autointro]    % prevent orphan list intro
\setupitemize[indentnext=no]

\setupfloat[figure][default={here,nonumber}]
\setupfloat[table][default={here,nonumber}]

\setupthinrules[width=15em] % width of horizontal rules

\setupxtable[frame=off]
\setupxtable[head][topframe=on,bottomframe=on]
\setupxtable[body][]
\setupxtable[foot][bottomframe=on]


\starttext
\startalignment[middle]
  {\tfd Phaedrus}
  \smallskip
  {\tfa Plato}
  \smallskip
  {\tfa 2008-12-04}
  \bigskip
\stopalignment

\reference[cover.htmlux23pha0000001]{}%

{\externalfigure[images/9780140449747_001.jpg]}

PHAEDRUS

PLATO stands, with his teacher Socrates and his pupil Aristotle, as one
of the shapers of the entire intellectual tradition of the West. Born
{\em c}. 427 BC, he came from a family that had long played a prominent
part in Athenian politics, and it would have been natural for him to
follow the same course; the reason for his not doing so, according to
the seventh of the collection of letters attributed to him (all of them
almost certainly inauthentic), was his disillusionment with the kind of
politics that could lead, among other things, to the execution -- in 399
-- of Socrates. Rather less plausibly, the same letter suggests that
Plato's several visits to the court of Dionysius II, tyrant of Syracuse
in Sicily, were motivated by a desire to put his political theories --
as developed above all in his masterwork, {\em Republic} -- into
practice. The reform of society on an ethical basis certainly remained
one of his central theoretical concerns. However, the focus of his
thinking was on ethics itself, in which he first followed and then went
beyond Socrates, and on metaphysics and the understanding of reality. In
the mid-380s, in Athens, he founded the Academy, the first permanent
institution devoted to philosophical research and teaching, and an
institution to which all Western universities like to trace their
origins.

Plato wrote more than twenty philosophical dialogues, appearing in none
himself (most have Socrates as chief speaker). His activity as a writer
seems to have lasted over half a century; few authors in any language
could claim to rival his particular combination of brilliant artistry
and intellectual power. He died in 347 BC.

CHRISTOPHER ROWE is Professor of Greek in the University of Durham, and
from 1999 until 2004 held a Leverhulme Personal Research Professorship.
Until 1995 he was H. O. Wills Professor of Greek in the University of
Bristol, and he has served at various times as President of the
International Plato Society, Chair of the Council of University
Classical Departments, Chair of the Council of the Classical
Association, President of the Society for the Promotion of Hellenic
Studies and joint editor of {\em Phronesis: A Journal for Ancient
Philosophy} (Leiden). His books include {\em Plato} (second edition,
2003), {\em Reading the Statesman} (edited volume, 1995), {\em The
Cambridge History of Greek and Roman Thought} (edited with Malcolm
Schofield, 2000) and {\em New Perspectives on Plato, Modern and Ancient}
(edited with Julia Annas, 2002). In Sarah Broadie and Christopher Rowe,
{\em Aristotle, Nicomachean Ethics} (2002), Sarah Broadie's
philosophical commentary is accompanied by his translation. He has also
translated, and/or written commentaries on, Plato's {\em Phaedrus}
(1986), {\em Phaedo} (1993), {\em Statesman} (1995) and {\em Symposium}
(1998) and, with Terry Penner, has written a monograph on the Lysis. His
present project is a comprehensive treatment of Plato's strategies as a
writer of philosophy.

\section[title={PLATO},reference={title.htmlux23pha0000008}]

\section[title={Phaedrus},reference={title.htmlux23pha0000009}]

{\em Translated with an Introduction and Notes by}\crlf
CHRISTOPHER ROWE

{\bf PENGUIN BOOKS}

PENGUIN CLASSICS

Published by the Penguin Group

Penguin Books Ltd, 80 Strand, London WC2R 0RL, England

Penguin Group (USA) Inc., 375 Hudson Street, New York, New York 10014,
USA

Penguin Group (Canada), 10 Alcorn Avenue, Toronto, Ontario, Canada M4V
3B2 (a division of Pearson Penguin Canada Inc.)

Penguin Ireland, 25 St Stephen's Green, Dublin 2, Ireland (a division of
Penguin Books Ltd)

Penguin Group (Australia), 250 Camberwell Road, Camberwell, Victoria
3124, Australia (a division of Pearson Australia Group Pty Ltd)

Penguin Books India Pvt Ltd, 11 Community Centre, Panchsheel Park, New
Delhi -- 110 017, India

Penguin Group (NZ), cnr Airborne and Rosedale Roads, Albany, Auckland
1310, New Zealand (a division of Pearson New Zealand Ltd)

Penguin Books (South Africa) (Pty) Ltd, 24 Sturdee Avenue, Rosebank
2196, South Africa

Penguin Books Ltd, Registered Offices: 80 Strand, London WC2R 0RL,
England

\useURL[url1][http://www.penguin.com][][www.penguin.com]\from[url1]

This translation first published in 2005

5

Introduction and Further Reading © Christopher Rowe, 2005

All rights reserved

The moral right of the translator has been asserted

Except in the United States of America, this book is sold subject to the
condition that it shall not, by way of trade or otherwise, be lent,
re-sold, hired out, or otherwise circulated without the publisher's
prior consent in any form of binding or cover other than that in which
it is published and without a similar condition including this condition
being imposed on the subsequent purchaser

ISBN: 9781101488782

\section[title={Contents},reference={contents.htmlux23pha0000030}]

\goto{Acknowledgements}[preface001.html]

\goto{Chronology}[preface002.html]

\goto{Introduction}[introduction.html]

\goto{Further Reading}[bibliography.html]

\goto{A Note on the Text and Translation}[preface003.html]

\goto{Phaedrus}[chapter-001.html]

\goto{Appendix: The Structure of the {\em Phaedrus}}[appendix.html]

\goto{Notes}[notes.html]

\section[title={Acknowledgements},reference={preface001.htmlux23pha0000039}]

I offer thanks to Adrian and Lucinda Phillips (of Aris & Phillips), who
first invited me to tackle the {\em Phaedrus}, and who published the
volume that is the direct ancestor of the present one ({\em Plato:
Phaedrus}, 1986; second edition, 2000); to David Brown of Oxbow Books,
who as the new owner of the Aris & Phillips imprint gave permission for
me to reuse my translation (though in the event I have significantly
modified it) and scattered material from the commentary; to David
Hopkins, who was largely responsible for the translation of a number of
short verse passages Plato includes in his text; to the Leverhulme
Trust, without whose grant to me of a Personal Research Professorship
this second journey of mine through the {\em Phaedrus} would not have
been possible; to Terry Penner, in whose company my appreciation of
Plato's philosophical acuity and literary sophistication continues to
develop; to my wife Heather, for perpetual support (and tolerance) and
for reading the proofs; and to Laura Barber, for proposing to me the
project of a new Penguin {\em Phaedrus}.

CHRISTOPHER ROWE\crlf
June 2004

\section[title={Chronology},reference={preface002.htmlux23pha0000042}]

We know as little about Plato's life as we do about most of his peers.
Plato is chiefly known as a writer, of course, but what passed for
‘publication' in ancient Athens was so far removed from the modern world
that it is hard to say even that a dialogue was published around
such-and-such a year. The relative dating of the dialogues is often
controversial too, and many scholars nowadays prefer to study each
dialogue on its own, rather than as part of a corpus. The situation is
confused by the presence of a number of works which are spurious or at
least of uncertain authorship: they have been omitted below.
Nevertheless, study of Plato's writing style has resulted in a broad
division of the dialogues into three groups, which is reflected in the
(very rough) chronology below. For this division, see especially Charles
Kahn, ‘On Platonic Chronology', in Julia Annas and Christopher Rowe
(eds), {\em New Perspectives on Plato, Ancient and Modern}) Cambridge,
MA: Harvard University Press, 2002), pp. 93 -- 127.

{\bf {\em {\em c}. 427}} BC Birth of Plato, son of Ariston, in Athens
into a well-connected and noble family. The story that the name ‘Plato'
derived from the adjective {\em platus}, ‘broad' (referring to the width
of the great man's shoulders; or his intellectual capacity; or the
‘breadth' of his style) has been discredited.

{\bf 418 -- 416} Dramatic date for the dialogue between Socrates and
Phaedrus.

{\bf 404} The defeat of Athens in the Peloponnesian War (which started
in 431) signals the temporary end of the democracy which had governed
Athens for most of the last hundred years. It is replaced by a cruel
junta of oligarchs, at least two of whom are members of Plato's
immediate family. The Thirty Tyrants, as they become known, last only a
few months before being overthrown in a civil war and replaced by a
revived democracy.

{\bf 399} The new democratic government puts Socrates, Plato's mentor
for the past few years, on trial. The charges are: failing to recognize
the gods recognized by the State, but introducing new deities, and
corrupting the young. Socrates' condemnation and subsequent execution by
drinking hemlock are the last straw: Plato more or less withdraws from
the world of practical politics.

{\bf 390s} Plato and other disciples find it expedient to leave Athens
for a while and stay in nearby Megara. Plato may also have travelled
around the eastern Mediterranean. Meanwhile, he is beginning to write.

{\bf 390s -- 380s} Plato composes a large and varied group of dialogues:
{\em Defence of Socrates (‘Apology'), Charmides, Cratylus, Crito,
Euthydemus, Euthyphro, Gorgias, Hippias Minor, Ion, Laches, Lysis,
Menexenus, Meno, Phaedo, Protagoras, Symposium}.

{\bf 389 -- 388} Plato visits Sicily and southern Italy and establishes
contact with the Pythagorean schools there.

{\bf {\em c}. 387} Plato founds the Academy, a research institute,
adjacent to one of the main gymnasia of Athens, a little to the
northwest of the ancient city, which was sacred to the local hero
Academus.

{\bf 380s -- 370s} Second group of dialogues: {\em Parmenides, Phaedrus,
Republic, Theaetetus}.

{\bf 367} Second visit to Sicily, at the invitation of Dion, uncle of
Dionysius 11, tyrant of Syracuse. Some have supposed that Plato had
hopes of making Dionysius an ideal ruler, a philosopher-king; if so,
they were soon dashed. Plato evidently had some difficulty getting home.

{\bf {\em c}. 365} Arrival in the Academy of its most eminent student,
Aristotle.

{\bf 361} Plato visits Sicily once more, for unknown reasons; in any
case this visit too seems to have ended badly.

{\bf 360s -- 350s} Third group of dialogues: {\em Philebus, Sophist,
Statesman, Timaeus-Critias, Laws} (known to be Plato's last work; he may
have been still working on it when he died).

{\bf 347} Plato dies.

\section[title={Introduction},reference={introduction.htmlux23pha0000058}]

At the beginning of the
{\em Phaedrus},\goto{\high{1}}[introduction.htmlux5cux23pla_0000006] the
philosopher Socrates meets
Phaedrus,\goto{\high{2}}[introduction.htmlux5cux23pla_0000008] an
amateur rhetorical enthusiast passionately devoted to professional
displays, and the conversation between the two of them lasts for the
whole of the work. If the dramatic date of the dialogue is somewhere
around 418 -- 416
BC,\goto{\high{3}}[introduction.htmlux5cux23pla_0000010] Socrates is in
his fifties (he was born in 469), Phaedrus in his mid-twenties; despite
the claims of some scholars, there is no evidence in the work of
anything beyond friendship between them. This is a point that needs
making, for the Socrates of Plato's dialogues is usually represented as
someone with a keen interest in beautiful young men or boys (though as
he says in the {\em Charmides}, ‘pretty well all of them, at the right
age, appear beautiful to me' (154b9 -- 10) -- and he then goes on to
say, typically, that he won't know whether anyone is really beautiful
until he knows the state of his {\em soul}). Moreover, one of the chief
topics of the dialogue is {\em erôs} -- i.e. passionate or (as it would
normally be understood) romantic love. So it would not have been in the
least surprising if the {\em Phaedrus} had involved some sort of erotic
by-play, perhaps the suggestion of an erotic relationship, between its
characters, making the action of the dialogue -- as so often in Plato --
mirror its content. The {\em Symposium}, in many ways the companion
dialogue of the {\em Phaedrus}, is full of such by-play, while the
{\em Lysis}, a third, shorter and more puzzling treatment of love,
starts and ends with Hippothales' passion for Lysis. In the
{\em Phaedrus}, by contrast, the characters -- Socrates and Phaedrus --
are content merely to {\em talk} about love, and lovers, as observers.

For in fact, both Phaedrus and Socrates have other kinds of love, other
kinds of obsession. They are both mad about {\em logoi}, the most
general meaning of which would be ‘words', or ‘things said'. But the two
men are mad about rather different kinds of ‘things said': in Phaedrus'
case, it is {\em speeches}, whereas for Socrates it is {\em talk} and,
as it will turn out, ‘talk' of a rather special (philosophical) kind.
When the pair of them meet, Phaedrus is fresh from hearing a rhetorical
performance by his ‘darling' Lysias, someone of about the same age as
himself\goto{\high{4}}[introduction.htmlux5cux23pla_0000012] but
evidently already making his mark as an orator. This speech was a
display on a paradoxical theme: an imaginary appeal by a man to an
adolescent boy to grant him sexual favours, but with the novelty that
the man in question is (so he pretends, at any rate) not in love with
the boy. Socrates proceeds to tease Phaedrus about his feigned
reluctance to rehearse the speech to him:

\reference[introduction.htmlux23pha0000061]{}%
Phaedrus -- if I don't know Phaedrus, I've forgotten even who I am. But
I do, and I haven't; I know perfectly well that when he heard Lysias'
speech he did not hear it just once but repeatedly asked him to go
through it for him, and Lysias responded readily. But for Phaedrus not
even that was enough, and in the end he borrowed the book and examined
the things in it which he was most eager to look at, and doing this he
sat from sun-up until he was tired and went for a walk, as I think --
{\em I}'ll swear by the Dog it's true -- knowing the speech quite off by
heart, unless it was a rather long one. He was going outside the wall to
practise it, when he met the very person who is sick with passion for
hearing people speak -- and ‘seeing, seeing him', he was glad, because
he would have a companion in his manic frenzy, and he told him to lead
on. Then when the one in love with speeches asked him to speak, he put
on a pose, as if not eager to speak; but in the end, even if no one
wanted to listen, he meant to use force, and {\em would} speak. So you,
Phaedrus, you just ask him to do here and now what he will soon do
anyway. (228a5 -- c5)

Right from the beginning, then, we know something about where the two
men's passions lie. At this stage, Socrates makes out that it is
Phaedrus' kind of {\em logoi} that he is ‘sick with passion' about; he
discovers that Phaedrus actually has the written version of the speech
hidden on his person and eventually makes him read it out. (The speech
just might be a genuine product of Lysias' but is more likely to be an
imitation by Plato himself. It contains just the sorts of things Lysias
would have said, piling point upon point in the way he did -- for the
lawcourts.)

But Lysias' speech does not impress Socrates, and he goes on to give a
rival speech, on the same subject, of his own -- or rather, as he
claims, not of his own: he must have got it from someone or somewhere
else,\goto{\high{5}}[introduction.htmlux5cux23pla_0000014] because he
knows nothing at all about speech-making, or about the subject. He is
still concerned -- so he suggests -- about putting on a poor showing in
front of Phaedrus, and so he gives the speech with his head covered,
allegedly to prevent his feeling ashamed as he sees Phaedrus' reactions
(237a). However, he has hardly finished when it suddenly emerges that he
has a much more important reason to be ashamed: it was just a ‘dreadful'
speech, ‘foolish and somewhat impious' (242d4, 7), because of what it
said about -- that is, against -- love. There follows the famous
‘palinode', or speech of repentance, which praises love, and the god of
love, as the greatest of good things, and describes the winged soul
traversing the cosmos: a speech of extraordinary range and virtuosity
that has always been, for ancient as well as modern readers, the main
focus of interest in the {\em Phaedrus}. What Socrates offers us in this
new, second, speech -- so he gives us to understand -- is more like the
truth of the matter: love may be a form of madness, as Lysias had
originally claimed (who would prefer to deal with a mad person when he
could deal with a sane one?), but when it comes from the gods, madness
is actually preferable to sanity.

How so? Because the madness that Socrates has in mind is the madness of
{\em philosophia}, the ‘love of wisdom'. And in the picture he gives of
the ideal pair of lover and beloved, we have a glimpse of what he means
by calling himself ‘sick with passion for hearing people speak'; for
what this ideal pair does is to talk, as their means of access to, or
(in terms of the special theory to which Socrates refers) as the
recovery of, the knowledge or wisdom they both desire.

After Socrates‘\goto{\high{6}}[introduction.htmlux5cux23pla_0000016]
second speech, he initiates a long discussion of what it is for a
{\em logos} (‘speech', ‘talk', ‘discussion', ‘discourse' or just ‘thing
said') to be a good or a bad one, and then of what one's attitude should
be towards one's verbal ‘offspring', whether written or spoken.
Fundamentally, the conclusions are that what distinguishes a good speech
from a bad one is whether or not it is based on knowledge (and also
whether it is framed appropriately for its audience); and that one
should never be too attached to what one has written, or said, because
one should always be able to {\em improve} on it -- something which
itself seems to be an essential feature of philosophical dialogue
(conversation) as Socrates understands it. So the dialogue ends as it
began, on the subject of {\em logoi}; but now we have moved a world away
from the supposedly virtuosic displays of speakers like Lysias, in which
truth plays no role, to ‘talk' of a quite different kind. Accordingly,
in the dying moments of their conversation, Socrates instructs Phaedrus
to convey their second main conclusion, about a speaker's proper
attitude towards his speeches, to Lysias (now merely Phaedrus'
{\em friend}, no longer his ‘darling') (278e4). Phaedrus retorts by
suggesting that Socrates do the same to {\em his} friend Isocrates, thus
showing how little he has learned. As Socrates shows by his response, it
is rather their {\em other} conclusion that should be reported to
Isocrates, about the importance, for the expert orator, of knowledge. In
short, Socrates hopes that Isocrates will become a philosopher. In fact,
though Isocrates was to
become\goto{\high{7}}[introduction.htmlux5cux23pla_0000018] one of the
most outstanding and influential intellectual figures of the fourth
century BC, he did so in part in direct and open opposition to Plato,
constructing his own, rival, notion of what it was to do ‘philosophy'.
While Lysias may have been an outstanding courtroom orator, Isocrates
pioneered a whole system of education -- but through rhetoric; neither
in his voluminous writings nor in his speaking did he live up to what
Socrates claims to hope for him.

This will give a sense of the general structure of the dialogue. What it
illustrates is a typical Platonic strategy: that of starting -- or
rather, of having Socrates start -- by giving the appearance of sharing
the very premises that he means to question, and then, little by little,
clarifying the terms of the argument so that, finally, we understand
just what it was that Socrates was really saying at the beginning
(though many readers will already have had their suspicions). Thus
Socrates {\em appears}, at the beginning of the conversation with
Phaedrus, to share exactly Phaedrus' own interests; that, at any rate,
is how Phaedrus understands it, and, as I have already suggested, he
probably still half thinks the same thing at the end. But the reader is
left in no doubt that Socrates' passion is not for {\em speeches} at
all.

So in that case, one may reasonably ask, what on earth is he doing when
he gives not just one but two speeches, together taking up a fair
proportion of the whole of the {\em Phaedrus}? No reader is going to be
fooled by Socrates' repeated denials that they are his own speeches,
especially when he frequently appears in other dialogues as sponsoring
ideas that are either very like or identical to many of those that he,
or whoever it is that is supposedly inspiring him, offers us in the
second speech. It could perhaps be that the {\em Phaedrus} marks his (or
rather Plato's) {\em repudiation} of those
ideas:\goto{\high{8}}[introduction.htmlux5cux23pla_0000020] what will be
substituted, from now on, for all that talk about the soul's encounter
with True Reality, beyond the heavens, is some businesslike,
down-to-earth application of proper philosophical method, and hard
analysis. To that one might raise the objection that, later on in the
conversation with Phaedrus, Socrates will use both his speeches as
themselves {\em examples} of expert {\em logos}-making; in particular,
he will praise them for starting with a definition of their subject
({\em erôs}) in a way that Lysias' speech did not (263d -- 265c). What
is more, he will by implication identify the method they used with the
very philosophical
method\goto{\high{9}}[introduction.htmlux5cux23pla_0000022] that, he
will say, is a {\em sine qua non} for the truly expert speaker/writer.
However, given that Plato places such importance on the speaker's
knowing the truth, he cannot in any case want to write off Socrates'
speeches as actually
false:\goto{\high{10}}[introduction.htmlux5cux23pla_0000024] they are,
at worst, a brilliantly suggestive story about the nature and fate of
the soul, its desires and aspirations. Part of Socrates' own verdict on
his second speech goes like this:

\ldots{} the madness of love we said was best, and by expressing the
experience of love through some kind of simile, which allowed us perhaps
to grasp some truth, though maybe also it took us in a wrong direction,
and mixing together a not wholly implausible speech, we sang a playful
hymn in the form of a story\ldots{} (265b5 -- c1)

So perhaps the message is: Even though it was all well done, we
shouldn't take it too seriously (for I, Socrates/Plato,
don't).\goto{\high{11}}[introduction.htmlux5cux23pla_0000026] Socrates
gives no indication as to what he got right and where he went off
(maybe) in a wrong direction; but his general message is that we always
need to move on and should never be content to be identified with
anything we have written (279b -- d). We may even take the end of the
{\em Phaedrus} as expressing Plato's view on his own written
products.\goto{\high{12}}[introduction.htmlux5cux23pla_0000028] Indeed
it is hard, though many have
tried,\goto{\high{13}}[introduction.htmlux5cux23pla_0000030] not to take
it this way, so insistent does Socrates appear to be on applying his
conclusion about the value of speech-making and speech-writing to
{\em every} genre of writing.

If that is the case, however, it will also be hard to suppose that he
means to be disowning any particular ideas in the speech. What the
argument of the {\em Phaedrus} is here pointing to is perhaps just the
danger of relying on any medium that does not allow for questioning,
challenge and the prospect of progress to a better understanding.
Socrates' description of his own performance, that it ‘allowed us
perhaps to grasp some truth, though maybe also it took us in a wrong
direction', will in that case apply to any {\em logos}, if taken just by
itself; and it will apply even to the {\em best} kind -- which, I myself
propose, the combination of Socrates' two speeches in the {\em Phaedrus}
is ultimately designed to exemplify. ‘If you are going to make speeches,
do it this way,' Plato will be saying (‘but always beware of taking your
products too seriously; the truth is too important for that'). And as a
matter of fact, the Socrates of the {\em Phaedrus} seems to be committed
to saying that there will be occasions for speech-making; for how else
are we to take his proposals for a reformed rhetoric, based on truth and
adapted to the specific nature of the audience?

\reference[introduction.htmlux23pha0000072]{}%
Since the power of speech is in fact a leading of the soul, the man who
means to be an expert in rhetoric must know how many forms soul has.
Thus their number is so and so, and they are of such and such kinds,
which is why some people are like this, and others like that; and these
having been distinguished in this way, then again there are so many
forms of speeches, each one of such and such a kind. People of one kind
are easily persuaded for one sort of reason by one kind of speech to
hold one kind of opinion, while people of another kind are for some
other sorts of reasons difficult to persuade.

Having then grasped these things satisfactorily, after that the student
must observe them as they are in real life, and actually being put into
practice, and be able to follow them with keen perception, or otherwise
be as yet no further on from the things he heard earlier when he was
with me. But when he both has sufficient ability to say what sort of man
is persuaded by what sorts of things, and is capable of telling himself
when he observes him that {\em this} is the man, {\em this} the nature
of person that was discussed before, now actually present in front of
him, to whom he must now apply {\em these} kinds of speech in {\em this}
way in order to persuade him of {\em this} kind of thing; when he now
has all of this, and has also grasped the occasions for speaking and for
holding back, and again for speaking concisely and piteously and in an
exaggerated fashion, and for all the forms of speeches he may learn,
recognizing the right and the wrong time for these, {\em then} his grasp
of the science will be well and completely finished, but not before
that; but in whichever of these things someone is lacking when he speaks
or teaches or writes, and says that he speaks scientifically, the person
who disbelieves him is in the stronger position. (Socrates at 271c10 --
272b2; put in the mouth of an imagined writer of the ideal rhetoric
textbook)

Fitting what he says to the nature of his hearers: this might not be a
natural role for Socrates himself, who is happiest (so he might claim)
telling things
straight.\goto{\high{14}}[introduction.htmlux5cux23pla_0000032] Yet
here, on this occasion, with Phaedrus, circumstances contrive to make
him take the role on, and -- as he says -- his speech is ‘forced to use
somewhat poetical language {\em because of Phaedrus}': is that, in part,
because he thinks that kind of language appropriate to the nature of
Phaedrus' soul?\goto{\high{15}}[introduction.htmlux5cux23pla_0000034]
The subsequent discussion certainly shows Phaedrus to be less than the
ideal recipient for straight (philosophical)
talking.\goto{\high{16}}[introduction.htmlux5cux23pla_0000036]

Thus, on the account I myself prefer, Socrates' second speech is a
statement of things that he is -- in some form or other -- supposed to
believe, even if he will stake nothing on
it.\goto{\high{17}}[introduction.htmlux5cux23pla_0000038] This will help
to explain its sheer {\em length}: among other things, it gives us a
sketch of the kind of view of the whole -- the ‘babbling and lofty talk
about nature' (270a1) that Socrates will propose as a requirement of any
science. In that sense, it will do no harm to treat the speech as a kind
of compendium of Platonic ideas, a role to which it is, up to a point,
ideally suited. There are, however, two dangers about this. The first,
which by now speaks for itself, is that it will have become detached
from the warnings Plato has so carefully attached to it: ‘{\em only} a
sketch'; ‘no detail to be pressed'; ‘right in a way, perhaps, but maybe
also misleading'.\goto{\high{18}}[introduction.htmlux5cux23pla_0000040]
It will then suffer the fate that Socrates predicts for all written
documents:

\reference[introduction.htmlux23pha0000077]{}%
\ldots{} when once it is written, every composition trundles about
everywhere in the same way, in the presence both of those who know about
the subject and of those who have nothing at all to do with it, and it
does not know how to address those it should address and not those it
should not. When it is ill treated and unjustly abused, it always needs
its father to help it; for it is incapable of either defending or
helping itself. (275d9 -- e5)

‘Ill treatment' or ‘abuse', in this case, will be a matter of treating
the speech as an account of Platonic {\em doctrine} (an approach that
would misrepresent Plato himself, not just the palinode of the
{\em Phaedrus}: Plato no doubt has plenty of firmly held convictions,
but he is perfectly serious -- as any reading across his dialogues will
show -- about wanting in every case to leave room for re-statement and,
in general, for progress). But there is also a second danger involved in
lifting Socrates' second speech from its context in the {\em Phaedrus}
as a whole. This is that we are likely to forget, or play down, two
facts about it: firstly, that it is intended also, and in the first
place, as an account of love ({\em erôs}), in response to Lysias' own;
secondly, and more importantly, that we are likely to miss that this
account of Socrates', in responding to Lysias', is full of deliberate
{\em paradox}. Lysias set out to dazzle by taking on an absurd and even
self-contradictory thesis; Socrates out-dazzles him, first by treating
the philosopher, that exemplar (surely) of rationality, as {\em mad},
then -- even more extraordinarily -- by giving Phaedrus, and us, an
ideal {\em erôs} that seems to overturn the whole idea of {\em erôs.
(Erôs} without sex? What on earth next? Isn't sex what {\em erôs} is
ultimately about?)

There is, however, a difference between Lysianic paradox and Socratic
paradox. Lysianic paradox merely allows Lysias to appear clever.
Socratic paradoxes, by contrast, even while being {\em para doxan}
(literally ‘contrary to belief' -- i.e. contrary to all our
expectations), are things Socrates actually believes are
{\em true}.\goto{\high{19}}[introduction.htmlux5cux23pla_0000042] That
is, they give us ways of seeing how the world actually is, without
excluding the possibility of other, complementary, ways of seeing it. So
the philosopher {\em is} crazy, not because most people think he is (on
the superficial grounds that he is not interested in the things that
interest them), but because he obsesses about truth with the same
intensity that an ordinary lover obsesses about his darling. And ideal
{\em erôs must} be without sex, if it is -- ideally -- a passion for
Beauty (as the speech suggests), not for the bit of beauty that happens
to be present in this particular body (and soul), since sex with this
body will merely distract from the search for Beauty, about which the
ideal beloved will be equally
obsessive.\goto{\high{20}}[introduction.htmlux5cux23pla_0000044]

We may, if we like, call this a kind of appropriation. But it can also
be seen as part of that Platonic strategy I described earlier: appearing
to start where others are, when actually, all the time, being somewhere
else. Just as Socrates seemed, at any rate to Phaedrus, to be sharing
Phaedrus' enthusiasm for speech-making while actually being focused on
something else, so he gives his first speech as if wholly endorsing
Lysias' view of {\em erôs} as culpable madness, and as being concerned
solely with sexual fulfilment. But when we reach his later assessment of
his two speeches, we find that he was not actually doing that at all,
for the first speech was, all along, just a description of one part of
{\em erôs} and one part of madness. It is not that he has changed his
mind about that first speech: that he speaks with his head covered shows
that he {\em already} knows what he is doing -- i.e. preparing to give a
one-sided (‘impious') account of {\em erôs}, as if that were the whole
of it.\goto{\high{21}}[introduction.htmlux5cux23pla_0000046] For him,
the lover is actually mad, only not -- if he's the ideal lover -- in the
way Lysias thinks; and the lover is actually obsessed with sexual
fulfilment, but only if he's a non-ideal
lover.\goto{\high{22}}[introduction.htmlux5cux23pla_0000048]

There is another extended example of the same
phenomenon\goto{\high{23}}[introduction.htmlux5cux23pla_0000050] in the
second speech. Towards the end, Socrates embarks on a vivid description
of the sprouting of feathers all over the surface of the soul of the
ideal lover, as he observes his beautiful beloved, and remembers -- or
half-remembers -- Beauty Itself, which he once glimpsed, at some divine
feast in the
heavens\ldots{}\goto{\high{24}}[introduction.htmlux5cux23pla_0000052]
This passage (250c -- 252c) is the most deliberately {\em erotic} in the
whole dialogue, evoking at every step ordinary -- male -- sexual
excitement, from the first stirrings of arousal through to ejaculation,
along with the behaviour of the ordinary, common-or-garden lover. Now
this is, to be sure, in part a kind of playfulness. But it also has a
serious point: it is not merely the case (so I suggest Plato is saying)
that there is a general kind of parallel between erotic passion and the
love of wisdom, but that the actual experience of encountering some
aspect, some dimly realized part of the truth in the company of someone
else,\goto{\high{25}}[introduction.htmlux5cux23pla_0000054] is just like
that; it makes one's {\em hair stand on end}, as we might put it, only
Plato chooses to put it more graphically. ‘Intercourse' with the objects
of knowledge, in other words, is no {\em mere} metaphor, metaphor though
it still is. In this way he makes the strong claim that in denying
himself ordinary sexual gratification, the philosopher actually denies
himself nothing, for when it comes to it he will get something that is
the same, only better. It is not so much that {\em erôs} has been
de-sexualised; rather, sex has been injected into
philosophy.\goto{\high{26}}[introduction.htmlux5cux23pla_0000056]

Much ink has been
spilled\goto{\high{27}}[introduction.htmlux5cux23pla_0000058] on the
question of the {\em unity} of the {\em Phaedrus}. At one point Socrates
expresses his view, and expects Phaedrus to accept, ‘that every speech
should be put together like a living creature, as it were with a body of
its own, so as not to lack either a head or feet, but to have both
middle parts and extremities, so written as to fit both each other and
the whole' (264c2 -- 5); and yet the {\em Phaedrus} itself has seemed,
to many readers, to have nothing at all organic about it. In particular,
the second part of the dialogue (from the end of Socrates' second
speech) has seemed insufficiently connected with the first. However, if
what I have said in the preceding paragraphs takes us at least broadly
in the right direction, the problem largely disappears. True, the second
part has nothing directly to say about erôs, which appears to be the
main subject of the first part; and the second part has little or
nothing of the wit and subtlety that adorn the first -- again true. But
it is no less true to say that the dialogue remains focused throughout
on the subject of {\em logoi} (‘speeches', ‘speaking', ‘talk' and so
on), and that erôs, as Socrates understands it, is itself all about
{\em logoi} -- the ‘talk' that leads us to the ultimate object of
{\em erôs} (Beauty or, more generally, knowledge). But this means that
the choice facing Phaedrus, and us, is not just between different kinds
of talk but between different kinds of {\em life}.

In other words, the real subject of the {\em Phaedrus} is pretty much
that of the {\em Symposium} (and also that of the {\em Lysis}): the
nature of human motivation. However, the Socrates of the {\em Phaedrus}
takes a radically different view on this subject from his counterpart in
the {\em Symposium}. In the {\em Symposium}, there are no irrational
parts to the soul: all desire is for the (real) good, and the only
difference between individuals has to do with their beliefs about what
that good consists in. (Extraordinarily, nowhere in the account of
{\em erôs} that Socrates offers on the authority of the priestess
Diotima is there any mention, or even hint, of irrational desire; we are
assumed to love good and beautiful things, and to go for whatever we
identify as good and
beautiful.\goto{\high{28}}[introduction.htmlux5cux23pla_0000060]) In the
{\em Phaedrus}, however, the soul is divided into three parts, as in the
{\em Republic}: one rational (which Socrates compares to the charioteer
of a two-horse team) and two irrational (the two horses, a white one on
the right, and a black one on the left).

Socrates may even implicitly acknowledge this difference in the
{\em Phaedrus} when he replies to a question Phaedrus asks him about his
attitude to the old myths:

\reference[introduction.htmlux23pha0000086]{}%
... I am not yet capable of ‘knowing myself', in accordance with the
Delphic inscription; so it seems absurd to me that while I am still
ignorant of this subject I should inquire into things which do not
belong to me. So then saying goodbye to these things, and believing what
is commonly thought about them, what I was saying just now, I inquire
not into these but into myself, to see whether I am actually a beast
more complex {[}i.e divided?{]} and more typhonic than Typhon, or both a
tamer and a simpler {[}i.e. unified, rational?{]} creature, sharing some
divine and un-Typhonic portion by nature. (229e6 -- 230a6)

However, to see this as a reference to the two alternative models of
soul or mind is somewhat
speculative,\goto{\high{29}}[introduction.htmlux5cux23pla_0000062] not
least because the {\em Phaedrus} will later distinguish on its own terms
between simple and complex souls: (probably) those which are
predominantly rational, and those in which ‘rule' is uncertainly shared
by the different parts (277c (cf. 269d -- 272b)). What is beyond
reasonable doubt, or so I hold, is that the {\em Symposium} takes
seriously the idea that there is only one object of desire -- our good,
and our happiness -- so that there is no room for conflict between
desire and reason. (If we go off in different directions, that is not
because of our {\em desires} but because of our decisions about which
way to go, based upon our beliefs plus our universal desire for the
good.)\goto{\high{30}}[introduction.htmlux5cux23pla_0000064] Meanwhile,
beyond {\em any} doubt, the {\em Phaedrus} proposes a view of things
that makes conflict endemic, at least in any erotic context. Here
Socrates recognizes distinct elements in us that naturally tend in
different directions: appetite pulls towards immediate pleasure, and
excess, while reason pulls away from that towards restraint, under the
guidance of reflections about what is best.

This is the schema on which Socrates builds his first
speech;\goto{\high{31}}[introduction.htmlux5cux23pla_0000066] in the
second, he adds the white horse as the instinctive, but not wholly
reliable, ally of the charioteer of
reason.\goto{\high{32}}[introduction.htmlux5cux23pla_0000068] It is this
radical division of the soul, and its attribution to the soul of more
than one object of desire, that (so one supposes) causes souls to be
‘complex', or ‘variegated' ({\em poikilos}) (277c2), and makes the new
rhetoric -- which knows how to persuade different types of soul -- so
important. Socrates may have preferred to tell things straight; on the
{\em Symposium} model, it also makes sense for him to do so, for on that
model we are all essentially rational beings. What we do is determined
by our beliefs, and what other way is there of reliably changing our
beliefs than trying (rationally) to persuade us? But the presence, or
arrival, of the black horse changes everything. In the image, it is only
a mixture of force and habituation that will finally calm him down and
stop him demanding immediate gratification. This is why even the ideal
orator will need to be versatile: to ‘grasp the occasions for speaking
and for holding back, and again for speaking concisely and piteously and
in an exaggerated fashion, and for all the forms of speeches he may
learn, recognizing the right and the wrong time for these' (272a4 -- 7).
In short, in addition to the truth he will need all those special skills
that ordinary orators
have.\goto{\high{33}}[introduction.htmlux5cux23pla_0000070] Once again,
the topics of ‘speech(es)' and love, {\em logos} and {\em erôs}, turn
out to be vitally connected.

\subsection[title={NOTES},reference={introduction.htmlux23pha0000090}]

\goto{1}[introduction.htmlux5cux23pla_0000005]. In order to refer to
particular passages in the dialogue, this volume uses -- as do all
modern translations and editions -- the page numbers and page sections
(usually five, marked a -- e) as fixed by the Stephanus edition of
Plato, dating from the Renaissance. (Line numbers vary between different
editions of the Greek text; the ones given in this volume are those of
Burnet's 1903 Oxford edition.) Thus ‘227b' in note 3 below means
‘section b of page 227 of the relevant Stephanus volume' -- as marked in
the margin of the translation.

\goto{2}[introduction.htmlux5cux23pla_0000007]. The modern reader who
hears the name ‘Phaedrus' is likely to think of Robert M. Pirsig's
excellent {\em Zen and the Art of Motorcycle Maintenance: An Inquiry
into Value} (first published in Britain by The Bodley Head Ltd in 1974).
But Pirsig's Phaedrus is rather more of a philosopher than Plato's; see
below.

\goto{3}[introduction.htmlux5cux23pla_0000009]. I here follow Nails
2003, Appendix 1, abandoning the view that the dialogue has no possible
dramatic date. (I endorsed the latter view in 1986, on the basis that
Lysias was still living in Thurii in the 410s; but there is no reason
why he should not have been visiting Athens, as is actually suggested by
the fact that the {\em Phaedrus} has him staying at Epicrates' house:
227b. The signs in the dialogue generally seem to point to a date not
long before 415, when Phaedrus was exiled for his alleged part in the
notorious profanation of the Eleusinian mysteries.)

\goto{4}[introduction.htmlux5cux23pla_0000011]. And not {\em literally}
his darling (236b); Phaedrus is only interested in Lysias' speeches.

\goto{5}[introduction.htmlux5cux23pla_0000013]. From Sappho or Anacreon,
or some prose writers (235c) ? Or because he is possessed by the nymphs
of the place (madness again: 238d, 241e) ?

\goto{6}[introduction.htmlux5cux23pla_0000015]. Once again Socrates does
his best to disclaim responsibility for what he says. And indeed
speeches of such length are not his usual style, which is rather
question-and-answer, with himself typically in the questioning role --
in other words, ‘dialogue', or ‘conversation' ({\em dialegesthai}). In
the second half of the {\em Phaedrus} he is himself again (in dialogue).

\goto{7}[introduction.htmlux5cux23pla_0000017]. And indeed had become,
by the time of writing of the {\em Phaedrus} (perhaps in the third
decade of the fourth century?).

\goto{8}[introduction.htmlux5cux23pla_0000019]. This is essentially the
view of Nehamas and Woodruff (in the Introduction to the Hackett
translation of the {\em Phaedrus}, 1995).

\goto{9}[introduction.htmlux5cux23pla_0000021]. I.e. ‘collection and
division', which means, roughly, finding the genus under which the thing
to be defined falls, then dividing the genus into its species, until one
reaches the {\em definiendum}, the thing to be defined, itself: so
{\em erôs} belongs to the genus madness, but its true nature only
emerges (so the second speech makes clear) when one knows what
{\em species} of madness it is.

\goto{10}[introduction.htmlux5cux23pla_0000023]. The first {\em is}
initially represented as false -- but only insofar as it pretends, and
actually pretended (for a moment), to give the whole of what love is.
When taken together with the second speech (as in 263 -- 5), the first
-- so Socrates finally suggests -- gives part of the truth.

\goto{11}[introduction.htmlux5cux23pla_0000025]. And no reader of the
speech is likely to miss the fact that the speech is by turns serious,
witty and playful, and sometimes both at once (when irony -- always one
of Plato's favourite weapons -- sets in).

\goto{12}[introduction.htmlux5cux23pla_0000027]. Some (one of whom is
Szlezák 1999) have seen Plato here as signalling to, or reminding, an
inner circle of readers, perhaps in or around the newly founded Platonic
Academy, that they needed to go behind the written dialogues, to
whatever was in Plato's {\em oral} teaching. This remains a minority
interpretation.

\goto{13}[introduction.htmlux5cux23pla_0000029]. E.g.by stressing the
un-speech-like character of Plato's dialogues; yet the poets are
certainly included, and some of them (the ones who wrote plays) also
used dialogue.

\goto{14}[introduction.htmlux5cux23pla_0000031]. Cf. {\em Gorgias} 521d
-- 522a, a famous passage in which he identifies himself as perhaps the
only true statesman in existence, precisely on the grounds that he tells
people the truth whether they like it or not. The theory of rhetoric in
the {\em Phaedrus} might -- though it need not -- be taken as
identifying ways of persuading people of the truth {\em without}
upsetting them.

\goto{15}[introduction.htmlux5cux23pla_0000033]. In that case, I take
it, Phaedrus' will be a ‘complex' soul, which requires ‘complex speeches
containing all the modes' (277c2 -- 3): see further below.

\goto{16}[introduction.htmlux5cux23pla_0000035]. Why should Plato
portray Socrates faced with so unreceptive and unpromising an
interlocutor? The suggestion I find most attractive is one put forward
by Jonathan Lear in a session (on, as it happened, Blondell 2002) at the
2004 meeting of the Pacific Division of the American Philosophical
Association: namely (if I may so summarize Lear's view) that as we
observe Phaedrus' superficial relationship with {\em logoi} (one which
puts him on a par with the ‘lovers of sights and sounds' at the end of
{\em Republic} V), we are, as a result, meant to be inoculated against
taking a similarly limited view of things ourselves.

\goto{17}[introduction.htmlux5cux23pla_0000037]. I.e. because -- to be
true to his principles, as later enunciated -- he would be content to be
asked to reconsider (try to improve on) any part of it; and also to the
extent that the statement may be specifically formulated {\em for
Phaedrus} (cf. n. 15 above). However, a passage like 271c -- 272b, cited
in the text above (‘Since the power of speech is in fact a leading of
the soul, the man who means to be an expert in rhetoric must know how
many forms soul has', and so on), nowhere suggests that the ideal orator
will ever {\em lie} to his audience.

\goto{18}[introduction.htmlux5cux23pla_0000039]. We shall also tend to
miss the {\em wit} in Socrates' performance (see n. 11 above); compendia
might be expected to be read po-faced. (As a matter of fact, many modern
readers, and probably many ancient ones too, {\em have} missed this
aspect of the speech -- a singular loss.)

\goto{19}[introduction.htmlux5cux23pla_0000041]. Only up to a point,
then, is Socratic paradox itself a matter of play; or perhaps, in the
end, it is not play at all.

\goto{20}[introduction.htmlux5cux23pla_0000043]. Thus, to spell it out:
bizarrely, what the ideal lover is in love with is not actually the one
he calls his ‘beloved' but something else. The same sort of idea appears
both in the {\em Symposium} and in the {\em Lysis}; it is not something
invented just for the occasion of the {\em Phaedrus} -- that is, if we
suppose, as most do, that the {\em Phaedrus} was written later than at
least one of the other two.

\goto{21}[introduction.htmlux5cux23pla_0000045]. Of course he has an
excuse ready for covering his head: Phaedrus will scoff at his
amateurism. (But why should we accept that explanation from someone who
has already shown no qualms about expressing firm views on the failings
of Lysias' effort?)

\goto{22}[introduction.htmlux5cux23pla_0000047]. In Socrates' second
speech, even those pairs who {\em occasionally} give in to the black
horses of appetite (i.e. plain lust) will be disqualified from the prize
available to those who never do (256b -- e).

\goto{23}[introduction.htmlux5cux23pla_0000049]. Or, to put it more
carefully, a similar one.

\goto{24}[introduction.htmlux5cux23pla_0000051]. The soul's
wings/feathers are what will, or may, ultimately carry this soul up, and
back, to Beauty.

\goto{25}[introduction.htmlux5cux23pla_0000053]. It is not clear whether
the experience depends on the presence of someone else; but then
Socrates is specifically describing relationships.

\goto{26}[introduction.htmlux5cux23pla_0000055]. Much here depends on
how we take 252b1 -- 9, which I translate this way: ‘This experience, my
beautiful boy, the one to whom my speech is addressed, men term love;
but when you hear what gods call it I expect you will laugh, because of
your youth.' I think some Homeric experts cite two verses to Love from
the less well-known poems, the second of which is quite outrageous
({\em hubristikon}) and not very metrical; they celebrate Love like
this: ‘We mortals call him Mighty Love, a winged power of great renown,
/ Immortals call him Fledgeling Dove -- since Eros' wings lack down.' By
contrast, Nehamas and Woodruff 1995 give ‘less well known poems, of
which the second {\em is quite indecent} (my italics) and does not scan
very well. They praise Love this way: “Yes, mortals call him powerful
winged ‘Love'; / But because of his need to thrust out his wings, the
gods call him ‘Shove'”~' -- explaining in a footnote that ‘the indecency
is in the word {\em pterophutôr} (“wing-thrusting”)'. If they are right,
Plato will presumably be drawing attention to the ‘obscenity' of the
preceding description, which the two made-up verses round off; that
description will, then (so far as I can see), merely be teasing, even
pornographic. However, there is nothing inherently obscene or indecent
about {\em pterophutôr}, which simply means ‘wing-sprouting'. Rather, I
suggest, Plato teases us by merely {\em appearing}, momentarily, to be
about to bring in something obscene (this with the adjective
{\em hubristikon}: since Socrates' first speech,{\em hubris} has been
specifically associated with lust), and then immediately explaining the
‘outrage', or ‘excess', as mere metrical ‘outrage'.

\goto{27}[introduction.htmlux5cux23pla_0000057]. By myself, among
others: see Further Reading.

\goto{28}[introduction.htmlux5cux23pla_0000059]. Cf. Nehamas and
Woodruff 1995, Introduction, p. xxxix. (It seems to me wrong, however,
to describe the {\em Symposium} as saying that ‘(a) n undivided soul,
all of it always desiring {\em what it considers best}, is subject to no
such conflicts {[}sc. between appetite and reason{]}': what every soul
wants, according to Diotima, is presumably real, not merely apparent,
happiness.)

\goto{29}[introduction.htmlux5cux23pla_0000061]. As would be the
suggestion that it might have been partly this change of view that led
Plato to write another dialogue about {\em erôs}, although I am myself
content to speculate in that way. (It is around here that I finally lose
any sympathy with the view, still held in one or two quarters, that the
{\em Phaedrus} could have been Plato's first dialogue. This proposal
stems especially from the point that the dialogue is, in a way, or can
be seen as, {\em programmatic}; and shouldn't programmes come before,
rather than during, the event? However, there are just too many of what
look like cross-references to other dialogues; not to see them as such
seems to imply a Plato who, implausibly, began with most of his most
characteristic ideas already fully formed in his head.

\goto{30}[introduction.htmlux5cux23pla_0000063]. See e.g.
{\em Symposium} 205d1 -- 7, with surrounding context.

\goto{31}[introduction.htmlux5cux23pla_0000065]. I.e. even before we
reach the simile of the charioteer and his horses, which dresses up the
same ideas in more colourful clothes.

\goto{32}[introduction.htmlux5cux23pla_0000067]. The white horse
corresponds to the ‘spirited', or {\em thumoeides}, part as described in
the {\em Republic}; by comparison, it is signally under-described in the
{\em Phaedrus}, which may or may not be a reason for suspecting an
intertextual reference.

\goto{33}[introduction.htmlux5cux23pla_0000069]. For the full
significance of the difference between what I here identify as the
{\em Symposium} and {\em Phaedrus} models of the soul, see Rowe in
Reshotko (ed.) 2003, and Penner and Rowe forthcoming. (The {\em Lysis}
contains the fullest account anywhere in the dialogues of the
‘{\em Symposium} model'.) To the modern mind, as indeed to Aristotle's,
the {\em Phaedrus} model is the obvious choice between the two: see e.g.
Price 1995 and 1997, and Lear 1993. Penner and Rowe forthcoming,
however, will take a different line.

\section[title={Further
Reading},reference={bibliography.htmlux23pha0000124}]

Annas, Julia, and Christopher Rowe (eds), {\em New Perspectives on
Plato, Modern and Ancient} (Cambridge, MA: Harvard University Press,
2002)

Blondell, Ruby, {\em The Play of Character in Plato's Dialogues}
(Cambridge: Cambridge University Press, 2002). Thought-provoking on the
general issue of the relationship between the philosophical, literary
and dramatic elements in the dialogues, though has disappointingly
little to say about the {\em Phaedrus} itself.

Burger, Ronna, {\em Plato's Phaedrus: A Defense of a Philosophic Art of
Writing} (Birmingham: University of Alabama Press, 1980

Burnet, John, {\em Platonis opera}, 5 vols. (Oxford Classical Texts),
(Oxford: Clarendon Press, 1903), 11

Burnyeat, Myles F. ‘Socratic Midwifery, Platonic Inspiration',
{\em Bulletin of the Institute of Classical Studies} 24 (1977), pp. 7 --
17; reprinted in Hugh H. Benson (ed.), {\em Essays on the Philosophy of
Socrates} (Oxford: Oxford University Press, 1992), pp. 53 -- 65

Cole, Thomas, {\em The Origins of Rhetoric in Ancient Greece}
(Baltimore: Johns Hopkins University Press, 1991)

Cooper, John M., {\em Plato: Complete Works} (Indianapolis: Hackett,
1997)

de Vries, G. J., {\em A Commentary on the Phaedrus of Plato} (Amsterdam:
Hakkert, 1969)

Derrida, Jacques, ‘Plato's Pharmacy' = {\em Dissemination}, trans.
Barbara Johnson (London: Athlone Press, 1981), pp. 61 -- 171. An essay
that in sum -- despite its influence among some readers of the
{\em Phaedrus} -- tells us rather more about Derrida than about the
{\em Phaedrus}, or about Plato; see Seán Burke, {\em The Death and
Return of the Author}, second edn (Edinburgh: Edinburgh University
Press, 1998), chap. 3

Dover, K. J., {\em Greek Homosexuality}, second edn (Cambridge, MA:
Harvard University Press, 1986)

Ferrari, G. R. F., {\em Listening to the Cicadas: A Study of Plato's
Phaedrus} (Cambridge Classical Studies), (Cambridge: Cambridge
University Press, 1987)

---------, ‘Platonic Love', in Richard Kraut (ed.), {\em The Cambridge
Companion to Plato} (Cambridge: Cambridge University Press, 1992), pp.
248 -- 76

Fine, Gail, {\em Plato I} and {\em II} (Oxford: Oxford University Press,
1999). A useful general collection of articles on Plato, including one
by Richard Bett, ‘Immortality and the Nature of the Soul in the
{\em Phaedrus}', which helps analyse the {\em Phaedrus}'s argument for
the soul's immortality.

Gill, Christopher, ‘Platonic Love and Individuality', in A. Loizou and
H. Lesser (eds), {\em Polis and Politics: Essays in Greek Moral and
Political Philosophy} (Aldershot: Avebury Series in Philosophy, 1990),
pp. 69 -- 88

------ (trans.), {\em Plato: Symposium} (Harmondsworth: Penguin Books,
1999). The companion to the present volume.

Griswold, Charles L., Jr, {\em Sélf-Knowledge in Plato's Phaedrus} (New
Haven: Yale University Press, 1986). New paperback edition published by
Pennsylvania State University Press, University Park, PA, in 1996

Hackforth, R., {\em Plato's Phaedrus} (Cambridge: Cambridge University
Press, 1932). Translation and commentary.

Halliwell, Stephen ‘Forms of Address: Socratic Vocatives in Plato', in
F. de Martino and A. H. Sommerstein (eds), {\em Lo spettacolo delle
voci} (Bari: Laterza, 1995), Pt 2, pp. 87 -- 121

Halperin, David M., ‘Plato and Erotic Reciprocity', {\em Classical
Antiquity} 5 (1986), pp. 60 -- 80

---, ‘Plato and the Metaphysics of Desire', {\em Proceedings of the
Boston Area Colloquium for Ancient Philosophy} 5 (1989), pp. 27 -- 52

------, ‘Platonic {\em Erôs} and What Men Call Love', {\em Ancient
Philosophy} 5 (1985), pp. 161 -- 204

Hamilton, Walter (trans.), {\em Plato, Phaedrus & Letters VII and VIII}
(Harmondsworth: Penguin Books, 1973). The precursor of the present
volume. The latter excludes the two letters not just because of doubts
about their authenticity (in fact I regard them both as certainly
spurious), but because to include them tends to suggest that the, or a,
primary focus of the {\em Phaedrus} too is somehow Plato himself -- when
he has done his level best to make himself invisible. ‘The seventh and
eighth letters\ldots{} provide fascinating glimpses into the
contemporary power struggle in Sicily and evidence his failure to put
into practice his theory of the philosopher-king' (Hamilton, on the back
cover of the 1973 Penguin): but even if Plato had a ‘theory of the
philosopher-king', that he ever seriously contemplated trying to ‘put
{[}it{]} into practice' in Syracuse, and in the person of Dionysius II,
seems in the highest degree implausible. However, the problems with the
seventh letter do not start, or end, here.

Heath, Malcolm, ‘The Unity of Plato's {\em Phaedrus', Oxford Studies in
Ancient Philosophy 7} (1987), pp. 150 -- 73, 189 -- 91. See also Rowe
1987 below.

Irwin, Terence, {\em Plato's Ethics} (Oxford: Oxford University Press,
1995)

Janaway, Christopher, {\em Images of Excellence: Plato's Critique of the
Arts} (Oxford: Clarendon Press, 1995)

Kahn, Charles H., ‘Plato's Theory of Desire', {\em Review of
Metaphysics} 41 (1987), pp. 77 -- 103

Kennedy, George, {\em The Art of Persuasion in Greece} (Princeton:
Princeton University Press, 1963)

Kosman, L. A., ‘Platonic Love', in W. Werkmeister (ed.), {\em Facets of
Plato's Philosophy (Phronesis} Supplementary Volume 2) (Assen: Van
Gorcum, 1976), pp. 53 -- 69

Lear, Jonathan, ‘Plato's Politics of Narcissism', in Terence Irwin and
Martha C. Nussbaum (eds), {\em Virtue, Love and Form: Essays in Memory
of Gregory Vlastos, Apeiron} 26/3 -- 4 (1993), pp. 137 -- 59

Mackenzie, Mary Margaret, ‘Paradox in Plato's {\em Phaedrus',
Proceedings of the Cambridge Philological Society} N. S. 28 (1982), pp.
64 -- 76

Morgan, Kathryn, ‘Socrates and Gorgias at Delphi and Olympia:
{\em Phaedrus} 235d6 -- 236b4,' {\em Classical Quarterly}, N. S. 44/2
(1994), pp. 375 -- 86

Moravcsik, Julius, and Philip Temko, {\em Plato on Beauty, Wisdom and
the Arts} (Totowa, NJ: Rowman and Littlefield, 1982)

Nails, Debra, {\em The People of Plato} (Indianapolis: Hackett 2003). A
complete account of what we can reconstruct about the people who figure
in Plato's dialogues; indispensable.

Nehamas, Alexander, and Paul Woodruff (trans.), {\em Plato, Phaedrus}
(Indianapolis: Hackett, 1995)

Nightingale, Andrea W., {\em Genres in Dialogue: Plato and the Construct
of Philosophy} (Cambridge: Cambridge University Press, 1995)

Nussbaum, Martha C., {\em The Fragility of Goodness: Luck and Ethics in
Greek Tragedy and Philosophy} (Cambridge: Cambridge University Press,
1986)

------, and Juha Sihvola (eds), {\em The Sleep of Reason: Erotic
Experience and Sexual Ethics in Ancient Greece and Rome} (Chicago:
University of Chicago Press, 2002)

Osborne, Catherine, {\em Eros Unveiled: Plato and the God of Love}
(Oxford: Clarendon Press, 1994)

Pender, E. E., {\em Images of Persons Unseen: Plato's Metaphors for the
Gods and the Soul} (Sankt Augustin: Academia Verlag (International
Studies 11), 2000)

Penner, Terry, {\em The Ascent from Nominalism: Some Existence Arguments
in Plato's Middle Dialogues} (Dordrecht: Reidel (Philosophical Studies
Series 37), 1987)

------, ‘Socrates and the Early Dialogues', in Richard Kraut (ed.),
{\em The Cambridge Companion to Plato} (Cambridge: Cambridge University
Press, 1992), pp. 121 -- 69. An excellent short account of the Socratic
theory of desire and action from which the {\em Phaedrus} seems
radically to distance itself.

------, and Christopher Rowe, {\em Plato's Lysis} (Cambridge: Cambridge
University Press, forthcoming). An interpretation of a puzzling dialogue
-- one of the ‘early' ones referred to in the title of the preceding
item -- that deals with some of the main topics of both the
{\em Phaedrus} and the {\em Symposium}.

Price, A. W., {\em Love and Friendship in Plato and Aristotle} (Oxford:
Clarendon Press, 1997). With important Afterword added to original 1989
edition.

------, {\em Mental Conflict} (London: Routledge (Issues in Ancient
Philosophy), 1995)

Rowe, Christopher J., ‘The Argument and Structure of Plato's
{\em Phaedrus', Proceedings of the Cambridge Philological Society} 32
(1986), pp. 106 -- 25

------, ‘Philosophy, Love, and Madness', in Christopher Gill (ed.),
{\em The Person and the Human Mind} (Oxford: Clarendon Press, 1990), pp.
227 -- 46

------, {\em Plato}, second edn (London: Bristol Classical Paperbacks,
2003). An introduction to Plato that attempts to acknowledge the fact
that he wrote {\em dialogues}.

------, ‘Plato', in David Sedley, {\em Companion to Greek and Roman
Philosophy} (Cambridge: Cambridge University Press, 2003), pp. 98 -- 124

------, {\em Plato: Phaedrus} (Warminster: Aris & Phillips, 1986).
Translation with facing Greek text and commentary.

------, ‘Plato, Socrates and Developmentalism', in Naomi Reshotko (ed.),
{\em Desire, Identity and Existence: Essays in Honour of T. M. Penner}
(Kelowna, BC: Academic Printing and Publishing, 2003)

------, {\em Plato: Symposium} (Warminster: Aris & Phillips, 1996).
Translation with facing Greek text and commentary.

---, ‘The Unity of the {\em Phaedrus}: A Reply to Heath', {\em Oxford
Studies in Ancient Philosophy} 7 (1987), pp. 175--88

Santas, Gerasimos X., {\em Goodness and Justice} (Oxford: Blackwell,
2001)

------, {\em Plato and Freud} (Oxford: Blackwell, 1988) Szlezák, Thomas,
{\em Reading Plato} (London: Routledge, 1999)

Vlastos, Gregory, {\em Socrates: Ironist and Moral Philosopher} (Ithaca,
NY: Cornell University Press, 1991)

White, David A., {\em Rhetoric and Reality in Plato's Phaedrus} (Albany:
State University of New York Press, 1993)

\section[title={A Note on the Text and
Translation},reference={preface003.htmlux23pha0000181}]

The Greek text translated in this volume is the same -- barring some
further changes to punctuation -- as that translated in Rowe,
{\em Plato: Phaedrus} (1986) -- i.e. Burnet's Oxford text with a number
of modifications.

The translation itself is also fundamentally the same as in the 1986
volume, but a close comparison will show that there have been numerous
small-scale changes. The original version was written to go with a
facing Greek text and a commentary. It was designed partly to help
readers of the original find their way through what is frequently
{\em difficult} Greek, partly to enable those without Greek to see
something of the structure of the original. Further, with a commentary
in support, the translation could sometimes share with it the job of
communicating the sense of Plato's text to the reader. A translation
that stands by itself -- even with endnotes -- needs to be a creature of
a rather different, and more independent, sort. The new translation is
also more like English than much of the 1986 version. However, I have
still tried to keep something of the shape of the Greek, not least
because in the course of the dialogue Plato adopts several different
styles, which need to be presented to the reader if he or she is to have
any chance of grasping properly what exactly Socrates -- the main
character -- is up to at any given point, and in particular what
{\em tone} he is using. Adapting the Greek, for example, to standard
English sentence lengths, tends to obscure such variations. I make no
apology, then, if the translation sometimes does not read quite like
ordinary English prose. And it would be pointless to {\em apologize} if
the English fails by a long way to reproduce the brilliance of Plato's
language; anyone who wants properly to appreciate that had better learn
to read Greek.

I add, finally, that the translation, along with the Introduction and
the notes, builds at every stage on my commentary in the 1986 volume, to
which I refer for detailed justification for choices made in translation
or, more generally, in interpretation. Inevitably, there are some
issues, both particular and more general, on which I have changed my
mind over the last twenty or so years; in such cases I regard my 1986
thoughts as superseded, and I am grateful to have had the opportunity to
move on to a sounder and clearer view. But this for the most part
applies to relatively minor points of translation -- where I have had
the advantage of being able to consult and cross-check with another
modern English translation, that of Nehamas and Woodruff (1995), now
also installed in the Hackett {\em Plato: Complete Works} (see Further
Reading). On some occasions, I have moved in their direction; where we
disagree, that is a matter of decision and not of oversight. (One such
disagreement has already been noted in the Introduction.)

\section[title={PHAEDRUS},reference={chapter-001.htmlux23pha0000185}]

SOCRATES My dear Phaedrus, where is it you're going, and {[}227a{]}
where have you come from?

PHAEDRUS From Lysias, son of
Cephalus,\goto{\high{1}}[notes.htmlux5cux23pla_0000072] Socrates; and
I'm going for a walk outside the
wall,\goto{\high{2}}[notes.htmlux5cux23pla_0000074] because I spent a
long time sitting there -- since sun-up. I'm doing what your friend and
{[}a5{]} mine, Acumenus,\goto{\high{3}}[notes.htmlux5cux23pla_0000076]
advises, and taking my walks along the country roads; he says that
walking here is more refreshing than in the colonnades. {[}b1{]}

SOCRATES He's right to say so, my friend. So it seems Lysias was in the
city.

PHAEDRUS Yes, at Epicrates' house, the one Morychus used to live
in,\goto{\high{4}}[notes.htmlux5cux23pla_0000078] near the temple of
Olympian Zeus.\goto{\high{5}}[notes.htmlux5cux23pla_0000080] {[}b5{]}

SOCRATES So then how did you spend your time? Obviously Lysias was
feasting you all with his
speeches?\goto{\high{6}}[notes.htmlux5cux23pla_0000082]

PHAEDRUS You'll find out about that if you have the leisure to walk and
listen.

SOCRATES What? Don't you think I shall be likely to regard it -- to
quote Pindar\goto{\high{7}}[notes.htmlux5cux23pla_0000084] -- as ‘a
thing above even want of leisure', to {[}b10{]} hear how you and Lysias
spent your time?

PHAEDRUS Well then -- lead on. {[}c1{]}

SOCRATES Please tell me.

PHAEDRUS Certainly, Socrates, and it will be pretty appropriate for you
to hear, because the speech on which we were spending our time was, I
tell you, in a certain sort of way about
{\em love}.\goto{\high{8}}[notes.htmlux5cux23pla_0000086] {[}c5{]}
Lysias has represented someone beautiful being propositioned but not by
a lover -- indeed, that's just the subtlety of his invention: he says
that favours should be
granted\goto{\high{9}}[notes.htmlux5cux23pla_0000088] to a man who is
not in love rather than to one who is.

SOCRATES How admirable of Lysias! I only wish he would {[}c10{]} write
that it should be to a poor man rather than a rich one, and an older
rather than a younger man, and all the other {[}d1{]} things which
belong to me and to most of us; {\em then} his speeches would be urbane,
and for the general good.\goto{\high{10}}[notes.htmlux5cux23pla_0000090]
I for one am so eager to hear it, in any case, that if your walk takes
you to Megara, and you touch the wall with your foot and come back
{[}d5{]} again, as Herodicus
recommends,\goto{\high{11}}[notes.htmlux5cux23pla_0000092] I certainly
won't be left behind.

PHAEDRUS Socrates, my good fellow, what do you mean? Do {[}228a{]} you
think that I, an amateur, will be able to repeat from memory in a way
worthy of Lysias what he, the cleverest of present writers, has put
together at leisure over a long period of time? Far from it; though I'd
like to be able to, more than I'd want to come into a stack of money.

{[}a5{]} SOCRATES Phaedrus -- if I don't know Phaedrus, I've forgotten
even who I am. But I do, and I haven't; I know perfectly well that when
he heard Lysias' speech he did not hear it just once, but repeatedly
asked him to go through it for him, and Lysias {[}b1{]} responded
readily. But for Phaedrus not even that was enough, and in the end he
borrowed the book and examined the things in it which he was most eager
to look at, and doing this he sat from sun-up until he was tired and
went for a walk, as {\em I} think {[}b5{]} -- I'll swear by the
Dog\goto{\high{12}}[notes.htmlux5cux23pla_0000094] it's true -- knowing
the speech quite off by heart, unless it was a rather long one. He was
going outside the wall to practise it, when he met the very person who
is sick with passion for hearing people
speak\goto{\high{13}}[notes.htmlux5cux23pla_0000096] -- and ‘seeing,
seeing him',\goto{\high{14}}[notes.htmlux5cux23pla_0000098] he was glad,
because he would have a {[}c1{]} companion in his manic frenzy, and he
told him to lead on. Then when the one in love with speeches asked him
to speak, he put on a pose, as if not eager to speak; but in the end,
even if no one wanted to listen, he meant to use force, and {\em would}
speak. So you, Phaedrus, you just ask him to do here and now {[}c5{]}
what he will soon do anyway.

PHAEDRUS For me, really much the best thing is to speak as I can, since
it seems to me you won't let me go until I speak, somehow or other.

SOCRATES You have just the right idea about me.

PHAEDRUS So that's what I'll do. Nothing could be truer, {[}d1{]}
Socrates -- I didn't learn it word for word; but I shall run through the
purport of just about everything in which he said the situation of the
lover was different from that of the non-lover, giving a summary of each
point in turn, beginning with {[}d5{]} the first.

SOCRATES Yes, my dear fellow, after you've first shown me just what it
is you have in your left hand under your cloak; for I suspect you have
the speech itself. If you have, you must know this about me, that fond
as I am of you, if Lysias is here as well, {[}e1{]} I am not really
inclined to offer myself to you to practise on. Come on, show me!

PHAEDRUS Stop! I'd hoped to flex my muscles on you, and now you've
foiled me!\goto{\high{15}}[notes.htmlux5cux23pla_0000100] Well, where
would you like us to sit {[}e5{]} down and read?

SOCRATES Let's turn off here and go along the
Ilissus;\goto{\high{16}}[notes.htmlux5cux23pla_0000102] then {[}229a{]}
we'll sit down quietly wherever we think best.

PHAEDRUS It seems it's just as well I happen to be barefoot; you always
are. So we can very easily go along the stream with {[}a5{]} our feet in
the water; and it won't be unpleasant, particularly at this time of year
and of the day.

SOCRATES Lead on, then, and keep a lookout for a place for us to sit
down.

PHAEDRUS Well, you see that very tall plane-tree?

SOCRATES Of course.

PHAEDRUS There's shade and a moderate breeze there, and {[}b1{]} grass
to sit on, or lie on, if we like.

SOCRATES Please lead on.

PHAEDRUS Tell me, Socrates, wasn't it from somewhere here that Boreas is
said to have seized
Oreithuia\goto{\high{17}}[notes.htmlux5cux23pla_0000104] from the
Ilissus? {[}b5{]}

SOCRATES Yes so it's said.

PHAEDRUS Well, was it from here? The rivulets look attractively pure and
clear -- just right for young girls to play beside.

SOCRATES No, it was from a place two or three stades lower {[}c1{]}
down, where one crosses over to the district of
Agra;\goto{\high{18}}[notes.htmlux5cux23pla_0000106] and there,
somewhere, there's an altar of Boreas.

PHAEDRUS I've not really noticed it. But do tell me, Socrates, {[}c5{]}
for goodness' sake,\goto{\high{19}}[notes.htmlux5cux23pla_0000108] do
you believe this fairy-tale to be true?

SOCRATES If I disbelieved it, as wise
people\goto{\high{20}}[notes.htmlux5cux23pla_0000110] do, I'd not be
extraordinary; then I'd use their wisdom and say that a blast of Boreas
pushed her down from the nearby rocks while she was playing with
Pharmaceia, and when she met her death in this {[}d1{]} way she was said
to have been snatched up by Boreas -- or else it was from the Areopagus;
for this too is something people say, that it was from there and not
from here that she was seized. But, Phaedrus, while I think such
explanations attractive in other respects, they belong in my view to an
over-clever and {[}d5{]} laborious person who is not altogether
fortunate; just because after that he must set the shape of the Centaurs
to rights, and again that of the Chimaera, and a mob of such things --
Gorgons {[}e1{]} and Pegasuses -- and strange hordes of other
intractable and portentous kinds of creatures flock in on him; if
someone is sceptical about these, and tries with his boorish kind of
wisdom to reduce each to what is likely, he'll need a good deal of
leisure. {[}e5{]} As for me, there's no way I have leisure for it all,
and the reason for it, my friend, is this. I am not yet capable of
‘knowing myself', in accordance with the Delphic
inscription;\goto{\high{21}}[notes.htmlux5cux23pla_0000112] so it seems
{[}230a{]} absurd to me that while I am still ignorant of this subject I
should inquire into things which do not belong to me. So then saying
goodbye to these things, and believing what is commonly thought about
them, as I was saying just now, I inquire not into these but into
myself, to see whether I am actually a beast {[}a5{]} more complex and
more typhonic than
Typhon,\goto{\high{22}}[notes.htmlux5cux23pla_0000114] or both a tamer
and a simpler creature, sharing by nature some divine and un-typhonic
portion. But, my friend, to interrupt our
conversation,\goto{\high{23}}[notes.htmlux5cux23pla_0000116] wasn't this
the tree you were taking us to?

{[}b1{]} PHAEDRUS It's the very one.

SOCRATES By Hera, a beautiful
stopping-place!\goto{\high{24}}[notes.htmlux5cux23pla_0000118] The
plane-tree here is altogether spreading and tall, and the tallness and
shadiness of the {\em agnus
castus}\goto{\high{25}}[notes.htmlux5cux23pla_0000120] are quite lovely;
it's at the peak {[}b5{]} of its flowering and gives the place the
sweetest perfume it could. The stream, too, flows very attractively
under the plane, with the coolest water, to judge by my foot. To judge
by the figurines and statuettes, the spot seems to be sacred to some
{[}c1{]} nymphs and to
Achelous.\goto{\high{26}}[notes.htmlux5cux23pla_0000122] Then again, if
you like, how welcome it is, the freshness of the place, and very
pleasant; it echoes with a summery shrillness to the cicadas' song. Most
charming of all is the matter of the grass, growing on a gentle slope
and thick enough to be just right to rest one's head upon. So you've
{[}c5{]} been the best of guides for a stranger, my dear Phaedrus.

PHAEDRUS You, my friend, really appear the most
extraordi-nary\goto{\high{27}}[notes.htmlux5cux23pla_0000124] sort of
person. You behave like someone being led around a strange place, as you
say, and not like a local. It comes {[}d1{]} of your not leaving the
city to cross the border or even, it seems to me, to go outside the wall
at all.

SOCRATES Forgive me, my good man. You see, I'm a lover of learning, and
country places and trees won't teach me anything, {[}d5{]} which the
people of the city\goto{\high{28}}[notes.htmlux5cux23pla_0000126] will.
But you seem to have found the
prescription\goto{\high{29}}[notes.htmlux5cux23pla_0000128] to get me to
go out. Just like people who lead hungry animals on by waving a branch
or some kind of vegetable in front of them, so you seem to me to be
going to lead me round all of Attica and wherever else you please by
{[}e1{]} doing as you are now and proffering me speeches in
books.\goto{\high{30}}[notes.htmlux5cux23pla_0000130] In any case, now
that I've got here, I think I'm going to lie down for the present, and
you choose whatever pose you think easiest for reading, and read.

PHAEDRUS Listen, then.\goto{\high{31}}[notes.htmlux5cux23pla_0000132]
{[}e5{]}

‘How it is with me, you know, and how I think it is to our advantage
that these things\goto{\high{32}}[notes.htmlux5cux23pla_0000134] should
happen, you have heard me say; and I claim that I should not fail to
achieve the things {[}231a{]} I ask for because I happen not to be in
love with you. Those in love repent of whatever services they do at the
point they cease from their desire; for the others, there is no time
appropriate for repentance. For it is not under compulsion but at their
own {[}a5{]} choosing, and in accordance with the way they would best
look after their own affairs, that they render {\em their} services, in
proportion to their own
capacity.\goto{\high{33}}[notes.htmlux5cux23pla_0000136] Again, those
who are in love consider the damage they did to their own interests
because of their love and the services they have performed and, adding
in the labour they have put in, they think they have long since {[}b1{]}
given return enough to the objects of their love; whereas those not in
love cannot allege neglect of their own interests because of it, nor
reckon up their past labours, nor put the blame on quarrels with their
relatives. So with all these bad things {[}b5{]} removed, there is
nothing left but to perform readily whatever actions they think will
please the other party.\goto{\high{34}}[notes.htmlux5cux23pla_0000138]
Again, if it is {[}c1{]} worth putting a high value on those in love
because they say they show the greatest degree of affection to those
they are in love with, and are ready to incur the enmity of everyone
else for their words\goto{\high{35}}[notes.htmlux5cux23pla_0000140] and
actions if it only pleases their beloved, it {[}c5{]} is easy to see, if
they are telling the truth, that they'll put a higher value on those
they fall in love with later than they put on {\em them}, and clear too
that they will maltreat them at the bidding of their new loves. Yet how
is it reasonable to give {[}d1{]} away such a
thing\goto{\high{36}}[notes.htmlux5cux23pla_0000142] to someone in so
unfortunate a condition -- one that no person with experience of it
would even try to prevent? For the ones who suffer it agree themselves
that they are sick rather than in their right mind, and that they
{\em know} they are out of their mind but cannot control themselves; so
{[}d5{]} how, when they come to their senses, could they approve of the
decisions they make when in this condition? Moreover, if you were to
choose the best one out of those in love with you, your choice would be
only from a few, while if you chose the most suitable to yourself out of
everybody else, you would be choosing {[}e1{]} from many; so that you
would have a much greater expectation of chancing on the man worthy of
your affection\goto{\high{37}}[notes.htmlux5cux23pla_0000144] among the
many.

‘Now if you are afraid of established convention, that if {[}232a{]}
people find out you will be subject to censure, the likelihood is that
those in love, thinking they would be envied by everyone else, too, just
as they envy themselves, will be on tiptoe with talking about it and
boastfully display to all and sundry that they have not laboured in
vain; whereas those not in love, {[}a5{]} because they are in control of
themselves, will choose what is best rather than to have people think
highly of them. And again, many are bound to find out about those in
love because they see them following their loved ones around and making
a practice {[}b1{]} of it, so that when they are seen in conversation
with each other, people think that they are together in the context of
passion spent or soon to be spent; whereas no one even tries to blame
those not in love for their being together, because they {[}b5{]} know
people have to talk if they are friends or to get any other sort of
pleasure. Moreover, if you are frightened by the thought that it is
difficult for affection to last, and that while under other
circumstances the occurrence of a quarrel is a misfortune shared by both
parties, if you have given away what you value {[}c1{]} most it is on
{\em you} that great injury would be inflicted, in that case you will
have reason to fear those in love more, for there are many things that
cause them pain, and everything, they think, is done in order to inflict
injury on them. It is for this {[}c5{]} very reason that they divert
their loved ones from associating with others, fearing that those who
possess wealth will outdo them with their money, and that the educated
will come off better in terms of intellect; and they are on their guard
against the power of anyone who possesses any other sort of advantage.
{[}d1{]} So by having persuaded you to become an object of hatred to
these people, they isolate you from any friends and, if you consider
your own interest and show more sense than they do, you will come into
conflict with them; whereas those who happened not to be in love, but
achieved what they asked {[}d5{]} through merit, would not begrudge
those who associate with the objects of their attentions but would hate
those who did not wish to do so, thinking that they were being looked
down on by the latter but benefited by the presence of the former, so
that there is much greater expectation that the other party will
{[}e1{]} gain friends than enemies from the affair.

‘Moreover, many of those in love desire a person's body before they know
his ways and before they have experience of {[}e5{]} the other aspects
belonging to him, so that it is unclear to them if they will still want
to be friends with him when they cease to desire him; whereas for those
not in love, since they were friends {[}233a{]} with each other even
before they did what they did, whatever benefits they
receive\goto{\high{38}}[notes.htmlux5cux23pla_0000146] are not likely to
make their friendship less but rather to be left as reminders of what is
still to come. Moreover, you should expect to become a better person if
you {[}a5{]} listen to my arguments than if you listen to a lover's. For
lovers praise words and actions even if it means disregarding what is
best, in part because they are afraid of being hated, in part because
their own judgement is weakened as a result of their {[}b1{]} desire.
For such are the ways that love displays itself: if lovers are
unsuccessful, it makes them regard as distressing the sorts of things
that cause pain to no one else; if they are successful, love compels
them to praise even things which ought not to {[}b5{]} cause pleasure at
all; so that it is much more fitting for their loved ones to pity them
than to want to emulate them. But if you listen to me, in the first
place shall give you my company {[}c1{]} with an eye not to present
pleasure but also to the benefit that is to come, not being overcome by
love but mastering myself, and not starting violent hostility because of
small things but feeling slight anger slowly because of big ones,
forgiving the {[}c5{]} unintentional and trying to prevent the
intentional; for these are signs of a friendship that will last for a
long time. But if, after all, you have the thought that strong
friendship cannot {[}d1{]} occur unless a man is actually in love, you
should bear in mind that in that case we would neither value our sons
nor our fathers and mothers, nor would we have trustworthy friends, who
are the product not of desire of this sort but of practices of a
different kind.

{[}d5{]} ‘And again, if it were the rule that one should grant favours
most to those who are most in need of them, then the rest of mankind too
ought to benefit not the best people but the most helpless; for since
they will have been released from the greatest sufferings, they will be
the most grateful to their benefactors. {[}e1{]} Moreover, when it comes
to private expenditure too, it will be right to invite, not one's
friends, but those who beg for their share and those who need filling
up; for they will treat their benefactors fondly, attend on them, call
at their doors, feel the e5 most delight and not the least gratitude,
and pray for many good things for them. Yet perhaps the fitting thing is
rather to grant favours not to those who stand in great need of them but
to those who are most able to pay a favour back; not to those {[}234a{]}
who are merely in love with you but to those who deserve the thing you
have to give; not to the sort who will take advantage of your youthful
beauty but to the ones who will share their own advantages with you when
you become older; not to those who after they have achieved their aim
will boast of it to {[}a5{]} everyone else but to the ones who will say
nothing to anyone, out of a sense of shame; not to those who are devoted
to you for a short time but to those whose friendship for you will
remain unaltered throughout their whole life; not to the ones who will
look for an excuse for hostilities with you when they cease to desire
you but those who will display their own excellence at that very moment
when you cease to be in the {[}b1{]} prime of youth. So I say to you:
Remember what has been said, and bear this in mind: that those in love
are admonished by their friends on the basis that what they do is bad,
whereas those not in love have never been blamed by anyone close to them
for making bad decisions because of that about their own {[}b5{]}
interests.

‘You will perhaps ask me, then, whether I advise you to grant favours to
all those who are not in love with you. I for my part think that not
even the man who was in love with you would tell you to take this
attitude to all those who were. For neither {[}c1{]} would it merit
equal gratitude from the receiver nor would it be possible for you to
keep things secret from everyone else in the same way, if you wished to
do so; but from the thing\goto{\high{39}}[notes.htmlux5cux23pla_0000148]
no harm should come, only benefit to both parties.

‘So I think what I have said is sufficient; but if there is {[}c5{]}
something you miss\goto{\high{40}}[notes.htmlux5cux23pla_0000150] in my
arguments and think I have left out, ask me about it.'

How does the speech seem to you, Socrates? Doesn't it seem to you to be
extraordinarily well done, especially in its language?

SOCRATES Superhumanly, in fact, my friend; enough to make {[}d1{]} me
beside myself. And it was because of you, Phaedrus, that I felt as I
did, as I looked at you, because you seemed to me to be positively
beaming with delight at the speech as you read it; for I followed your
lead, thinking that you are more of an {[}d5{]} expert about such things
than me, and I joined in the ecstasy with your
inspired\goto{\high{41}}[notes.htmlux5cux23pla_0000152] self.

PHAEDRUS Just stop. Do you mean to joke about it like this?

SOCRATES Do I really seem to you to be joking and not serious?

PHAEDRUS Don't do that, Socrates. Tell me really -- in the name {[}e1{]}
of Zeus, the god of friendship -- do you think any other Greek who gave
his own speech on the same subject would have weightier and more
numerous things to say?

{[}e5{]} SOCRATES What? Should you and I also praise the speech on the
grounds that its creator\goto{\high{42}}[notes.htmlux5cux23pla_0000154]
has said what he should, and not just because he has said things clearly
and in a well-rounded fashion, and each and every word of his is
precisely turned? If we should, then I must go along with your
judgement, for your {[}235a{]} sake, though in fact I missed
it\goto{\high{43}}[notes.htmlux5cux23pla_0000156] through my feebleness;
for I was only paying attention to the rhetorical aspect of the speech.

In this other respect I didn't think even Lysias himself thought the
speech adequate; and in fact he seemed to me, Phaedrus, unless you say
otherwise, to have said the same things two or {[}a5{]} three times
over, as if he wasn't altogether well off when it came to saying many
things about the same subject, or else perhaps because he didn't care at
all about this sort of thing; indeed he seemed to me to be behaving with
a youthful swagger, showing off his ability to say the same things now
in this way and now in that, and to say them excellently either way.

{[}b1{]} PHAEDRUS You're talking nonsense, Socrates; the very thing you
mention is in fact the main feature of the speech. It has left out
nothing that was waiting in the subject to be expressed in a way worthy
of it; so that no one could ever say other things {[}b5{]} which were
more numerous and of greater worth than what {\em he} said.

SOCRATES That's where I shall no longer be able to go along with you;
men and women of old, wise people who have spoken and written about the
subject, will refute me if I agree as a favour to you.

{[}c1{]} PHAEDRUS Who are these people? And where have you heard better
things than there are in Lysias' speech?

SOCRATES At the moment I can't say, just like that, but clearly I
{\em have} heard
something,\goto{\high{44}}[notes.htmlux5cux23pla_0000158] either --
maybe -- from the beautiful Sappho, or from Anacreon the wise, or indeed
from some {[}c5{]}
prose-writers.\goto{\high{45}}[notes.htmlux5cux23pla_0000160] On what
evidence do I say this? My breast is full, if I may say so, my fine
fellow, and I see that I would have other things to say beyond what
Lysias says, and no worse either. I am well aware that I have thought up
none of them from within {\em my} resources, because I am conscious of
my own ignorance; the only alternative, then, I think, is that I have
been {[}d1{]} filled up through my ears, like a vessel, from someone
else's streams. But dullness again has made me forget this very thing,
how I heard it and from whom.

PHAEDRUS Absolutely
excellent!\goto{\high{46}}[notes.htmlux5cux23pla_0000162] I love what I
hear. Don't {[}d5{]} you tell me from whom and how you heard it, not
even if I tell you to, but do exactly as you say: you've promised to say
better things and no fewer than those in the book -- different things,
and keeping away from what Lysias says; and I in my turn promise you
that like the nine
archons\goto{\high{47}}[notes.htmlux5cux23pla_0000164] I'll dedicate a
golden statue of equal weight at Delphi, not just of me but of you as
{[}e1{]} well.

SOCRATES You are a very dear man, and truly made of gold, Phaedrus, if
you think I mean that Lysias has completely missed the mark, and that
I'm actually able to say different things, beyond everything he says;
that couldn't, I think, happen even {[}e5{]} to the worst writer. To
begin with, on the topic of the speech, who do you think -- if he is
saying that one should grant favours to the one who is not in love
rather than to the one who is -- would be able not to laud the good
sense of the one and censure {[}236a{]} the lack of sense of the other,
these being indispensable points, and then have something further to
say? In my view such points must be allowed, and one should be forgiven
for making them; with such things, what should be praised is not so much
the invention as the arrangement, whereas with things that are not
indispensable, and are difficult to invent, we should praise the
{[}a5{]} invention as well as the arrangement.

PHAEDRUS I agree with what you say; it seems a reasonable statement. So
for my part, I'll behave like this: I'll allow you to make it an
assumption that the man in love is more sick than {[}b1{]} the man not
in love; but when you've made a speech different from Lysias' in all
other respects, and one that contains more points and of greater worth,
then you'll stand in hammered metal beside the dedication of the
Cypselids at Olympia.\goto{\high{48}}[notes.htmlux5cux23pla_0000166]

SOCRATES Have you been taking me seriously, Phaedrus, {[}b5{]} because I
made my teasing attack on your darling? Do you think I would really try
to say something different, of greater variety, to set beside his
wisdom?

PHAEDRUS Now here, my friend, you've really let me catch {[}c1{]} you.
You'll have to say your piece, however you can, to avoid our being
forced to behave in the vulgar way we see on the comic stage, exchanging
jibes; watch out, and don't deliberately {[}c5{]} make me give you a
‘Socrates, if I don't know Socrates, I've even forgotten who {\em I}
am,' or a ‘he was desperate to speak, but put on a pose.' Just make up
your mind that we won't leave this spot until you say what you were
claiming you had ‘in your {[}d1{]} breast'. We're alone in a deserted
place, and I'm stronger and younger than you; from all of which ‘grasp
the meaning of my words',\goto{\high{49}}[notes.htmlux5cux23pla_0000168]
and make sure you're not forced to speak when you can do it voluntarily.

SOCRATES But, Phaedrus, my fine friend -- I shall be a laughing-stock
{[}d5{]} if I improvise as a layman in competition with an expert
craftsman\goto{\high{50}}[notes.htmlux5cux23pla_0000170] on the same
subjects.

PHAEDRUS I warn you, stop being coy with me. I've got something to say
which will pretty well force you to speak.

SOCRATES Then don't say it.

{[}d10{]} PHAEDRUS No, I shall say it, and what I say will be an oath. I
{[}e1{]} swear to you -- but by whom, by which god? What about this
plane-tree here? I swear that if you don't make your speech {\em in}
{\em the presence of this tree}, I shall neither display nor report to
you any speech of anyone's ever again.

SOCRATES You wretch, you! How well you've found the way {[}e5{]}to force
a lover of speeches to do whatever you tell him to do.

PHAEDRUS So why go on twisting and turning?

SOCRATES Not any longer, now you've sworn what you've sworn. How would I
be able to keep myself away from feasts of that sort?

{[}237a{]} PHAEDRU Speak then.

SOCRATES Do you know what I shall do, then?

PHAEDRUS About what?

SOCRATES I shall speak with my head covered, so that I can {[}a5{]} rush
through my speech as quickly as I can and not lose my way through shame,
from looking at you.

PHAEDRUS Just speak; for the rest, do as you like.

SOCRATES Come then, you Muses, whether you are ‘clear-voiced' because of
the beauty of your song, or whether you acquired this epithet through
the musical race of the
Ligurians,\goto{\high{51}}[notes.htmlux5cux23pla_0000172] {[}a10{]}
‘take part with me'\goto{\high{52}}[notes.htmlux5cux23pla_0000174] in
the story this excellent fellow here forces me to tell, so that his
friend,\goto{\high{53}}[notes.htmlux5cux23pla_0000176] who seemed to him
to be {[}b1{]} wise even before, may seem even more so now.

‘Once upon a time, then, there was a boy, or rather a young lad, and
very beautiful he was; and he had a very large number of lovers. And one
of them was cunning, because although he was as much in love as any of
them, he had convinced the boy that he was not in love with him. And
once in pressing his {[}b5{]} claims he tried to convince him of just
this, that one ought to grant favours to one not in love rather than to
the one in love; and he spoke like this:

‘~“In everything, my boy, there is one starting-point for those who are
going to deliberate successfully: they must know what {[}c1{]} they are
deliberating about, or they will inevitably miss their target
altogether. Most people are unaware that they do not know what each
thing really is.\goto{\high{54}}[notes.htmlux5cux23pla_0000178] So then,
assuming that they know what it is, they fail to reach agreement about
it at the beginning of their enquiry, and, having gone forward on this
basis, they pay the penalty one would expect: they agree neither
{[}c5{]} with themselves nor with each other. So let us, you and I,
avoid having happen to us what we find fault with in others: since the
discussion before you and me is whether one should rather enter into
friendship with lover or with non-lover, let us establish an agreed
definition of love, about what sort of thing it is {[}d1{]} and what
power it possesses, and look to this as our point of reference while we
make our enquiry as to whether it brings help or harm.

‘~“Well then, that love is some sort of
desire\goto{\high{55}}[notes.htmlux5cux23pla_0000180] is clear to
everyone; and again we know that men desire the
beautiful\goto{\high{56}}[notes.htmlux5cux23pla_0000182] {[}d5{]} even
if they are not in love. By what then shall we distinguish the man in
love and the man who is not? Our next step is to observe that in each of
us there are two kinds of thing which rule and lead us, which we follow
wherever they may lead, the one an inborn desire for pleasures, the
other an acquired judgement that aims at the best. These two things in
us are sometimes in accord, but there are times when they are at
{[}e1{]} variance; and sometimes the one, at other times the second, has
control. Now when judgement leads us by reason towards the best and is
in control, its control over us has the name of {[}238a{]}
restraint;\goto{\high{57}}[notes.htmlux5cux23pla_0000184] when desire
drags us irrationally towards pleasures and has established rule within
us, its rule is called by the name of
excess.\goto{\high{58}}[notes.htmlux5cux23pla_0000186] Excess is
something which has many names, for it has many limbs and many forms;
and whichever of these forms {[}a5{]} happens to stand out in any case,
it gives its possessor its own name, which is neither an admirable one
nor one worth the acquisition. When it is in connection with food that a
desire has achieved control over both reasoning for the best and the
{[}b1{]} other desires, it is called gluttony, and will give its
possessor this same name; again, when it has become a tyrant in
connection with drinking, leading the man who has acquired it in this
direction, it is plain what appellation he will receive; and as
for{[}b5{]} the other related names of related desires, we can see
already that a person will be called by the appropriate one, that of
whichever desire happens at any time to be in power. As for the desire
for the sake of which all the foregoing has been said, it is already
pretty evident what one should say; but everything is in a way clearer
when said than when unsaid: the irrational desire that has gained
control over any judgement urging a man {[}c1{]} towards what is
correct, and that is carried towards pleasure in beauty -- in turn being
forcefully reinforced by the desires related to it in its pursuit of the
beauty of bodies -- and that wins victory by its drive, taking its name
from its very force: this is called
love.”~'\goto{\high{59}}[notes.htmlux5cux23pla_0000188]

{[}c5{]} Well then, my dear Phaedrus, do you think, as I do myself, that
something more than
human\goto{\high{60}}[notes.htmlux5cux23pla_0000188a] has happened to
me?

PHAEDRUS I certainly agree, Socrates, that you've been seized by a
fluency greater than normal.

SOCRATES Then listen to me in silence. For the spot seems {[}d1{]}
really to be a divine one, so if by any chance I become possessed by
Nymphs as my speech proceeds, don't be surprised; as it is, I'm already
close to uttering in
dithyrambs.\goto{\high{61}}[notes.htmlux5cux23pla_0000190]

PHAEDRUS Very true.

{[}d5{]} SOCRATES For that you're to blame. But listen to what remains;
perhaps the threat might be averted. That, though, will be a matter for
god; we must return to the boy with our speech.

“~‘Well, my brave friend:\goto{\high{62}}[notes.htmlux5cux23pla_0000192]
we have stated, then, and defined what it really is that is to be
deliberated about; so, looking towards that, let us say, for the rest,
what help or harm will be {[}e1{]} likely to accrue to the person
granting favours, from lover and non-lover. Now it is
necessarily\goto{\high{63}}[notes.htmlux5cux23pla_0000194] the case, I
suppose, that the man who is ruled by desire and enslaved to pleasure
will make the one he loves as pleasing to himself as possible; and to a
sick man anything which does not resist him is pleasant, while {[}e5{]}
anything which is stronger than he is or equal to him is hateful.

So a lover will not willingly put up with his beloved's being {[}239a{]}
stronger than him or matching him but always tries to make him weaker
and less self-sufficient; and an ignorant man is weaker than a wise one,
a coward than a brave man, a poor speaker than an expert in rhetoric, a
slow-witted man than a quick one. When all these faults and more besides
make their {[}a5{]} appearance or are present by nature in the mind of a
loved one, a lover will necessarily delight in these and procure others,
or else he will be deprived of what is immediately pleasant.

Necessarily, then, he will be jealous, and by keeping him from {[}b1{]}
many other forms of association, of a beneficial kind, which would most
make a man of him, he will be a cause of great harm to him; and he will
be the cause of the greatest harm by keeping him from that association
from which he would become
{\em wisest}.\goto{\high{64}}[notes.htmlux5cux23pla_0000196] This is
what that divine thing, philosophy, is, from which the lover must
necessarily keep his beloved far {[}b5{]} away, out of a dread of being
despised; and he must contrive in everything else that the boy should be
in complete ignorance and looking for everything to his lover, which is
the condition in which he will offer most pleasure to the other but most
harm to himself. So, in respect of the mind, there is no profit at all
in {[}c1{]} a man as guardian and partner if he is in love.

‘~“What we must look at after this is the condition of the body and its
treatment: what sort of physical condition will the man who is under
compulsion to pursue pleasure in preference to good aim to produce in
anyone under his charge, and what {[}c5{]} treatment will he apply? And
he will be observed pursuing someone soft and not tough, brought up not
in the full light of the sun but in a dappled shade, unversed in manly
exertions and harsh, sweated labour but fully versed in a soft and
effeminate way of life, decking himself out in borrowed colours and d1
ornaments for lack of his own, and resorting to all the other practices
that go along with these, which are obvious and are not worth listing
further but will allow us to go on to another matter after we have laid
down one summary point: a body in {[}d5{]} such condition is one that in
war and in other times of great crisis gives heart to the enemy, and
creates alarm in one's friends, and in one's lovers themselves.

‘~“This, then, we should dismiss as obvious, and pass on to {[}e1{]} the
point that comes next: what help or what harm to us in respect of our
possessions the society and guardianship of the man in love will bring.
This at least is clear enough to everyone, and especially to the lover:
that he would pray above all for the {[}e5{]} one he loves to be bereft
of his dearest and best-intentioned and most divine possessions; for he
would be happy for him to be {[}240a{]} deprived of father and mother,
relations and friends, thinking them likely to prevent and censure the
most pleasant kind of intercourse he has with him. Further, if his loved
one possesses property, in the form of gold or any other possession, he
will think him neither as easy to catch nor as manageable once caught;
as a result, there is every necessity that the lover should {[}a5{]}
begrudge his beloved the possession of his property, and delight in his
loss of it. So too the lover would pray that his beloved should be
without wife, without children, without home for the longest possible
time, because he desires to reap the sweetness of his own enjoyment for
as long as possible.

‘~“There are indeed other bad things in life, but with most of them
{[}b1{]} some divine agency mixes a pleasure of the moment: so with the
flatterer, a formidable beast and a source of great harm, nature has
nevertheless mixed in a certain pleasure that is not entirely gross; and
one might object to a courtesan as something harmful, and many other
similarly endowed creatures {[}b5{]} and their practices, which have the
feature of being very pleasant, at least to meet the needs of the day.
But for the beloved, the lover, over and above his harmfulness, is the
least {[}c1{]} pleasant of all things to spend the day with. As the
proverb has it too, ‘young delights young' -- for I suppose matching
years draw people to matching pleasures and so makes them friends on the
grounds of likeness; yet all the same, even these are bound to have
enough of being together. What is more, in every sphere what is
{\em compulsory} is said to be oppressive to everyone; {[}c5{]} and this
element is especially present in the relation of lover to beloved, in
addition to their dissimilarity. The older man does not willingly let
the younger one leave his company, whether by day or by night, but is
driven by a frenzied compulsion that {[}d1{]} draws him on, by giving
{\em him} pleasures all the time, as he sees, hears, touches,
experiences his loved one through all the senses, so that pleasure makes
him press his services on him; but as for the one who is loved, what
kind of solace or what pleasures {[}d5{]} will the lover give {\em him},
to prevent {\em him}, when he is with him over that same period of time,
from experiencing extreme disgust -- when he sees a face that is old and
past its prime, along with everything else which follows on that, which
it is no pleasure {[}e1{]} even to hear talked about, let alone be
continually compelled actually to deal with; when he is guarded
suspiciously all the time and in all his relationships; and when he
hears himself praised at the wrong times, and too much, and reproached
in just the same way, which is intolerable when his lover is sober but
{[}e5{]} shaming as well as intolerable when he is drunk and speaking
with an unrestrained and barefaced licence?

‘~“And while he is in love, the lover is harmful and unpleasant, but
when he ceases to be in love there is no trusting him in relation to the
future, for which he promised many things with {[}e10{]} many oaths and
entreaties, so barely prevailing on the other {[}241a{]} one in that
previous time to put up with his company, painful as it was, through
hope of goods to come. Then, at the point when he should be paying back
what he owes, he substitutes a different ruler and champion in himself,
sense and sanity in place of love and madness, and has become a
different person without his beloved's realizing it. And the beloved
asks for {[}a5{]} something in return for what happened
before,\goto{\high{65}}[notes.htmlux5cux23pla_0000198] giving reminders
of what was done and said then, thinking that he is talking to the same
man; while the other through shame cannot either bring himself to say
that he has become a different person or see his way to making good the
oaths and promises of his previous mindless regime, having now come to
his senses and {[}b1{]} sobered
up\goto{\high{66}}[notes.htmlux5cux23pla_0000200] -- for fear that if he
did the same things as his previous self, he would become like that self
again, the same person. A fugitive, then, is what he becomes from all of
this, {[}b5{]} and, compelled to default, the former lover changes
direction and launches himself into flight as the sherd flips on to its
other side;\goto{\high{67}}[notes.htmlux5cux23pla_0000202] and the other
one is compelled to run after him, angrily invoking the gods, ignorant
of everything from the beginning: that in fact he ought never to have
granted favours to one in {[}c1{]} love and
necessarily\goto{\high{68}}[notes.htmlux5cux23pla_0000204] mindless but
much rather to one who was not in love and who was in possession of his
senses; and that otherwise he was necessarily surrendering himself to
someone untrustworthy, peevish, jealous, disagreeable, harmful to
property, harmful to his physical condition, but by far most harmful
{[}c5{]} to the education of his soul, than which in truth there neither
is nor ever will be anything more valuable in the eyes either of men or
of gods. So these, my boy, are the things you must bear in mind, and you
must understand that the friendship of a lover does not come with
goodwill; it's like an appetite for food, for {[}d1{]} the purpose of
filling up -- as wolves love lambs, so is lovers' affection for a
boy.”~'\goto{\high{69}}[notes.htmlux5cux23pla_0000206]

There, Phaedrus, it's as I said it would
be.\goto{\high{70}}[notes.htmlux5cux23pla_0000208] You'll hear nothing
further than that from me; please let my speech end here.

PHAEDRUS But I thought it was just in the middle, and would {[}d5{]} go
on to say an equal amount about the non-lover, to the effect that one
should rather grant him favours, saying all the good things he has on
{\em his} side; why are you stopping now?

{[}e1{]} SOCRATES Haven't you noticed, my fine fellow, that I'm already
uttering epic verses, no longer dithyrambs now, even though I'm playing
the critic?\goto{\high{71}}[notes.htmlux5cux23pla_0000210] What do you
think I'll produce if I begin praising the other man? Don't you know
I'll patently be {[}e5{]} possessed by the Nymphs, to whom {\em you}
deliberately exposed me? So, in a word, I say that the other man has the
good points that are opposed to all the things for which we've abused
the first. And why indeed make a long speech of it? Enough has been said
about both. So whatever fate should befall my story {[}242a{]} will
befall it without me;\goto{\high{72}}[notes.htmlux5cux23pla_0000212] I'm
off across the river here before I'm forced by you into something
bigger.

PHAEDRUS Don't go yet, Socrates, not until the heat of the day has
passed. Don't you see that it's just about midday, the
time\goto{\high{73}}[notes.htmlux5cux23pla_0000214] when we say
everything stands still? Let's wait and discuss {[}a5{]} what's been
said, and then we'll go, when it's cooler.

SOCRATES You've a superhuman
capacity\goto{\high{74}}[notes.htmlux5cux23pla_0000216] when it comes to
speeches, Phaedrus; you're simply amazing. Of the speeches there have
been during your lifetime, I think no one has brought {[}b1{]} more into
existence than you, either by making them yourself or by forcing others
to make them, in one way or another. Simmias the
Theban\goto{\high{75}}[notes.htmlux5cux23pla_0000218] is the one
exception; the rest you beat by a long way. Just so, now, I think you've
again become the cause of my making a speech. {[}b5{]}

PHAEDRUS No bad thing! But how do you mean? What speech is this?

SOCRATES When I was about to cross the river, my good man, I had that
supernatural experience, the
sign\goto{\high{76}}[notes.htmlux5cux23pla_0000220] that I am accustomed
to having -- on each occasion, you understand, it {[}c1{]} holds me back
from whatever I am about to do -- and I seemed to hear a kind of voice
from the very spot, forbidding me to leave until I make expiation,
because I have committed an offence against what belongs to the
gods.\goto{\high{77}}[notes.htmlux5cux23pla_0000222] Well, I am a seer;
not a very good one, but like people who are poor at reading {[}c5{]}
and writing, just good enough for my own purposes; so I already clearly
understand what my offence is. For the fact is, my friend, that the soul
too is something which has divinatory powers; for something certainly
troubled me some while ago as I was making the speech, and I had a
certain feeling of unease, as Ibycus says (if I remember rightly), ‘that
for offences against {[}d1{]} the gods, I win renown from all my fellow
men'.\goto{\high{78}}[notes.htmlux5cux23pla_0000224] But now I realize
my offence.

PHAEDRUS Just what do you mean?

SOCRATES A dreadful speech it was, Phaedrus, dreadful, both the one you
brought with you and the one you compelled me {[}d5{]} to make.

PHAEDRUS How so?

SOCRATES It was foolish and somewhat impious; what speech could be more
dreadful than that?

PHAEDRUS None -- if you're right in what you say.

SOCRATES What? Don't you think Love to be the son of Aphrodite, and a
god?

{[}d10{]} PHAEDRUS So it is said.

SOCRATES Not, I think, by Lysias, at any rate, nor by your {[}e1{]}
speech, which came from my mouth, bewitched as it was by your
potion.\goto{\high{79}}[notes.htmlux5cux23pla_0000226] But if Love is,
as indeed he is, a god, or something divine, he would not be anything
bad; whereas the two speeches {[}e5{]} we had just now spoke of him as
if he were like that. So this was their offence in relation to Love; and
besides, their {[}243a{]} foolishness was really quite refined --
parading themselves as if they were worth something while actually
saying nothing healthy or true, in case they might deceive some poor
specimens of humanity and win praise from them. So I, my friend, must
purify myself, and for those who offend in the telling of stories there
is an ancient method of purification, which Homer was {[}a5{]} not aware
of, but Stesichorus\goto{\high{80}}[notes.htmlux5cux23pla_0000228] was.
For when he was deprived of his sight because of his libel against
Helen, he did not fail to recognize the cause, like Homer; because he
was a true follower of the
Muses,\goto{\high{81}}[notes.htmlux5cux23pla_0000230] he knew it, and
immediately composed the verses

This tale I told is false. There is no doubt:

You made no journey in the well-decked ships

{[}b1{]} Nor voyaged to the citadel of
Troy.\goto{\high{82}}[notes.htmlux5cux23pla_0000232]

And after composing the whole of the so-called
{\em Palinode},\goto{\high{83}}[notes.htmlux5cux23pla_0000234] he at
once regained his sight. So I shall follow a wiser course than
Stesichorus and Homer in just this respect: I shall try to render
{[}b5{]} my palinode to Love before anything happens to me because of my
libel against him, with my head bare, and not covered as it was before,
for shame.

PHAEDRUS There's nothing, Socrates, you could have said that would have
given me more pleasure.

{[}c1{]} SOCRATES Yes, my good Phaedrus, for you see how shamelessly
said the speeches were, this second one and the one from the book. If we
were being listened to by someone of a noble and gentle character who
was in love with someone else of the same sort, or else had ever been in
love with someone like that before, {[}c5{]} and he heard us saying that
lovers start large-scale hostilities because of small things, and adopt
a jealous and harmful attitude towards their beloved, surely you think
he would suppose himself to be listening to people who had perhaps been
brought up among sailors, and who had never seen a love of the sort that
belongs to free men,\goto{\high{84}}[notes.htmlux5cux23pla_0000236] and
would be far from agreeing with {[}d1{]} the things we find to blame in
Love?

PHAEDRUS Zeus! Socrates, perhaps he would.

SOCRATES Then out of shame for what this man would think, and out of
fear of Love himself, I for my part am anxious to wash away the bitter
taste, as it were, of the things we have heard said, with a wholesome
speech; and I advise Lysias too {[}d5{]} to put it in writing as quickly
as possible that one should grant favours to a lover rather than to one
not in love, in return for favours
received.\goto{\high{85}}[notes.htmlux5cux23pla_0000238]

PHAEDRUS You can be sure that's how it will be: once you have spoken
your praise of the lover, there'll be nothing for it but for me to
compel Lysias to write a speech in his turn on the {[}e1{]} same subject

SOCRATES I believe you'll do it; that's the sort of person you are.

PHAEDRUS Then you can give your speech with full confidence.

SOCRATES Where, then, is that boy I was talking to? I want him to hear
this speech too; if he doesn't, he may go ahead and {[}e5{]} grant
favours to the non-lover before we can stop him.

PHAEDRUS Here he is right next to you, whenever you
wish.\goto{\high{86}}[notes.htmlux5cux23pla_0000240]

SOCRATES Well then, my beautiful boy, you should take note of this --
that the previous speech belonged to Phaedrus son of {[}244a{]}
Pythocles, of the deme Myrrhinous; while the one I am going to make
belongs to Stesichorus son of Euphemus, of
Himera.\goto{\high{87}}[notes.htmlux5cux23pla_0000242] It must go like
this: “The story is not true” if it says that when a lover is there for
the having, one should rather grant favours to the one not in love, on
the grounds that the first is mad, {[}a5{]} while the second is sane.
That would be rightly said if it were a simple truth that madness is a
bad thing; but as it is, the greatest of goods come to us through
madness, provided that it is bestowed by divine gift. The prophetess at
Delphi, no less, and the priestesses at Dodona do many fine things for
Greece {[}b1{]} when mad, both on a private and on a public level,
whereas when sane they achieve little or nothing; and if we speak of the
Sibyl and of others who by means of inspired prophecy foretell {[}b5{]}
many things to many people and set them on the right track with respect
to the future, we would spin the story out by saying things that are
obvious to everyone. But it is worthwhile adducing this point: that
among the ancients, too, those who gave things their names did not
regard madness as shameful {[}c1{]} or a matter for reproach; for
otherwise they would not have connected this very word with the finest
of the sciences, that by which the future is judged, and named it the
“manic” art. No, they gave it this name thinking madness a fine thing
when it comes by divine dispensation; whereas people now crudely
{[}c5{]} throw in the extra “t” and call it
“mantic”.\goto{\high{88}}[notes.htmlux5cux23pla_0000244] So too when the
ancients gave a name to the investigation sane men make into the future
by means of birds and the other signs they use, they called it
“oionoistic”, because its proponents in a rational way provide insight
({\em nous}) and information ({\em historia}) for human {[}d1{]}
thinking ({\em oiêsis}); while moderns now call it “oiônistic”, making
it more high-sounding with the long “o”. So the ancients testify to the
fact that god-sent madness is a finer thing than man-made sanity, by the
very degree that mantic is a more perfect and more valuable thing than
oionistic, both when name is {[}d5{]} measured against name and when
effect is measured against effect. But again, in the case of the
greatest maladies and sufferings that occur in certain families from
some ancient causes of divine anger, madness comes about in them and
acts as {[}e1{]} interpreter, finding the necessary means of relief by
recourse to prayers and forms of service to the gods; as a result of
which it hits upon secret rites of purification and puts the man who is
touched by it\goto{\high{89}}[notes.htmlux5cux23pla_0000246] out of
danger for both the present and the future, so finding a release from
his present evils for the one {[}245a{]} who is rightly maddened and
possessed. A third kind of possession and madness comes from the
Muses:\goto{\high{90}}[notes.htmlux5cux23pla_0000248] taking a soft,
virgin soul and arousing it to a Bacchic frenzy of expression in lyric
and the other forms of poetry, it educates succeeding {[}a5{]}
generations by glorifying myriad deeds of those of the past; while the
man who arrives at the doors of poetry without madness from the Muses,
convinced that after all expertise will make him a good poet, both he
and his poetry -- the poetry of the sane -- are eclipsed by that of the
mad, remaining imperfect and unfulfilled.

‘All these and still more are the fine achievements I can relate
{[}b1{]} to you of madness that comes from the gods. So let us have no
fears about {\em that}, and let us not be alarmed by any argument that
tries to frighten us into supposing that we should prefer the sane man
as friend to the one who is disturbed; let it carry {[}b5{]} off the
prize of victory only if it has shown this too -- that love is not sent
from the gods to help lover and beloved. We in our turn must prove the
reverse, that such madness is given by the {[}c1{]} gods to allow us to
achieve the greatest good fortune; and the proof will be disbelieved by
the clever, believed by the wise.

‘Well then: first, we must comprehend the truth about the nature of
soul, both divine and human, by observing experiences and actions
belonging to it; and the beginning of our {[}c5{]} proof is this:

‘All soul is immortal. For that which is always in movement is immortal;
that which moves something else, and is moved by something else, in
ceasing from movement ceases from living. So only that which moves
itself, because it does not abandon itself, never stops moving. But it
is also source and first principle {[}c10{]} of movement for the other
things which move. Now a first {[}d1{]} principle is something which
does not come into being. For all that comes into being must come into
being from a first principle, but a first principle itself cannot come
into being from anything at all; for if a first principle came into
being from anything, it would not do so from a first
principle.\goto{\high{91}}[notes.htmlux5cux23pla_0000250] Since it is
something that does not come into being, it must also be something which
does not perish. For if a first principle is destroyed, neither will it
{[}d5{]} ever come into being from anything itself nor will anything
else come into being from it, given that all things must come into being
from a first principle. It is in this way, then, that that which moves
{[}246a{]} itself is a first principle of movement. It is not possible
for this either to be destroyed or to come into being, or else the whole
universe and the whole of that which comes {[}e1{]} to be might collapse
together and come to a halt, and never again have a source from which
things will be moved and come to be. And since that which is moved by
itself has been shown to be immortal, it will incur no shame to say that
this is the {[}e5{]} essence and the definition of soul. For all body
which has its source of motion outside itself is soulless, whereas that
which has it within itself, from itself, is ensouled, this being the
nature of soul; and if this is the way it is that that which moves
itself is nothing other than soul -- then soul will necessarily be
something that neither comes into being nor dies.

‘About its immortality, then, enough has been said. About its form we
must say the following: {\em that what kind of thing it} {[}a5{]}
{\em is} belongs to a completely and utterly
superhuman\goto{\high{92}}[notes.htmlux5cux23pla_0000252] exposition,
and a long one; to say {\em what it resembles} requires a lesser one,
one within human capacities. So let us speak in the latter way. Let it
then resemble the
combined\goto{\high{93}}[notes.htmlux5cux23pla_0000254] power of a
winged team of horses and their charioteer. Now in the case of gods,
horses {[}b1{]} and charioteers are all both good themselves and of good
stock; whereas in the case of the rest, there is a mixture. In the first
place, our driver\goto{\high{94}}[notes.htmlux5cux23pla_0000256] has
charge of a pair; secondly, one of them he finds noble and good, and of
similar stock, while the other is of the opposite stock, and opposite in
its nature; so that the {[}b5{]} driving in our case is necessarily
difficult and troublesome. How it is, then, that some living creatures
are called mortal and some immortal, we must now try to say. All soul
has the care of all that is soulless, and ranges about the whole
universe, coming {[}c1{]} to be now in one form, now in another. Now
when it is perfectly winged, it travels above the earth and governs the
whole cosmos; but the soul that has lost its wings is swept along until
it lays hold of something solid, where it settles down, taking on an
earthly body that seems to move itself because of the power{[}c5{]} of
soul, and the whole is called a living creature, soul and body fixed
together, and acquires the name “mortal”; immortal it is not, on the
basis of any argument which has been reasoned through, but because we
have neither seen nor adequately {[}d1{]} conceived of a god, we imagine
a kind of immortal living creature which has a soul and has a body, and
we imagine these combined for all time. But let this, and our account of
it, be as is pleasing to
god;\goto{\high{95}}[notes.htmlux5cux23pla_0000258] let us grasp the
cause of the loss of wings -- {[}d5{]} why they fall from a soul. It is
something like this.

‘The natural property of a wing is to carry what is heavy upwards,
lifting it aloft to the region where the race of the gods resides, and
in a way,\goto{\high{96}}[notes.htmlux5cux23pla_0000260] of all the
things belonging to the sphere of the body, it has the greatest share in
the divine, the divine {[}e1{]} being beautiful, wise, good and
everything which is of that
kind;\goto{\high{97}}[notes.htmlux5cux23pla_0000262] so it is by these
things that the plumage of the soul is nourished and increased most of
all, while the shameful, the bad and in general the opposites of the
other things make it waste away and perish. First in the heavens travels
Zeus, the {[}e5{]} great leader, driving a winged chariot, putting all
things in order and caring for all; after him there follows an army of
gods and divinities, ordered in eleven companies. For Hestia {[}247a{]}
alone remains\goto{\high{98}}[notes.htmlux5cux23pla_0000264] in the
house of the gods; of the rest, all those who have their place among the
number of the twelve take the lead as commanders in the station given to
each. Many, then, and blessed are the paths to be seen along which the
happy race {[}a5{]} of gods turns within the heavens, each of them
performing what belongs to him; and after them follows anyone who wishes
and is able to do so, for jealousy is excluded from the divine chorus.
But when they go to their feasting and to banquet, then they travel to
the summit of the arch of heaven, and the climb is {[}b1{]} steep: the
chariots of the gods travel easily, being well balanced and easily
controlled, while the rest do so with difficulty; for the horse that is
partly bad weighs them down, inclining them towards the earth through
its weight, if any of the charioteers has not trained him well. Here it
is that the final labour,\goto{\high{99}}[notes.htmlux5cux23pla_0000266]
{[}b5{]} the final contest, awaits a soul. Those souls that are called
immortal,\goto{\high{100}}[notes.htmlux5cux23pla_0000268] when they are
at the top, travel outside and take their stand upon the outer part of
the heavens, and positioned {[}c1{]} like this they are carried round by
its revolution, and gaze on the things outside the heavens.

‘Now the region above the heavens has never yet been celebrated as it
deserves to be by any earthly poet, nor will it ever be. But it is like
this -- for one must be bold enough to say what {[}c5{]} is true,
especially when speaking about truth. This region is occupied by being
which really is, which is without colour or shape, intangible,
observable by the steersman of the soul alone, by intellect, and to
which the class of true knowledge
relates.\goto{\high{101}}[notes.htmlux5cux23pla_0000270] {[}d1{]} Thus
because the mind of a god is nourished by intellect and knowledge
unmixed, and so too that of every soul which is concerned to receive
what is appropriate to it, it is glad at last to see what is and is
nourished and made happy by gazing on {[}d5{]} what is true, until the
revolution of the whole brings it around in a circle to the same point.
In its circuit it sees justice itself, sees
self-control,\goto{\high{102}}[notes.htmlux5cux23pla_0000272] sees
knowledge -- not that knowledge to which coming into being attaches, nor
the knowledge that {[}e1{]} strangely differs in different items among
the things that we now say
are,\goto{\high{103}}[notes.htmlux5cux23pla_0000274] but that which is
in what really is and which is really knowledge; and having feasted its
gaze in the same way on the other things that really are, it descends
back into the region within the heavens and goes home. When it arrives
there, {[}e5{]} the charioteer stations his horses at their manger,
throwing them ambrosia and giving them nectar to drink down with the
ambrosia.\goto{\high{104}}[notes.htmlux5cux23pla_0000276]

{[}248a{]} ‘This is the life of gods; of the other souls, the one that
follows god best and has come to resemble him most raises the head of
its charioteer into the region outside and is carried round with the
revolution, meanwhile being disturbed by its {[}a5{]} horses and
scarcely seeing the things that are; while another now rises, now sinks,
and because of the force exerted by its horses sees some things but not
others. The remaining souls follow after them, all straining to reach
the place above but unable to do so, and are carried round together
under the {[}b1{]} surface, trampling and jostling one another, each
trying to get ahead of the next. So there ensues the greatest confusion
among the sweating competitors, and in all of it, through their
charioteers' incompetence, many souls are maimed, and many have their
wings all broken; all of them with great labour depart {[}b5{]} without
achieving a sight of what is, and afterwards feed on what only appears
to nourish them.\goto{\high{105}}[notes.htmlux5cux23pla_0000278] The
cause of their great eagerness to see the plain of truth where it lies
is that the {[}c1{]} pasturage that is fitting for the best part of the
soul\goto{\high{106}}[notes.htmlux5cux23pla_0000280] really comes from
the meadow there, and that it is the nature of the wing that lifts up
the soul to be nourished by this. And the ordinance of
Destiny\goto{\high{107}}[notes.htmlux5cux23pla_0000282] is this: that
whichever soul follows in the train of god and catches some sight of
what is true shall {[}c5{]} remain free from sorrow until the next
circuit, and if it is always able to do this, it shall always remain
free from harm; but whenever through inability to follow it fails to
see, and through some mischance is weighed down by being filled with
forgetfulness and incompetence, and because of the weight loses its
wings and falls to the earth, then it is the law that this soul shall
not be planted in any wild creature at its first
birth;\goto{\high{108}}[notes.htmlux5cux23pla_0000284] rather, {[}d1{]}
the one that saw most shall be planted in the seed of a man who will
become a lover of wisdom, or of beauty, or devoted to the Muses and to
love;\goto{\high{109}}[notes.htmlux5cux23pla_0000286] the second in the
seed of a law-abiding king, or someone fit for generalship and ruling;
the {[}d5{]} third in that of a man who devotes himself to the affairs
of a city, or some expert in household or business affairs; the fourth
in that of an exercise-loving trainer in the gymnasium, or of someone
who will be concerned with healing the body; the fifth will have the
life of a seer or of some expert in mystic rites; for {[}e1{]} the
sixth, the fitting life will be that of a
poet\goto{\high{110}}[notes.htmlux5cux23pla_0000288] or of some other
type concerned with imitation; for the seventh that of a craftsman or
farmer; for the eighth that of sophist or demagogue; for the ninth that
of a tyrant. Among all these kinds, whoever lives justly receives a
better portion, whoever lives {[}e5{]} unjustly receives a worse. For
each soul only returns to the place from which it has come after ten
thousand years;\goto{\high{111}}[notes.htmlux5cux23pla_0000290] it
{[}249a{]} does not become winged before then, except in the case of the
soul of the man who has lived the philosophical life without guile or
who has united his love of boys with philosophy. These souls, with the
third circuit of a thousand years, if they choose this life three times
in succession, on that condition become winged and depart, in the
three-thousandth year. But the rest, {[}a5{]} when they finish their
first life, undergo judgement, and after judgement some of them go to
the places of correction under the earth and pay their penalty, while
others are lifted up by Justice into some region of the heavens and live
a life of a kind merited by their life in human
form.\goto{\high{112}}[notes.htmlux5cux23pla_0000292] In the thousandth
year, {[}b1{]} both sorts come to an allotment and
choice\goto{\high{113}}[notes.htmlux5cux23pla_0000294] of their second
life, and each chooses whichever it wishes: then a human soul passes
even into the life of a wild animal, and what was once a {[}b5{]} man
back into a man from a wild animal. For the soul which has never seen
the truth shall not enter this shape of ours. A human being must
comprehend what is said universally, arising {[}c1{]} from many
sensations and being collected together into one through reasoning; and
this is a recollection\goto{\high{114}}[notes.htmlux5cux23pla_0000296]
of those things which our soul once saw when it travelled in company
with god and treated with contempt the things we now say
are,\goto{\high{115}}[notes.htmlux5cux23pla_0000298] and when it poked
its head up into what really is. Hence it is {[}c5{]} with justice that
only the thought of the philosopher becomes winged; for so far as it can
it is close, through memory, to those things his closeness to which
gives a god his divinity. Thus if a man uses such
reminders\goto{\high{116}}[notes.htmlux5cux23pla_0000300] rightly, being
continually initiated in perfect rites, he alone achieves real
perfection;\goto{\high{117}}[notes.htmlux5cux23pla_0000302] and standing
{[}d1{]} aside from human concerns, and coming close to the divine, he
is admonished by the many for being disturbed, when his real state is
one of possession, which goes unrecognized by the many.

‘Well then, this is the outcome of my whole account of the {[}d5{]}
fourth kind of madness -- the madness of the man who, on seeing beauty
here on earth, and being reminded of true beauty, becomes winged and,
fluttering with eagerness to fly upwards but unable to leave the ground,
looking upwards like a bird, and taking no heed of the things below,
causes him to be {[}e1{]} regarded as
mad:\goto{\high{118}}[notes.htmlux5cux23pla_0000304] the outcome is that
this in fact reveals itself as the best of all the kinds of divine
possession and from the best of sources for the man who is subject to it
and shares in it, and that it is when he partakes in this madness that
the man who loves the
beautiful\goto{\high{119}}[notes.htmlux5cux23pla_0000306] is called a
lover. For as has been said, every soul of a human being has by the law
of its nature {[}250a{]} observed the things that are, or else it would
not have entered this creature, man; but it is not easy for every soul
to gain from things here a recollection of those other things, either
for those which only briefly saw the things there at that earlier time,
or for those which fall to earth and have the misfortune to be turned to
injustice by keeping certain kinds of company, forgeting {[}a5{]} the
holy things they saw then. Few souls are left who have sufficient
memory; and these, when they see some likeness of the things there, are
driven out of their wits with amazement and lose control of themselves,
though they do not know what {[}b1{]} has happened to them because they
cannot properly see through it. Now in the earthly likenesses of justice
and self-control and the other things that are of value to souls, there
is no illumination, but through dulled organs just a few individuals
approach their images and with difficulty observe the nature of what is
{[}b5{]} imaged in them; but in that earlier time beauty was there to
see, blazing out, when with a happy company --
ourselves\goto{\high{120}}[notes.htmlux5cux23pla_0000308] following with
Zeus, others with different gods -- they {\em saw} a blessed sight there
before them, and were initiated into what it is right to call most
blessed of rites, which we celebrated, whole {[}c1{]} in ourselves, and
untouched by the evils that awaited us in a later time, with our gaze
turned in our final initiation towards whole, simple, unchanging and
blissful revelations, in a pure light, pure ourselves and not entombed
in this thing which we {[}c5{]} now carry round with us and call body,
imprisoned like oysters.

‘Let this, then, be our concession to memory, which has made me speak
now at some length out of longing for what was
before;\goto{\high{121}}[notes.htmlux5cux23pla_0000310] but on the
subject of beauty -- as we said, it shone out {[}d1{]} when in company
with those other things, and now that we have come to earth we have
found it gleaming most clearly through the clearest of the senses that
we have. For of all the sensations coming to us through the body, sight
is the keenest: wisdom we do not see with it -- the feelings of love it
would {[}d5{]} cause in us would be terrible, if it allowed some such
clear image of itself to reach our sight, and so too with the other
objects of love;\goto{\high{122}}[notes.htmlux5cux23pla_0000312] but as
it is, beauty alone has acquired this privilege, of being most evident
and most loved. Thus the man {[}e1{]} whose initiation was not recent,
or who has been corrupted, does not move keenly from here to there, to
beauty itself, when he gazes on its namesake here, so that he does not
revere it when he looks at it but, surrendering himself to pleasure,
does his best to go on four feet like an animal and father offspring
{[}e5{]} and, keeping close company with
excess,\goto{\high{123}}[notes.htmlux5cux23pla_0000314] has no fear or
shame {[}251a{]} in pursuing pleasure contrary to nature; while the
newly initiated, the man who observed much of what was visible to him
before, whenever he sees a godlike face or some form of body which
imitates beauty well, first shudders, feeling something of the fears he
had before, then reveres what he sees like {[}a5{]} a god as he gazes at
it and, if he were not afraid of appearing thoroughly mad, would
sacrifice to his beloved as if to a statue of a god. When he has seen
him, the expected change comes {[}b1{]} over him following the
shuddering -- sweating and a high fever; for he is warmed by receiving
the effluence of beauty that is the natural nourishment of his plumage,
and with that warming there is a melting of the parts around its base,
which have {[}b5{]} long since become hard and closed up, so preventing
it from sprouting, and with the incoming stream of nourishment the
quills of the feathers swell and set to growing from their roots under
the surface of the whole form of the soul; for formerly {[}c1{]} the
whole of it was winged. Meanwhile all of it throbs and palpitates, and
the experience is like that of cutting teeth, the itching and the aching
that occur around the gums when the teeth are just coming through: such
is the state of the soul of {[}c5{]} the man who is beginning to sprout
wings -- it throbs and aches and tickles as it grows its feathers. So
when it gazes at the boy's beauty, and is nourished and warmed by
receiving particles ({\em merê}) which come to it ({\em epionta}) in a
flood ({\em rheonta}) from there (hence, of course, the name we give
them, desire
({\em himeros})\goto{\high{124}}[notes.htmlux5cux23pla_0000316]), it
{[}d1{]} experiences relief from its anguish and is filled with joy; but
when it is apart and becomes parched, the openings of the passages
through which the feathers push their way out are dried up and closed,
so shutting off their shoots, and these, shut {[}d5{]} in with the
desire,\goto{\high{125}}[notes.htmlux5cux23pla_0000318] throb like
pulsing arteries, each of them pricking at the outlet corresponding to
it, so that the entire soul, stung all over, goes mad with pain; but
then, remembering the boy with his
beauty,\goto{\high{126}}[notes.htmlux5cux23pla_0000320] it rejoices
again. The mixture of both these states makes it despair at the
strangeness of its {[}e1{]} condition, raging in its perplexity, and in
its madness it can neither sleep at night nor keep still where it is by
day, but in its yearning runs to wherever it thinks it will see the
possessor of the beauty it longs for; and, when it has seen him and
channelled desire\goto{\high{127}}[notes.htmlux5cux23pla_0000320a] in to
itself, it releases what was pent up before, and, {[}e5{]} finding a
breathing space, it ceases from its stinging birth-pains, {[}252a{]}
once more enjoying this for the moment as the sweetest pleasure. This it
does not willingly give up, nor does it value anyone above the one with
beauty, but quite forgets mother, brothers, friends, all together, loses
wealth through neglect without caring a jot about it, and, feeling
contempt for all the accepted {[}a5{]} standards of propriety and good
taste in which it previously prided itself, it is ready to act the part
of a slave and sleep wherever it is allowed to do so, provided it is as
close as possible to the object of its yearning; for in addition to its
reverence for the one who possesses the beauty, it has found him to be
the {[}b1{]} sole healer of its greatest labours. This experience, my
beautiful boy, the one to whom my speech is addressed, men term love;
but when you hear what gods call it I expect you will laugh, because of
your youth. I think some Homeric experts cite two {[}b5{]} verses to
Love from the less well-known poems, the second of which is quite
outrageous and not very metrical; they celebrate him like this: We
mortals call him Mighty Love, a winged power of great renown, Immortals
call him Fledgeling Dove -- since Eros' wings lack
down.\goto{\high{128}}[notes.htmlux5cux23pla_0000322] You may believe
this or you may not; but at any rate the cause {[}c1{]} of the lover's
experience and the experience itself are as I have described.

‘If the man who is taken by Love belongs among the followers of Zeus, he
is able to bear the burden of the Feathery One with some sedateness; but
as for those who were attendants of
Ares\goto{\high{129}}[notes.htmlux5cux23pla_0000324] {[}c5{]} and made
the circuit with him, when they are captured by Love and think that they
are being wronged in some way by the one they love, they become
murderous and ready to sacrifice both themselves and their beloved. Just
so each lives after the pattern {[}d1{]} of the god in whose chorus he
was, honouring him by imitating him in his life so far as he can,
provided that he is uncorrupted and living out the life following his
first birth here on earth; and he behaves in this way in his
associations both with those he loves and with everyone else. So each
selects his love from {[}d5{]} the ranks of the beautiful according to
his own disposition and, as if that love were the very god he followed,
fashions and adorns him like a statue for himself, in order to honour
him {[}e1{]} and celebrate his mystic rites. Thus those who belong to
Zeus seek that the one loved by themselves should be Zeus-like in
respect of his soul; so they look to see whether he is naturally
disposed towards philosophy and leadership, and when they have found him
and fallen in love, they do everything to make {[}e5{]} him like this.
So if they haven't embarked on this
practice\goto{\high{130}}[notes.htmlux5cux23pla_0000326] before now, now
they do undertake it, both learning from wherever they can and finding
out for themselves; and as they follow the {[}253a{]} scent from within
themselves to the discovery of the nature of their own god, they find
the means to it through the compulsion on them to gaze intensely on the
god, and grasping him through memory, and possessed by him, it is from
him that they take their habits and ways, to the extent that it is
possible for man {[}a5{]} to share in god; and because they count their
beloved responsible for these very things, they love him even more, and
if it is from Zeus that they draw, like Bacchants, they pour the draught
{[}b1{]} over the soul of their loved one and make him as like their god
as possible. As for those who followed with Hera, they seek someone
regal in nature, and when they have found him they do all the same
things in respect of him. Those who belong to Apollo and each of the
other gods proceed in the same way in accordance with their god and seek
that their boy should be of {[}b5{]} the same nature; and when they
acquire him, imitating the god themselves and persuading and
disciplining their beloved, they draw him into the way of life and
pattern of the god, to the extent that each is able, without showing
jealousy or mean {[}c1{]} ill-will towards their beloved; rather, they
act as they do because they are trying as much as they can, in every
way, to draw him into complete resemblance to themselves and to
whichever god they honour. The eagerness of those who are truly in love,
then, and its outcome\goto{\high{131}}[notes.htmlux5cux23pla_0000328] --
if, that is, they manage to achieve what they eagerly desire in the way
I have said -- are thus rendered {[}c5{]} beautiful and bring happiness
from the friend who is maddened through love to the object of his
affection, if he is caught; and one who is caught is captured in the
following way.

‘Just as at the beginning of this story we divided each soul into three
forms,\goto{\high{132}}[notes.htmlux5cux23pla_0000330] two like horses
and the third with the role {[}d1{]} of charioteer, let this still stand
now. Of the horses, one, we say, is good, the other not; but we did not
describe what the excellence of the good horse was, or the badness of
the bad horse, and now we must. Well then, the first of the two, which
is on the nobler
station,\goto{\high{133}}[notes.htmlux5cux23pla_0000332] is erect in
form and clean-limbed, {[}d5{]} high-necked, nose somewhat hooked, white
in colour, with black eyes, a lover of honour when joined with restraint
and a sense of shame, and a companion of true glory, needing no whip,
responding to spoken
orders\goto{\high{134}}[notes.htmlux5cux23pla_0000334] alone; the other
is crooked {[}e1{]} in shape, gross, a random collection of parts, with
a short, powerful neck, flat-nosed, black-skinned, grey-eyed, bloodshot,
companion of excess\goto{\high{135}}[notes.htmlux5cux23pla_0000336] and
boastfulness, shaggy around the ears, deaf, hardly yielding to whip and
goad together. Now {[}e5{]} when the charioteer first catches sight of
the light of his love, warms the whole soul with the seeing of it, and
begins to be filled with tickling and pricks of longing, the horse that
is {[}254a{]} obedient to the charioteer, constrained then as always by
shame, holds itself back from leaping on the loved one; while the other
no longer takes any heed of goading or the whip from the charioteer but
springs powerfully forward and, causing all {[}a5{]} kinds of trouble to
his yoke-mate and the charioteer, forces them to move towards the
beloved and mention to him the delights of sex. At the start, the two of
them resist, indignant {[}b1{]} at the idea of being forced to do
terrible and improper things; but finally, when there is no limit to the
trouble it causes, they follow its lead, giving in and agreeing to do
what it tells them to do. And now they are close to the beloved, and
they see the beloved's face, flashing like lightning. As the charioteer
sees it, {[}b5{]} his memory is carried back to the nature of beauty and
again sees it standing together with self-control on a holy pedestal; at
the sight it becomes frightened, and in sudden reverence falls on its
back, and is forced at the same time to pull back the reins {[}c1{]} so
violently as to bring both horses down on their haunches, the one
willingly, because of its lack of resistance to him, but the horse of
excess\goto{\high{136}}[notes.htmlux5cux23pla_0000338] much against its
will. When they have backed off a little way, the first horse drenches
the whole soul {[}c5{]} with sweat from shame and alarm, while the
other, when it has recovered from the pain caused to it by the bit and
its fall, scarcely gets its breath back before it breaks into angry
abuse, repeatedly reviling the charioteer and its yoke-mate for cowardly
and unmanly desertion of their agreed position; and again {[}d1{]} it
tries to compel them to approach, unwilling as they are, and barely
concedes when they beg him to postpone it until a later time. When the
agreed time comes, and they pretend not to {[}d5{]} remember, it reminds
them; struggling, neighing, pulling, it forces them to approach the
beloved again to make the same proposition, and as soon as they are
close to him, head down and tail outstretched, teeth clamped on its bit,
it pulls shamelessly; {[}e1{]} but the same thing happens to the
charioteer as before, only even more violently, as he falls back as if
from a starting-barrier;
\goto{\high{137}}[notes.htmlux5cux23pla_0000340] still more violently,
he wrenches the bit back and forces it from the teeth of the horse of
excess, spattering its evil-speaking tongue and its jaws with blood and,
thrusting its {[}e5{]} legs and haunches to the ground, ‘gives it over
to pains'.\goto{\high{138}}[notes.htmlux5cux23pla_0000342] When the bad
horse has had the same thing happen to it repeatedly and it ceases from
its excess, now humbled it allows the charioteer with his foresight to
lead, and when it sees the boy in his beauty, it nearly dies of fright;
and the result is that then {[}255a{]} the soul of the lover follows the
beloved in reverence and awe. So because he receives every kind of
service, as if equal to the gods, from a lover who is not pretending to
be in love but genuinely in this state, and because he naturally feels
friendship for the man who renders him service, even if perhaps in the
{[}a5{]} past he has been prejudiced against him by hearing his
schoolfellows or others say that it is shameful to associate with a
lover, and repulses the one in love for that reason, as time goes on he
{[}b1{]} is led both by his age, and by necessity, towards admitting him
to his company; for it is surely against fate that bad be friend to bad,
or that good not be friend to good. Once he has admitted him, and
accepted his conversation and his company, the goodwill that he
experiences at close quarters from the one in love {[}b5{]} astounds the
beloved, as he clearly sees that not even all his other friends and his
relations together have anything to offer by way of friendship in
comparison with the friend who is divinely possessed. And when he
continues doing this, and association is combined with physical contact
in the gymnasium {[}c1{]} and on the other occasions when people come
together, then it is that the springs of that stream which Zeus when in
love with Ganymede named
‘desire'\goto{\high{139}}[notes.htmlux5cux23pla_0000344] flow in
abundance upon the lover, some sinking within him and some flowing off
outside him as he brims over; and as a breath of wind or some echo
rebounds from smooth, hard surfaces and returns to the source from
{[}c5{]} which it issued, so the stream of beauty passes back into its
possessor\goto{\high{140}}[notes.htmlux5cux23pla_0000346] through his
eyes, which is its natural route to the soul; arriving there and setting
him all aflutter, it waters the {[}d1{]} passages of the feathers and
causes the wings to grow, and fills the soul of the loved one in his
turn with love. So he is in love, but as to what he is in love with, he
is at a loss; and he neither knows what has happened to him nor can he
even begin to express what it is, but -- like a man who has caught
eye-disease {[}d5{]} from someone -- he can give no account of it and is
unaware that he is seeing himself in the one who loves as if in a
mirror. And when his lover is with him, like him he ceases from his
anguish; when he is absent, again like him he longs and is longed for,
because he is feeling love back, an image of the {[}e1{]} lover's love,
though he calls what he has and thinks of it not as love but as
friendship.\goto{\high{141}}[notes.htmlux5cux23pla_0000348] His desire
is similar to his lover's but weaker: to see, touch, kiss and lie down
together; and indeed, as one might expect, soon afterwards he does just
that. So as {[}e5{]} they lie together, the lover's licentious horse has
something to say to the charioteer and claims the right to a little
enjoyment {[}256a{]} as recompense for many labours endured; while its
counterpart in the beloved has nothing to say, but, swelling with
confused passion, it embraces the lover and kisses him, welcoming him as
someone full of goodwill, and when they lie down together, it is ready
not to refuse to do its own part in granting favours {[}a5{]} to the one
in love, should he beg to receive them; but its yoke-fellow, for its
part, together with the charioteer, resists this with a reasoned sense
of shame. And then, well, if the better elements of their minds get the
upper hand by drawing them to a well-ordered life, and to philosophy,
they pass their life here in {[}b1{]} blessedness and harmony, masters
of themselves and orderly in their behaviour, having enslaved that part
through which badness attempted to enter the soul and having freed that
part through which goodness enters; and when they die they become winged
and light, and have won one of their three {[}b5{]}
submissions\goto{\high{142}}[notes.htmlux5cux23pla_0000350] in these,
the true Olympic games -- and neither human sanity nor divine madness
has any greater good to offer {[}c1{]} a man than this. But if they live
a coarser way of life, devoted not to wisdom but to honour, then
perhaps, I suppose, when they are drinking or in some other moment of
carelessness, the licentious horses in the two of them catch them off
their guard, bring them together and make that choice which is called
{[}c5{]} blessed by the many, and carry it through; and, once having
done so, they continue with that choice, but sparingly, because what
they are doing has not been approved by their whole mind. So these too
spend their lives as mutual friends, though {[}d1{]} not to the same
degree as the other pair, both during the course of their love and when
they have passed beyond it, believing that they have given and received
the most binding pledges, which it would be against piety to break by
ever becoming {[}d5{]} enemies. On their death they leave the body
without wings but with the impulse to gain them, so that they carry off
no small reward for their lovers' madness; for it is ordained that those
who have already begun on the journey under the heavens shall no longer
pass into the darkness of the journey under the earth but shall rather
live in the light and be happy as they travel {[}e1{]} with each other,
and acquire matching plumage, when they acquire it, because of their
love.

‘These are the blessings, my boy, so great as to be counted divine, that
will come to you from the friendship of a lover, in the way I have
described; whereas the acquaintance of the one {[}e5{]} not in love,
which is diluted with a merely mortal good sense, dispensing miserly
benefits of a mortal kind, engenders in the soul that is the object of
its attachment a meanness that, though {[}257a{]} praised by the many as
a virtue, will cause it to wallow mindlessly around the earth and under
the earth for nine thousand years.'

This, dear god of love, is offered and paid to you as the finest
{[}a5{]} and best palinode of which I am capable, especially given that
it was forced to use somewhat poetical language because of
Phaedrus.\goto{\high{143}}[notes.htmlux5cux23pla_0000352] Forgive what
went before and regard this with favour; be kind and gracious -- do not
in anger take away or maim the expertise in love that you gave
me,\goto{\high{144}}[notes.htmlux5cux23pla_0000354] and grant that
{[}b1{]} I be valued still more than now by the beautiful. If in our
earlier speech Phaedrus and I said anything harsh against you, blame
Lysias as the instigator of the speech, and make him cease from speeches
of that kind, turning him instead, as his brother
Polemarchus\goto{\high{145}}[notes.htmlux5cux23pla_0000356] has been
turned, to philosophy, so that his lover here too may no longer waver,
as he does now, between the {[}b5{]} two choices but may single-mindedly
direct his life towards love accompanied by
talk\goto{\high{146}}[notes.htmlux5cux23pla_0000356a] of a philosophical
kind.

PHAEDRUS I pray with you, Socrates: if indeed that is better {[}c1{]}
for us, that may we have. But as for your speech, for some time I have
been amazed at how much finer you managed to make it than the one
before; so that I have a suspicion Lysias will appear wretched to me in
comparison, if he really does consent to put up another in competition
with it. Indeed, my amazing friend, just recently one of the politicians
was using this very {[}c5{]} reproach to abuse him, and all through the
abuse kept calling him a ‘speech-writer'; so perhaps we shall find him
refraining from writing out of concern for his reputation.

SOCRATES An absurd idea, young man; you much mistake your {[}d1{]}
friend, if you think him so frightened of mere noise. But perhaps you
think that the man who was abusing him really meant what he said.

PHAEDRUS He seemed to, Socrates; and I think you know {[}d5{]} yourself
that the men with the most power and dignity in our cities are ashamed
to write speeches and leave compositions of theirs behind them, for fear
of what posterity will think of them -- they're afraid they'll be called
sophists.

SOCRATES Phaedrus, you don't know the expression ‘pleasant
bend';\goto{\high{147}}[notes.htmlux5cux23pla_0000358] and besides the
bend you're missing the point that the politicians who have the highest
opinion of themselves are most in love with speech-writing and with
leaving compositions behind them, to judge at any rate from the fact
that whenever they write a speech, they are so pleased with those who
commend {[}e5{]} mend it that they add in at the beginning the names of
those who commend them on each occasion.

PHAEDRUS What do you mean by that? I don't understand.

SOCRATES You don't understand that at the beginning of a {[}258a{]}
politician's composition the commender's name is written first?

PHAEDRUS How so?

SOCRATES The writer says, I think, ‘It was resolved by the {[}a5{]}
council,' or ‘by the people' or both, and ‘So-and-so said', refer-ring
to his own dear self with great pomposity and self-eulogy; then he
proceeds with what he has to say, demonstrating his own wisdom to those
commending him, sometimes making a very long composition of it; or does
such a thing seem to you to be anything other than a written speech?

{[}b1{]} PHAEDRUS Not to me.

SOCRATES Then if it stays written down, the
author\goto{\high{148}}[notes.htmlux5cux23pla_0000360] leaves the
theatre delighted; but if it is rubbed out and he loses his chance of
being a speech-writer and of being recognized as a {[}b5{]} writer, he
and his friends go into mourning.

PHAEDRUS Very much so.

SOCRATES And clearly they behave like this not because they despise the
profession, but because they regard it with admiration.

PHAEDRUS Yes indeed.

{[}b10{]} SOCRATES Well then: when a person becomes a good enough
{[}c1{]} orator or king to acquire the capacity of a Lycurgus, a Solon
or a Darius\goto{\high{149}}[notes.htmlux5cux23pla_0000362] and achieve
immortality as a speech-writer in a city, doesn't he think himself equal
to the gods even while he is alive, and don't those who come later think
the very same of {[}c5{]} him, when they observe his compositions?

PHAEDRUS Very much so.

SOCRATES So do you think anyone of that kind, whoever he is and however
ill disposed towards Lysias, reproaches him on this count -- that he is
a writer?

{[}c10{]} PHAEDRUS It's not very likely, from what you say; if he did,
it seems he would be reproaching what he himself desires.

{[}d1{]} SOCRATES This much, then, is clear to everyone, that writing
speeches is not {\em itself} something shameful.

PHAEDRUS How could it be?

SOCRATES But what is shameful, I think, is speaking and writing {[}d5{]}
and doing it not well but shamefully and badly.

PHAEDRUS Clearly.

SOCRATES So what is the way to write well or badly? Do we need,
Phaedrus, to examine Lysias, perhaps, on this subject, and anyone else
who has so far written anything, or will write anything, thing,
{[}d10{]} whether it's a political composition or a private one, and
whether he writes it as a poet, in verse, or in plain man's prose?

PHAEDRUS You really ask if we need to? What would anyone {[}e1{]} live
for, if I may put it as strongly as that, if not for such pleasures as
this? Not, I think, for those which have to be preceded by pain if one
is to enjoy pleasure at all -- a feature possessed by nearly all the
pleasures relating to the body; which is why in fact they are called
slavish, and justly so.\goto{\high{150}}[notes.htmlux5cux23pla_0000364]
{[}e5{]}

SOCRATES We have plenty of time, it seems; and there's something else: I
think that as the cicadas sing above our heads in their usual fashion in
the heat, and converse
with\goto{\high{151}}[notes.htmlux5cux23pla_0000366] each other,
{[}259a{]} they are also watching us. So if they saw us behaving like
most people at midday, and not conversing but nodding off under their
spell through lazy-mindedness, they would justly laugh at us, thinking
that some slaves had come to their gathering-place {[}a5{]} and were
having their midday sleep around the spring, like sheep; but if they see
us conversing and sailing past them unbe-witched by their Siren song,
perhaps they may respect us and {[}b1{]} give us that gift which they
have from the gods to give to men.

PHAEDRUS What is this gift they have? I don't seem to have heard of it.

SOCRATES A man who loves the Muses really ought to have {[}b5{]} heard
of things like this. The story is that these creatures were once human
beings, belonging to a time before the Muses were born, and that with
the birth of the Muses and the appearance of song some of the people of
the time were so unhinged by pleasure that in their singing they
neglected to eat and drink, {[}c1{]} and failed to notice that they had
died. From them the race of cicadas later sprang, with this gift from
the Muses, that from their birth they have no need of sustenance but
immediately start singing, with no food and no drink, and sing until
they {[}c5{]} die; then they go and report to the Muses which among
those here honours which of them. To Terpsichore they report those who
have honoured her in the choral dance, and so make them {[}d1{]} dearer
to her; to Erato those who have honoured her in the affairs of love; and
to the other Muses similarly, according to the form of honour belonging
to each; but to Calliope, the eldest, and to Ourania, who comes after
her, they announce those who spend their time in philosophy and honour
the music {[}d5{]} that belongs to the two of them -- who, most of all
the Muses, are concerned both with the heavens and with
speech,\goto{\high{152}}[notes.htmlux5cux23pla_0000368] both divine and
human, and whose voices carry most beautifully. So there are many
reasons why we should say something and not sleep in the midday heat.

PHAEDRUS Yes, we should.

{[}e1{]} SOCRATES Then we should consider what we proposed just now:
speeches -- in what way they will be well said and written, and in what
way they will not.

PHAEDRUS Clearly.

SOCRATES Well then, for things that are going to be said {\em well}, and
beautifully, mustn't there be knowledge in the mind of the speaker of
the truth about whatever he means to speak of?

PHAEDRUS What I have heard about this, my dear Socrates, is {[}260a{]}
that there is no necessity for the man who means to be an orator to
understand what is really just but only what would appear so to the
majority of those who will give judgement; and not what is really good
or beautiful but whatever will appear so; because persuasion comes from
that and not from the truth.

{[}a5{]} SOCRATES Whatever wise people say, Phaedrus, is ‘a word not to
be cast aside',\goto{\high{153}}[notes.htmlux5cux23pla_0000370] and we
should always look to see whether they may not be right; what you just
said, particularly, must not be dismissed.

PHAEDRUS Quite right.

SOCRATES Let us consider it like this.

{[}b1{]} PHAEDRUS How?

SOCRATES If I were persuading you to defend yourself against the enemy
by getting a horse, and neither of us knew what a horse was, but I
happened to know just so much about you, that Phaedrus thinks a horse is
that tame animal which has the largest ears --

PHAEDRUS It would be ridiculous, Socrates. {[}b5{]}

SOCRATES Not so ridiculous yet; but it would be when I tried in earnest
to persuade you by putting together a speech in praise of the donkey,
labelling it a horse and saying that the beast would be an invaluable
acquisition both at home and on active {[}c1{]} service, useful to fight
from and capable too of carrying baggage, and good for many other
purposes.

PHAEDRUS Then it would be thoroughly ridiculous.

SOCRATES Well then, isn't it better to be ridiculous and a friend than
to be clever and an enemy?

PHAEDRUS It seems so. {[}c5{]}

SOCRATES So when an expert in rhetoric who is ignorant of good and bad
finds a city in the same condition and tries to persuade it, by making
his eulogy not about a miserable donkey as if it were a horse but about
what is bad as if it were good, and -- having applied himself to what
the masses think -- actually persuades the city to do something bad
instead of good, what {[}c10{]} sort of harvest do you think rhetoric
reaps after that from the {[}d1{]} seed it sowed?

PHAEDRUS Not a very good one.

SOCRATES Well, my good friend, have we abused the science of speaking
more coarsely than we should? She might perhaps say ‘What nonsense is
this you're talking, you fine people? {[}d5{]} I don't insist that
anyone who learns how to speak should be ignorant of the truth; on the
contrary, if I advise anything, it is that he should acquire the truth
first and then get hold of me. But this at any rate is my boast, that
without me the man who knows what is true will be quite unable to
persuade scientifically.'

PHAEDRUS So will she be right in saying this? {[}e1{]}

SOCRATES I say she will; if, that is, the
arguments\goto{\high{154}}[notes.htmlux5cux23pla_0000372] advancing on
her testify that she is a science. For it seems to me as if I am hearing
certain arguments approaching and solemnly protesting even before the
case comes to court that she is lying, and is not a science but an
unscientific knack; without a grasp of truth, {[}e5{]} saith the
Laconian,\goto{\high{155}}[notes.htmlux5cux23pla_0000374] a genuine
science of speaking neither exists nor will come into existence in the
future.

PHAEDRUS We need these arguments, Socrates; bring them {[}261a{]} here
before us and examine what they say and how they say it.

SOCRATES Come here then, you noble beasts, and persuade Phaedrus of the
beautiful offspring\goto{\high{156}}[notes.htmlux5cux23pla_0000376] that
unless he engages in philosophy sufficiently well, neither will he ever
be a sufficiently {[}a5{]} good speaker about anything. Let Phaedrus
answer you.

PHAEDRUS ({\em addressing the Arguments}) Ask your questions.

SOCRATES/ARGUMENTS Well then, will not the science of rhetoric as a
whole be a kind of leading of the soul by means of
speech,\goto{\high{157}}[notes.htmlux5cux23pla_0000378] not only in
law-courts and other kinds of public gatherings but in private ones too
-- the same science, whether {[}b1{]} it is concerned with small matters
or large ones, and something which possesses no more value, if properly
understood, when it comes into play in relation to things of importance
than when it does with things of no importance? Is this what you've
heard about it?

PHAEDRUS Zeus! No, not quite that, I must say. A science of {[}b5{]}
speaking and writing is perhaps especially employed in lawsuits, though
scientific speaking is also involved in public addresses; I have not
heard of any extension of it beyond that.

SOCRATES/ARGUMENTS What? Have you only heard of the manuals on rhetoric
by Nestor and Odysseus, the ones they composed at Troy when they had
nothing to do? You haven't heard of those of
Palamedes?\goto{\high{158}}[notes.htmlux5cux23pla_0000380]

{[}c1{]} PHAEDRUS Neither -- Zeus! -- have I heard of Nestor's, unless
you're dressing up Gorgias as a kind of Nestor, or maybe a Thrasymachus
or Theodorus as
Odysseus.\goto{\high{159}}[notes.htmlux5cux23pla_0000382]

SOCRATES/ARGUMENTS Perhaps. But anyway let them pass. {[}c5{]} Now you
tell us this: What do opposing parties in law-courts do? Don't they give
opposing speeches? Or what shall we say?

PHAEDRUS Just that.

SOCRATES/ARGUMENTS About the just and unjust?

PHAEDRUS Yes.

{[}c10{]} SOCRATES/ARGUMENTS So the man who does this scientifically
{[}d1{]} will make the same thing appear to the same people at one time
just and, whenever he wishes, unjust?

PHAEDRUS Of course.

SOCRATES/ARGUMENTS And in public addresses he will make the same things
seem to the city at one time good, at another the
opposite?\goto{\high{160}}[notes.htmlux5cux23pla_0000384]

{[}d5{]} PHAEDRUS Just so.

SOCRATES/ARGUMENTS Well, don't we recognize the Eleatic Palamedes as
speaking scientifically so as to make the same things appear to his
hearers to be like and unlike, one and many, at rest and in
motion?\goto{\high{161}}[notes.htmlux5cux23pla_0000386]

PHAEDRUS Yes indeed.

SOCRATES/ARGUMENTS Then the science of giving opposing {[}d10{]}
speeches is not restricted to law-courts and public addresses, {[}e1{]}
but, it seems, there will be this single science -- if indeed it
{\em is} a science -- in relation to everything that is said: the
science that enables one to make everything which is capable of being
made to resemble something else resemble everything which it is capable
of being made to resemble, and to bring it to light when someone else
makes one thing resemble another and tries to disguise it.

PHAEDRUS What sort of thing do you mean? {[}e5{]}

SOCRATES/ARGUMENTS I think it will become clear if we direct our search
this way: Does deception occur more in the case of things that are
widely different or in those that differ little?

PHAEDRUS In those that differ little. {[}262a{]}

SOCRATES/ARGUMENTS At any rate, when you are passing over from one thing
to its opposite you will be more likely to escape detection if you take
small steps than if you take large ones.

PHAEDRUS Certainly.

SOCRATES/ARGUMENTS In that case the person who means to {[}a5{]} deceive
someone else, but be undeceived himself, must have a precise knowledge
of the likeness and unlikeness of the things that
are.\goto{\high{162}}[notes.htmlux5cux23pla_0000388]

PHAEDRUS Yes, necessarily.

SOCRATES/ARGUMENTS So will he be able, if he is ignorant of the truth of
each thing, to identify the likeness, whether small or {[}a10{]} great,
that the other things have to the thing he does not know?

PHAEDRUS Impossible. {[}b1{]}

SOCRATES/ARGUMENTS Then clearly those who hold beliefs contrary to what
is the case and are deceived have this kind of thing creeping in on them
through certain likenesses.

PHAEDRUS It does happen that way.

SOCRATES/ARGUMENTS So is there any way in which a man {[}b5{]} will be a
scientific expert at making others cross over little by little from what
is the case on each occasion, via the likenesses, leading them off
towards the opposite, or at escaping this himself, if he has not
recognized what each of the things that are actually is?

PHAEDRUS No, never.

{[}c1{]} SOCRATES/ARGUMENTS In that case, my friend, anyone who does not
know the truth, but has made it his business to hunt down appearances,
will give us a science of speech that will, so it seems, be ridiculously
unscientific.

PHAEDRUS You may be right.

{[}c5{]} SOCRATES ({\em returning to his own persona}) So do you want to
take the speech of Lysias you're carrying, and the ones you and I
made,\goto{\high{163}}[notes.htmlux5cux23pla_0000390] and see in them
something of the features we say are scientific and unscientific?

PHAEDRUS Yes, I think so, more than anything; as things are, our
discussion is somewhat bare, because we do not have sufficient examples.

{[}c10{]} SOCRATES What's more, by some chance, it seems, the pair of
{[}d1{]} speeches\goto{\high{164}}[notes.htmlux5cux23pla_0000392] as
they were given do have in them an example of a sort of how someone who
knows the truth can mislead his audience by playing with
them.\goto{\high{165}}[notes.htmlux5cux23pla_0000394] I myself,
Phaedrus, blame the gods of the place; and perhaps too the spokesmen of
the Muses {[}d5{]} who sing over our heads may have breathed this gift
upon us -- for I don't think {\em I} share in any science of speaking.

PHAEDRUS So be it; only make clear what you're saying.

SOCRATES Well, read me the beginning of Lysias' speech.

{[}e1{]} PHAEDRUS ‘How it is with me, you know, and how I think it is to
our advantage that these things should happen, you have heard me say;
and I claim that I should not fail to achieve the things I ask for
because I happen not to be in love with you. Those in love repent of
whatever services they do at the point --'

{[}e5{]} SOCRATES Stop! We need to say, then, where the author goes
wrong and what he does unscientifically -- am I right?

{[}263a{]} PHAEDRUS Yes.

SOCRATES Isn't this sort of thing, at least, clear to anyone: that we're
of one mind about some things like this, and at odds about others?

{[}a5{]} PHAEDRUS I think I understand what you mean, but tell me still
more clearly.

SOCRATES When someone utters the name of iron, or of silver, don't we
all have the same thing in mind?

PHAEDRUS Absolutely.

SOCRATES What about the names of just, or
good?\goto{\high{166}}[notes.htmlux5cux23pla_0000396] Doesn't one of us
go off in one direction, another in another, so that {[}a10{]} we
disagree both with each other and with ourselves?

PHAEDRUS We certainly do.

SOCRATES Then we are in accord in some cases, not in others. {[}b1{]}

PHAEDRUS Just so.

SOCRATES So in which of the two cases are we easier to deceive, and in
which does rhetoric have the greater power?

PHAEDRUS Clearly in those cases where we go off in different {[}b5{]}
directions.

SOCRATES So the one who means to pursue a science of rhetoric must first
have divided these up methodically and grasped some mark which
distinguishes each of the two kinds, those in which most
people\goto{\high{167}}[notes.htmlux5cux23pla_0000398] are bound to
tread uncertainly, and those in which they are not.

PHAEDRUS A fine kind of thing he will have identified, Socrates,
{[}c1{]} if he grasps this!

SOCRATES Then, I think, as he comes across each thing, he must not be
caught unawares but look sharply to see which of the two types the thing
he is going to speak about belongs to. {[}c5{]}

PHAEDRUS Right.

SOCRATES Well then, are we to say that love belongs with the disputed
cases or the undisputed ones?

PHAEDRUS With the disputed, surely; otherwise, do you think it would
have been possible for you to say what you said about {[}c10{]} it just
now, both that it is harmful to beloved and lover, and then on the other
hand that it is really the greatest of goods?

SOCRATES Admirably said; but tell me this too -- for of course {[}d1{]}
because of my inspired condition then, I don't quite remember -- whether
I defined love when beginning my speech.

PHAEDRUS Zeus! Yes, indeed you did, most emphatically.

SOCRATES Dear me! How much more scientific you're saying {[}d5{]} the
Nymphs, daughters of Achelous, and Pan, son of Hermes, are than Lysias,
son of Cephalus, in the business of speaking! Or am I wrong? Did Lysias
too compel us when beginning his speech on love to take love as one
definite thing that he himself {[}e1{]} had in mind, and did he then
bring the whole speech that followed to its conclusion by ordering it in
relation to that? Shall we read the beginning again?

PHAEDRUS If you think we should; but what you're looking for isn't
there.

{[}e5{]} SOCRATES Quote it, so I can hear it from the man himself.

PHAEDRUS ‘How it is with me, you know, and how I think it is to our
advantage that these things should happen, you have {[}264a{]} heard me
say; and I claim that I should not fail to achieve the things I ask for
because I happen not to be in love with you. Those in love repent of
whatever services they do at the point they cease from their desire --'

SOCRATES He does indeed seem to be a long way from doing {[}a5{]} what
we're looking for, since he doesn't even begin at the beginning but from
the end, trying to swim through his speech in reverse, on his back, and
begins from the things the lover would say to his beloved when he'd
already finished loving. Or am I wrong, Phaedrus, dear
thing?\goto{\high{168}}[notes.htmlux5cux23pla_0000398a]

{[}b1{]} PHAEDRUS What he makes his speech about, Socrates, is certainly
an ending.\goto{\high{169}}[notes.htmlux5cux23pla_0000400]

SOCRATES What about the rest? Don't the elements of the speech seem to
have been thrown in a random heap? Or do you {[}b5{]} think the second
thing he said had to be placed second for some essential reason, or any
of the others where {\em they} were? It seemed to me, as one who knows
nothing about it, that the writer had said just what happened to occur
to him, in a not ignoble way; but do you know of any constraint deriving
from the science of speech-writing which made him place these thoughts
one beside another in this order?

PHAEDRUS You're kind to think me competent to understand {[}c1{]} so
precisely what he has done.

SOCRATES But this much I think you would say: that every speech should
be put together like a living creature, as it were with a body of its
own, so as not to lack either a head or feet {[}c5{]} but to have both
middle parts and extremities, so written as to fit both each other and
the whole.

PHAEDRUS Yes indeed.

SOCRATES Well then, ask if your friend's speech is like this or if it's
some other way, and you'll find it exactly like the epigram that some
say is inscribed on the tomb of Midas the
Phrygian.\goto{\high{170}}[notes.htmlux5cux23pla_0000402]

PHAEDRUS What's this epigram, and what feature of it are you {[}d1{]}
talking about?

SOCRATES The poem's this:

A bronze-clad maid I stand on Midas' tomb,

As long as rivers run and trees grow tall,

A guardian of this much-lamented grave, {[}d5{]}

I'll tell the traveller: Midas rests within.

I think you see that it makes no difference whether any part of {[}e1{]}
it is put first or last.

PHAEDRUS You're making fun of our speech, Socrates.

SOCRATES Well, to avoid your becoming upset, let's leave this speech to
one side -- though it does seem to me to contain plenty {[}e5{]} of
examples which someone could glance at with profit, if not exactly by
trying to imitate them -- and pass on to the others. For in my view
there was something in them which should be noticed by those who wish to
enquire into speeches.

PHAEDRUS What sort of thing do you mean? {[}265a{]}

SOCRATES They were, I think, opposites: one of them said that favours
should be granted to the one in love, the other to the one not.

PHAEDRUS And very manfully too.

SOCRATES I thought you were going to speak the truth, and {[}a5{]} say
‘madly', which in fact was the very thing I was looking for. We said,
didn't we, that love was a kind of madness?

PHAEDRUS Yes.

SOCRATES And that there were two kinds of madness, the one caused by
sicknesses of a human sort, the other coming about {[}a10{]} from a
divinely caused reversal of our customary ways of behaving?

PHAEDRUS Certainly. {[}b1{]}

SOCRATES And of the divine kind we distinguished four parts, belonging
to four gods, taking the madness of the seer as Apollo's inspiration,
that of mystic rites as Dionysus', poetic madness, for its part, as the
Muses', and the fourth as that belonging to Aphrodite and Love. The
madness of love we said {[}b5{]} was best, and -- by expressing the
experience of love through some kind of simile, which allowed us perhaps
to grasp some truth, though maybe also it took us in a wrong direction,
and {[}c1{]} mixing together a not wholly implausible speech -- we sang
a playful hymn in the form of a story, in a fittingly quiet way, to my
master and yours, Phaedrus, Love, watcher over beautiful boys.

PHAEDRUS And it gave me great pleasure to hear it.

{[}c5{]} SOCRATES Well then, let's take up this point from it: how the
speech\goto{\high{171}}[notes.htmlux5cux23pla_0000404] was able to pass
over from censure to praise.

PHAEDRUS Precisely what aspect are you referring to?

SOCRATES To me it seems that the rest really was playfully done, by way
of amusement; but by chance two kinds of
thing\goto{\high{172}}[notes.htmlux5cux23pla_0000406] found expression,
whose significance it would be gratifying to grasp in a scientific way.

PHAEDRUS What were these?

SOCRATES First, there is perceiving together and bringing into one form
items which are scattered in many
places,\goto{\high{173}}[notes.htmlux5cux23pla_0000408] in order that
one may define each thing and make clear whatever it is {[}d5{]} that
one wishes to instruct\goto{\high{174}}[notes.htmlux5cux23pla_0000410]
one's audience about on any given occasion. Just so with the things we
said just now about what love amounts to when defined: whether what was
said was right or wrong, because of it the
speech\goto{\high{175}}[notes.htmlux5cux23pla_0000412] was able to say
what was at any rate clear and self-consistent.

PHAEDRUS And what's the second kind of thing you're talking about,
Socrates?

SOCRATES Being able to cut up whatever it is again, kind by
kind,\goto{\high{176}}[notes.htmlux5cux23pla_0000414] according to its
natural joints, and not to try to break any part into pieces, like an
inexpert butcher; as just now the two speeches took the unreasoning
aspect of the mind as one {[}266a{]} form together, and in the way that
a single body naturally has its parts in pairs, with both members of
each pair having the same name, and labelled respectively left and
right, so too the speeches regarded derangement as naturally a single
form in us, and the one cut off the part on the left-hand side, then
cutting {[}a5{]} it again and not giving up until it had found among the
parts a love that is, as we say, ‘left-handed', and abused it with full
justice, while the other speech led us to the parts of madness on the
right-hand side, and discovering and setting forth a love that shares
the same name as the other but is divine, it praised {[}b1{]} it as the
cause of our greatest goods.

PHAEDRUS Very true.

SOCRATES Now I am myself, Phaedrus, a lover of these divisions and
collections, so that I may be able both to speak and {[}b5{]} to think;
and if I find anyone else who I think has the natural capacity to look
to one and to many,\goto{\high{177}}[notes.htmlux5cux23pla_0000416] I
pursue him ‘in his footsteps, behind him, as if he were a
god'.\goto{\high{178}}[notes.htmlux5cux23pla_0000418] And the name I
give those who can do this -- whether it's the right one or not, god
knows,\goto{\high{179}}[notes.htmlux5cux23pla_0000420] but at any rate
up till now I have called them {[}c1{]} ‘experts in
dialectic'.\goto{\high{180}}[notes.htmlux5cux23pla_0000422] But now tell
me what we should have to call them if we learned from you and Lysias;
or is this that very thing, the science of speaking, by means of which
Thrasymachus and the rest have become clever at speaking themselves, and
make others the same, if they are willing to {[}c5{]} bring them gifts
as if they were kings?

PHAEDRUS Royal these people
are,\goto{\high{181}}[notes.htmlux5cux23pla_0000424] but they certainly
don't possess knowledge of the things you're asking about. You do seem,
though, to be calling this kind of thing by the right name when you call
it dialectical; the rhetorical kind seems to me still to be eluding
us.\goto{\high{182}}[notes.htmlux5cux23pla_0000426]

SOCRATES What do you mean? Could there perhaps be something {[}d1{]}
thing fine that's divorced from the principles in question and is
nonetheless grasped in a scientific way? We must certainly not treat it
without proper respect, you and I, and we must say just what that part
of rhetoric is which is being left out. PHAEDRUS There are a great many
things left, I think, Socrates: {[}d5{]} the things in the books that
have been written on the science of speaking.

SOCRATES A timely reminder. First of all, I think, there's the point
that a ‘preamble' must be given at the beginning of a speech; these are
the things you mean, aren't they -- the refinements of the science?

PHAEDRUS Yes. {[}e1{]}

SOCRATES In second place, there is to be something called an
‘exposition', with ‘testimonies' hard on its heels; thirdly ‘proofs',
fourthly ‘probabilities'; and I think ‘confirmation' and ‘further
confirmation' are mentioned, at least by that excellent {[}e5{]}
Byzantine artist in speeches.

PHAEDRUS You mean the worthy
Theodorus?\goto{\high{183}}[notes.htmlux5cux23pla_0000428]

{[}267a{]} SOCRATES Of course; and he tells us we must put in a
‘refutation' and ‘further refutation' both when prosecuting and when
defending. And must we not give public recognition to that most
admirable Parian, Evenus, for being the first to discover ‘covert
allusion' and ‘indirect praise'? Some say he also {[}a5{]} utters
‘indirect censures' in verse as an aid to memory; he's a clever one. And
shall we leave Tisias and Gorgias to their sleep, when they saw that
probabilities were to be given precedence over truths, and when they
make small things appear large and {[}b1{]} large things small by
force\goto{\high{184}}[notes.htmlux5cux23pla_0000430] of speech, and put
new things in an old way and things of the opposite sort in a new way,
and discovered conciseness of speech and infinite length on every
subject? Though when once Prodicus heard me talking like this, he
laughed and said that he alone had discovered what kind of speeches are
needed: what are needed, he said, are neither long {[}b5{]} speeches nor
short ones but ones of a fitting length.

PHAEDRUS Masterly, Prodicus!

SOCRATES And must we not mention Hippias? I think our friend from Elis
would cast his vote with Prodicus.

PHAEDRUS Certainly.

{[}b10{]} SOCRATES And how then are we to tell of the terms Polus
{[}c1{]} has enshrined -- terms like ‘speaking with reduplication' and
‘speaking with maxims' and ‘speaking with images' -- and the names that
Licymnius gave him as a present for the production of fine diction?

{[}c5{]} PHAEDRUS And weren't there some such things that belonged to
Protagoras?

SOCRATES Yes, my boy, there was a ‘correctness of diction', and many
other fine things. Then again, the scientific mastery of wailing
speeches dragged out in connection with old age and poverty seems to me
to belong to the might of the
Chalcedonian,\goto{\high{185}}[notes.htmlux5cux23pla_0000432] and the
man has also become clever at rousing anger in {[}d1{]} large numbers of
people all at once, and again, when once they are angry, at charming
them with incantations, as he put it; and at both devising and refuting
calumnies, from whatever source, he is unbeatable. As for the ending of
speeches, everyone seems to be in complete agreement; some call it
‘recapitulation', while others call it by other names.

PHAEDRUS You mean summarizing the points at the end, and {[}d5{]} so
reminding the audience of what has been said?

SOCRATES That's what I mean -- and anything else you can add on the
subject of speaking scientifically.

PHAEDRUS Only small things, and not worth mentioning.

SOCRATES Then let's leave the small points; let's hold what we
{[}268a{]} have more closely up to the light, and see just what the
power of the science is that's contained in them.

PHAEDRUS A very forceful power it is, Socrates, when it's a question of
mass gatherings.

SOCRATES You're right. But see, my fine friend, whether after {[}a5{]}
all you don't think, as I do, that their warp has some gaps in it.

PHAEDRUS Do show me.

SOCRATES Tell me then: if someone came up to your friend Eryximachus or
his father, Acumenus,\goto{\high{186}}[notes.htmlux5cux23pla_0000434]
and said, ‘I know how {[}a10{]} to apply certain things to people's
bodies so as to make them warm, if I want to, and to cool them down and,
if I see fit, to {[}b1{]} make them vomit, or again make their bowels
move, and all sorts of things like that; and because I know all that, I
claim to be an expert doctor and to be able to make an expert of anyone
else to whom I impart knowledge of these things' -- when they heard him
say that, what do you think {\em they} would say? {[}b5{]}

PHAEDRUS What else but to ask him whether he also knew both to whom he
ought to do all these things and when, and to what extent?

SOCRATES So if he said ‘No, not at all; but I expect someone to be able
to do the things you ask about by himself, if he has {[}c1{]} learned
the things I teach'?

PHAEDRUS I think they'd say the man is mad, and thinks he's become a
doctor from having heard something somewhere from a book, or from having
stumbled across some common-or-garden remedies, when he has no knowledge
of the science itself.

SOCRATES And what about if someone came up to Sophocles {[}c5{]} or
Euripides and said that he knew how to compose very long passages about
a small subject and very short ones about a large one, and piteous
speeches, when he wished, or again {[}d1{]} frightening and threatening
ones, and everything else like that, and that he thought that by
teaching these things he was passing on the making of tragedy?

PHAEDRUS They too, I think, Socrates, would laugh if anyone thought that
tragedy was anything other than the arrangement {[}d5{]} of these things
-- their being put together so as to fit both each other and the whole.

SOCRATES But I don't think they'd abuse him coarsely; just as a musical
expert, if he met someone who thought he knew all about harmony just
because he happened to know how to {[}e1{]} produce the highest and the
lowest notes with strings, would not say savagely ‘You're off your head,
you wretch,' but, being a musician, more gently, ‘My dear fellow, the
person who means to be an expert in harmony must certainly know that
{[}e5{]} too, but there is nothing to prevent someone in your condition
from having not the slightest understanding of harmony; for what you
know is what has to be learned before harmony itself, not the elements
of harmony as such.'

PHAEDRUS Quite right.

{[}269a{]} SOCRATES So Sophocles too would say that the man displaying
himself to him and Euripides knew the preliminaries to tragedy and not
its elements, and Acumenus that the individual in his case knew the
preliminaries to medicine but not the elements of medicine.

PHAEDRUS Absolutely.

{[}a5{]} SOCRATES And what do we think, if the ‘honey-toned Adrastus',
or Pericles,\goto{\high{187}}[notes.htmlux5cux23pla_0000436] heard of
some of those really fine techniques we were going through just now --
things like ‘speaking with brevity' and ‘speaking with images', and all
the other things we went through and said we should look at under the
{[}b1{]} light -- do we think that they, like you and me, would coarsely
utter some uneducated expression at those who have written these things
up and teach them as a science of rhetoric, or, because they are wiser
than us, do we think they would {[}b5{]} reproach us and say, ‘Phaedrus
and Socrates, one should not get angry but be forgiving, if some people
who do not know how to
converse\goto{\high{188}}[notes.htmlux5cux23pla_0000438] prove unable to
give a definition of what rhetoric is, and as a result of being in this
state think that they have discovered rhetoric when they have merely
learned the necessary preliminaries to the science, believing that when
they {[}c1{]} teach these things to other people they have given them a
complete course in rhetoric; and that the matter of putting all of these
things persuasively, and of arranging the whole, as something involving
no difficulty, their pupils must supply in their speeches from their own
resources'? {[}c5{]}

PHAEDRUS I rather think, Socrates, that the substance of the science
that these men teach and write up as rhetoric is somehing like that, and
to me, at any rate, you seem to be right; but how and from where can one
provide for oneself the science {[}d1{]} belonging to the real expert in
rhetoric and the really persuasive speaker?

SOCRATES As for the ability to acquire it, Phaedrus, so as to become a
complete performer, probably -- perhaps even necessarily -- the matter
is as it is in all other cases: if it is naturally in you to be a good
orator, a notable orator you will be when you have acquired knowledge
and practice besides, {[}d5{]} and whichever you lack of these, you will
be incomplete in this respect. But as for the part of it that is a
science, the way of proceeding seems to me not to be the one that Lysias
and Thrasymachus choose.

PHAEDRUS Then how should one proceed?

SOCRATES I suppose it's no surprise, my good fellow, that {[}e1{]}
Pericles turned out to be the most complete of all with respect to
rhetoric.

PHAEDRUS Why do you say that?

SOCRATES All sciences of importance require the addition of {[}270a{]}
babbling and lofty talk\goto{\high{189}}[notes.htmlux5cux23pla_0000440]
about nature; for the relevant high-mindedness and effectiveness in all
directions seem to come from some such source as that. This is something
that Pericles acquired in addition to his natural ability; for I think
because he fell in with Anaxagoras, who was just such a person, so
becoming filled with lofty talk, and arriving at the nature of {[}a5{]}
mind and the absence of mind, which were the very subjects about which
Anaxagoras\goto{\high{190}}[notes.htmlux5cux23pla_0000442] used to talk
so much, he was able to draw from there and apply to the science of
speaking what was applicable to it.

PHAEDRUS What do you mean by that?

{[}b1{]} SOCRATES The method of the science of medicine is, I suppose,
the same as that of the science of rhetoric.

PHAEDRUS How is that?

SOCRATES In both sciences it is necessary to determine the {[}b5{]}
nature of something, in the one science the nature of body, in the other
the nature of soul, if you are to proceed scientifically, and not merely
by knack and experience,\goto{\high{191}}[notes.htmlux5cux23pla_0000444]
to produce health and strength in the one by applying medicines and diet
to it, and to pass on to the other whatever conviction you wish, along
with excellence, by applying
words\goto{\high{192}}[notes.htmlux5cux23pla_0000446] and practices in
conformance with law and custom.

{[}b10{]} PHAEDRUS Probably it is like that, Socrates.

{[}c1{]} SOCRATES Do you think, then, that it's possible to understand
the nature of soul satisfactorily without understanding the nature of
the whole?\goto{\high{193}}[notes.htmlux5cux23pla_0000448]

PHAEDRUS If one is to place any reliance on Hippocrates the
Asclepiad,\goto{\high{194}}[notes.htmlux5cux23pla_0000450] one can't
understand about the body either {[}c5{]} without proceeding in this
way.

SOCRATES And he's right, my friend; but besides Hippocrates we should
examine the argument\goto{\high{195}}[notes.htmlux5cux23pla_0000452] to
see if it agrees with him.

PHAEDRUS I accept that.

SOCRATES Well then, on the subject of nature, see what {[}c10{]}
Hippocrates and the true argument say about it. Shouldn't one {[}d1{]}
reflect about the nature of anything like this: First, is the thing
about which we shall want to be experts ourselves and be capable of
making others expert about something that is simple or complex? Next, if
it is simple, we should consider, shouldn't {[}d5{]} we, what natural
capacity it has for acting, and on what, or what capacity it has for
being acted upon, and by what; and if it has more
forms\goto{\high{196}}[notes.htmlux5cux23pla_0000454] than one, we
should count these, and see in the case of each, as in the case where it
had only one, with which of them it is its nature to do what, or with
which to have what done to it by what?

PHAEDRUS Probably, Socrates.

SOCRATES At any rate, proceeding without doing these things would seem
to be just like a blind man's progress. But on no {[}e1{]} account must
we represent the man who pursues anything scientifically as like someone
blind, or deaf; it's clear that if anyone teaches anyone speech-making
in a scientific way, he will reveal precisely the essential nature of
that thing to which his pupil will apply his speeches; and that, I
think, will be soul. {[}e5{]}

PHAEDRUS Of course.

SOCRATES Then all his effort is concentrated on that; for it is
{[}271a{]} in that that he tries to produce conviction. True?

PHAEDRUS Yes.

SOCRATES In that case, it is clear that both Thrasymachus and anyone
else who seriously teaches a science of rhetoric will first {[}a5{]}
write with complete accuracy and enable us to see whether soul is
something that is one and uniform in nature or complex like the form of
the body; for this is what we say is to reveal the nature of something.

PHAEDRUS Yes, absolutely.

SOCRATES And in the second place, he will show with which {[}a10{]} of
its forms it is its nature to do what, or to have what done to it by
what.

PHAEDRUS Of course.

SOCRATES And then, thirdly, having classified the
kinds\goto{\high{197}}[notes.htmlux5cux23pla_0000456] of {[}b1{]}
speeches and of soul, and the ways in which these are affected, he will
go through all the causes, fitting each to each and explaining what sort
of soul's being subjected to what sorts of speeches necessarily results
in one being convinced and another not, giving the cause in each case.
{[}b5{]}

PHAEDRUS It would certainly seem to be best like that.

SOCRATES Indeed, my friend, if a model
speech\goto{\high{198}}[notes.htmlux5cux23pla_0000458] or a real one is
ever spoken or written in any way other than this, it will never be
given or written scientifically -- not on any other {[}c1{]} subject,
and not on this one.\goto{\high{199}}[notes.htmlux5cux23pla_0000460] But
those who now write speech manuals, the people you have listened to, are
cunning, and keep the secret to themselves, although they know perfectly
well about soul; so until they both speak and write in the following
way, let's not believe their claim that they write scientifically.

PHAEDRUS What way is this? {[}c5{]}

SOCRATES To give the actual words would not be easy; but I'm willing to
say how one should write\goto{\high{200}}[notes.htmlux5cux23pla_0000462]
if it's to be as scientific as it is possible to be.

PHAEDRUS Say it then.

{[}c10{]} SOCRATES ‘Since the power of speech is in fact a leading of
the soul, {[}d1{]} the man who means to be an expert in rhetoric must
know how many forms soul has. Thus their number is so and so, and they
are of such and such kinds, which is why some people are like this, and
others like that; and these having been distinguished in this way, then
again there are so many forms of {[}d5{]} speeches, each one of such and
such a kind. People of one kind are easily persuaded for one sort of
reason by one kind of speech to hold one kind of opinion, while people
of another kind are for some other sorts of reasons difficult to
persuade.

‘Having then grasped these things satisfactorily, after that {[}e1{]}
the student must observe them as they are in real life, and actually
being put into practice, and be able to follow them with keen
perception, or otherwise be as yet no further on from the things he
heard earlier when he was with me. But when he both has sufficient
ability to say what sort of man is persuaded by what sorts of things,
and is capable of telling himself when {[}272a{]} he observes him that
{\em this} is the man, {\em this} the nature of person that was
discussed before, now actually present in front of him, to whom he must
now apply {\em these} kinds of speech in {\em this} way in order to
persuade him of {\em this} kind of thing; when he now has all of this,
and has also grasped the occasions for speaking {[}a5{]} and for holding
back, and again for speaking concisely and piteously and in an
exaggerated fashion, and for all the forms of speeches he may learn,
recognizing the right and the wrong time for these, {\em then} his grasp
of the science will be well and completely finished, but not before
that; but in whichever of {[}b1{]} these things someone is lacking when
he speaks or teaches or writes, and says that he speaks scientifically,
the person who disbelieves him is in the stronger position.' ‘Well then,
Phaedrus and Socrates,' perhaps our writer will say, ‘do you agree, or
should we accept it if the science of speaking is stated in some other
way?'

{[}b5{]} PHAEDRUS It's impossible, I think, Socrates, to accept any
other description; yet it seems no light undertaking.

SOCRATES You're right. It's just for this reason that we must turn all
our arguments upside down in order to see whether some easier and
shorter route to the science doesn't show up some- {[}c1{]} where, so
that a person doesn't waste his time going off on a long and rough road
when he could take a short and smooth one. But if you have any help to
give from what you have heard from Lysias or anyone else, try to
remember it and tell me.

PHAEDRUS If it depended on trying, I would; but as things are, {[}c5{]}
I'm just not in a position to help.

SOCRATES Then would you like me to mention something I've heard from
some of those who make these things their business?

PHAEDRUS Of course

SOCRATES The saying goes, Phaedrus, that it's right to give the
{[}c10{]} wolf's side of the case as well.

PHAEDRUS Then you do just that. {[}d1{]}

SOCRATES Well then, they say that there is no need to treat these things
so portentously, or carry them back to general principles, going the
long way round; for it's just what we said at the very beginning of this
discussion -- that the person who means to be competent at rhetoric need
have nothing to do {[}d5{]} with the truth about just or good things, or
indeed about people who are such by nature or upbringing. For, they say,
in the law-courts no one cares in the slightest for the truth about
these things but only for what is convincing; and what is convincing
{[}e1{]} is what is {\em probable}, which is what the person who means
to speak scientifically must pay attention to. They go on to say that in
fact sometimes one should not even say what was actually done, if it is
improbable, but rather what is probable, both when accusing and when
defending; whatever one's purpose when speaking, the {\em probable} is
what must be pursued, and {[}e5{]} that means frequently saying goodbye
to the truth. When this happens throughout one's entire speech, it gives
one the entire {[}273a{]} science.

PHAEDRUS You've stated just what those who profess to be experts in
speaking say; for I'm reminded, now you say it, that we did touch
briefly on this sort of thing before, and it seems {[}a5{]} a point of
crucial significance to those concerned with these things.

SOCRATES But you've gone over the man Tisias himself carefully; so let
Tisias tell us this too: doesn't he say that the {[}b1{]} probable is
just what most people think to be the case?

PHAEDRUS Just that.

SOCRATES I suppose it was on making this clever, and scientific,
discovery that he wrote to the effect that if a weak but brave {[}b5{]}
man beats up a strong coward and steals his cloak or something else of
his, and is taken to court for it, then neither party should speak the
truth; the coward should say that he wasn't beaten up by the brave man
single-handed, while the other man should establish that they were on
their own together, and should {[}c1{]} resort to the well-known
argument, ‘How could a man like me have assaulted a man like him?' The
coward will certainly not admit his cowardice but will try to invent
some other lie and so perhaps offer an opening for his opponent to
refute him. And in all other cases too the way to speak ‘scientifically'
will {[}c5{]} be something like this. True, Phaedrus?

PHAEDRUS Of course.

SOCRATES Hey! How cleverly hidden a science Tisias seems to have
discovered -- or whoever else it really was, and wherever he pleases to
borrow his name from.\goto{\high{201}}[notes.htmlux5cux23pla_0000464]
Still, my friend, should {[}c10{]} we or should we not say to him --

{[}d1{]} PHAEDRUS What?

SOCRATES This: ‘Tisias, we have for some time been saying, before you
came along, that this “probability” comes about in the minds of ordinary
people because of a likeness to the truth; {[}d5{]} and we showed only a
few moments ago that in every case it is the man who knows the truth who
knows best how to discover these likenesses. So if you have anything
else to say on the subject of a science of speaking, we'll gladly hear
it; if not, we'll believe what we showed just now, that unless someone
counts {[}e1{]} up the various natures of those who are going to listen
to him, and is capable both of dividing
things\goto{\high{202}}[notes.htmlux5cux23pla_0000466] up according to
their forms and of including each thing, one by one, under one kind, he
will never be an expert in the science of speaking to the degree
possible for humankind. This ability he will never {[}e5{]} acquire
without a great deal of application -- a labour that the sensible person
ought to undertake not for the purpose of speaking and acting in
relation to human beings but in order to be able both to say what is
gratifying\goto{\high{203}}[notes.htmlux5cux23pla_0000468] to the gods,
and to act in everything, so far as he can, in a way that is gratifying
to them. For you see, Tisias -- so say wiser people than us -- no one in
his right mind should practise at gratifying his fellow-slaves,
{[}274a{]} except as a secondary consideration, but rather at gratifying
good masters, of noble
stock.\goto{\high{204}}[notes.htmlux5cux23pla_0000470] So if the way
round is a long one, don't be surprised; for the journey is to be made
for the sake of important things, not for the things you have in mind.
Yet those too, as our argument asserts, if that is what one {[}a5{]}
wants, will come about best as an outcome of the
others.'\goto{\high{205}}[notes.htmlux5cux23pla_0000472]

PHAEDRUS I think that what you say is very fine, Socrates, if only one
had the capacity for it.

SOCRATES But surely if one merely tries for the
beautiful,\goto{\high{206}}[notes.htmlux5cux23pla_0000474] to put up
with what it takes is beautiful too. {[}b1{]}

PHAEDRUS Indeed.

SOCRATES So let that be enough on the subject of the scientific and
unscientific aspects of speaking.

PHAEDRUS By all means. {[}b5{]}

SOCRATES What we have left is the subject of propriety and impropriety
in writing:\goto{\high{207}}[notes.htmlux5cux23pla_0000476] in what way,
when it is done, it will be done well, and in what way improperly. True?

PHAEDRUS Yes.

SOCRATES So do you know how you will most gratify god in relation to
speaking, whether actually doing it or talking about {[}b10{]} it?

PHAEDRUS Not at all; do you?

SOCRATES At least I can tell you something I've heard, from {[}c1{]}
people before me; only they know the truth of it. But if we were to find
this out for ourselves, would we care any longer at all about mere human
conjectures?\goto{\high{208}}[notes.htmlux5cux23pla_0000478]

PHAEDRUS What an absurd question! Tell me what you say you have heard.

SOCRATES Well, what I heard was that one of the ancient gods {[}c5{]} of
Egypt was at Naucratis in that country, the god to whom the sacred bird
they call the ibis belongs; the divinity's own name was Theuth. The
story was that he was the first to discover number and calculation, and
geometry and astronomy, as well {[}d1{]} as the games of draughts and
dice and, to cap it all, letters. King of all Egypt at that time was
Thamus -- of all of it, that is, that surrounds the great city of the
upper region, which the {[}d5{]} Greeks call Egyptian Thebes; Thamus
they call Ammon. Theuth came to him and displayed his technical
inventions, saying that they should be passed on to the rest of the
Egyptians; and Thamus asked what benefit each brought. As Theuth went
{[}e1{]} through them, Thamus criticized or praised whatever he seemed
to be getting right or wrong. It is reported that Thamus expressed many
views to Theuth about each science, both for and against; it would take
a long time to go through them in detail, but when it came to the
subject of letters, Theuth said, {[}e5{]} ‘But {\em this} study, King
Thamus, will make the Egyptians wiser and improve their memory; what I
have discovered is an
elixi\goto{\high{209}}[notes.htmlux5cux23pla_0000480] of memory and
wisdom.' Thamus replied, ‘Most scientific Theuth, one man has the
ability to beget the elements of a science, but it belongs to a
different person to be able to judge what measure of harm and help it
contains for those who are {[}275a{]} going to make use of it; so now
you, as the father of letters, have been led by your affection for them
to describe them as having the opposite of their real effect. For your
invention will produce forgetfulness in the souls of those who have
learned it, through lack of practice at using their memory, as through
reliance on writing they are reminded from outside by alien {[}a5{]}
marks, not from within, themselves by
themselves.\goto{\high{210}}[notes.htmlux5cux23pla_0000482] So you have
discovered an elixir not of memory but of reminding. To your students
you give an appearance of wisdom, not the reality of it; thanks to you,
they will hear many things without being {[}b1{]} taught them, and will
appear to know much when for the most part they know nothing, and they
will be difficult to get along with because they have acquired the
appearance of wisdom instead of wisdom itself.'

PHAEDRUS Socrates, how easily you make up stories, from Egypt or from
anywhere else you like!

{[}b5{]} SOCRATES Well, my friend, those in the sanctuary of Zeus of
Dodona claimed that words from an oak were the first prophetic
utterances. So the men of those days, because they were not wise like
you moderns, were content because of their simplicity to listen to oak
and rock,\goto{\high{211}}[notes.htmlux5cux23pla_0000484] provided only
that they said {[}c1{]} what was true; but for you, Phaedrus, perhaps it
makes a difference who the speaker is and where he comes from: you don't
just consider whether things are as he says or not. PHAEDRUS You're
right to rebuke me, and it seems to me to be as your Theban says about
letters.

SOCRATES So the man who thinks that he has left behind him {[}c5{]} a
science in writing, and no less the man who receives it from him, in the
belief that anything clear or certain will come from what is written
down, would be full of simplicity and would be really ignorant of
Ammon's prophetic utterance -- thinking that written words were anything
more than a reminder to the man {[}d1{]} who knows the subjects to which
the things written relate.

PHAEDRUS Quite right.

SOCRATES Yes, Phaedrus, because I think writing has this strange
feature, which makes it truly like painting. The offspring {[}d5{]}
spring of painting stand there as if alive, but if you ask them
something, they preserve a quite solemn silence. Similarly with written
words: you might think that they spoke as if they had some thought in
their heads, but if you ever ask them about any of the things they say
out of a desire to learn, they point to just one thing, the same each
time. And when once it is written, {[}e1{]} every composition trundles
about everywhere in the same way, in the presence both of those who know
about the subject and of those who have nothing at all to do with it,
and it does not know how to address those it should address and not
those it should not. When it is ill treated and unjustly abused, it
always needs its father to help it; for it is incapable of either
defending {[}e5{]} or helping itself.

PHAEDRUS You're quite right about that too.

SOCRATES Now then, do we see another kind of
speech,\goto{\high{212}}[notes.htmlux5cux23pla_0000486] a {[}276a{]}
legitimate brother of this last one? Do we see both how it comes into
being and how much better and more capable it is from its birth?

PHAEDRUS What kind are you referring to, and how does it ‘come into
being'?

SOCRATES The kind of speech that is written together with {[}a5{]}
knowledge in the soul of the learner, capable of defending itself, and
knowing how to speak and keep silent in relation to the people it
should.

PHAEDRUS You mean the living,
animate\goto{\high{213}}[notes.htmlux5cux23pla_0000488] speech of the
man who knows, of which written speech would rightly be called a kind of
phantom.

{[}b1{]} SOCRATES Absolutely. So tell me this: the sensible farmer who
had some seeds he cared about and wanted to bear fruit -- would he sow
them with serious purpose during the summer in some garden of
Adonis,\goto{\high{214}}[notes.htmlux5cux23pla_0000490] and delight in
watching the garden become {[}b5{]} beautiful in eight days, or would he
do that for the sake of amusement on a feast-day, if he did it at all;
whereas for the purposes about which he was in earnest, would he make
use of the science of farming and sow them in appropriate soil, being
content if what he sowed reached maturity in the eighth month?

{[}c1{]} PHAEDRUS Just so, I think, Socrates: he would do the one sort
of thing in earnest, the other in the other way, the way you say.

SOCRATES And are we to say that the man who has pieces of
knowledge\goto{\high{215}}[notes.htmlux5cux23pla_0000492] about what is
just, beautiful\goto{\high{216}}[notes.htmlux5cux23pla_0000494] and good
has a {[}c5{]} less sensible attitude towards his seeds than the farmer?

PHAEDRUS Hardly!

SOCRATES In that case he will not be in earnest about writing them in
water -- black water, sowing them through a pen with words that are
incapable of speaking in their own support, and incapable of adequately
teaching what is true.

{[}c10{]} PHAEDRUS It certainly isn't likely.

{[}d1{]} SOCRATES No, it isn't; but his gardens of letters, it seems, he
will sow and write for amusement, when he does write, laying up a store
of reminders both for himself, for when he ‘reaches a forgetful old
age',\goto{\high{217}}[notes.htmlux5cux23pla_0000496] and for anyone
following the same track, {[}d5{]} and he will be pleased as he watches
their tender growth; and when others resort to other sorts of
amusements, watering themselves with drinking-parties and the other
things that go along with these, then he, it seems, will spend his time
amusing himself with the things I say, instead of those others.

{[}e1{]} PHAEDRUS It's a quite beautiful form of amusement you're
talking of, Socrates, in contrast with a worthless one: if someone is
able to amuse himself with words, telling stories about justice and the
other subjects you speak
of.\goto{\high{218}}[notes.htmlux5cux23pla_0000498]

SOCRATES Yes, Phaedrus, just so. But I think it is far finer if one
{[}e5{]} is in earnest about those subjects: when one makes use of the
science of dialectic and, taking a fitting soul, plants and sows in it
words accompanied by knowledge, which are sufficient to help themselves
and the one who planted them, and are not without {[}277a{]} fruit but
contain a seed from which others grow in other soils, capable of
rendering that seed for ever immortal, and making the one who has it as
happy as it is possible for a man to be.

PHAEDRUS This is indeed much finer still. {[}a5{]}

SOCRATES So now, Phaedrus, since we've agreed about these issues we can
decide those others.

PHAEDRUS Which ones?

SOCRATES The ones we wanted to look into, and so got ourselves to the
present point: how we were to weigh up the reproach aimed {[}a10{]} at
Lysias about his writing of speeches, and about speeches themselves,
{[}b1{]} which were written scientifically and which not. Well then,
what is scientific and what is unscientific seems to me to have been
demonstrated in fair measure.

PHAEDRUS I thought so; but remind me again how.

SOCRATES Until a person knows the truth about each of the {[}b5{]}
things about which he speaks or writes, and becomes capable of defining
the whole by itself, and, having defined it, knows how to cut it up
again according to its forms until it can no longer be cut; and until he
has reached an understanding of the nature of soul along the same lines,
discovering the form of {[}c1{]} speech that fits each nature, and so
arranges and orders what he says, offering a
complex\goto{\high{219}}[notes.htmlux5cux23pla_0000500] soul complex
speeches containing all the modes, and simple speeches to a simple soul:
not until then will he be capable of pursuing the making of speeches as
a whole in a scientific way, to the degree that its nature allows,
{[}c5{]} whether for the purposes of teaching or for those of persuading
either, as the whole of our previous argument has indicated.

PHAEDRUS Absolutely; that was just about how it appeared to us.

SOCRATES And what about the matter of its being fine or {[}d1{]}
shameful to give speeches and write them, and the circumstances under
which it would rightly be called a disgrace or not? Hasn't what we said
a little earlier shown --

{[}d5{]} PHAEDRUS Shown what?

SOCRATES That whether Lysias or anyone else ever wrote or writes in the
future, either for private purposes or publicly, in the course of
proposing laws, so writing a political composition, and thinks there is
any great certainty or clarity in it, then it is {[}d10{]} a reproach to
its writer, whether anyone says so or not; for to {[}e1{]} be ignorant,
whether awake or asleep, about the nature of just and unjust and bad and
good cannot truly escape being a matter of reproach, even if the whole
mob applauds it.

PHAEDRUS No indeed.

{[}e5{]} SOCRATES But the person who thinks that there is necessarily
much that is merely for amusement in a written speech on any subject,
and that none has ever yet been written, whether in verse or in prose,
which is worth much serious attention, or indeed spoken, in the way that
rhapsodes\goto{\high{220}}[notes.htmlux5cux23pla_0000502] speak theirs,
to produce conviction without questioning or teaching, but that
{[}278a{]} the best of them have really been a way of reminding people
who know; who thinks that clearness and completeness and seriousness
exist only in those things that are taught about what is just and
beautiful and good, and are said for the purpose of {[}a5{]} someone's
learning from them, and genuinely written in the soul; who thinks that
discourses\goto{\high{221}}[notes.htmlux5cux23pla_0000504] of that kind
should be said to be as it were his legitimate sons, first of all the
one within {[}b1{]} him, if it is found there, and in second place any
offspring and brothers of this one that have sprung up simultaneously,
in the way they should, in other souls, other men; and who says goodbye
to the other kind -- {\em this}, surely, Phaedrus, will be the sort of
person you and I would pray that we both might come to be.

{[}b5{]} PHAEDRUS Yes, absolutely. I wish and pray for what you say.
SOCRATES So now let that count as our due amusement from the subject of
speaking. And as for you, Phaedrus, you go and tell Lysias that we two
came down to the spring and the sacred {[}c1{]} place of the Nymphs and
heard arguments\goto{\high{222}}[notes.htmlux5cux23pla_0000506] that
instructed us to tell this to Lysias and anyone else who composes
speeches, and to Homer and anyone else who has composed verses, whether
without music or to be sung, and, thirdly, to Solon and whoever writes
compositions in the form of political speeches, {[}c5{]} which he calls
laws: if he has composed these things knowing how the truth is, able to
help his composition when he is challenged on its subjects, and with the
capacity, speaking in his own person, to show that what he has written
is of little worth,\goto{\high{223}}[notes.htmlux5cux23pla_0000508] then
such a man ought not to derive his title from these, and be called after
them, but rather from those things in {[}d1{]} which he is seriously
engaged.

PHAEDRUS What are the titles you assign him, then? SOCRATES To call him
wise seems to me to be too much, and to be fitting only in the case of a
god; to call him either a
philosopher\goto{\high{224}}[notes.htmlux5cux23pla_0000510] or something
like that would both fit him more {[}d5{]} and be in better taste.

PHAEDRUS And not at all inappropriate.

SOCRATES On the other hand, the man who doesn't possess things of more
value than the things he composed or wrote, turning them upside down
over a long period of time, sticking {[}e1{]} them together and taking
them apart -- him, I think, you'll rightly call a poet or author of
speeches or writer of laws?

PHAEDRUS Of course.

SOCRATES Then tell that to your friend.

PHAEDRUS And what of you? What will you do? For we certainly {[}e5{]}
shouldn't pass over your friend, either.

SOCRATES Who do you mean?

PHAEDRUS The beautiful
Isocrates.\goto{\high{225}}[notes.htmlux5cux23pla_0000512] What will you
report to {\em him}, Socrates? What title shall we give him?

SOCRATES Isocrates is still young, Phaedrus; but I'm willing to
{[}e10{]} say what I prophesy for him. {[}279a{]}

PHAEDRUS What's that?

SOCRATES He seems to me to be on a level superior to Lysias and his
speeches in terms of his natural endowment, and to have a greater
nobility in the blend of his character; so there {[}a5{]} would be no
surprise, as he grows older, if the very speeches he works at now turned
out to make those of any other speech-writer worse than puerile by
comparison. Still more so, were he to be dissatisfied with what he does
now, and some diviner impulse led him to more important things; for
there is a certain innate philosophical instinct in the man's mind. So
that is the {[}b1{]} report I take from the gods here to Isocrates as my
beloved, and you take the other to Lysias as yours.

{[}b5{]} PHAEDRUS I'll do it. But let's go, now that the heat has become
milder.

SOCRATES Shouldn't we pray to the gods here before we go?

PHAEDRUS Of course.

SOCRATES Dear Pan and all you gods of this place, grant me that I may
become beautiful within; and that what is in my {[}c1{]} possession
outside me may be in friendly accord with what is inside. And may I
count the wise man as rich; and may my pile of gold be of a size that no
one but a man of moderate
desires\goto{\high{226}}[notes.htmlux5cux23pla_0000514] could bear or
carry it.

{[}c5{]} Do we still need anything else, Phaedrus? For me that prayer is
enough.

PHAEDRUS Make it a prayer for me too; for what friends have they share.

SOCRATES Let's go.

\section[title={Appendix:\crlf
The Structure of the
{\em Phaedrus}},reference={appendix.htmlux23pha0000599}]

The translation in this volume offers the reader no resting-points in
the form of chapters or any other form of subdivision; Plato did not
divide his text in this way, and, after all, by modern standards, the
{\em Phaedrus} is not very long -- maybe the size of a novella (though
infinitely more complex than any novella known to me). Still, the
following rough sketch of the structure of the dialogue may be found
useful:

\startplacetable[location=none]
\startxtable
\startxtablebody[body]
\startxrow
\startxcell[width={0.30\textwidth}] 227a -- 23oe: \stopxcell
\startxcell[width={0.70\textwidth}] Socrates meets Phaedrus; preliminary
conversation. \stopxcell
\stopxrow
\startxrow
\startxcell[width={0.30\textwidth}] 230e--234c: \stopxcell
\startxcell[width={0.70\textwidth}] Phaedrus reads Lysias'
speech. \stopxcell
\stopxrow
\startxrow
\startxcell[width={0.30\textwidth}] 234c -- 237b: \stopxcell
\startxcell[width={0.70\textwidth}] Transition to Socrates' first
speech. \stopxcell
\stopxrow
\startxrow
\startxcell[width={0.30\textwidth}] 237b -- 241d: \stopxcell
\startxcell[width={0.70\textwidth}] Socrates' first speech on
{\em erôs}. \stopxcell
\stopxrow
\startxrow
\startxcell[width={0.30\textwidth}] 241d-243e: \stopxcell
\startxcell[width={0.70\textwidth}] Transition to Socrates' second
speech. \stopxcell
\stopxrow
\startxrow
\startxcell[width={0.30\textwidth}] 243e -- 245c: \stopxcell
\startxcell[width={0.70\textwidth}] Socrates' second speech
begins. \stopxcell
\stopxrow
\startxrow
\startxcell[width={0.30\textwidth}] 245c -- 249d: \stopxcell
\startxcell[width={0.70\textwidth}] ‘Experiences and actions' of divine
and human souls. \stopxcell
\stopxrow
\startxrow
\startxcell[width={0.30\textwidth}] 249d -- 257a: \stopxcell
\startxcell[width={0.70\textwidth}] The blessings of the madness of
{\em erôs}. \stopxcell
\stopxrow
\startxrow
\startxcell[width={0.30\textwidth}] 257a -- b: \stopxcell
\startxcell[width={0.70\textwidth}] A prayer to Love. \stopxcell
\stopxrow
\startxrow
\startxcell[width={0.30\textwidth}] 257b -- 259d: \stopxcell
\startxcell[width={0.70\textwidth}] Transition to a discussion of
speaking and writing. \stopxcell
\stopxrow
\startxrow
\startxcell[width={0.30\textwidth}] 259e -- 274b: \stopxcell
\startxcell[width={0.70\textwidth}] Rhetoric -- as it should be, and as
it is. \stopxcell
\stopxrow
\startxrow
\startxcell[width={0.30\textwidth}] 274b -- 277a: \stopxcell
\startxcell[width={0.70\textwidth}] How useful is the written text (or
the set speech) as a medium of communication and teaching? \stopxcell
\stopxrow
\startxrow
\startxcell[width={0.30\textwidth}] 277a -- 279b: \stopxcell
\startxcell[width={0.70\textwidth}] Conclusions. \stopxcell
\stopxrow
\stopxtablebody
\startxtablefoot[foot]
\startxrow
\startxcell[width={0.30\textwidth}] 279b -- c: \stopxcell
\startxcell[width={0.70\textwidth}] A final prayer, after which Socrates
and Phaedrus leave. \stopxcell
\stopxrow
\stopxtablefoot
\stopxtable
\stopplacetable

\section[title={Notes},reference={notes.htmlux23pha0000644}]

\goto{1}[chapter-001.htmlux5cux23pla_0000071]. {\em Cephalus}: Cephalus
plays a prominent role in the {\em Republic}, the main action of which
takes place at his house.

\goto{2}[chapter-001.htmlux5cux23pla_0000073]. {\em outside the wall}:
The Lysis too is set outside the walls of Athens but right beneath them;
Socrates evidently only left the city when he had to (on military
service: see e.g. Alcibiades' speech at the end of the {\em Symposium}),
and -- as we shall soon be told -- is a stranger even in the countryside
of Attica.

\goto{3}[chapter-001.htmlux5cux23pla_0000075]. {\em Acumenus}: Father
(if the same person is meant as at 268a) of another medical doctor,
Eryximachus, one of the party-goers and speakers in the {\em Symposium}
(Phaedrus is his friend and near contemporary).

\goto{4}[chapter-001.htmlux5cux23pla_0000077]. {\em Epicrates' house}:
Evidently Epicrates (see Nails 2003) was himself, like Lysias, an
orator, who shared Lysias' democratic leanings; Morychus, whose house he
owned, has a name in comedy for his extravagance. (So evidently Lysias
mixes with the very wealthy.)

\goto{5}[chapter-001.htmlux5cux23pla_0000079]. {\em temple of Olympian
Zeus}: Still one of the landmarks of Athens, near Syntagma Square.

\goto{6}[chapter-001.htmlux5cux23pla_0000081]. {\em Obviously\ldots{}
speeches}: ‘Obviously', because of Lysias' fame as an
orator/speech-writer. ‘Speeches', or more generally {\em types of
discourse}, will be one of the key subjects of the {\em Phaedrus}.

\goto{7}[chapter-001.htmlux5cux23pla_0000083]. {\em to quote Pindar}:
Pindar, {\em Isthmians} 1.2 (adapted, as quotations often are in Plato).

\goto{8}[chapter-001.htmlux5cux23pla_0000085]. {\em love}: ‘Love' here
is {\em erôs}, which will be another key subject of the dialogue -- that
is, passionate love, of which consensual sex would be a normal and
expected (or hoped-for) component -- at least as normally understood.
Phaedrus' remark here (that Lysias' speech will be an ‘appropriate' one
for Socrates to hear) reflects what is -- at any rate in the world of
the Platonic dialogues -- a familiar feature of Socrates' character,
namely his eroticism, in particular his attraction to beautiful boys.
And yet, later in the {\em Phaedrus}, as in the {\em Symposium} and the
{\em Lysis}, he will be found advocating a kind of erotic relationship
that in its ideal form actually {\em excludes} sexual intercourse as a
diversion from the real object, and goal, of {\em erôs}: knowledge.

\goto{9}[chapter-001.htmlux5cux23pla_0000087]. {\em favours}: I.e., as
will become patently obvious, sexual favours. Throughout the dialogue,
sexual activity will be referred to only indirectly, in Lysias' case as
a kind of show of good taste; Socrates will then follow suit in his
first speech. Paradoxically, the closest the {\em Phaedrus} comes to the
overt use of sexual language will be when Socrates, in his second
speech, describes the true lover's passion as he remembers Beauty
Itself, and his soul sprouts its wings (so taking it {\em away} from the
physical aspects of existence). See Introduction, p.
\goto{xxii}[introduction.html].

\goto{10}[chapter-001.htmlux5cux23pla_0000089]. {\em the general good}:
As so often in Plato, there is a point behind the joke: Socrates will
ultimately suggest that all truly scientific rhetoric will be ‘for the
general good'.

\goto{11}[chapter-001.htmlux5cux23pla_0000091]. {\em as Herodicus
recommends}: Is this a reference to training for {\em sprinting}? The
run to Megara would be more like a Marathon, but then Socrates/Plato
evidently did not think much of Herodicus (see {\em Republic} 406a --
b).

\goto{12}[chapter-001.htmlux5cux23pla_0000093]. {\em I'll swear by the
Dog}: A favourite, characteristic oath of Socrates'.

\goto{13}[chapter-001.htmlux5cux23pla_0000095]. {\em hearing people
speak}: Or ‘hearing speeches', which is how Phaedrus might understand
Socrates here. But the {\em logoi} Socrates is most interested in, as we
shall discover by the end of the {\em Phaedrus}, are not ‘speeches' but
talk of a quite different sort. ({\em Logos}: ‘speech', ‘discourse',
‘word', ‘thing said'; also ‘account', ‘reason'.)

\goto{14}[chapter-001.htmlux5cux23pla_0000097]. {\em seeing, seeing
him}: Probably another poetic reference; in any case Socrates' language
half-mimics Phaedrus' supposed ‘manic frenzy'. (Madness in various forms
will be another central topic of the {\em Phaedrus}.)

\goto{15}[chapter-001.htmlux5cux23pla_0000099]. {\em you've foiled me}:
Writing contrasted with memory: cf. the story of Theuth and Thamus at
{\em Phaedrus} 274c -- 275b.

\goto{16}[chapter-001.htmlux5cux23pla_0000101]. {\em the Ilissus}: No
longer, sadly, part of an idyllic landscape, in modern Athens (nor
usually even flowing in high summer).

\goto{17}[chapter-001.htmlux5cux23pla_0000103]. {\em Boreas\ldots{}
Oreithuia}: Boreas is the god who is the North Wind; Oreithuia was
daughter of Erechtheus, primeval King of Athens.

\goto{18}[chapter-001.htmlux5cux23pla_0000105]. {\em Agra}: One of the
‘demes', or basic administrative districts, of Athens.

\goto{19}[chapter-001.htmlux5cux23pla_0000107]. {\em for goodness'
sake}: What Phaedrus actually says is ‘by Zeus!', which in English fails
quite to convey his tone.

\goto{20}[chapter-001.htmlux5cux23pla_0000109]. {\em wise people}: Of
which, of course, Socrates is not one; his typical claim is that he
knows {\em nothing}.

\goto{21}[chapter-001.htmlux5cux23pla_0000111]. {\em Delphic
inscription}: The inscription would famously have been found on, or in,
the temple of Apollo at Delphi.

\goto{22}[chapter-001.htmlux5cux23pla_0000113]. {\em Typhon}: A
hundred-headed dragon, the last obstacle between Zeus and kingship over
the gods (Hesiod, {\em Theogony} 82off). Socrates will later give an
answer to his present question here (am I like the Typhon, or a simpler
creature?) by comparing the human soul to a charioteer and his two
horses, one wild, the other of a nobler sort; for the full significance
of this question, see Introduction, pp.
\goto{xxiii-xxv}[introduction.html].

\goto{23}[chapter-001.htmlux5cux23pla_0000115]. {\em conversation:
Logoi} again (see n. 13 above).

\goto{24}[chapter-001.htmlux5cux23pla_0000117]. {\em By Hera\ldots{}
stopping-place}: The tone and language of the description that follows
are those of a poet's ideal landscape. It is open to question how
seriously we should take Socrates' enthusiasm; later, certainly, he will
raise questions about the nature and location of true beauty.

\goto{25}[chapter-001.htmlux5cux23pla_0000119]. {\em agnus castus: Vitex
agnus castus}, a plant of the vervain genus.

\goto{26}[chapter-001.htmlux5cux23pla_0000121]. {\em Achelous}: ‘The
river god {\em par excellence}' (Hamilton 1973).

\goto{27}[chapter-001.htmlux5cux23pla_0000123]. {\em extraordinary}:
Literally ‘out of place' ({\em atopos}); similarly at 229c6 above.

\goto{28}[chapter-001.htmlux5cux23pla_0000125]. {\em people in the
city}: At any rate, the Socrates of the dialogues spends all his time
talking to them, and asking questions.

\goto{29}[chapter-001.htmlux5cux23pla_0000127]. {\em prescription}: The
{\em pharmakon} -- a term that can be used of either remedial drugs or
toxic ones.

\goto{30}[chapter-001.htmlux5cux23pla_0000129]. {\em speeches in books}:
Just what Socrates really thinks about {\em logoi} in books we shall
discover in the concluding parts of the {\em Phaedrus}.

\goto{31}[chapter-001.htmlux5cux23pla_0000131]. {\em Listen, then}: Is
the speech authentically Lysianic or not? Despite divisions in scholarly
opinion, it is in the end hard to believe that this {\em Erotic Essay}
(we have an example by the most outstanding of all Attic orators,
Demosthenes) is not a Platonic parody of the genre. Such speeches would
typically have been displays of virtuosity by professional writers, and
typically, like the present example, they would have been on paradoxical
themes. (If the speaker is really not in the grip of erotic passion, why
should he want sex with the boy in the first place?)

\goto{32}[chapter-001.htmlux5cux23pla_0000133]. {\em these things}: I.e.
sex.

\goto{33}[chapter-001.htmlux5cux23pla_0000135]. {\em For it\ldots{} own
capacity}: I.e. it will all be done in a wholly businesslike way, to fit
in with the non-lover's normal schedule and strictly without any damage
to his own interests (unlike the other man, driven by the ‘compulsion'
of love).

\goto{34}[chapter-001.htmlux5cux23pla_0000137]. {\em please the other
party}: And, presumably, the speaker: the term used for ‘to please' is
{\em charizesthai}, which is what the speaker wants the boy to do to him
(see 227c7). But since he's trying to persuade the boy, he specifically
mentions only the non-lover's attempt to please the {\em boy}
({\em autois}, 231c6).

\goto{35}[chapter-001.htmlux5cux23pla_0000139]. {\em words: Logoi again}
(n. 13 above).

\goto{36}[chapter-001.htmlux5cux23pla_0000141]. {\em such a thing}: I.e.
sex.

\goto{37}[chapter-001.htmlux5cux23pla_0000143]. {\em affection}: I.e.
{\em philia}, the term that most closely corresponds to English
‘friendship'; {\em philia} is the most a boy would normally be expected
to feel in response to the (normal) erotic attentions of an older man --
i.e. {\em philia} rather than {\em erôs}. (So the boy would be expected
to derive other benefits from the relationship, not sexual pleasure;
‘Lysias' will shortly give us a clearer notion of what these benefits
might be.)

\goto{38}[chapter-001.htmlux5cux23pla_0000145]. {\em whatever benefits
they receive}: I.e. from ‘doing what they did' (pleasure, presumably).

\goto{39}[chapter-001.htmlux5cux23pla_0000147]. {\em from the thing}:
I.e. from ‘granting favours' (the Greek has just {\em apautou}, ‘from
it').

\goto{40}[chapter-001.htmlux5cux23pla_0000149]. {\em miss}: ‘Miss' is
{\em pothein}, which is what the lover feels for the loved one when
separated from him; and it may also be worth noticing that the final
word of the speech, {\em erôta} (‘ask'), has the same letters as one
form (the accusative) of {\em erôs}, though it would have sounded
differently. An appropriately showy ending, perhaps, for an
{\em erôtikos logos}?

\goto{41}[chapter-001.htmlux5cux23pla_0000151]. {\em inspired}: In Greek
{\em theios}, ‘divine'; so ‘{\em divinely} inspired'.

\goto{42}[chapter-001.htmlux5cux23pla_0000153]. {\em its creator}: Its
‘maker', or ‘poet': {\em poiêtês}.

\goto{43}[chapter-001.htmlux5cux23pla_0000155]. {\em I missed it}: I.e.
the author's saying what he ought to say on his subject.

\goto{44}[chapter-001.htmlux5cux23pla_0000157]. {\em clearly I have
heard something}: I.e. because it can't have come from {\em him} -- he
knows nothing (see below).

\goto{45}[chapter-001.htmlux5cux23pla_0000159]. {\em Sappho\ldots{}
prose-writers}: It is not immediately clear what Socrates could have
learned from pre-eminent {\em love}-poets like Sappho and Anacreon about
the {\em non}-lover; or is it just about what hopeless cases (ordinary)
lovers are? Still more mysterious is the reference to prose-writers, for
the only one we know of who wrote in praise of the non-lover is Plato
himself ({\em his} version is just about to come). Or is this just the
point? Is Plato perhaps, for once, allowing himself a sly
self-reference?

\goto{46}[chapter-001.htmlux5cux23pla_0000161]. {\em Absolutely
excellent}!: What Phaedrus actually says is ‘Most noble (Socrates)'; I
take it that this is just another way of expressing his excitement at
the prospect of hearing a new speech.

\goto{47}[chapter-001.htmlux5cux23pla_0000163]. {\em nine archons}: The
{\em Constitution of the Athenians}, attributed to Aristotle, tells us
that the nine archons -- key officers of state in the Athenian system --
swore to ‘dedicate a golden statue, if they should contravene any of the
laws'. Thus Phaedrus is implying the following: ‘I'll dedicate statues
if I'm wrong, and you do better than Lysias.' However, the significance
of what he says (and of Socrates' response: ‘You are\ldots{} truly made
of gold') is immediately increased as soon as we know that the virtuoso
fifth--fourth-century practitioner and teacher of rhetoric after whom
Plato's {\em Gorgias} is named -- and whom Socrates briefly mentions
later in the {\em Phaedrus} (261b, 267a); he would have been about
seventy at the dramatic date of the dialogue, and lived to be well over
a hundred -- famously dedicated a golden statue of {\em himself} at
Delphi; and that a (much later) report has Plato scoffing at him on his
return, as ‘the beautiful and {\em golden} Gorgias'. For these and
further aspects of the whole context, see the brilliant account in
Morgan 1994.

\goto{48}[chapter-001.htmlux5cux23pla_0000165]. {\em dedication of the
Cypselids at Olympia}: Once again there is a connection with Gorgias,
who had a statue at Olympia as well (see preceding note), this time
dedicated by his grand-nephew Eumolpus; but {\em Socrates}' statue will
be ‘beside the dedication of the Cypselids', which probably refers to a
colossal statue of Zeus (whom Socrates will himself later associate with
philosophy). Here too I refer to Morgan 1994, which also provides, among
other things, a convincing explanation of the detail that Socrates'
statue will be ‘of hammered metal'. Morgan's article as a whole
illustrates just how rich Plato's text can be; we modern readers will no
doubt very often miss a great deal of what the text would have offered
to an alert contemporary, i.e. fourth-century, reader.

\goto{49}[chapter-001.htmlux5cux23pla_0000167]. {\em grasp\ldots{}
words}: The words are attributed to Pindar at Plato, {\em Meno} 76d.

\goto{50}[chapter-001.htmlux5cux23pla_0000169]. {\em craftsman}: The
word is {\em poiêtês} again (see n. 42 above).

\goto{51}[chapter-001.htmlux5cux23pla_0000171]. {\em musical race of the
Ligurians}: Apparently a deliberately silly etymology for {\em ligeiai}
(‘clear-voiced') as an epithet of the Muses: the only known connection
between those Greeks in the western Mediterranean called Ligurians and
music is actually what Plato -- it seems -- is inventing here. Socrates
is parodying the invocations of the Muses traditionally made by poets
(cf. the opening lines of both the {\em Iliad} and the {\em Odyssey}),
perhaps thereby indicating something about his attitude towards the
performance he is about to give.

\goto{52}[chapter-001.htmlux5cux23pla_0000173]. {\em take part with me}:
Another poeticism, if not a quotation.

\goto{53}[chapter-001.htmlux5cux23pla_0000175]. {\em his friend}: I.e.
Lysias, who was just now Phaedrus' ‘darling' (236b5).

\goto{54}[chapter-001.htmlux5cux23pla_0000177]. {\em what each thing
really is}: The starting-point for numerous Platonic dialogues is the
question of what something or other (often one or other of the
‘virtues') is.

\goto{55}[chapter-001.htmlux5cux23pla_0000179]. {\em desire}: I.e.
{\em epithumia}, a term often, but not exclusively, used in connection
with irrational wants; for Socrates, paradoxically, ideal {\em erôs}
will turn out to be {\em both} an expression of the highest rationality
{\em and} a kind of irrationality (in fact, madness).

\goto{56}[chapter-001.htmlux5cux23pla_0000181]. {\em the beautiful}:
‘The beautiful' is here plural in Greek, and either (or both) neuter and
masculine.

\goto{57}[chapter-001.htmlux5cux23pla_0000183]. {\em restraint}: I.e.
{\em sôphrosunê}, traditionally and unhelpfully translated as
‘temperance'; self-mastery or, more generally, being in one's right
mind.

\goto{58}[chapter-001.htmlux5cux23pla_0000185]. {\em excess}: I.e.
{\em hubris}.

\goto{59}[chapter-001.htmlux5cux23pla_0000187]. {\em this is called
love}: The leaden etymology -- {\em erôslrhômê} -- recalls the earlier
one (see n. 51 above), and surely matches the tone of the whole. (No
wonder Socrates was keen to claim that he got it from someone else.)

\goto{60}[chapter-001.htmlux5cux23pla_0000187a]. {\em something more
than human: Something divine}, in fact ({\em theios} again: cf. n. 41
above).

\goto{61}[chapter-001.htmlux5cux23pla_0000189]. {\em uttering in
dithyrambs}: Dithyrambic poetry, about which -- in its classical form --
we know relatively little, seems to be associated in Plato particularly
with mere artificial invention; he certainly seems to have had little
time for it.

\goto{62}[chapter-001.htmlux5cux23pla_0000191]. {\em Well, my brave
friend}: I borrow this translation from Nehamas and Woodruff 1995, who
point out the peculiar epic form of address Socrates uses here ({\em ô
pheriste}): ‘\ldots{} probably signals Socrates' parody of overblown
rhetoric'.

\goto{63}[chapter-001.htmlux5cux23pla_0000193]. {\em it is necessarily}:
Throughout this part of his speech, Socrates emphasizes what (he claims)
will follow {\em necessarily} from his definition of erôs. His approach
is systematic in a way in which -- as he has already begun to suggest --
Lysias' was not.

\goto{64}[chapter-001.htmlux5cux23pla_0000195]. {\em he will\ldots{}
become wisest}: Once again, in the midst of parody, we find a genuinely
Socratic element (cf. n. 54 above). For Socrates, the greatest good, if
not the only true good (the only thing {\em always} good), is wisdom;
depriving someone of wisdom will correspondingly be the way of doing
them the greatest harm.

\goto{65}[chapter-001.htmlux5cux23pla_0000197]. {\em what happened
before}: Sex.

\goto{66}[chapter-001.htmlux5cux23pla_0000199]. {\em having\ldots{}
sobered up}: ‘Having become {\em sôphrôn}': cf. n. 57 above.

\goto{67}[chapter-001.htmlux5cux23pla_0000201]. {\em as the sherd
flips\ldots{} side}: An ancient source makes this a reference to a game
of tag, in which the fall of a sherd (like the flip of a coin)
determined who would be ‘it'.

\goto{68}[chapter-001.htmlux5cux23pla_0000203]. {\em necessarily}: I.e.
given what {\em erôs} essentially is.

\goto{69}[chapter-001.htmlux5cux23pla_0000205]. {\em as\ldots{} boy}:
Socrates ends with a flourish, and with what is almost a hexameter
(epic) line, and possibly a reference to a proverb.

\goto{70}[chapter-001.htmlux5cux23pla_0000207]. {\em as I said it would
be}: I.e. as inexpert(?).

\goto{71}[chapter-001.htmlux5cux23pla_0000209]. {\em even though I'm
playing the critic}: Epic poetry -- or Homer -- typically sings the
{\em praise} of men, {\em kleos andrôn}.

\goto{72}[chapter-001.htmlux5cux23pla_0000211]. {\em whatever
fate\ldots{} without me}: Cf. e.g. 275e3 -- 5, on the fate of
compositions left without a father to defend them.

\goto{73}[chapter-001.htmlux5cux23pla_0000213]. {\em the time}: The
words ‘the time\ldots{} stands still' may be a gloss, and their meaning
is far from certain; the translation given may be the best that can be
made of them.

\goto{74}[chapter-001.htmlux5cux23pla_0000215]. {\em a superhuman
capacity: A divine (theios)} capacity.

\goto{75}[chapter-001.htmlux5cux23pla_0000217]. {\em Simmias the
Theban}: This Simmias is one of Socrates' two interlocutors in the
{\em Phaedo}; see especially {\em Phaedo} 85c--d.

\goto{76}[chapter-001.htmlux5cux23pla_0000219]. {\em that supernatural
experience, the sign}: For another, and very similar, description of
Socrates' famous {\em daimonion}, see Apology 31c d. Here in the
{\em Phaedrus}, however, there is a clear sense of a {\em literary} use
of the idea; Socrates hardly needs a divine sign to tell him to make
amends for the preceding speech, when he was already ashamed even at the
prospect of giving it (237a; cf. 243b); it isn't just that his ‘soul was
troubled' while he was making it, as he is about to say -- surely
disingenuously (242c6 -- 8).

\goto{77}[chapter-001.htmlux5cux23pla_0000221]. {\em against\ldots{}
gods}: I.e. against the divine ({\em to theion}).

\goto{78}[chapter-001.htmlux5cux23pla_0000223]. {\em that\ldots{} men}:
In fact (it seems) a Platonic adaptation from Ibycus.

\goto{79}[chapter-001.htmlux5cux23pla_0000225]. {\em potion: The Greek
is katapharmakeuthentos; cf. pharmakon} at 230d6.

\goto{80}[chapter-001.htmlux5cux23pla_0000227]. {\em Stesichorus}:
Another lyric poet (sixth century).

\goto{81}[chapter-001.htmlux5cux23pla_0000229]. {\em true follower of
the Muses}: Or ‘a musical expert' ({\em mousikos}); but see the story
about the cicadas at 259a -- d.

\goto{82}[chapter-001.htmlux5cux23pla_0000231]. {\em This\ldots{} Troy}:
This is evidently a fragment of a poem now lost. (As the plot of
Euripides' {\em Helen} has it, the real Helen was in Egypt throughout
the war, and it was a phantom Helen who went to Troy.)

\goto{83}[chapter-001.htmlux5cux23pla_0000233]. {\em Palinode}: Or
‘taking-it-back poem' (Nehamas and Woodruff 1995).

\goto{84}[chapter-001.htmlux5cux23pla_0000235]. {\em free men}: Not that
Athenian sailors, especially rowers in the fleet, would not for the most
part have been free men in the literal sense (citizens); but not all
free men -- so the claim is -- behave as free men should.

\goto{85}[chapter-001.htmlux5cux23pla_0000237]. {\em one\ldots{}
received}: Because after all, on the Socratic model of {\em erôs}, what
the lover brings to the relationship will be just as important as what
the beloved brings; it will be much more of an equal relationship than
that between an ordinary lover and his boy. (The Greek is {\em ek tôn
homoiôn}, which might be more literally rendered as ‘on equal terms'.)

\goto{86}[chapter-001.htmlux5cux23pla_0000239]. {\em Here\ldots{} wish}:
I.e. Phaedrus will stand in for him. (There is no suggestion in the
{\em Phaedrus} of anything beyond simple friendship between Socrates and
Phaedrus himself.)

\goto{87}[chapter-001.htmlux5cux23pla_0000241]. {\em Euphemus, of
Himera}: ‘Euphemus' means something like ‘speaking with good omen'
({\em euphêmein} is actually to say nothing, the best way of keeping out
of trouble); while ‘Himera' suggests {\em himeros}, another word for
‘desire' or ‘longing'.

\goto{88}[chapter-001.htmlux5cux23pla_0000243]. {\em mantic}: That is,
the art of the {\em mantis}, or seer, with which Socrates suggested he
had a passing acquaintance back at 242c--d. Another fanciful etymology,
shortly to be followed by another; yet at {\em some} level Socrates must
presumably be serious about the basic claim he is now making, that under
the right conditions madness is beneficial rather than harmful.

\goto{89}[chapter-001.htmlux5cux23pla_0000245]. {\em touched by it}:
I.e. by madness.

\goto{90}[chapter-001.htmlux5cux23pla_0000247]. {\em A third\ldots{}
Muses}: This, obviously enough, is poetic inspiration (see Plato's
{\em Ion}, which suggests at best a rather mixed view about its real
usefulness); it is rather less clear what exactly is meant to fall under
the previous (second) kind of ‘madness'.

\goto{91}[chapter-001.htmlux5cux23pla_0000249]. {\em a first principle}:
A reading preferred by many editors, and evidently known to Cicero more
than two thousand years ago, would give the easier sense ‘\ldots{} it
would not {\em be a first principle}'; but our manuscripts give us the
text I have translated, and I have preferred to stay with that more
difficult, and more interesting, text. (Here is one consideration: the
verb {\em gignesthai}, which is used here, {\em can} mean ‘be'; but
since it has just been used twice, and will fairly soon be used again,
in the sense of ‘{\em come-to-be}', it is not easy to suppose that Plato
would have used it to mean ‘be' -- instead of {\em einai}, the standard
verb for ‘be' -- just {\em here}.)

\goto{92}[chapter-001.htmlux5cux23pla_0000251]. {\em superhuman}: I.e.
divine (only a god could do it).

\goto{93}[chapter-001.htmlux5cux23pla_0000253]. {\em combined}:
Literally ‘grown together' ({\em sumphutos}); the point is that it is in
the nature of the things combined to be so. (Nehamas and Woodruff 1995
have ‘natural union' for the complete phrase {\em sumphutos dunamis};
this leaves out the {\em dunamis}, ‘force' or ‘power', but may still be
the better translation.)

\goto{94}[chapter-001.htmlux5cux23pla_0000255]. {\em driver}: Or our
‘ruler', {\em archôn}, a term that recalls the language Socrates used in
describing the relationship between reason and desire in his first
speech (‘in each of us there are two kinds of thing that rule
({\em archein}) and lead us', and so on).

\goto{95}[chapter-001.htmlux5cux23pla_0000257]. {\em god}: I.e.
whichever god it may be appropriate to refer to; ‘a very frequent
formula for expressing pious reservation' (de Vries 1969).

\goto{96}[chapter-001.htmlux5cux23pla_0000259]. {\em in a way}: An
important qualification, reminding us that the context is one of a
{\em simile}. (Human beings, on Socrates' reckoning, are closer to the
divine than birds.)

\goto{97}[chapter-001.htmlux5cux23pla_0000261]. {\em beautiful\ldots{}
kind}: These are not, it should be said, typical attri-butes of the
divine, especially as represented by the poets. Socrates' treatment of
the gods is radical in other respects too: he has already denied them
bodies, and will shortly associate them -- without identifying them --
with astronomical entities (planets and/or constellations).

\goto{98}[chapter-001.htmlux5cux23pla_0000263]. {\em Hestia alone
remains}: As goddess of the hearth, where else would she be but at home?
(This is, I suggest, one of a number of places where the general tone of
apparent seriousness in the speech is relaxed; it is for the reader to
judge what the effect of these moments may be on the whole.)

\goto{99}[chapter-001.htmlux5cux23pla_0000265]. {\em the final labour}:
The term {\em labour} here recalls the Labours of Heracles: the issue,
for each soul, will be whether it manages to achieve even the slightest
share in the divine feast -- which Socrates is about to describe.

\goto{100}[chapter-001.htmlux5cux23pla_0000267]. {\em that are called
immortal}: I.e. the gods; all souls, as we have been told, are immortal,
but we call the gods ‘the immortals', and they are pure souls.

\goto{101}[chapter-001.htmlux5cux23pla_0000269]. {\em This\ldots{}
relates}: This is the language Platonic dialogues typically reserve for
those entities that go under the names of (Platonic) ‘Forms' or ‘Ideas'
({\em eidê, ideai}); so when, a few lines below, a divine soul is said
to ‘see justice itself\ldots{} self-control\ldots{} knowledge', it is
presumably ‘the Form of Justice', ‘the Form of Self-Control' and so on.
These, roughly speaking, represent the essence, or what-it-is-to-be, of
justice, self-control, etc., conceived of as existing separately from
their instantiations -- i.e. particular instances of them. Platonic
Forms are not normally represented as existing in space, and indeed in
the Platonic universe it is not clear that there is any space outside
the universe (which is ‘the all', {\em to pan}, everything there is, so
that it would be surprising to find that there was anything {\em beyond}
it). But neither, of course, is the soul literally composed of a
charioteer and two horses; nor can there be any question of the Forms
being ‘seen' in anything but a metaphorical sense. The general (and more
prosaic) point is that the mind or soul has to grasp Forms -- however
these are to be understood -- in order to reach full, or real, knowledge
of anything. Socrates will towards the end of the dialogue describe the
means by which mere human souls may perhaps achieve that end:
‘dialectic', or the ‘science of (philosophical) conversation', which
understands how to ‘collect' and ‘divide' in order to reach the essence
of things (265c ff.).

\goto{102}[chapter-001.htmlux5cux23pla_0000271]. {\em self-control}:
I.e. {\em sôphrosunê}, previously rendered as ‘restraint' (see n. 5
above).

\goto{103}[chapter-001.htmlux5cux23pla_0000273]. {\em things that we now
say are}: I.e. particular instantiations of Forms, like the things
around us that we can see, hear, etc. -- and that have the habit of
{\em changing} (‘to which coming-into-being attaches'), and in which
properties will show up in what appear to be widely different ways
(‘that which seems to be different\ldots{}': what is it, exactly, that
is in common between the beauty of this boy, for example, and the beauty
of that piece of music?).

\goto{104}[chapter-001.htmlux5cux23pla_0000275]. {\em nectar\ldots{}
ambrosia}: Once again the tone suddenly lightens; high metaphysical
seriousness is followed by {\em play}.

\goto{105}[chapter-001.htmlux5cux23pla_0000277]. {\em what only appears
to nourish them}: Or ‘feed on the nourishment of (mere) opinion
({\em trophê doxastê})': in Plato, {\em doxa}, ‘opinion', is typically
contrasted with knowledge, not least because it can be false as well as
true. ‘Opinion' is the state of mind of non-philosophers, who fail to
look further than (or rise above) ordinary appearances.

\goto{106}[chapter-001.htmlux5cux23pla_0000279]. {\em best part of the
soul}: I.e., clearly, its rational part (‘part', however, is imported by
the translation; the Greek has just the definite article + superlative
adjective, both in the neuter: ‘the best (?)' of the soul).

\goto{107}[chapter-001.htmlux5cux23pla_0000281]. {\em Destiny}: I.e. the
way things are, which is fixed and immovable for ever. For a similar
idea (and a similar account of {\em what} is fixed, i.e. an ordering of
types of life in terms of worth: see below), see the concluding myth
(the story of Er) in the {\em Republic}; later in the {\em Phaedrus},
the present discourse will itself be described as a kind of
{\em muthos}, or story (265b -- c).

\goto{108}[chapter-001.htmlux5cux23pla_0000283]. {\em this soul\ldots{}
birth}: The idea, originally Pythagorean, that our souls will enter new
bodies at some time after our death -- human or animal, depending on our
behaviour in this life -- is one that appears in several Platonic
dialogues, not always in a mythical context.

\goto{109}[chapter-001.htmlux5cux23pla_0000285]. {\em devoted\ldots{}
love}: These are not real alternatives: the true lover will turn out to
be a philosopher (a lover of wisdom), drawn to true Beauty, and beloved
of the Muses.

\goto{110}[chapter-001.htmlux5cux23pla_0000287]. {\em poet}: The
(ordinary) poet, then, even if inspired (the seer, who is certainly
inspired, comes only just above), comes strikingly low in the list --
even after the gymnastic trainer. But then hardly any of them will even
have heard of the ‘region above the heavens' (cf. 247c3 -- 4); so there
will be few {\em philosophical} poets. (Parmenides might be the sole
exception.)

\goto{111}[chapter-001.htmlux5cux23pla_0000289]. {\em For\ldots{}
years}: I.e. one's chance of real bliss -- which means staying
(temporarily) in the company of the gods -- comes only once every ten
thousand years.

\goto{112}[chapter-001.htmlux5cux23pla_0000291]. {\em live a
life\ldots{} human form}: Probably a reference to the fate e.g. of
Heracles, who -- according to one version of the story -- actually did
join the company of the stars after death; Socrates' story manages to
pretend to being traditional at the same time as being utterly radical.

\goto{113}[chapter-001.htmlux5cux23pla_0000293]. {\em allotment and
choice}: The same apparently puzzling combination occurs at
{\em Republic} x, 617d ff., where it seems that souls cast lots for the
{\em order} in which they choose among a limited number of lives of each
sort (how else would the world order be maintained?).

\goto{114}[chapter-001.htmlux5cux23pla_0000295]. {\em recollection}: For
the Platonic theory of learning as ‘recollection' (i.e. of things our
souls ‘saw' before birth), see {\em Meno} 80d ff. and {\em Phaedo} 72e
ff.

\goto{115}[chapter-001.htmlux5cux23pla_0000297]. {\em things we now say
are}: I.e., again, the familiar things around us, as opposed to those
things that ‘really are' (the Forms).

\goto{116}[chapter-001.htmlux5cux23pla_0000299]. {\em reminders}: I.e.
(presumably) his sensations, which give him what he needs to ‘collect
together into one through reasoning' in order to ‘comprehend what is
said universally' -- the whole being properly described as
‘recollection' (b4 -- c4, just above).

\goto{117}[chapter-001.htmlux5cux23pla_0000301]. {\em perfection}: There
is a verbal play here, on {\em teleos}, ‘perfect', and {\em teletai},
‘rites' (cf. {\em neotelês}, ‘newly initiated', at 250e1); the point is
not so much that philosophers become literally {\em perfect} -- because
after all their souls remain human souls -- but rather that they are
initiated into the highest rites.

\goto{118}[chapter-001.htmlux5cux23pla_0000303]. {\em regarded as mad}:
But of course he is also mad; people call him mad for the wrong reasons
(that he is truly possessed goes ‘unrecognized by the many', d3 above).

\goto{119}[chapter-001.htmlux5cux23pla_0000305]. {\em beautiful}: ‘The
beautiful' is a genitive plural in the Greek that could be either
masculine or neuter; the masculine would perhaps be more to the point
here -- except that what the (true) lover really loves, according to
Socrates, is actually Beauty, not any particular beautiful person.

\goto{120}[chapter-001.htmlux5cux23pla_0000307]. {\em ourselves}: The
‘we' here is perhaps most naturally taken as refer-ring to Socrates
himself and the boy he is addressing: according to what he will say at
252c and e, it is philosophers who follow in Zeus' train, and
philosophers will look for boys who are ‘naturally disposed towards
philosophy' (252e3).

\goto{121}[chapter-001.htmlux5cux23pla_0000309]. {\em longing for what
was before}: So the preceding description was itself a case of
recollection; Socrates was himself truly inspired, or maddened, in his
description of the ‘perfect rites'.

\goto{122}[chapter-001.htmlux5cux23pla_0000311]. {\em objects of love}:
I.e., presumably, ‘justice and self-control and the other things which
are of value to souls' (b1 -- 2 above).

\goto{123}[chapter-001.htmlux5cux23pla_0000313]. {\em excess}: This is
{\em hubris} again (238a2, etc.).

\goto{124}[chapter-001.htmlux5cux23pla_0000315]. {\em himeros}: Yet
another fanciful etymology (and no more possible to render in English).

\goto{125}[chapter-001.htmlux5cux23pla_0000317]. {\em desire}: I.e. the
{\em himeros}.

\goto{126}[chapter-001.htmlux5cux23pla_0000319]. {\em boy with his
beauty}: Or is it that {\em other} beautiful thing, Beauty Itself, that
he remembers? The Greek leaves it open.

\goto{127}[chapter-001.htmlux5cux23pla_0000319a]. {\em desire}: Again,
{\em himeros}.

\goto{128}[chapter-001.htmlux5cux23pla_0000321]. {\em I think\ldots{}
down}: I have taken it that the lines are outrageous (it should be
‘excessive': the word is {\em hubristikon}), {\em because} they are --
or rather the second is -- unmetrical (it breaks with the restraints of
metre, as it were). But see Introduction.

\goto{129}[chapter-001.htmlux5cux23pla_0000323]. {\em Ares}: The
traditional god of war (cf. n. 112 above).

\goto{130}[chapter-001.htmlux5cux23pla_0000325]. {\em this practice}:
‘This practice' must be that of leading, helping a boy towards
philosophy: a reference to Socrates' own position, perhaps, in relation
to the boy he is addressing -- for after all, Socrates is no expert in
anything, and any skills the present speech may suggest come from a
source outside him (see especially 257a7 -- 8, where he asks Love not to
take away ‘the expertise in love you gave me', i.e. for the duration of
the speech?).

\goto{131}[chapter-001.htmlux5cux23pla_0000327]. {\em outcome}: Or its
‘ending', {\em teleutê}: presumably sharing in the same love of wisdom.
Socrates carefully avoids presuming that his speech will persuade the
boy (but we should also note the contrast with the kind of ‘ending' that
Lysias had in mind in his speech, i.e. sex: see 264b1 -- 2, where
Phaedrus makes a joke of it).

\goto{132}[chapter-001.htmlux5cux23pla_0000329]. {\em Into three forms}:
The Greek here has just ‘into three' ({\em trichêi}), but the addition
of ‘forms' is justified by what follows, which is, more literally:
‘(\ldots{} into three:) two horse-shaped (?) kinds of forms/sorts of
things ({\em hippomorphô men duo tine eidê}), and a third, charioteer's,
form ({\em hêniochon de eidos triton})'.

\goto{133}[chapter-001.htmlux5cux23pla_0000331]. {\em on the nobler
station}: I.e. on the right.

\goto{134}[chapter-001.htmlux5cux23pla_0000333]. {\em spoken orders}: Or
‘orders and reason' ({\em keleusmati\ldots{} kai logôi}).

\goto{135}[chapter-001.htmlux5cux23pla_0000335]. {\em excess: Hubris}.

\goto{136}[chapter-001.htmlux5cux23pla_0000337]. {\em excess:
Hubristês}.

\goto{137}[chapter-001.htmlux5cux23pla_0000339]. {\em a
starting-barrier}: I.e. on a racecourse. Falling back from the start
might look an unlikely thing for a chariot-horse to do but perhaps not
if he is having to be reined in like {\em this} horse?

\goto{138}[chapter-001.htmlux5cux23pla_0000341]. {\em gives it over to
pains}: As Nehamas and Woodruff 1995 point out, this is a Homeric
phrasing. Now we are, finally, in a kind of epic context (cf. 241e) --
though an odd one, with the soul as battlefield.

\goto{139}[chapter-001.htmlux5cux23pla_0000343]. {\em desire: Himeros}
-- so apparently it was Zeus who coined the word (see 251c).

\goto{140}[chapter-001.htmlux5cux23pla_0000345]. {\em into its
possessor}: Literally ‘into the beautiful (one)'.

\goto{141}[chapter-001.htmlux5cux23pla_0000347]. {\em not as love but as
friendship}: I.e. he interprets his extraordinary experience in the
conventional terms, of a boy who feels friendship for a lover in return
for benefits received (when the immediate cause of everything is
actually his own beauty, acting as a reminder of that other Beauty).

\goto{142}[chapter-001.htmlux5cux23pla_0000349]. {\em three
submissions}: The metaphor is from wrestling.

\goto{143}[chapter-001.htmlux5cux23pla_0000351]. {\em forced\ldots{}
Phaedrus}: I.e. he had to play the poet; he's no good as a poet; but it
was all Phaedrus' fault anyway? (But later on in the dialogue, Socrates
will have much to say about the need to adapt one's speeches to the
nature of one's audience; is that, perhaps, the deeper point behind what
he says here? See Introduction.)

\goto{144}[chapter-001.htmlux5cux23pla_0000353]. {\em the
expertise\ldots{} that you gave me}: Sc. ‘and so enabled me to say what
I said'?

\goto{145}[chapter-001.htmlux5cux23pla_0000355]. {\em Polemarchus}:
Polemarchus plays a small role at the beginning of the conversation in
the {\em Republic}.

\goto{146}[chapter-001.htmlux5cux23pla_0000355a]. {\em talk}: ‘Talk' is
logoi in the Greek; clearly, in light of what is to follow (which will
be an example of, and will end by discussing, philosophical
{\em exchange}), not in this case just ‘speeches'.

\goto{147}[chapter-001.htmlux5cux23pla_0000357]. {\em you\ldots{} bend}:
What the Greek text gives us is ‘you don't know {[}that{]} the
expression “pleasant bend” {[}comes from the long bend of the Nile{]}';
the bracketed parts are usually treated as an addition by a copyist
trying to explain ‘pleasant bend' (or ‘sweet elbow'); whatever the
origin of the expression, it evidently referred to forms of words that
manage to say something other than what the speaker intends -- the
elbow/bend is precisely not sweet/pleasant. In correspondence, Alan
Griffiths has suggested an alternative, and much more attractive,
scenario for the origin of the phrase: that {\em glukus a(n) gkôn} was a
rhapsode's garbling of a Homeric phrase -- {\em kateibeto glukus aiôn}:
‘his sweet {\em life} was dripping away' (for ‘rhapsodes', see n. 220
below). The description is of Odysseus pining away on Kalypso's beach at
{\em Odyssey} 5.152. The rhapsode (Griffiths suggests), getting to this
line, mixed up its ending with that of the next but one, {\em kai ana(n)
gkêi}, either because he misremembered the line or -- more likely --
because he looked down for a prompt and got the wrong place ({\em a(n)
gkôn} being a garbled combination of {\em aiôn} and {\em ana(n) gkei}).
In any case he ended up saying ‘his (the great hero Odysseus') sweet
{\em elbow} was dripping away.' The audience, Griffiths surmises,
dissolved into laughter, and ‘~“sweet elbow” became irresistible
shorthand for non-correspondence between intention and
utterance\ldots{}' If this is not the right explanation (and we shall
never know), it is certainly beautiful (sweet) enough.

\goto{148}[chapter-001.htmlux5cux23pla_0000359]. {\em author}: Or ‘poet'
-- the word is {\em poiêtês}; hence the following metaphor of the
{\em theatre}.

\goto{149}[chapter-001.htmlux5cux23pla_0000361]. {\em a Lycurgus, a
Solon or a Darius}: Respectively, lawgivers of Athens and Sparta, and
Great King of Persia.

\goto{150}[chapter-001.htmlux5cux23pla_0000363]. {\em What would\ldots{}
justly so}: A somewhat odd set of things for Phaedrus to say, and just
here; either something has been added to Plato's text -- or, possibly,
Phaedrus is doing some philosophical posing (but still, why just here?).

\goto{151}[chapter-001.htmlux5cux23pla_0000365]. {\em converse with}:
The verb is {\em dialegesthai}, which is what (Platonic)
{\em philosophers} typically do.

\goto{152}[chapter-001.htmlux5cux23pla_0000367]. {\em speech}: Or
‘talk'; but shortly, in e1, {\em logos} must again be translated as ‘(a)
speech'. (The difficulty for the translator throughout is that
{\em logoi} can be {\em either} speeches {\em or} philosophical talk,
which will usually not take the form of, and may actually be opposed to,
the making of speeches.)

\goto{153}[chapter-001.htmlux5cux23pla_0000369]. {\em a\ldots{} aside}:
Homer, {\em Iliad} 2.361.

\goto{154}[chapter-002.htmlux5cux23pla_0000371]. {\em arguments: Logoi}
yet again.

\goto{155}[chapter-002.htmlux5cux23pla_0000373]. {\em the Laconian}:
I.e. the Spartan; is Socrates saying just ‘to put it bluntly'? (The
Spartans, supposedly, were characterized by their {\em laconic} speech.)

\goto{156}[chapter-002.htmlux5cux23pla_0000375]. {\em beautiful
offspring}: Phaedrus is bringing yet more {\em logoi} into existence
(cf. 242a -- b), though now of a different kind (arguments).

\goto{157}[chapter-002.htmlux5cux23pla_0000377]. {\em speech: Logoi}
(which will not exclude written ‘speech').

\goto{158}[chapter-002.htmlux5cux23pla_0000379]. {\em Palamedes}: An
epic hero proverbial for his cleverness.

\goto{159}[chapter-002.htmlux5cux23pla_0000381]. {\em Gorgias}\ldots{}
{\em Odysseus}: For Gorgias, see n. 47 above; Thrasymachus of Chalcedon
and Theodorus of Byzantium were well-known rhetorical theorists
(Thrasymachus is also the aggressive opponent of Socrates in
{\em Republic} 1 who provides the conception of justice that the rest of
the {\em Republic} is in large part designed to refute). Phaedrus'
identifications may possibly indicate his different valuations of the
three figures: Nestor in Homer is an upright speaker, while Odysseus, at
least later, became synonymous with sly cunning.

\goto{160}[chapter-002.htmlux5cux23pla_0000383]. {\em he will\ldots{}
the opposite}: Cf. the formulae at 258a4 -- 5, which can be rendered ‘It
seemed (good) to the council/people\ldots{}'

\goto{161}[chapter-002.htmlux5cux23pla_0000385]. {\em so as\ldots{} in
motion}: See Plato, {\em Parmenides} 127d -- 128a, where the ‘Eleatic
Palamedes' (Parmenides' clever pupil/follower, Zeno) puts forward just
these sorts of theses.

\goto{162}[chapter-002.htmlux5cux23pla_0000387]. {\em things that are}:
‘The things that are' could just mean ‘things'; but readers are likely
to remember, in such a context, expressions like ‘being that really is'
from Socrates' second speech (247c7) -- and they should presumably not
be discouraged from doing so.

\goto{163}[chapter-002.htmlux5cux23pla_0000389]. {\em ones you and I
made}: So Socrates is still holding Phaedrus jointly responsible for the
two speeches he gave.

\goto{164}[chapter-002.htmlux5cux23pla_0000391]. {\em pair of speeches}:
I.e. (what we would call) Socrates' pair of speeches.

\goto{165}[chapter-002.htmlux5cux23pla_0000393]. {\em someone\ldots{}
them}: The point, as it will emerge, is that the first of Socrates'
speeches pretended to give an account of the {\em whole} of {\em erôs}
when in fact only describing an {\em aspect} of it (i.e. its
‘left-handed' aspect).

\goto{166}[chapter-002.htmlux5cux23pla_0000395]. {\em names of just, or
good}: I.e. the names that belong to what is (actually) just and good.

\goto{167}[chapter-002.htmlux5cux23pla_0000397]. {\em most people}: Or
‘the masses' ({\em to plêthos}).

\goto{168}[chapter-002.htmlux5cux23pla_0000397a]. {\em dear thing}: The
Greek is {\em philê kephalê} (literally ‘dear head'), a poetic
expression, here presumably used ironically (‘dear thing' is perhaps the
closest English can get to the same effect, insofar as {\em kephalê} is
essentially a place-filler

\goto{169}[chapter-002.htmlux5cux23pla_0000399]. {\em ending}: Cf. n.
131 above.

\goto{170}[chapter-002.htmlux5cux23pla_0000401]. {\em Midas the
Phrygian}: The legendary King Midas (eighth century BC).

\goto{171}[chapter-002.htmlux5cux23pla_0000403]. {\em the speech}: ‘The
speech' is either Socrates' second speech by itself (‘passing over' to
praise after the censure of the first), or both speeches together, now
treated as one -- which they will immediately be in any case (see 265d7;
though they are two again in 265e -- 266b).

\goto{172}[chapter-002.htmlux5cux23pla_0000405]. {\em two kinds of
thing}: I.e. {\em eidê}: as it turns out, the two sides of a particular
kind of {\em method} (‘collection and division').

\goto{173}[chapter-002.htmlux5cux23pla_0000407]. {\em perceiving
together\ldots{} many places}: ‘Collection' -- probably to be
distinguished from, but analogous in structure to, the more basic
process of building up universal concepts that Socrates described at
249b -- c.

\goto{174}[chapter-002.htmlux5cux23pla_0000409]. {\em instruct}:
‘Instruct', or ‘teach' ({\em didaskein}), will now be the appropriate
term, if -- as Socrates has argued -- the speaker is always to be
concerned with the {\em truth}.

\goto{175}[chapter-002.htmlux5cux23pla_0000411]. {\em speech}: See n.
171 above.

\goto{176}[chapter-002.htmlux5cux23pla_0000413]. {\em kind by kind}:
I.e. ‘kind of thing by kind of thing': {\em eidos} again. These
{\em eidê} (kinds, forms) will also be Platonic Forms (see n. 101
above). But although Phaedrus had an introduction of sorts to these
entities in Socrates' speech (those things like Beauty mysteriously
located in the ‘region above the heavens'), he is no metaphysician, and
the whole context is framed so far as possible in plain speech.

\goto{177}[chapter-002.htmlux5cux23pla_0000415]. {\em to look to one and
to many}: I.e. to collect and divide in the way specified.

\goto{178}[chapter-002.htmlux5cux23pla_0000417]. {\em in his\ldots{}
god}: Adapted from Homer ({\em Odyssey} 2.402).

\goto{179}[chapter-002.htmlux5cux23pla_0000419]. {\em And\ldots{}
knows}: With these words Socrates comes as close as he can to treating
the person in question (the ‘dialectician') as if he were a god: ‘if I
have got your name right' is a typical apologetic formula used in
addressing a god -- but then ‘god (only) knows' makes it clear that he
{\em isn't} in fact divine.

\goto{180}[chapter-002.htmlux5cux23pla_0000421]. {\em experts in
dialectic}: I.e. {\em dialektikoi}, experts in the science of
{\em dialegesthai}, (philosophical) conversation, where the special
nature of the conversation in question is marked by its use of
‘collection and division'.

\goto{181}[chapter-002.htmlux5cux23pla_0000423]. {\em Royal these people
are}: For Phaedrus, at least, they perhaps are real kings (of their art)
?

\goto{182}[chapter-002.htmlux5cux23pla_0000425]. {\em the
rhetorical\ldots{} us}: So Phaedrus resists Socrates' invitation to
assimilate rhetoric, the ‘science of speaking', as a whole to dialectic;
and Socrates himself will go on to allow that there is rather more to
speaking than just grasping, and telling, the truth.

\goto{183}[chapter-002.htmlux5cux23pla_0000427]. {\em the worthy
Theodorus}: Cf. 261c. Socrates goes on to give something of a roll-call
of rhetorical experts (most of whom appear as characters elsewhere in
Plato); his tone unmistakably suggests the same light irony as his
treatment here of Theodorus.

\goto{184}[chapter-002.htmlux5cux23pla_0000429]. {\em force}: The term
({\em rhômê}) is the same as the one Socrates used for the definition of
{\em erôs} at 238b -- c.

\goto{185}[chapter-002.htmlux5cux23pla_0000431]. {\em the Chalcedonian}:
I.e. Thrasymachus (see 261c, and n. 159 above); Socrates uses a Homeric
turn of phrase to refer to someone who was -- to judge by his
performance in the first book of the {\em Republic} -- a powerful
presence.

\goto{186}[chapter-002.htmlux5cux23pla_0000433]. {\em Acumenus}: See
227a.

\goto{187}[chapter-002.htmlux5cux23pla_0000435]. {\em Pericles}: The
outstanding Athenian statesman and orator (immortalized by the great
funeral speech given to him in Thucydides' {\em History of the
Peloponnesian War}) is compared, via a reminiscence of Tyrtaeus, the
Spartan poet, to Adrastus, a king of Argos who led the Seven against
Thebes.

\goto{188}[chapter-002.htmlux5cux23pla_0000437]. {\em who\ldots{}
converse}: I.e., in light of what has preceded, ‘who are ignorant of
dialectic'; but it must be open to question whether Pericles, let alone
Adrastus, would have matched up to Socrates' specifications for
expertise in dialectic.

\goto{189}[chapter-002.htmlux5cux23pla_0000439]. {\em babbling and lofty
talk}: These are the sorts of things philosophers tend to be described
as indulging in by non-philosophers -- so, especially, in Aristophanes'
{\em Clouds}, in which Socrates himself is a main character; Socrates in
Plato's {\em Apology} refers to Aristophanes' treatment of him (as
meddling with ‘things under the earth and in the heavens', 19b); and
then in the parable of the ship in the {\em Republic}, the true
steersman, standing in for the philosopher, is said by the ordinary
sailors (the people) to be ‘a star-gazer' ({\em meteôroskopos}), a
babbler and a good-for-nothing' (Republic 488e -- 489e; cf.
{\em Statesman} 299b). But here in the {\em Phaedrus}, ‘lofty talk',
meteôrologia (literally, pretty much what the {\em meteôroskopos} does,
i.e. look at {\em ta meteôra}, ‘things on high/ in the heavens'), will
surely have special resonance -- recalling the heavenly experiences and
aspirations of the disembodied soul.

\goto{190}[chapter-002.htmlux5cux23pla_0000441]. {\em Anaxagoras}:
Socrates seems to be punning: the natural philosopher Anaxagoras of
Clazomenae, apparently an intimate of Pericles', gave {\em nous}, ‘Mind'
or ‘Intelligence', a leading role in his account of the
coming-into-being of the universe (and in the process talked about what
things were like in the beginning, before Mind intervened); what
Anaxagoras should have taught Pericles and what we have no evidence
Plato thought he did teach him -- is the difference between intelligent
{\em thinking} and the lack of it.

\goto{191}[chapter-002.htmlux5cux23pla_0000443]. {\em by knack and
experience}: Cf. 260e5.

\goto{192}[chapter-002.htmlux5cux23pla_0000445]. {\em by
applying\ldots{} words}: ‘Medicines' translates {\em pharmaka},
‘excellence' is {\em aretê} (traditionally translated as ‘virtue'),
‘words' are logoi (of course). The redefinition of rhetoric continues
apace: no Lysias, no Gorgias could have compared it to medicine, a
science of {\em improving souls} (minds).

\goto{193}[chapter-002.htmlux5cux23pla_0000447]. {\em the whole}: The
whole universe, or the whole soul? Socrates leaves it open: in the
context of what follows, it ought to be the latter, but in light of what
has {\em just} been said (about the need for ‘lofty talk'), it could
still be the former.

\goto{194}[chapter-002.htmlux5cux23pla_0000449]. {\em Hippocrates the
Asclepiad}: Asclepius was the mythical founder of medicine (of the
physicians' guild, as it were), Hippocrates the medical writer {\em par
excellence}: medical opinion could come no higher.

\goto{195}[chapter-002.htmlux5cux23pla_0000451]. {\em argument: So
argument (logos} yet again) is what matters, not (mere) authority.

\goto{196}[chapter-002.htmlux5cux23pla_0000453]. {\em forms}: Another
non-technical use of {\em eidos} (see n. 176 above); complex', just
above, is, more literally, ‘many-formed', {\em polueides}.

\goto{197}[chapter-002.htmlux5cux23pla_0000455]. {\em kinds}: The word
is {\em genos}, as it happens, but it might just as well have been
{\em eidos}.

\goto{198}[chapter-002.htmlux5cux23pla_0000457]. {\em model speech}:
Lysias' speech at the beginning of the {\em Phaedrus} is one such, given
for display.

\goto{199}[chapter-002.htmlux5cux23pla_0000459]. {\em not on this one}:
I.e. rhetoric.

\goto{200}[chapter-002.htmlux5cux23pla_0000461]. {\em how one should
write}: I.e. (as the immediate sequel shows) in a rhetorical handbook.

\goto{201}[chapter-002.htmlux5cux23pla_0000463]. {\em to borrow his name
from}: Socrates is probably making fun here of Tisias' teacher Corax,
whose name means ‘Crow'.

\goto{202}[chapter-002.htmlux5cux23pla_0000465]. {\em things}: Or (if we
were still in the context of Socrates' second speech) ‘the things that
are': {\em ta onta}.

\goto{203}[chapter-002.htmlux5cux23pla_0000467]. {\em gratifying}: The
verb is the same as in the original ‘favours should be granted to a man
who is not in love\ldots{}' at 227c7--8.

\goto{204}[chapter-002.htmlux5cux23pla_0000469]. {\em of noble stock}:
Compare 246a7 -- 8: ‘Now in the case of gods, horses and charioteers are
all both good themselves and of good stock\ldots{}'; surely a deliberate
reminiscence. Similarly, ‘the way round' ({\em periodos}), in the next
sentence, must surely be meant to recall that other {\em periodos}, the
soul's circuit of the heavens (247d5).

\goto{205}[chapter-002.htmlux5cux23pla_0000471]. {\em Yet\ldots{}
others}: What this seems to mean is that ordinary, every-day goals can
best be achieved by acquiring knowledge first; but where has this been
argued for? Should we perhaps see here a covert reference to the goals
of the lover (and of Lysias' supposed non-lover), combined with the
description in Socrates' second speech of the pair who fall just short
of the ideal -- because they give in to their black horses, and have
sex?

\goto{206}[chapter-002.htmlux5cux23pla_0000473]. {\em the beautiful}:
Neuter or masculine (plural) ? Again, the Greek leaves it open.

\goto{207}[chapter-002.htmlux5cux23pla_0000475]. {\em writing}: It may
look as if we are turning from speaking to writing; in fact, as soon
becomes clear, we are turning from the subject of what is scientific or
not in {\em logoi} (spoken or written) to what is appropriate or not in
{\em logoi} (spoken or written).

\goto{208}[chapter-002.htmlux5cux23pla_0000477]. {\em conjectures}: I.e.
{\em doxasmata}; the word is closely related to {\em doxa}, ‘opinion',
on which see n. 105 above.

\goto{209}[chapter-002.htmlux5cux23pla_0000479]. {\em elixir:
Pharmakon.}

\goto{210}[chapter-002.htmlux5cux23pla_0000481]. {\em they are\ldots{}
themselves}: The verb for ‘reminding' here, {\em anami-mnêiskesthai},
corresponds to the noun {\em anamnêsis}, translated as ‘recollection' at
249c2.

\goto{211}[chapter-002.htmlux5cux23pla_0000483]. {\em oak and rock}:
I.e. to {\em any} source (‘oak or rock' is proverbial; see e.g. Homer,
{\em Odyssey} 19.162 -- 3 -- which may actually be in Plato's mind
here).

\goto{212}[chapter-002.htmlux5cux23pla_0000485]. {\em another kind of
speech}: I.e. another kind of speaking ({\em logos}).

\goto{213}[chapter-002.htmlux5cux23pla_0000487]. {\em animate}: I.e.
‘ensouled' ({\em empsuchon}).

\goto{214}[chapter-002.htmlux5cux23pla_0000489]. {\em garden of Adonis}:
Plants would be forced in pots during the festival of Adonis.

\goto{215}[chapter-002.htmlux5cux23pla_0000491]. {\em pieces of
knowledge}: I.e., as I have taken it, bits of knowledge about the
important subjects in question (the Greek has just the plural of
{\em epistême}, ‘knowledge'). However, Terry Penner has almost succeeded
in persuading me that Socrates means, or might mean, just plural
‘knowledges': knowledge of the just, knowledge of the beautiful (or
fine), knowledge of the good. In that case the reference here would be
to the ideal, expert knower, corresponding to the expert farmer -- and
specifically not to the mere dialectician, who at least by implication
will be denied the title of ‘wise' or knowing (279d).

\goto{216}[chapter-002.htmlux5cux23pla_0000493]. {\em beautiful}: Or
‘fine' (noble): the Greek word is {\em kalos}, which covers both -- and
for Socrates/Plato, the fine is the beautiful.

\goto{217}[chapter-002.htmlux5cux23pla_0000495]. {\em reaches\ldots{}
age}: Probably a quotation, from an unknown poetic source.

\goto{218}[chapter-002.htmlux5cux23pla_0000497]. {\em if someone\ldots{}
speak of}: Phaedrus here picks up on Socrates' idea of writing for
amusement, stories being the obvious example. But there is perhaps also
a covert reference, on Plato's part, either to Socrates' second speech
(which, of course, Plato {\em wrote}) or to the defence of justice in
the {\em Republic}.

\goto{219}[chapter-002.htmlux5cux23pla_0000499]. {\em complex}:
‘Complex' is now {\em poikilos}, ‘many-coloured', ‘variegated'.

\goto{220}[chapter-002.htmlux5cux23pla_0000501]. {\em rhapsodes}:
Professional performers of poetry (Ion, in Plato's {\em Ion}, is one).

\goto{221}[chapter-002.htmlux5cux23pla_0000503]. {\em discourses}:
Another rendering of {\em logoi} (I tried ‘speeches' again here in the
original version of the present translation, but that no longer seems
helpful to me).

\goto{222}[chapter-002.htmlux5cux23pla_0000505]. {\em arguments}: Or
(and) ‘speeches' ({\em logoi} again). Perhaps ‘things said' would be
enough.

\goto{223}[chapter-002.htmlux5cux23pla_0000507]. {\em of little worth}:
Or even ‘of no worth' -- that is, by comparison with what he has to say
{\em now}, on revisiting the same subject.

\goto{224}[chapter-002.htmlux5cux23pla_0000509]. {\em philosopher}: I.e.
again, a lover of wisdom ({\em philo-sophos}).

\goto{225}[chapter-002.htmlux5cux23pla_0000511]. {\em Isocrates}: A
brilliant rhetorician, teacher and speech-writer of Plato's own
generation, who also wrote against Plato -- and who would no longer have
been young by the time of writing of the {\em Phaedrus}. This is
something we need to bear in mind when reading what Socrates goes on to
say about him: whether or not he went on to ‘greater things', from
Plato's point of view, must be open to question, at least insofar as it
is certain that he never became a {\em dialectician} of the sort
Socrates has described in the {\em Phaedrus} (and would have been
appalled at the suggestion that he should).

\goto{226}[chapter-002.htmlux5cux23pla_0000513]. {\em desires}: I.e. the
{\em sôphrôn}, the self-controlled, restrained person. (For gold: cf.
235d --.)

\stoptext
