\part[symposium]{Symposium}

\blank[8*line,force]
\noindent\centerline{\definedfont[Serif at 26pt] S\kern 5.5pt%
                                 Y\kern 5.5pt%
                                 M\kern 6pt%
                                 P\kern 4pt%
                                 O\kern 4.5pt%
                                 S\kern 6pt%
                                 I\kern 5.5pt%
                                 U\kern 5pt%
                                 M}

\startchapter[title=Foreword]

This dialogue, Plato's poetic and dramatic masterpiece, relates the
events of a ‘symposium' or formal drinking party held in honor of
the tragedian Agathon's first victorious production. To gratify
Phaedrus (the passionate admirer of speeches and rhetoric in the
dialogue named after him), who indignantly regrets the neglect by
Greek poets and writers of the god of Love, the company agree to
give speeches in turn, while they all drink, in praise of Love. ‘Love'
(Greek {\em erôs}) covers sexual attraction and gratification
between men and women and between men and teenage boys, but the
focus here is also and especially on the adult male's role as
ethical and intellectual educator of the adolescent that was
traditional among the Athenians in the latter sort of relationship,
whether accompanied by sex or not. There are six speeches---plus a
seventh delivered by an uninvited and very drunk latecomer, the
Athenian statesman and general Alcibiades. In his youth Alcibiades
had been one of Socrates' admiring followers, and he now reports
in gripping detail the fascinating reversal Socrates worked upon
him in the erotic roles of the older and the younger man usual
among the Greeks in a relationship of ‘love': Socrates became the
pursued, Alcibiades the pursuer. Appropriately enough, all the
speakers, with the interesting exception of the comic poet
Aristophanes, are mentioned in {\em Protagoras} as among those who
flocked to Callias' house to attend the sophists gathered there
(all experts on speaking): as he enters Callias' house, Socrates spots 
four of the {\em Symposium} speakers---Phaedrus and Eryximachus in
a crowd round Hippias, and Agathon and Pausanias (his lover)
hanging on the words of Prodicus; Alcibiades joins the company
shortly afterwards.

Socrates' own speech is given over to reporting a discourse on love
he says he once heard from Diotima, a wise woman from Mantinea.
This Diotima seems an invention, contrived by Socrates (and Plato)
to distance Socrates in his report of it from what she says. In
any event, Diotima herself is made to say that Socrates can
probably not follow her in the ‘final and highest mystery' of the
‘rites of love'---her account of the ascent in love, beginning with love
for individual young men, ending with love for the Form of Beauty,
which ‘always {\em is} and neither comes to be nor passes away, neither
waxes nor wanes', and is ‘not beautiful this way and ugly that
way, nor beautiful at one time and ugly at another, nor beautiful
in relation to one thing and ugly in relation to another' but is
‘just what it is to be beautiful'. In this way Plato lets us know
that this theory of the Beautiful is his own contrivance, not
really an idea of Socrates (whether the historical philosopher or
the philosopher of the ‘Socratic' dialogues). Readers will want to
compare Diotima's speech on Love with those of Socrates in
{\em Phaedrus}, and also with Socrates' discussion on friendship with 
the boys in the {\em Lysis}.

The events of this evening at Agathon's house are all reported long
afterward by a young friend of Socrates' in his last years,
Apollodorus. Apparently they had become famous among Socrates'
intimates and others who were interested in hearing about him.
That, at any rate, is the impression Apollodorus leaves us with:
he has himself taken the trouble to learn about it all from
Aristodemus, who was present on the occasion, and he has just
reported on it to Glaucon (Socrates' conversation partner in the
{\em Republic}). He now reports again to an unnamed friend who has
asked to hear about it all---and to us readers of Plato's
dialogue.

\blank[line]
\hfill ---John M.~Cooper\par
\stopchapter

\startchapter[title=Symposium]

\sayapollodorus In\inouter{172} fact, your question does not find me
unprepared. Just the other day, as it happens, I was walking to the city
from my home in Phaleron when a man I know, who was making his way
behind me, saw me and called from a distance:

“The gentleman from Phaleron!” he yelled, trying to be funny. “Hey,
Apollodorus, wait!”

So I stopped and waited.

“Apollodorus\inouter{b}, I've been looking for you!” he said. “You know
there once was a gathering at Agathon's when Socrates, Alcibiades, and
their friends had dinner together; I wanted to ask you about the
speeches they made on Love. What were they? I heard a version from a man
who had it from Phoenix, Philip's son, but it was badly garbled, and he
said you were the one to ask. So please, will you tell me all about it?
After all, Socrates is your friend---who has a better right than you to
report his conversation? But before you begin,” he added, “tell me this:
were you there yourself?”

“Your\inouter{c} friend must have really garbled his story,” I replied, “if
you think this affair was so recent that I could have been there.”

“I did think that,” he said.

“Glaucon, how could you? You know very well Agathon hasn't lived in
Athens for many years, while it's been less than three that I've been
Socrates' companion and made it my job to know exactly what he says
and\inouter{173} does each day. Before that, I simply drifted
aimlessly. Of course, I used to think that what I was doing was
important, but in fact I was the most worthless man on earth---as bad as
you are this very moment: I used to think philosophy was the last thing
a man should do.”

“Stop joking, Apollodorus,” he replied. “Just tell me when the party
took place.”

“When we were still children, when Agathon won the prize with his first
tragedy. It was the day after he and his troupe held their victory
celebration.”

“So it really was a long time ago,” he said. “Then who told you about
it? Was it Socrates himself?”

“Oh\inouter{b}, for god's sake, of course not!” I replied. “It was the very same
man who told Phoenix, a fellow called Aristodemus, from
Cydatheneum, a real runt of a man, who always went barefoot. He went to
the party because, I think, he was obsessed with Socrates---one of the
worst cases at that time. Naturally, I checked part of his story with
Socrates, and Socrates agreed with his account.”

“Please tell me, then,” he said. “You speak and I'll listen, as we walk
to the city. This is the perfect opportunity.”

So\inouter{c} this is what we talked about on our way; and that's why, as I said
before, I'm not unprepared. Well, if I'm to tell {\em you} about
it too---I'll be glad to. After all, my greatest pleasure comes from
philosophical conversation, even if I'm only a listener, whether or not
I think it will be to my advantage. All other talk, especially the talk
of rich businessmen like you, bores me to tears, and I'm sorry for you
and your friends because you think your affairs are important when
really\inouter{d} they're totally trivial. Perhaps, in your turn, you think
I'm a failure, and, believe me, I think that what you think is true. But
as for all of you, I don't just {\em think} you are failures---I know it
for a fact.

\sayfriend You'll never change, Apollodorus! Always nagging, even at
yourself! I do believe you think everybody---yourself first of all---is
totally worthless, except, of course, Socrates. I don't know exactly how
you came to be called “the maniac,” but you certainly talk like one,
always furious with everyone, including yourself---but not with
Socrates!

\sayapollodorus Of course, my dear friend\inouter{e}, it's perfectly obvious why I 
have these views about us all: it's simply because I'm a maniac,
and I'm raving!

\sayfriend It's not worth arguing about this now, Apollodorus. Please do as
I asked: tell me the speeches.

\sayapollodorus All right \ldots{} Well, the speeches went something like
this---but I'd better\inouter{174} tell you the whole story from the very beginning,
as Aristodemus told it to me.

He said, then, that one day he ran into Socrates, who had just bathed
and put on his fancy sandals---both very unusual events. So he asked him
where he was going, and why he was looking so good.

Socrates replied, “I'm going to Agathon's for dinner. I managed to avoid
yesterday's victory party---I really don't like crowds---but I promised
to be there today. So, naturally, I took great pains with my appearance:
I'm going to the house of a good-looking man; I had to look my best. But
let me ask you this,” he added, “I know you haven't been invited to the
dinner; how would you like to come anyway?”\inouter{b}

And Aristodemus answered, “I'll do whatever you say.”

“Come with me, then,” Socrates said, “and we shall prove the proverb
wrong; the truth is, ‘Good men go uninvited to Goodman's
feast.'\endnote{Agathon's name could be
translated “Goodman.” The proverb is, “Good men go uninvited to an
inferior man's feast” (Eupolis fr. 289 Kock).} Even Homer
himself, when you think about it, did not much like this proverb;
he\inouter{c} not only disregarded it, he violated it. Agamemnon, of
course, is one of his great warriors, while he describes Menelaus as a
‘limp spearman.' And yet, when Agamemnon offers a sacrifice and gives a
feast, Homer has the weak Menelaus arrive uninvited at his superior's
table.”\endnote{Menelaus calls on Agamemnon
at {\em Iliad} ii.408. Menelaus is called a limp spearman at
xvii.587--88}

Aristodemus replied to this, “Socrates, I am afraid Homer's description
is bound to fit me better than yours. Mine is a case of an obvious
inferior arriving uninvited at the table of a man of letters. I think
you'd better figure out a good excuse for bringing me along, because,
you know, I \inouter{d} won't admit I've come without an invitation. I'll
say I'm your guest.”

“Let's go,” he said. “We'll think about what to say ‘as we proceed the
two of us along the way.'
”\endnote{An allusion to {\em Iliad}
x.224, “When two go together, one has an idea before the other.”}

With these words, they set out. But as they were walking, Socrates began
to think about something, lost himself in thought, and kept lagging
behind. Whenever Aristodemus stopped to wait for him, Socrates would
urge him \inouter{e} to go on ahead. When he arrived at Agathon's he found
the gate wide open, and that, Aristodemus said, caused him to find
himself in a very embarrassing situation: a household slave saw him the
moment he arrived and took him immediately to the dining room, where the
guests were already lying down on their couches, and dinner was about to
be served.

As soon as Agathon saw him, he called:

“Welcome, Aristodemus! What perfect timing! You're just in time for
dinner! I hope you're not here for any other reason---if you are, forget
it. I looked all over for you yesterday, so I could invite you, but I
couldn't find you anywhere. But where is Socrates? How come you didn't
bring him along?”

So I turned around (Aristodemus said), and Socrates was nowhere to be
seen. And I said that it was actually Socrates who had brought {\em me}
along as his guest.

“I'm\inouter{175} delighted he did,” Agathon replied. “But where is
he?”

“He was directly behind me, but I have no idea where he is now.”

“Go look for Socrates,” Agathon ordered a slave, “and bring him in.
Aristodemus,” he added, “you can share Eryximachus' couch.”

A slave brought water, and Aristodemus washed himself before he lay
down. Then another slave entered and said: “Socrates is here, but he's
gone off to the neighbor's porch. He's standing there and won't come in
even though I called him several times.”

“How strange,” Agathon replied. “Go back and bring him in. Don't leave
him there.”

But Aristodemus stopped him.\inouter{b} “No, no,” he said. “Leave him alone. It's 
one of his habits: every now and then he just goes off like that
and stands motionless, wherever he happens to be. I'm sure he'll come in
very soon, so don't disturb him; let him be.”

“Well, all right, if you really think so,” Agathon said, and turned to
the slaves: “Go ahead and serve the rest of us. What you serve is
completely up to you; pretend nobody's supervising you---as if I ever
did! Imagine that we are all your\inouter{c} own guests, myself included. Give us
good reason to praise your service.”

So they went ahead and started eating, but there was still no sign of
Socrates. Agathon wanted to send for him many times, but Aristodemus
wouldn't let him. And, in fact, Socrates came in shortly afterward, as
he always did---they were hardly halfway through their meal. Agathon,
who, as it happened, was all alone on the farthest couch, immediately
called: “Socrates, come lie down next to me.\inouter{d} Who knows, if I touch you,
I may catch a bit of the wisdom that came to you under my
neighbor's porch. It's clear {\em you've} seen the light. If you hadn't,
you'd still be standing there.”

Socrates sat down next to him and said, “How wonderful it would be, dear
Agathon, if the foolish were filled with wisdom simply by touching the
wise. If only wisdom were like water, which always flows from a full cup
into an empty one when we connect\inouter{e} them with a piece of yarn---well, 
then I would consider it the greatest prize to have the chance
to lie down next to you. I would soon be overflowing with your wonderful
wisdom. My own wisdom is of no account---a shadow in a dream---while
yours is bright and radiant and has a splendid future. Why, young as you
are, you're so brilliant I could call more than thirty thousand Greeks
as witnesses.”

“Now you've gone {\em too} far, Socrates,” Agathon replied. “Well, eat
your dinner. Dionysus will soon enough\inouter{176} be the judge of our claims to
wisdom!”\endnote{Dionysus was the god of
wine and drunkenness.}

Socrates took his seat after that and had his meal, according to
Aristodemus. When dinner was over, they poured a libation to the god,
sang a hymn, and---in short---followed the whole ritual. Then they
turned their attention to drinking. At that point Pausanias addressed
the group:

“Well, gentlemen, how can we arrange to drink less tonight? To be
honest, I still have a terrible hangover from yesterday, and I could
really use a break. I daresay most of you could, too, since you were
also part of the celebration. So let's try not to overdo it.”\inouter{b}

Aristophanes replied: “Good idea, Pausanias. We've got to make a plan
for going easy on the drink tonight. I was over my head last night
myself, like the others.”

After that, up spoke Eryximachus, son of Acumenus: “Well said, both of
you. But I still have one question: How do {\em you} feel, Agathon? Are
you strong enough for serious drinking?”

“Absolutely not,” replied Agathon. “I've no strength left for anything.”

“What\inouter{c} a lucky stroke for us,” Eryximachus said, “for me, for
Aristodemus, for Phaedrus, and the rest---that you large-capacity
drinkers are already exhausted. Imagine how weak drinkers like ourselves
feel after last night! Of course I don't include Socrates in my claims:
he can drink or not, and will be satisfied whatever we do. But since
none of us seems particularly eager to overindulge, perhaps it would not
be amiss for me\inouter{d} to provide you with some accurate information as
to the nature of intoxication. If I have learned anything from medicine,
it is the following point: inebriation is harmful to everyone.
Personally, therefore, I always refrain from heavy drinking; and I
advise others against it---especially people who are suffering the
effects of a previous night's excesses.”

“Well,” Phaedrus interrupted him, “I always follow your advice,
especially when you speak as a doctor. In this case, if the others know
what's good for them, they too will do just as you say.”

\inouter{e}At that point they all agreed not to get drunk that evening;
they decided to drink only as much as pleased them.

“It's settled, then,” said Eryximachus. “We are resolved to force no one
to drink more than he wants. I would like now to make a further motion:
let us dispense with the flute-girl who just made her entrance; let her
play for herself or, if she prefers, for the women in the house. Let us
instead spend our evening in conversation. If you are so minded, I would
like to propose a subject.”\inouter{177}

They all said they were quite willing, and urged him to make his
proposal. So Eryximachus said:

“Let me begin by citing Euripides' {\em Melanippe:} ‘Not mine the tale.'
What I am about to tell belongs to Phaedrus here, who is deeply
indignant on this issue, and often complains to me about it:

“‘Eryximachus,' he says, ‘isn't it an awful thing! Our poets have
composed hymns in honor of just about any god you can think of; but has
a single\inouter{b} one of them given one moment's thought to the god of
love, ancient and powerful as he is? As for our fancy intellectuals,
they have written volumes praising Heracles and other heroes (as did the
distinguished Prodicus). Well, perhaps {\em that's} not surprising, but
I've actually read a book by\inouter{c} an accomplished author who saw fit
to extol the usefulness of salt! How {\em could} people pay attention to
such trifles and never, not even once, write a proper hymn to Love? How
could anyone ignore so great a god?'

“Now, Phaedrus, in my judgment, is quite right. I would like, therefore,
to take up a contribution, as it were, on his behalf, and gratify his
wish. Besides,\inouter{d} I think this a splendid time for all of us here
to honor the god. If you agree, we can spend the whole evening in
discussion, because I propose that each of us give as good a speech in
praise of Love as he is capable of giving, in proper order from left to
right. And let us begin with Phaedrus, who is at the head of the table
and is, in addition, the father of our subject.”

“No one will vote against that, Eryximachus,” said Socrates. “How could
{\em I} vote ‘No,'\inouter{e} when the only thing I say I understand is the
art of love? Could Agathon and Pausanias? Could Aristophanes, who thinks
of nothing but Dionysus and Aphrodite? No one I can see here now could
vote against your proposal.

“And though it's not quite fair to those of us who have to speak last,
if the first speeches turn out to be good enough and to exhaust our
subject, I promise we won't complain. So let Phaedrus begin, with the
blessing of Fortune; let's hear his praise of Love.”

They all agreed\inouter{178} with Socrates, and pressed Phaedrus to start. Of course,
Aristodemus couldn't remember exactly what everyone
said, and I myself don't remember everything he told me. But I'll tell
you what he remembered best, and what I consider the most important
points.

As I say, he said Phaedrus spoke first, beginning more or less like
this:
\blank[line]
Love is a great god, wonderful in many ways to gods and men, and most
marvelous of all is the way he came into being. We honor him as one of
the most ancient gods,\inouter{b} and the proof of his great age is this: the 
parents of Love have no place in poetry or legend. According to
Hesiod, the first to be born was Chaos,

\setbox0\hbox{\em Earth, broad-chested, a seat for all, forever safe,}
\setbox1\hbox{\em\ldots{}but then came}

\startlines
    \setupnarrower[middle=\dimexpr(\textwidth-\wd0)/2]
    \startnarrower
        \noindent\hskip \dimexpr(\wd0-\wd1) {\em\ldots{}but then came}
        {\em Earth, broad-chested, a seat for all, forever safe,} 
        {\em And Love.}
    \stopnarrower
\stoplines

And Acusilaus agrees with Hesiod: after Chaos came Earth and Love, these
two.\endnote{Acusilaus was an
early-fifth-century writer of genealogies} And Parmenides
tells of this beginning:

\blank[line]
\noindent \hskip \dimexpr(\textwidth-\wd0)/2 {\em The very first god [she] designed was Love.}
\blank[line]

All sides\inouter{c} agree, then, that Love is one of the most ancient gods. As
such, he gives to us the greatest goods. I cannot say what
greater good there is for a young boy than a gentle lover, or for a
lover than a boy to love. There is a certain guidance each person needs
for his whole life, if he is to live well; and nothing imparts this
guidance---not high kinship, not public honor, not wealth---nothing
imparts\inouter{d} this guidance as well as Love. What guidance do I mean?
I mean a sense of shame at acting shamefully, and a sense of pride in
acting well. Without these, nothing fine or great can be accomplished,
in public or in private.

What I say is this: if a man in love is found doing something shameful,
or accepting shameful treatment because he is a coward and makes no
defense, then nothing would give him more pain than being seen by the
boy he loves---not even being seen by his father\inouter{e} or his comrades. We see 
the same thing also in the boy he loves, that he is especially
ashamed before his lover when he is caught in something shameful. If
only there were a way to start a city or an army made up of lovers and
the boys they love! Theirs would be the best possible system of society,
for they would \inouter{179}hold back from all that is shameful, and
seek honor in each other's
eyes. Even a few of
them, in battle side by side, would conquer all the world, I'd say. For
a man in love would never allow his loved one, of all people, to see him
leaving ranks or dropping weapons. He'd rather die a thousand deaths!
And as for leaving the boy behind, or not coming to his aid in
danger---why, no one is so base that true Love could not inspire him
with \inouter{b} courage, and make him as brave as if he'd been born a hero.
When Homer says a god ‘breathes might' into some of the heroes, this is
really Love's gift to every
lover.\endnote{Cf. {\em Iliad} x.482,
xv.262; {\em Odyssey} ix.381.}

Besides, no one will die for you but a lover, and a lover will do this
even if she's a woman. Alcestis is proof to everyone in Greece that what
\inouter{c} I say is
true.\endnote{Alcestis was the
self-sacrificing wife of Admetus, whom Apollo gave a chance to live if
anyone would go to Hades in his place.} Only she was
willing to die in place of her husband, although his father and mother
were still alive. Because of her love, she went so far beyond his
parents in family feeling that she made them look like outsiders, as if
they belonged to their son in name only. And when she did this her deed
struck everyone, even the gods, as nobly done. The gods were so
delighted, in fact, that they gave her the prize they reserve for a
handful \inouter{d} chosen from the throngs of noble heroes---they sent her
soul back from the dead. As you can see, the eager courage of love wins
highest honors from the gods.

Orpheus, however, they sent unsatisfied from Hades, after showing him
only an image of the woman he came for. They did not give him the woman
herself, because they thought he was soft (he was, after all, a
cithara-player) and did not dare to die like Alcestis for Love's sake,
but contrived to enter living into Hades. So they punished him for that,
and \inouter{e} made him die at the hands of
women.\endnote{Orpheus was a musician of
legendary powers, who charmed his way into the underworld in search of
his dead wife, Eurydice.}

The honor they gave to Achilles is another matter. They sent him to the
Isles of the Blest because he dared to stand by his lover Patroclus and
\inouter{180} avenge him, even after he had learned from his mother
that he would die if he killed Hector, but that if he chose otherwise
he'd go home and end his life as an old man. Instead he chose to die for
Patroclus, and more than that, he did it for a man whose life was
already over. The gods were highly delighted at this, of course, and
gave him special honor, because he made so much of his lover. Aeschylus
talks nonsense when he claims Achilles was the
lover;\endnote{In his play, {\em The
Myrmidons}. In Homer there is no hint of sexual attachment between
Achilles and Patroclus.} he was more
beautiful than Patroclus, more beautiful than all the heroes, and still
beardless. Besides he was much younger, as Homer says.

In truth, the gods honor virtue most highly when it belongs to Love.
\inouter{b} They are more impressed and delighted, however, and are more
generous with a loved one who cherishes his lover, than with a lover who
cherishes the boy he loves. A lover is more godlike than his boy, you
see, since he is inspired by a god. That's why they gave a higher honor
to Achilles than to Alcestis, and sent him to the Isles of the Blest.

Therefore I say Love is the most ancient of the gods, the most honored,
and the most powerful in helping men gain virtue and blessedness,
whether they are alive or have passed away.

That was more or less what Phaedrus said according to Aristodemus.
\inouter{c} There followed several other speeches which he couldn't remember
very well. So he skipped them and went directly to the speech of
Pausanias.

\blank[line]

Phaedrus (Pausanias began), I'm not quite sure our subject has been well
defined. Our charge has been simple---to speak in praise of Love. This
would have been fine if Love himself were simple, too, but as a matter
of fact, there are two kinds of Love. In view of this, it might be
better to begin by making clear which kind of Love we are to praise. Let
me therefore \inouter{d} try to put our discussion back on the right track
and explain which kind of Love ought to be praised. Then I shall give
him the praise he deserves, as the god he is.

It is a well-known fact that Love and Aphrodite are inseparable. If,
therefore, Aphrodite were a single goddess, there could also be a single
Love; but, since there are actually two goddesses of that name, there
also are two kinds of Love. I don't expect you'll disagree with me about
the two goddesses, will you? One is an older deity, the motherless
daughter of Uranus, the god of heaven: she is known as Urania, or
Heavenly Aphrodite. The other goddess is younger, the daughter of Zeus
and Dione: her name is Pandemos, or Common Aphrodite. It follows,
therefore, that there \inouter{e} is a Common as well as a Heavenly Love,
depending on which goddess is Love's partner. And although, of course,
all the gods must be praised, we must still make an effort to keep these
two gods apart.

The reason for this applies in the same way to every type of action:
considered in itself, no action is either good or bad, honorable or
shameful. \inouter{181} Take, for example, our own case. We had a
choice between drinking, singing, or having a conversation. Now, in
itself none of these is better than any other: how it comes out depends
entirely on how it is performed. If it is done honorably and properly,
it turns out to be honorable; if it is done improperly, it is
disgraceful. And my point is that exactly this principle applies to
being in love: Love is not in himself noble and worthy of praise; that
depends on whether the sentiments he produces in us are themselves
noble.

\inouter{b} Now the Common Aphrodite's Love is himself truly common. As
such, he strikes wherever he gets a chance. This, of course, is the love
felt by the vulgar, who are attached to women no less than to boys, to
the body more than to the soul, and to the least intelligent partners,
since all they care about is completing the sexual act. Whether they do
it honorably or not is of no concern. That is why they do whatever comes
their way, sometimes good, sometimes bad; and which one it is is
incidental to their purpose. For the Love who moves them belongs to a
much younger goddess, \inouter{c} who, through her parentage, partakes of
the nature both of the female and the male.

Contrast this with the Love of Heavenly Aphrodite. This goddess, whose
descent is purely male (hence this love is for boys), is considerably
older and therefore free from the lewdness of youth. That's why those
who are inspired by her Love are attracted to the male: they find
pleasure in what is by nature stronger and more intelligent. But, even
within the group that \inouter{d} is attracted to handsome boys, some are
not moved purely by this Heavenly Love; those who are do not fall in
love with little boys; they prefer older ones whose cheeks are showing
the first traces of a beard---a sign that they have begun to form minds
of their own. I am convinced that a man who falls in love with a young
man of this age is generally prepared to share everything with the one
he loves---he is eager, in fact, to spend the rest of his own life with
him. He certainly does not aim to deceive him---to take advantage of him
while he is still young and inexperienced and \inouter{e} then, after
exposing him to ridicule, to move quickly on to someone else.

As a matter of fact, there should be a law forbidding affairs with young
boys. If nothing else, all this time and effort would not be wasted on
such an uncertain pursuit---and what is more uncertain than whether a
particular boy will eventually make something of himself, physically or
mentally? Good men, of course, are willing to make a law like this for
themselves, but those other lovers, the vulgar ones, need external
restraint. \inouter{182} For just this reason we have placed every
possible legal obstacle to their seducing our own wives and daughters.
These vulgar lovers are the people who have given love such a bad
reputation that some have gone so far as to claim that taking {\em any}
man as a lover is in itself disgraceful. Would anyone make this claim if
he weren't thinking of how hasty vulgar lovers are, and therefore how
unfair to their loved ones? For nothing done properly and in accordance
with our customs would ever have provoked such righteous disapproval.

I should point out, however, that, although the customs regarding Love
in most cities are simple and easy to understand, here in Athens (and in
\inouter{b} Sparta as well) they are remarkably complex. In places where the
people are inarticulate, like Elis or Boeotia, tradition
straightforwardly approves taking a lover in every case. No one there,
young or old, would ever consider it shameful. The reason, I suspect, is
that, being poor speakers, they want to save themselves the trouble of
having to offer reasons and arguments in support of their suits.

By contrast, in places like Ionia and almost every other part of the
Persian empire, taking a lover is always considered disgraceful. The
Persian empire is absolute; that is why it condemns love as well as
philosophy and sport. \inouter{c} It is no good for rulers if the people
they rule cherish ambitions for themselves or form strong bonds of
friendship with one another. That these are precisely the effects of
philosophy, sport, and especially of Love is a lesson the tyrants of
Athens learned directly from their own experience: Didn't their reign
come to a dismal end because of the bonds uniting Harmodius and
Aristogiton in love and
affection?\endnote{Harmodius and Aristogiton
attempted to overthrow the tyrant Hippias in 514 B.C. Although their
attempt failed, the tyranny fell three years later, and the lovers were
celebrated as tyrannicides.} \inouter{d}

So you can see that plain condemnation of Love reveals lust for power in
the rulers and cowardice in the ruled, while indiscriminate approval
testifies to general dullness and stupidity.

Our own customs, which, as I have already said, are much more difficult
to understand, are also far superior. Recall, for example, that we
consider it more honorable to declare your love rather than to keep it a
secret, especially if you are in love with a youth of good family and
accomplishment, even if he isn't all that beautiful. Recall also that a
lover is encouraged in every possible way; this means that what he does
is not considered shameful. On the contrary, conquest is deemed noble,
and failure shameful. \inouter{e} And as for {\em attempts} at conquest, our
custom is to praise lovers for totally extraordinary acts---so
extraordinary, in fact, that if they performed them \inouter{183} for
any other purpose whatever, they would reap the most profound contempt.
Suppose, for example, that in order to secure money, or a public post,
or any other practical benefit from another person, a man were willing
to do what lovers do for the ones they love. Imagine that in pressing
his suit he went to his knees in public view and begged in the most
humiliating way, that he swore all sorts of vows, that he spent the
night at the other man's doorstep, that he were anxious to provide
services even a slave would have refused---well, you can be sure that
everyone, his enemies no less than his friends, would stand in his way.
His enemies would jeer at \inouter{b} his fawning servility, while his
friends, ashamed on his behalf, would try everything to bring him back
to his senses. But let a lover act in any of these ways, and everyone
will immediately say what a charming man he is! No blame attaches to his
behavior: custom treats it as noble through and through. And what is
even more remarkable is that, at least according to popular wisdom, the
gods will forgive a lover even for breaking his vows---a lover's vow,
our people say, is no vow at all. The freedom given \inouter{c} to the lover
by both gods and men according to our custom is immense.

In view of all this, you might well conclude that in our city we
consider the lover's desire and the willingness to satisfy it as the
noblest things in the world. When, on the other hand, you recall that
fathers hire attendants for their sons as soon as they're old enough to
be attractive, and that an attendant's main task is to prevent any
contact between his charge and his suitors; when you recall how
mercilessly a boy's own friends tease him if they catch him at it, and
how strongly their elders approve and \inouter{d} even encourage such
mocking---when you take all this into account, you're bound to come to
the conclusion that we Athenians consider such behavior the most
shameful thing in the world.

In my opinion, however, the fact of the matter is this. As I said
earlier, love is, like everything else, complex: considered simply in
itself, it is neither honorable nor a disgrace---its character depends
entirely on the behavior it gives rise to. To give oneself to a vile man
in a vile way is truly disgraceful behavior; by contrast, it is
perfectly honorable to give oneself honorably to the right man. Now you
may want to know who \inouter{e} counts as vile in this context. I'll tell
you: it is the common, vulgar lover, who loves the body rather than the
soul, the man whose love is bound to be inconstant, since what he loves
is itself mutable and unstable. The moment the body is no longer in
bloom, “he flies off and
away,”\endnote{{\em Iliad} ii.71.} his promises
and vows in tatters behind him. How different from this is a man who
loves the right sort of character, and who remains its lover for
\inouter{184} life, attached as he is to something that is permanent.

We can now see the point of our customs: they are designed to separate
the wheat from the chaff, the proper love from the vile. That's why we
do everything we can to make it as easy as possible for lovers to press
their suits and as difficult as possible for young men to comply; it is
like a competition, a kind of test to determine to which sort each
belongs. This explains two further facts: First, why we consider it
shameful to yield too quickly: the passage of time in itself provides a
good test in these matters. \inouter{b} Second, why we also consider it
shameful for a man to be seduced by money or political power, either
because he cringes at ill-treatment and will not endure it or because,
once he has tasted the benefits of wealth and power, he will not rise
above them. None of these benefits is stable or permanent, apart from
the fact that no genuine affection can possibly be based upon them.

Our customs, then, provide for only one honorable way of taking a man
\inouter{c} as a lover. In addition to recognizing that the lover's total
and willing subjugation to his beloved's wishes is neither servile nor
reprehensible, we allow that there is one---and only one---further
reason for willingly subjecting oneself to another which is equally
above reproach: that is subjection for the sake of virtue. If someone
decides to put himself at another's disposal because he thinks that this
will make him better in wisdom or in any other part of virtue, we
approve of his voluntary subjection: we consider it neither shameful nor
servile. Both these principles---that is, both the principle governing
the proper attitude toward the lover of young men and the principle
governing the love of wisdom and of \inouter{d} virtue in general---must be
combined if a young man is to accept a lover in an honorable way. When
an older lover and a young man come together and each obeys the
principle appropriate to him---when the lover realizes that he is
justified in doing anything for a loved one who grants him favors, and
when the young man understands that he is justified in performing any
service for a lover who can make him wise and virtuous---and when the
\inouter{e} lover {\em is} able to help the young man become wiser and
better, and the young man {\em is} eager to be taught and improved by
his lover---then, and only then, when these two principles coincide
absolutely, is it ever honorable for a young man to accept a lover.

Only in this case, we should notice, is it never shameful to be
deceived; in every other case it is shameful, both for the deceiver and
the person he \inouter{185} deceives. Suppose, for example, that
someone thinks his lover is rich and accepts him for his money; his
action won't be any less shameful if it turns out that he was deceived
and his lover was a poor man after all. For the young man has already
shown himself to be the sort of person who will do anything for
money---and that is far from honorable. By the same token, suppose that
someone takes a lover in the mistaken belief that this lover is a good
man and likely to make him better himself, while in reality the man is
horrible, totally lacking in virtue; even so, it is noble for him to
\inouter{b} have been deceived. For he too has demonstrated something about
himself: that he is the sort of person who will do anything for the sake
of virtue---and what could be more honorable than that? It follows,
therefore, that giving in to your lover for virtue's sake is honorable,
whatever the outcome. And this, of course, is the Heavenly Love of the
heavenly goddess. Love's value to the city as a whole and to the
citizens is immeasurable, for he compels the lover and his loved one
alike to make virtue their central \inouter{c} concern. All other forms of
love belong to the vulgar goddess.

Phaedrus, I'm afraid this hasty improvisation will have to do as my
contribution on the subject of Love.

When Pausanias finally came to a pause (I've learned this sort of fine
figure from our clever rhetoricians), it was Aristophanes' turn,
according to Aristodemus. But he had such a bad case of the
hiccups---he'd probably stuffed himself again, though, of course, it
could have been anything---that making a speech was totally out of the
question. So he turned to the doctor, Eryximachus, who was next in line,
and said to him: \inouter{d}

“Eryximachus, it's up to you---as well it should be. Cure me or take my
turn.”

“As a matter of fact,” Eryximachus replied, “I shall do both. I shall
take your turn---you can speak in my place as soon as you feel
better---and I shall also cure you. While I am giving my speech, you
should hold your breath for as long as you possibly can. This may well
eliminate your \inouter{e} hiccups. If it fails, the best remedy is a
thorough gargle. And if even this has no effect, then tickle your nose
with a feather. A sneeze or two will cure even the most persistent
case.”

“The sooner you start speaking, the better,” Aristophanes said. “I'll
follow your instructions to the letter.”

This, then, was the speech of Eryximachus:

\blank[line]

Pausanias introduced a crucial consideration in his speech, though in my
opinion he did not develop it sufficiently. Let me therefore try to
carry \inouter{186} his argument to its logical conclusion. His
distinction between the two species of Love seems to me very useful
indeed. But if I have learned a single lesson from my own field, the
science of medicine, it is that Love does not occur only in the human
soul; it is not simply the attraction we feel toward human beauty: it is
a significantly broader phenomenon. It \inouter{b} certainly occurs within
the animal kingdom, and even in the world of plants. In fact, it occurs
everywhere in the universe. Love is a deity of the greatest importance:
he directs everything that occurs, not only in the human domain, but
also in that of the gods.

Let me begin with some remarks concerning medicine---I hope you will
forgive my giving pride of place to my own profession. The point is that
our very bodies manifest the two species of Love. Consider for a moment
the marked difference, the radical dissimilarity, between healthy and
diseased constitutions and the fact that dissimilar subjects desire and
love objects that are themselves dissimilar. Therefore, the love
manifested in health is fundamentally distinct from the love manifested
in disease. And \inouter{c} now recall that, as Pausanias claimed, it is as
honorable to yield to a good man as it is shameful to consort with the
debauched. Well, my point is that the case of the human body is strictly
parallel. Everything sound and healthy in the body must be encouraged
and gratified; that is precisely the object of medicine. Conversely,
whatever is unhealthy and unsound must be frustrated and rebuffed:
that's what it is to be an expert in medicine.

\inouter{d} In short, medicine is simply the science of the effects of Love
on repletion and depletion of the body, and the hallmark of the
accomplished physician is his ability to distinguish the Love that is
noble from the Love that is ugly and disgraceful. A good practitioner
knows how to affect the body and how to transform its desires; he can
implant the proper species of Love when it is absent and eliminate the
other sort whenever it occurs. The physician's task is to effect a
reconciliation and establish mutual love between the most basic bodily
elements. Which are those elements? They are, of course, those that are
most opposed to one another, as hot is to \inouter{e} cold, bitter to sweet,
wet to dry, cases like those. In fact, our ancestor Asclepius first
established medicine as a profession when he learned how to produce
concord and love between such opposites---that is what those poet
fellows say, and---this time---I concur with them.

\inouter{187} Medicine, therefore, is guided everywhere by the god of
Love, and so are physical education and farming as well. Further, a
moment's reflection suffices to show that the case of poetry and music,
too, is precisely the same. Indeed, this may have been just what
Heraclitus had in mind, though his mode of expression certainly leaves
much to be desired. The one, he says, “being at variance with itself is
in agreement with itself” “like the attunement of a bow or a
lyre.”\endnote{Heraclitus of Ephesus, a
philosopher of the early fifth century, was known for his enigmatic
sayings. This one is quoted elsewhere in a slightly different form, frg.
B 51 Diels-Kranz.} Naturally, it
is patently absurd to claim that an attunement or a harmony is in itself
discordant or that its elements are still in discord with one another.
Heraclitus probably meant that an expert musician creates a harmony by
resolving the prior discord between \inouter{b} high and low notes. For
surely there can be no harmony so long as high and low are still
discordant; harmony, after all, is consonance, and consonance is a
species of agreement. Discordant elements, as long as they are still in
discord, cannot come to an agreement, and they therefore cannot produce
a harmony. Rhythm, for example, is produced only when fast and slow,
\inouter{c} though earlier discordant, are brought into agreement with each
other. Music, like medicine, creates agreement by producing concord and
love between these various opposites. Music is therefore simply the
science of the effects of Love on rhythm and harmony.

These effects are easily discernible if you consider the constitution of
rhythm and harmony in themselves; Love does not occur in both his forms
in this domain. But the moment you consider, in their turn, the effects
of rhythm and harmony on their audience---either through composition,
\inouter{d} which creates new verses and melodies, or through musical
education, which teaches the correct performance of existing
compositions---complications arise directly, and they require the
treatment of a good practitioner. Ultimately, the identical argument
applies once again: the love felt by good people or by those whom such
love might improve in this regard must be encouraged and protected. This
is the honorable, heavenly species of Love, produced by the melodies of
Urania, the Heavenly Muse. \inouter{e} The other, produced by Polyhymnia,
the muse of many songs, is common and vulgar. Extreme caution is
indicated here: we must be careful to enjoy his pleasures without
slipping into debauchery---this case, I might add, is strictly parallel
to a serious issue in my own field, namely, the problem of regulating
the appetite so as to be able to enjoy a fine meal without unhealthy
aftereffects.

In music, therefore, as well as in medicine and in all the other
domains, in matters divine as well as in human affairs, we must attend
with the greatest possible care to these two species of Love, which are,
indeed, to \inouter{188} be found everywhere. Even the seasons of the
year exhibit their influence. When the elements to which I have already
referred---hot and cold, wet and dry---are animated by the proper
species of Love, they are in harmony with one another: their mixture is
temperate, and so is the climate. Harvests are plentiful; men and all
other living things are in good health; no harm can come to them. But
when the sort of Love that is crude and impulsive controls the seasons,
he brings death and destruction. He spreads the \inouter{b} plague and many
other diseases among plants and animals; he causes frost and hail and
blights. All these are the effects of the immodest and disordered
species of Love on the movements of the stars and the seasons of the
year, that is, on the objects studied by the science called astronomy.

\inouter{c} Consider further the rites of sacrifice and the whole area with
which the art of divination is concerned, that is, the interaction
between men and gods. Here, too, Love is the central concern: our object
is to try to maintain the proper kind of Love and to attempt to cure the
kind that is diseased. For what is the origin of all impiety? Our
refusal to gratify the orderly kind of Love, and our deference to the
other sort, when we should have been guided by the former sort of Love
in every action in connection with our parents, living or dead, and with
the gods. The task of divination is to keep watch over these two species
of Love and to doctor them as \inouter{d} necessary. Divination, therefore,
is the practice that produces loving affection between gods and men; it
is simply the science of the effects of Love on justice and piety.

Such is the power of Love---so varied and great that in all cases it
might be called absolute. Yet even so it is far greater when Love is
directed, in temperance and justice, toward the good, whether in heaven
or on earth: happiness and good fortune, the bonds of human society,
concord with the gods above---all these are among his gifts.

\inouter{e} Perhaps I, too, have omitted a great deal in this discourse on
Love. If so, I assure you, it was quite inadvertent. And if in fact I
have overlooked certain points, it is now your task, Aristophanes, to
complete the argument---unless, of course, you are planning on a
different approach. In any \inouter{189} case, proceed; your hiccups
seem cured.

\blank[line]

Then Aristophanes took over (so Aristodemus said): “The hiccups have
stopped all right---but not before I applied the Sneeze Treatment to
them. Makes me wonder whether the ‘orderly sort of Love' in the body
calls for the sounds and itchings that constitute a sneeze, because the
hiccups stopped immediately when I applied the Sneeze Treatment.”

“You're good, Aristophanes,” Eryximachus answered. “But watch what
you're doing. You are making jokes before your speech, and you're
forcing me to prepare for you to say something funny, and to put up my
guard \inouter{b} against you, when otherwise you might speak at peace.”

Then Aristophanes laughed. “Good point, Eryximachus. So let me ‘unsay
what I have said.' But don't put up your guard. I'm not worried about
saying something funny in my coming oration. That would be pure profit,
and it comes with the territory of my Muse. What I'm worried about is
that I might say something ridiculous.”

“Aristophanes, do you really think you can take a shot at me, and then
escape? Use your head! Remember, as you speak, that you will be called
\inouter{c} upon to give an account. Though perhaps, if I decide to, I'll
let you off.”

“Eryximachus,” Aristophanes said, “indeed I do have in mind a different
approach to speaking than the one the two of you used, you and
Pausanias. You see, I think people have entirely missed the power of
Love, because, if they had grasped it, they'd have built the greatest
temples and altars to him and made the greatest sacrifices. But as it
is, none of this is done for him, though it should be, more than
anything else! For he loves the human \inouter{d} race more than any other
god, he stands by us in our troubles, and he cures those ills we humans
are most happy to have mended. I shall, therefore, try to explain his
power to you; and you, please pass my teaching on to everyone
else.”

\blank[line]

First you must learn what Human Nature was in the beginning and what has
happened to it since, because long ago our nature was not what it is
now, but very different. There were three kinds of human beings, that's
my first point---not two as there are now, male and female. In \inouter{e}
addition to these, there was a third, a combination of those two; its
name survives, though the kind itself has vanished. At that time, you
see, the word “androgynous” really meant something: a form made up of
male and female elements, though now there's nothing but the word, and
that's used as an insult. My second point is that the shape of each
human being was completely round, with back and sides in a circle; they
had four hands each, as many legs as hands, and two faces, exactly
alike, on a rounded \inouter{190} neck. Between the two faces, which
were on opposite sides, was one head with four ears. There were two sets
of sexual organs, and everything else was the way you'd imagine it from
what I've told you. They walked upright, as we do now, whatever
direction they wanted. And whenever they set out to run fast, they
thrust out all their eight limbs, the ones they had then, and spun
rapidly, the way gymnasts do cartwheels, by bringing their legs around
straight.

Now here is why there were three kinds, and why they were as I \inouter{b}
described them: The male kind was originally an offspring of the sun,
the female of the earth, and the one that combined both genders was an
offspring of the moon, because the moon shares in both. They were
spherical, and so was their motion, because they were like their parents
in the sky.

In strength and power, therefore, they were terrible, and they had great
ambitions. They made an attempt on the gods, and Homer's story about
Ephialtes and Otus was originally about them: how they tried to make an
ascent to heaven so as to attack the
gods.\endnote{{\em Iliad} v.385,
{\em Odyssey} xi.305 ff.} Then Zeus and
the other gods \inouter{c} met in council to discuss what to do, and they
were sore perplexed. They couldn't wipe out the human race with
thunderbolts and kill them all off, as they had the giants, because that
would wipe out the worship they receive, along with the sacrifices we
humans give them. On the other hand, they couldn't let them run riot. At
last, after great effort, Zeus had an idea.

“I think I have a plan,” he said, “that would allow human beings to
exist and stop their misbehaving: they will give up being wicked when
\inouter{d} they lose their strength. So I shall now cut each of them in
two. At one stroke they will lose their strength and also become more
profitable to us, owing to the increase in their number. They shall walk
upright on two legs. But if I find they still run riot and do not keep
the peace,” he said, “I will cut them in two again, and they'll have to
make their way on one leg, hopping.”

\inouter{e} So saying, he cut those human beings in two, the way people cut
sorbapples before they dry them or the way they cut eggs with hairs. As
he cut each one, he commanded Apollo to turn its face and half its neck
towards the wound, so that each person would see that he'd been cut and
keep better order. Then Zeus commanded Apollo to heal the rest of the
wound, and Apollo did turn the face around, and he drew skin from all
sides over what is now called the stomach, and there he made one mouth,
as in a pouch with a drawstring, and fastened it at the center of the
stomach. \inouter{191} This is now called the navel. Then he smoothed
out the other wrinkles, of which there were many, and he shaped the
breasts, using some such tool as shoemakers have for smoothing wrinkles
out of leather on the form. But he left a few wrinkles around the
stomach and the navel, to be a reminder of what happened long ago.

Now, since their natural form had been cut in two, each one longed for
its own other half, and so they would throw their arms about each other,
weaving themselves together, wanting to grow together. In that condition
\inouter{b} they would die from hunger and general idleness, because they
would not do anything apart from each other. Whenever one of the halves
died and one was left, the one that was left still sought another and
wove itself together with that. Sometimes the half he met came from a
woman, as we'd call her now, sometimes it came from a man; either way,
they kept on dying.

Then, however, Zeus took pity on them, and came up with another plan: he
moved their genitals around to the front! Before then, you see, they
\inouter{c} used to have their genitals outside, like their faces, and they
cast seed and made children, not in one another, but in the ground, like
cicadas. So Zeus brought about this relocation of genitals, and in doing
so he invented interior reproduction, {\em by} the man {\em in} the
woman. The purpose of this was so that, when a man embraced a woman, he
would cast his seed and they would have children; but when male embraced
male, they would at least have the satisfaction of intercourse, after
which they could stop embracing, \inouter{d} return to their jobs, and look
after their other needs in life. This, then, is the source of our desire
to love each other. Love is born into every human being; it calls back
the halves of our original nature together; it tries to make one out of
two and heal the wound of human nature.

Each of us, then, is a “matching half” of a human whole, because each
was sliced like a flatfish, two out of one, and each of us is always
seeking the half that matches him. That's why a man who is split from
the double sort (which used to be called “androgynous”) runs after
women. Many \inouter{e} lecherous men have come from this class, and so do
the lecherous women who run after men. Women who are split from a woman,
however, pay no attention at all to men; they are oriented more towards
women, and lesbians come from this class. People who are split from a
male are male-oriented. While they are boys, because they are chips off
the male block, they love men and enjoy lying with men and being
embraced by men; \inouter{192} those are the best of boys and lads,
because they are the most manly in their nature. Of course, some say
such boys are shameless, but they're lying. It's not because they have
no shame that such boys do this, you see, but because they are bold and
brave and masculine, and they tend to cherish what is like themselves.
Do you want me to prove it? Look, these are the only kind of boys who
grow up to be real men in politics. When \inouter{b} they're grown men, they
are lovers of young men, and they naturally pay no attention to marriage
or to making babies, except insofar as they are required by local
custom. They, however, are quite satisfied to live their lives with one
another unmarried. In every way, then, this sort of man grows up as a
lover of young men and a lover of Love, always rejoicing in his own
kind.

And so, when a person meets the half that is his very own, whatever his
orientation, whether it's to young men or not, then something wonderful
happens: the two are struck from their senses by love, by a sense of
\inouter{c} belonging to one another, and by desire, and they don't want to
be separated from one another, not even for a moment.

These are the people who finish out their lives together and still
cannot say what it is they want from one another. No one would think it
is the intimacy of sex---that mere sex is the reason each lover takes so
great and deep a joy in being with the other. It's obvious that the soul
of every lover \inouter{d} longs for something else; his soul cannot say
what it is, but like an oracle it has a sense of what it wants, and like
an oracle it hides behind a riddle. Suppose two lovers are lying
together and
Hephaestus\endnote{Cf. {\em Odyssey}
viii.266 ff.} stands
over them with his mending tools, asking, “What is it you human beings
really want from each other?” And suppose they're perplexed, and he asks
them again: “Is this your heart's desire, then---for the two of you to
become parts of the same whole, as near as can be, and never to
separate, day or night? Because if that's your desire, I'd like to weld
you together and join you into something that is naturally whole, so
that the two of you are made \inouter{e} into one. Then the two of you would
share one life, as long as you lived, because you would be one being,
and by the same token, when you died, you would be one and not two in
Hades, having died a single death. Look at your love, and see if this is
what you desire: wouldn't this be all the good fortune you could want?”

Surely you can see that no one who received such an offer would turn it
down; no one would find anything else that he wanted. Instead, everyone
would think he'd found out at last what he had always wanted: to come
together and melt together with the one he loves, so that one person
emerged from two. Why should this be so? It's because, as I said, we
used to be complete wholes in our original nature, and now “Love” is the
name \inouter{193} for our pursuit of wholeness, for our desire to be
complete.

Long ago we were united, as I said; but now the god has divided us as
punishment for the wrong we did him, just as the Spartans divided the
Arcadians.\endnote{Arcadia included the city
of Mantinea, which opposed Sparta, and was rewarded by having its
population divided and dispersed in 385 B.C. Aristophanes seems to be
referring anachronistically to those events; such anachronisms are not
uncommon in Plato.} So there's
a danger that if we don't keep order before the gods, we'll be split in
two again, and then we'll be walking around in the condition of people
carved on gravestones in bas-relief, sawn apart between the nostrils,
like half dice. We should encourage all men, therefore, to treat \inouter{b}
the gods with all due reverence, so that we may escape this fate and
find wholeness instead. And we will, if Love is our guide and our
commander. Let no one work against him. Whoever opposes Love is hateful
to the gods, but if we become friends of the god and cease to quarrel
with him, then we shall find the young men that are meant for us and win
their love, as very few men do nowadays.

\inouter{c} Now don't get ideas, Eryximachus, and turn this speech into a
comedy. Don't think I'm pointing this at Pausanias and Agathon.
Probably, they both do belong to the group that are entirely masculine
in nature. But I am speaking about everyone, men and women alike, and I
say there's just one way for the human race to flourish: we must bring
love to its perfect conclusion, and each of us must win the favors of
his very own young man, so that he can recover his original nature. If
that is the ideal, then, of course, the nearest approach to it is best
in present circumstances, and that is to win the favor of young men who
are naturally sympathetic to us.

\inouter{d} If we are to give due praise to the god who can give us this
blessing, then, we must praise Love. Love does the best that can be done
for the time being: he draws us towards what belongs to us. But for the
future, Love promises the greatest hope of all: if we treat the gods
with due reverence, he will restore to us our original nature, and by
healing us, he will make us blessed and happy.

“That,” he said, “is my speech about Love, Eryximachus. It is rather
different from yours. As I begged you earlier, don't make a comedy of
it. \inouter{e} I'd prefer to hear what all the others will say---or,
rather, what each of them will say, since Agathon and Socrates are the
only ones left.”

“I found your speech delightful,” said Eryximachus, “so I'll do as you
say. Really, we've had such a rich feast of speeches on Love, that if I
couldn't vouch for the fact that Socrates and Agathon are masters of the
art of love, I'd be afraid that they'd have nothing left to say. But as
it is, I have no fears on this score.”

\inouter{194} Then Socrates said, “That's because {\em you} did
beautifully in the contest, Eryximachus. But if you ever get in my
position, or rather the position I'll be in after Agathon's spoken so
well, then you'll really be afraid. You'll be at your wit's end, as I am
now.”

“You're trying to bewitch me, Socrates,” said Agathon, “by making me
think the audience expects great things of my speech, so I'll get
flustered.” \inouter{b}

“Agathon!” said Socrates, “How forgetful do you think I am? I saw how
brave and dignified you were when you walked right up to the theater
platform along with the actors and looked straight out at that enormous
audience. You were about to put your own writing on display, and you
weren't the least bit panicked. After seeing that, how could I expect
you to be flustered by us, when we are so few?”

“Why, Socrates,” said Agathon. “You must think I have nothing but
theater audiences on my mind! So you suppose I don't realize that, if
you're intelligent, you find a few sensible men much more frightening
than a senseless crowd?”

“No,” he said, “It wouldn't be very handsome of me to think you crude
\inouter{c} in any way, Agathon. I'm sure that if you ever run into people
you consider wise, you'll pay more attention to them than to ordinary
people. But you can't suppose we're in that class; we were at the
theater too, you know, part of the ordinary crowd. Still, if you did run
into any wise men, other than yourself, you'd certainly be ashamed at
the thought of doing anything ugly in front of them. Is that what you
mean?”

‘That's true,” he said.

“On the other hand, you wouldn't be ashamed to do something ugly \inouter{d}
in front of ordinary people. Is that it?”

At that point Phaedrus interrupted: “Agathon, my friend, if you answer
Socrates, he'll no longer care whether we get anywhere with what we're
doing here, so long as he has a partner for discussion. Especially if
he's handsome. Now, like you, I enjoy listening to Socrates in
discussion, but it is my duty to see to the praising of Love and to
exact a speech from every one of this group. When each of you two has
made his offering to the god, then you can have your discussion.”
\inouter{e}

“You're doing a beautiful job, Phaedrus,” said Agathon. “There's nothing
to keep me from giving my speech. Socrates will have many opportunities
for discussion later.”

\blank[line]

I wish first to speak of how I ought to speak, and only then to speak.
In my opinion, you see, all those who have spoken before me did not so
much celebrate the god as congratulate human beings on the good things
that come to them from the god. But who it is who gave these gifts, what
he is like---no one has spoken about that. Now, only one method is
correct \inouter{195} for every praise, no matter whose: you must
explain what qualities in the subject of your speech enable him to give
the benefits for which we praise him. So now, in the case of Love, it is
right for us to praise him first for what he is and afterwards for his
gifts.

I maintain, then, that while all the gods are happy, Love---if I may say
so without giving offense---is the happiest of them all, for he is the
most beautiful and the best. His great beauty lies in this: First,
Phaedrus, he is \inouter{b} the youngest of the
gods.\endnote{Contrast 178b.} He proves my
point himself by fleeing old age in headlong flight, fast-moving though
it is (that's obvious---it comes after us faster than it should). Love
was born to hate old age and will come nowhere near it. Love always
lives with young people and is one of them: the old story holds good
that like is always drawn to like. And though on many other points I
agree with Phaedrus, I do not agree with this: that \inouter{c} Love is more
ancient than Cronus and Iapetus. No, I say that he is the youngest of
the gods and stays young forever.

Those old stories Hesiod and Parmenides tell about the gods---those
things happened under Necessity, not Love, if what they say is true. For
not one of all those violent deeds would have been done---no
castrations, no imprisonments---if Love had been present among them.
There would have been peace and brotherhood instead, as there has been
now as long as Love has been king of the gods.

\inouter{d} So he is young. And besides being young, he is delicate. It
takes a poet as good as Homer to show how delicate the god is. For Homer
says that Mischief is a god and that she is delicate---well, that her
feet are delicate, anyway! He says:

\blank[line]
\quad{\em \ldots{} hers are delicate feet: not on the ground}\par
\quad{\em Does she draw nigh; she walks instead upon the heads of
men.}\endnote{{\em Iliad} xix.92--93.
“Mischief” translates {\em Atē}.}
\blank[line]

\inouter{e} A lovely proof, I think, to show how delicate she is: she
doesn't walk on anything hard; she walks only on what is soft. We shall
use the same proof about Love, then, to show that he is delicate. For he
walks not on earth, not even on people's skulls, which are not really
soft at all, but in the softest of all the things that are, there he
walks, there he has his home. For he makes his home in the characters,
in the souls, of gods and men---and not even in every soul that comes
along: when he encounters a soul with a harsh character, he turns away;
but when he finds a soft and gentle character, he settles down in it.
Always, then, he is touching with his feet \inouter{196} and with the
whole of himself what is softest in the softest places. He must
therefore be most delicate.

He is youngest, then, and most delicate; in addition he has a fluid,
supple shape. For if he were hard, he would not be able to enfold a soul
completely or escape notice when he first entered it or withdrew.
Besides, his graceful good looks prove that he is balanced and fluid in
his nature. Everyone knows that Love has extraordinary good looks, and
between ugliness and Love there is unceasing war.

And the exquisite coloring of his skin! The way the god consorts with
\inouter{b} flowers shows that. For he never settles in anything, be it a
body or a soul, that cannot flower or has lost its bloom. His place is
wherever it is flowery and fragrant; there he settles, there he stays.

Enough for now about the beauty of the god, though much remains still to
be said. After this, we should speak of Love's moral
character.\endnote{“Moral character”:
{\em aretē}, i.e., virtue.} The main
point is that Love is neither the cause nor the victim of any injustice;
he does no wrong to gods or men, nor they to him. If anything has an
effect on him, it is never by violence, for violence never touches Love.
\inouter{c} And the effects he has on others are not forced, for every
service we give to love we give willingly. And whatever one person
agrees on with another, when both are willing, that is right and just;
so say “the laws that are kings of
society.”\endnote{A proverbial expression
attributed by Aristotle ({\em Rhetoric} 1406a17--23) to the
fourth-century liberal thinker and rhetorician Alcidamas.}

And besides justice, he has the biggest share of
moderation.\endnote{{\em Sōphrosunē}. The
word can be translated also as “temperance” and, most literally,
“sound-mindedness.” (Plato and Aristotle generally contrast
{\em sōphrosunē} as a virtue with self-control: the person with
{\em sōphrosunē} is naturally well-tempered in every way and so does not
need to control himself, or hold himself back.)} For
moderation, by common agreement, is power over pleasures and passions,
and no pleasure is more powerful than Love! But if they are weaker, they
are under the power of Love, and {\em he} has the power; and because he
has power over pleasures and passions, Love is exceptionally moderate.

And as for manly bravery, “Not even Ares can stand up to”
Love!\endnote{From Sophocles, fragment
234b Dindorf: “Even Ares cannot withstand Necessity.” Ares is the god of
war.} For \inouter{d}
Ares has no hold on Love, but Love does on Ares---love of Aphrodite, so
runs the tale.\endnote{See {\em Odyssey}
viii.266--366. Aphrodite's husband Hephaestus made a snare that caught
Ares in bed with Aphrodite.} But he
who has hold is more powerful than he who is held; and so, because Love
has power over the bravest of the others, he is bravest of them all.

Now I have spoken about the god's justice, moderation, and bravery; his
wisdom remains.\endnote{“Wisdom” translates
{\em sophia}, which Agathon treats as roughly equivalent to {\em technē}
(professional skill); he refers mainly to the ability to produce things.
Accordingly “wisdom” translates {\em sophia} in the first instance;
afterwards in this passage it is “skill” or “art.”} I
must try not to leave out anything that can be said on this. In the
first place---to honor {\em our} profession as Eryximachus \inouter{e} did
his\endnote{At 186b.}---the god is so
skilled a poet that he can make others into poets: once Love touches
him, {\em anyone} becomes a poet,

\blank[line]
\quad{\em \ldots{} howe'er uncultured he had been
before.}\endnote{Euripides,
{\em Stheneboea} (frg. 666 Nauck).}
\blank[line]

This, we may fittingly observe, testifies that Love is a good poet,
good, in sum, at every kind of artistic production. For you can't give
to another \inouter{197} what you don't have yourself, and you can't
teach what you don't know.

And as to the production of animals---who will deny that they are all
born and begotten through Love's skill?

And as for artisans and professionals---don't we know that whoever has
this god for a teacher ends up in the light of fame, while a man
untouched by Love ends in obscurity? Apollo, for one, invented archery,
\inouter{b} medicine, and prophecy when desire and love showed the way. Even
he, therefore, would be a pupil of Love, and so would the Muses in
music, Hephaestus in bronze work, Athena in weaving, and Zeus in “the
governance of gods and men.”

That too is how the gods' quarrels were settled, once Love came to be
among them---love of beauty, obviously, because love is not drawn to
ugliness. Before that, as I said in the beginning, and as the poets say,
many dreadful things happened among the gods, because Necessity was
king. \inouter{c} But once this god was born, all goods came to gods and men
alike through love of beauty.

This is how I think of Love, Phaedrus: first, he is himself the most
beautiful and the best; after that, if anyone else is at all like that,
Love is responsible. I am suddenly struck by a need to say something in
poetic meter,\endnote{After these two lines of
poetry, Agathon continues with an extremely poetical prose peroration.} that it
is he who---

\blank[line]
\quad{\em Gives peace to men and stillness to the sea,}\par
\quad{\em Lays winds to rest, and careworn men to sleep.}\inouter{d}
\blank[line]

Love fills us with togetherness and drains all of our divisiveness away.
Love calls gatherings like these together. In feasts, in dances, and in
ceremonies, he gives the lead. Love moves us to mildness, removes from
us wildness. He is giver of kindness, never of meanness. Gracious,
kindly\endnote{Accepting the emendation
{\em aganos} at d5.}---let wise men
see and gods admire! Treasure to lovers, envy to others, father of
elegance, luxury, delicacy, grace, yearning, desire. Love cares \inouter{e}
well for good men, cares not for bad ones. In pain, in fear, in desire,
or speech, Love is our best guide and guard; he is our comrade and our
savior. Ornament of all gods and men, most beautiful leader and the
best! Every man should follow Love, sing beautifully his hymns, and join
with him in the song he sings that charms the mind of god or man.

This, Phaedrus, is the speech I have to offer. Let it be dedicated to
the \inouter{198} god, part of it in fun, part of it moderately
serious, as best I could manage.

When Agathon finished, Aristodemus said, everyone there burst into
applause, so becoming to himself and to the god did they think the young
man's speech.

Then Socrates glanced at Eryximachus and said, “Now do you think I was
foolish to feel the fear I felt before? Didn't I speak like a prophet a
while ago when I said that Agathon would give an amazing speech and I
would be tongue-tied?”

“You were prophetic about one thing, I think,” said Eryximachus, “that
Agathon would speak well. But you, tongue-tied? No, I don't believe
that.” \inouter{b}

“Bless you,” said Socrates. “How am I not going to be tongue-tied, I or
anyone else, after a speech delivered with such beauty and variety? The
other parts may not have been so wonderful, but that at the end! Who
would not be struck dumb on hearing the beauty of the words and phrases?
Anyway, I was worried that I'd not be able to say anything that came
close to them in beauty, and so I would almost have run away and
escaped, \inouter{c} if there had been a place to go. And, you see, the
speech reminded me of Gorgias, so that I actually experienced what Homer
describes: I was afraid that Agathon would end by sending the Gorgian
head,\endnote{“Gorgian head” is a pun
on “Gorgon's head.” In his peroration Agathon had spoken in the style of
Gorgias, and this style was considered to be irresistibly powerful. The
sight of a Gorgon's head would turn a man to stone.}
awesome at
speaking in a speech, against my speech, and this would turn me to stone
by striking me dumb. Then I realized how ridiculous I'd been to agree to
join \inouter{d} with you in praising Love and to say that I was a master of
the art of love, when I knew nothing whatever of this business, of how
anything whatever ought to be praised. In my foolishness, I thought you
should tell the truth about whatever you praise, that this should be
your basis, and that from this a speaker should select the most
beautiful truths and arrange them most suitably. I was quite vain,
thinking that I would talk well and that I knew the truth about praising
anything whatever. But now it appears that this is not what it is to
praise anything whatever; rather, it is to apply \inouter{e} to the object
the grandest and the most beautiful qualities, whether he actually has
them or not. And if they are false, that is no objection; for the
proposal, apparently, was that everyone here make the rest of us think
he is praising Love---and not that he actually praise him. I think that
is why you stir up every word and apply it to Love; your description of
him and \inouter{199} his gifts is designed to make him look better
and more beautiful than anything else---to ignorant listeners, plainly,
for of course he wouldn't look that way to those who knew. And your
praise did seem beautiful and respectful. But I didn't even know the
method for giving praise; and it was in ignorance that I agreed to take
part in this. So “the tongue” promised, and “the mind” did
not.\endnote{The allusion is to
Euripides, {\em Hippolytus} 612.} Goodbye to that!
I'm not giving another eulogy using that method, not at all---I wouldn't
be able to do \inouter{b} it!---but, if you wish, I'd like to tell the truth
my way. I want to avoid any comparison with your speeches, so as not to
give you a reason to laugh at me. So look, Phaedrus, would a speech like
this satisfy your requirement? You will hear the truth about Love, and
the words and phrasing will take care of themselves.”

Then Aristodemus said that Phaedrus and the others urged him to speak in
the way he thought was required, whatever it was.

“Well then, Phaedrus,” said Socrates, “allow me to ask Agathon a few
\inouter{c} little questions, so that, once I have his agreement, I may
speak on that basis.”

“You have my permission,” said Phaedrus. “Ask away.”

After that, said Aristodemus, Socrates began: “Indeed, Agathon, my
friend, I thought you led the way beautifully into your speech when you
said that one should first show the qualities of Love himself, and only
then those of his deeds. I must admire that beginning. Come, then, since
\inouter{d} you have beautifully and magnificently expounded his qualities
in other ways, tell me this, too, about Love. Is Love such as to be a
love of something or of nothing? I'm not asking if he is born {\em of}
some mother or father, (for the question whether Love is love of mother
or of father would really be ridiculous), but it's as if I'm asking this
about a father---whether a father is the father {\em of} something or
not. You'd tell me, of course, if you wanted to give me a good answer,
that it's {\em of} a son or a daughter that a father is the father.
Wouldn't you?”

“Certainly,” said Agathon.

“Then does the same go for the mother?”

\inouter{e} He agreed to that also.

“Well, then,” said Socrates, “answer a little more fully, and you will
understand better what I want. If I should ask, ‘What about this: a
brother, just insofar as he {\em is} a brother, is he the brother of
something or not?' ”

He said that he was.

“And he's of a brother or a sister, isn't he?”

He agreed.

“Now try to tell me about love,” he said. “Is Love the love of nothing
or of something?”

\inouter{200} “Of something, surely!”

“Then keep this object of love in mind, and remember what it
is.\endnote{Cf. 197b.} But tell me this
much: does Love desire that of which it is the love, or not?”

“Certainly,” he said.

“At the time he desires and loves something, does he actually have what
he desires and loves at that time, or doesn't he?”

“He doesn't. At least, that wouldn't be likely,” he said.

“Instead of what's {\em likely},” said Socrates, “ask yourself whether
it's {\em necessary} \inouter{b} that this be so: a thing that desires
desires something of which it is in need; otherwise, if it were not in
need, it would not desire it. I can't tell you, Agathon, how strongly it
strikes me that this is necessary. But how about you?”

“I think so too.”

“Good. Now then, would someone who is tall, want to be tall? Or someone
who is strong want to be strong?”

“Impossible, on the basis of what we've agreed.”

“Presumably because no one is in need of those things he already has.”

“True.”

“But maybe a strong man could want to be strong,” said Socrates, “or a
fast one fast, or a healthy one healthy: in cases like these, you might
\inouter{c} think people really do want to be things they already are and do
want to have qualities they already have---I bring them up so they won't
deceive us. But in these cases, Agathon, if you stop to think about
them, you will see that these people are what they are at the present
time, whether they want to be or not, by a logical necessity. And who,
may I ask, would ever bother to desire what's necessary in any event?
But when someone says ‘I am healthy, but that's just what I want to be,'
or ‘I am rich, but that's just what I want to be,' or ‘I desire the very
things that I have,' let us say \inouter{d} to him: ‘You already have riches
and health and strength in your possession, my man, what you want is to
possess these things in time to come, since in the present, whether you
want to or not, you have them. Whenever you say, {\em I desire what I
already have}, ask yourself whether you don't mean this: {\em I want the
things I have now to be mine in the future as well}.' Wouldn't he
agree?”

According to Aristodemus, Agathon said that he would.

So Socrates said, “Then this is what it is to love something which is
not at hand, which the lover does not have: it is to desire the
preservation of what he now has in time to come, so that he will have it
then.” \inouter{e}

“Quite so,” he said.

“So such a man or anyone else who has a desire desires what is not at
hand and not present, what he does not have, and what he is not, and
that of which he is in need; for such are the objects of desire and
love.”

“Certainly,” he said.

“Come, then,” said Socrates. “Let us review the points on which we've
agreed. Aren't they, first, that Love is the love of something, and,
second, that he loves things of which he has a present need?”
\inouter{201}

“Yes,” he said.

“Now, remember, in addition to these points, what you said in your
speech about what it is that Love loves. If you like, I'll remind you. I
think you said something like this: that the gods' quarrels were settled
by love of beautiful things, for there is no love of ugly
ones.\endnote{197b3--5.} Didn't you say
something like that?”

“I did,” said Agathon.

“And that's a suitable thing to say, my friend,” said Socrates. “But if
this is so, wouldn't Love have to be a desire for beauty, and never for
ugliness?”

He agreed. \inouter{b}

“And we also agreed that he loves just what he needs and does not have.”

“Yes,” he said.

“So Love needs beauty, then, and does not have it.”

“Necessarily,” he said.

“So! If something needs beauty and has got no beauty at all, would you
still say that it is beautiful?”

“Certainly not.”

“Then do you still agree that Love is beautiful, if those things are
so?”

\inouter{c} Then Agathon said, “It turns out, Socrates, I didn't know what I
was talking about in that speech.”

“It was a beautiful speech, anyway, Agathon,” said Socrates. “Now take
it a little further. Don't you think that good things are always
beautiful as well?”

“I do.”

“Then if Love needs beautiful things, and if all good things are
beautiful, he will need good things too.”

“As for me, Socrates,” he said, “I am unable to contradict you. Let it
be as you say.”

“Then it's the truth, my beloved Agathon, that you are unable to
contradict,” he said. “It is not hard at all to contradict
Socrates.”

\blank[line]

\inouter{d} Now I'll let you go. I shall try to go through for you the
speech about Love I once heard from a woman of Mantinea, Diotima---a
woman who was wise about many things besides this: once she even put off
the plague for ten years by telling the Athenians what sacrifices to
make. She is the one who taught me the art of love, and I shall go
through her speech as best I can on my own, using what Agathon and I
have agreed to as a basis.

Following your lead, Agathon, one should first describe who Love is
\inouter{e} and what he is like, and afterwards describe his works---I think
it will be easiest for me to proceed the way Diotima did and tell you
how she questioned me.

You see, I had told her almost the same things Agathon told me just now:
that Love is a great god and that he belongs to beautiful
things.\endnote{The Greek is ambiguous
between “Love loves beautiful things” and “Love is one of the beautiful
things.” Agathon had asserted the former (197b5, 201a5), and this will
be a premise in Diotima's argument, but he asserted the latter as well
(195a7), and this is what Diotima proceeds to refute.} And she used
the very same arguments against me that I used against Agathon; she
showed how, according to my very own speech, Love is neither beautiful
nor good.

So I said, “What do you mean, Diotima? Is Love ugly, then, and bad?”

\inouter{202} But she said, “Watch your tongue! Do you really think
that, if a thing is not beautiful, it has to be ugly?”

“I certainly do.”

“And if a thing's not wise, it's ignorant? Or haven't you found out yet
that there's something in between wisdom and ignorance?”

“What's that?”

“It's judging things correctly without being able to give a reason.
Surely you see that this is not the same as knowing---for how could
knowledge be unreasoning? And it's not ignorance either---for how could
what hits the truth be ignorance? Correct judgment, of course, has this
character: it is {\em in between} understanding and ignorance.”

“True,” said I, “as you say.” \inouter{b}

“Then don't force whatever is not beautiful to be ugly, or whatever is
not good to be bad. It's the same with Love: when you agree he is
neither good nor beautiful, you need not think he is ugly and bad; he
could be something in between,” she said.

“Yet everyone agrees he's a great god,” I said.

“Only those who don't know?” she said. “Is that how you mean ‘everyone'?
Or do you include those who do know?”

“Oh, everyone together.”

And she laughed. “Socrates, how could those who say that he's not a
\inouter{c} god at all agree that he's a great god?”

“Who says that?” I asked.

“You, for one,” she said, “and I for another.”

“How can you say this!” I exclaimed.

“That's easy,” said she. “Tell me, wouldn't you say that all gods are
beautiful and happy? Surely you'd never say a god is not beautiful or
happy?”

“Zeus! Not I,” I said.

“Well, by calling anyone ‘happy,' don't you mean they possess good and
beautiful things?”

“Certainly.” \inouter{d}

“What about Love? You agreed he needs good and beautiful things, and
that's why he desires them---because he needs them.”

“I certainly did.”

“Then how could he be a god if he has no share in good and beautiful
things?”

“There's no way he could, apparently.”

“Now do you see? You don't believe Love is a god either!”

“Then, what could Love be?” I asked. “A mortal?”

“Certainly not.”

“Then, what is he?”

“He's like what we mentioned before,” she said. “He is in between mortal
and immortal.”

“What do you mean, Diotima?”

“He's a great spirit, Socrates. Everything spiritual, you see, is in
between \inouter{e} god and mortal.”

“What is their function?” I asked.

“They are messengers who shuttle back and forth between the two,
conveying prayer and sacrifice from men to gods, while to men they bring
commands from the gods and gifts in return for sacrifices. Being in the
middle of the two, they round out the whole and bind fast the all to
all. \inouter{203} Through them all divination passes, through them
the art of priests in sacrifice and ritual, in enchantment, prophecy,
and sorcery. Gods do not mix with men; they mingle and converse with us
through spirits instead, whether we are awake or asleep. He who is wise
in any of these ways is a man of the spirit, but he who is wise in any
other way, in a profession or any manual work, is merely a mechanic.
These spirits are many and various, then, and one of them is Love.”

\inouter{b} “Who are his father and mother?” I asked.

“That's rather a long story,” she said. “I'll tell it to you, all the
same.”

“When Aphrodite was born, the gods held a celebration. Poros, the son of
Metis, was there among
them.\endnote{{\em Poros} means “way,”
“resource.” His mother's name, {\em Mētis}, means “cunning.” {\em Penia}
means “poverty.”} When they had
feasted, Penia came begging, as poverty does when there's a party, and
stayed by the gates. Now Poros got drunk on nectar (there was no wine
yet, you see) and, feeling drowsy, went into the garden of Zeus, where
he fell asleep. Then \inouter{c} Penia schemed up a plan to relieve her lack
of resources: she would get a child from Poros. So she lay beside him
and got pregnant with Love. That is why Love was born to follow
Aphrodite and serve her: because he was conceived on the day of her
birth. And that's why he is also by nature a lover of beauty, because
Aphrodite herself is especially beautiful.

“As the son of Poros and Penia, his lot in life is set to be like
theirs. In the first place, he is always poor, and he's far from being
delicate and \inouter{d} beautiful (as ordinary people think he is);
instead, he is tough and shriveled and shoeless and homeless, always
lying on the dirt without a bed, sleeping at people's doorsteps and in
roadsides under the sky, having his mother's nature, always living with
Need. But on his father's side he is a schemer after the beautiful and
the good; he is brave, impetuous, and intense, an awesome hunter, always
weaving snares, resourceful in his pursuit of intelligence, a lover of
wisdom\endnote{I.e., a philosopher.} through all
his life, a genius with enchantments, potions, and clever pleadings.

\inouter{e} “He is by nature neither immortal nor mortal. But now he springs
to life when he gets his way; now he dies---all in the very same day.
Because he is his father's son, however, he keeps coming back to life,
but then anything he finds his way to always slips away, and for this
reason Love is never completely without resources, nor is he ever rich.

\inouter{204} “He is in between wisdom and ignorance as well. In fact,
you see, none of the gods loves wisdom or wants to become wise---for
they are wise---and no one else who is wise already loves wisdom; on the
other hand, no one who is ignorant will love wisdom either or want to
become wise. For what's especially difficult about being ignorant is
that you are content with yourself, even though you're neither beautiful
and good nor intelligent. If you don't think you need anything, of
course you won't want what you don't think you need.”

“In that case, Diotima, who {\em are} the people who love wisdom, if
they are \inouter{b} neither wise nor ignorant?”

“That's obvious,” she said. “A child could tell you. Those who love
wisdom fall in between those two extremes. And Love is one of them,
because he is in love with what is beautiful, and wisdom is extremely
beautiful. It follows that Love {\em must} be a lover of wisdom and, as
such, is in between being wise and being ignorant. This, too, comes to
him from his parentage, from a father who is wise and resourceful and a
mother who is not wise and lacks resource.

“My dear Socrates, that, then, is the nature of the Spirit called Love.
\inouter{c} Considering what you thought about Love, it's no surprise that
you were led into thinking of Love as you did. On the basis of what you
say, I conclude that you thought Love was {\em being loved}, rather than
{\em being a lover}. I think that's why Love struck you as beautiful in
every way: because it is what is really beautiful and graceful that
deserves to be loved, and this is perfect and highly blessed; but being
a lover takes a different form, which I have just described.”

So I said, “All right then, my friend. What you say about Love is
beautiful, but if you're right, what use is Love to human beings?”
\inouter{d}

“I'll try to teach you that, Socrates, after I finish this. So far I've
been explaining the character and the parentage of Love. Now, according
to you, he is love for beautiful things. But suppose someone asks us,
‘Socrates and Diotima, what is the point of loving beautiful things?'

“It's clearer this way: ‘The lover of beautiful things has a desire;
what does he desire?' ”

“That they become his own,” I said.

“But that answer calls for still another question, that is, ‘What will
this man have, when the beautiful things he wants have become his own?'
”

I said there was no way I could give a ready answer to that question.
\inouter{e}

Then she said, “Suppose someone changes the question, putting ‘good' in
place of ‘beautiful,' and asks you this: ‘Tell me, Socrates, a lover of
good things has a desire; what does he desire?' ”

“That they become his own,” I said.

“And what will he have, when the good things he wants have become his
own?”

“This time it's easier to come up with the answer,” I said. “He'll have
happiness.
\inouter{205}”\endnote{{\em Eudaimonia}: no
English word catches the full range of this term, which is used for the
whole of well-being and the good, flourishing life.}

“That's what makes happy people happy, isn't it---possessing good
things. There's no need to ask further, ‘What's the point of wanting
happiness?' The answer you gave seems to be final.”

“True,” I said.

“Now this desire for happiness, this kind of love---do you think it is
common to all human beings and that everyone wants to have good things
forever and ever? What would you say?”

“Just that,” I said. “It is common to all.”

\inouter{b} “Then, Socrates, why don't we say that everyone is in love,” she
asked, “since everyone always loves the same things? Instead, we say
some people are in love and others not; why is that?”

“I wonder about that myself,” I said.

“It's nothing to wonder about,” she said. “It's because we divide out a
special kind of love, and we refer to it by the word that means the
whole---‘love'; and for the other kinds of love we use other words.”

“What do you mean?” I asked.

“Well, you know, for example, that ‘poetry' has a very wide
range.\endnote{“Poetry” translates
{\em poiēsis}, lit. ‘making', which can be used for any kind of
production or creation. However, the word {\em poiētēs}, lit. ‘maker',
was used mainly for poets---writers of metrical verses that were
actually set to music.} After all,
everything that is responsible for creating something out of \inouter{c}
nothing is a kind of poetry; and so all the creations of every craft and
profession are themselves a kind of poetry, and everyone who practices a
craft is a poet.”

“True.”

“Nevertheless,” she said, “as you also know, these craftsmen are not
called poets. We have other words for them, and out of the whole of
poetry we have marked off one part, the part the Muses give us with
melody and rhythm, and we refer to this by the word that means the
whole. For this alone is called ‘poetry,' and those who practice this
part of poetry are called poets.”

\inouter{d} “True.”

“That's also how it is with love. The main point is this: every desire
for good things or for happiness is ‘the supreme and treacherous love'
in everyone. But those who pursue this along any of its many other
ways---through making money, or through the love of sports, or through
philosophy---we don't say that {\em these} people are in love, and we
don't call them lovers. It's only when people are devoted exclusively to
one special kind of love that we use these words that really belong to
the whole of it: ‘love' and ‘in love' and ‘lovers.' ”

“I am beginning to see your point,” I said.

\inouter{e} “Now there is a certain story,” she said, “according to which
lovers are those people who seek their other halves. But according to my
story, a lover does not seek the half or the whole, unless, my friend,
it turns out to be good as well. I say this because people are even
willing to cut off their own arms and legs if they think they are
diseased. I don't think an individual takes joy in what belongs to him
personally unless by ‘belonging to me' he means ‘good' and by ‘belonging
to another' he means ‘bad.' That's because what everyone loves is really
nothing other than the good. \inouter{206} Do you disagree?”

“Zeus! Not I,” I said.

“Now, then,” she said. “Can we simply say that people love the good?”

“Yes,” I said.

“But shouldn't we add that, in loving it, they want the good to be
theirs?”

“We should.”

“And not only that,” she said. “They want the good to be theirs forever,
don't they?”

“We should add that too.”

“In a word, then, love is wanting to possess the good forever.”

“That's very true,” I said. \inouter{b}

“This, then, is the object of
love,”\endnote{Accepting the emendation
{\em toutou} in b1.} she said.
“Now, how do lovers pursue it? We'd rightly say that when they are in
love they do something with eagerness and zeal. But what is it precisely
that they do? Can you say?”

“If I could,” I said, “I wouldn't be your student, filled with
admiration for your wisdom, and trying to learn these very things.”

“Well, I'll tell you,” she said. “It is giving birth in
beauty,\endnote{The preposition is
ambiguous between “within” and “in the presence of.” Diotima may mean
that the lover causes the newborn (which may be an idea) to come to be
within a beautiful person; or she may mean that he is stimulated to give
birth to it in the presence of a beautiful person.} whether in
body or in soul.”

“It would take divination to figure out what you mean. I can't.” \inouter{c}

“Well, I'll tell you more clearly,” she said. “All of us are pregnant,
Socrates, both in body and in soul, and, as soon as we come to a certain
age, we naturally desire to give birth. Now no one can possibly give
birth in anything ugly; only in something beautiful. That's because when
a man and a woman come together in order to give birth, this is a godly
affair. Pregnancy, reproduction---this is an immortal thing for a mortal
animal to do, and it cannot occur in anything that is out of harmony,
but ugliness \inouter{d} is out of harmony with all that is godly. Beauty,
however, is in harmony with the divine. Therefore the goddess who
presides at childbirth---she's called Moira or Eilithuia---is really
Beauty.\endnote{Moira is known mainly as
a Fate, but she was also a birth goddess ({\em Iliad} xxiv.209), and was
identified with the birth-goddess Eilithuia (Pindar, {\em Olympian Odes}
vi.42, {\em Nemean} {\em Odes} vii.1).} That's why,
whenever pregnant animals or persons draw near to beauty, they become
gentle and joyfully disposed and give birth and reproduce; but near
ugliness they are foulfaced and draw back in pain; they turn away and
shrink back and do not reproduce, and because they hold on to what they
carry inside them, the labor is painful. This is the source of the great
excitement about beauty \inouter{e} that comes to anyone who is pregnant and
already teeming with life: beauty releases them from their great pain.
You see, Socrates,” she said, “what Love wants is not beauty, as you
think it is.”

“Well, what is it, then?”

“Reproduction and birth in beauty.”

“Maybe,” I said.

“Certainly,” she said. “Now, why reproduction? It's because reproduction
\inouter{207} goes on forever; it is what mortals have in place of
immortality. A lover must desire immortality along with the good, if
what we agreed earlier was right, that Love wants to possess the good
forever. It follows from our argument that Love must desire
immortality.”

All this she taught me, on those occasions when she spoke on the art of
love. And once she asked, “What do you think causes love and desire,
Socrates? Don't you see what an awful state a wild animal is in when it
\inouter{b} wants to reproduce? Footed and winged animals alike, all are
plagued by the disease of Love. First they are sick for intercourse with
each other, then for nurturing their young---for their sake the weakest
animals stand ready to do battle against the strongest and even to die
for them, and they may be racked with famine in order to feed their
young. They would do anything for their sake. Human beings, you'd think,
would do this because \inouter{c} they understand the reason for it; but
what causes wild animals to be in such a state of love? Can you say?”

And I said again that I didn't know.

So she said, “How do you think you'll ever master the art of love, if
you don't know that?”

“But that's why I came to you, Diotima, as I just said. I knew I needed
a teacher. So tell me what causes this, and everything else that belongs
to the art of love.”

“If you really believe that Love by its nature aims at what we have
often \inouter{d} agreed it does, then don't be surprised at the answer,”
she said. “For among animals the principle is the same as with us, and
mortal nature seeks so far as possible to live forever and be immortal.
And this is possible in one way only: by reproduction, because it always
leaves behind a new young one in place of the old. Even while each
living thing is said to be alive and to be the same---as a person is
said to be the same from childhood till he turns into an old man---even
then he never consists of the same things, though he is called the same,
but he is always being renewed and \inouter{e} in other respects passing
away, in his hair and flesh and bones and blood and his entire body. And
it's not just in his body, but in his soul, too, for none of his
manners, customs, opinions, desires, pleasures, pains, or fears ever
remains the same, but some are coming to be in him while others are
passing away. And what is still far stranger than that is that not only
\inouter{208} does one branch of knowledge come to be in us while
another passes away and that we are never the same even in respect of
our knowledge, but that each single piece of knowledge has the same
fate. For what we call {\em studying} exists because knowledge is
leaving us, because forgetting is the departure of knowledge, while
studying puts back a fresh memory in place of what went away, thereby
preserving a piece of knowledge, so that it seems to be the same. And in
that way everything mortal is preserved, not, like the divine, by always
being the same in every way, but because \inouter{b} what is departing and
aging leaves behind something new, something such as it had been. By
this device, Socrates,” she said, “what is mortal shares in immortality,
whether it is a body or anything else, while the immortal has another
way. So don't be surprised if everything naturally values its own
offspring, because it is for the sake of immortality that everything
shows this zeal, which is Love.”

Yet when I heard her speech I was amazed, and spoke: “Well,” said I,
\inouter{c} “Most wise Diotima, is this really the way it is?”

And in the manner of a perfect sophist she said, “Be sure of it,
Socrates. Look, if you will, at how human beings seek honor. You'd be
amazed at their irrationality, if you didn't have in mind what I spoke
about and if you hadn't pondered the awful state of love they're in,
wanting to become famous and ‘to lay up glory immortal forever,' and how
they're ready to brave any danger for the sake of this, much more than
they are for their children; and they are prepared to spend money,
suffer through all sorts of ordeals, and even die for the sake of glory.
Do you really think that \inouter{d} Alcestis would have died for Admetus,”
she asked, “or that Achilles would have died after Patroclus, or that
your Codrus would have died so as to preserve the throne for his
sons,\endnote{Codrus was the legendary
last king of Athens. He gave his life to satisfy a prophecy that
promised victory to Athens and salvation from the invading Dorians if
their king was killed by the enemy.} if they hadn't
expected the memory of their virtue---which we still hold in honor---to
be immortal? Far from it,” she said. “I believe that anyone will do
anything for the sake of immortal virtue and the glorious fame that
follows; and the better the people, the \inouter{e} more they will do, for
they are all in love with immortality.

“Now, some people are pregnant in body, and for this reason turn more to
women and pursue love in that way, providing themselves through
childbirth with immortality and remembrance and happiness, as they
think, for all time to come; while others are pregnant in soul---because
there \inouter{209} surely {\em are} those who are even more pregnant
in their souls than in their bodies, and these are pregnant with what is
fitting for a soul to bear and bring to birth. And what is fitting?
Wisdom and the rest of virtue, which all poets beget, as well as all the
craftsmen who are said to be creative. But by far the greatest and most
beautiful part of wisdom deals with the proper ordering of cities and
households, and that is called moderation and justice. When someone has
been pregnant with these in his soul from \inouter{b} early youth, while he
is still a virgin, and, having arrived at the proper age, desires to
beget and give birth, he too will certainly go about seeking the beauty
in which he would beget; for he will never beget in anything ugly. Since
he is pregnant, then, he is much more drawn to bodies that are beautiful
than to those that are ugly; and if he also has the luck to find a soul
that is beautiful and noble and well-formed, he is even more drawn
\inouter{c} to this combination; such a man makes him instantly teem with
ideas and arguments about virtue---the qualities a virtuous man should
have and the customary activities in which he should engage; and so he
tries to educate him. In my view, you see, when he makes contact with
someone beautiful and keeps company with him, he conceives and gives
birth to what he has been carrying inside him for ages. And whether they
are together or apart, he remembers that beauty. And in common with him
he nurtures the newborn; such people, therefore, have much more to share
than do the parents of human children, and have a firmer bond of
friendship, because the children in whom they have a share are more
\inouter{d} beautiful and more immortal. Everyone would rather have such
children than human ones, and would look up to Homer, Hesiod, and the
other good poets with envy and admiration for the offspring they have
left behind---offspring, which, because they are immortal themselves,
provide their parents with immortal glory and remembrance. For example,”
she said, “those are the sort of children
Lycurgus\endnote{Lycurgus was supposed to
have been the founder of the oligarchic laws and stern customs of
Sparta.} left behind
in Sparta as the saviors of Sparta and virtually all of Greece. Among
you the honor goes \inouter{e} to Solon for his creation of your laws. Other
men in other places everywhere, Greek or barbarian, have brought a host
of beautiful deeds into the light and begotten every kind of virtue.
Already many shrines have sprung up to honor them for their immortal
children, which hasn't happened yet to anyone for human offspring.

\inouter{210} “Even you, Socrates, could probably come to be initiated
into these rites of love. But as for the purpose of these rites when
they are done correctly---that is the final and highest mystery, and I
don't know if you are capable of it. I myself will tell you,” she said,
“and I won't stint any effort. And you must try to follow if you can.

“A lover who goes about this matter correctly must begin in his youth to
devote himself to beautiful bodies. First, if the
leader\endnote{The leader: Love.} leads aright,
he should love one body and beget beautiful ideas there; then he should
\inouter{b} realize that the beauty of any one body is brother to the beauty
of any other and that if he is to pursue beauty of form he'd be very
foolish not to think that the beauty of all bodies is one and the same.
When he grasps this, he must become a lover of all beautiful bodies, and
he must think that this wild gaping after just one body is a small thing
and despise it.

“After this he must think that the beauty of people's souls is more
valuable than the beauty of their bodies, so that if someone is decent
in \inouter{c} his soul, even though he is scarcely blooming in his body,
our lover must be content to love and care for him and to seek to give
birth to such ideas as will make young men better. The result is that
our lover will be forced to gaze at the beauty of activities and laws
and to see that all this is akin to itself, with the result that he will
think that the beauty of bodies is a thing of no importance. After
customs he must move on to various kinds of knowledge. The result is
that he will see the beauty of knowledge and \inouter{d} be looking mainly
not at beauty in a single example---as a servant would who favored the
beauty of a little boy or a man or a single custom (being a slave, of
course, he's low and small-minded)---but the lover is turned to the
great sea of beauty, and, gazing upon this, he gives birth to many
gloriously beautiful ideas and theories, in unstinting love of
wisdom,\endnote{The leader: Love.} until, having
grown and been strengthened there, he catches sight of such \inouter{e}
knowledge, and it is the knowledge of such beauty \ldots{}

“Try to pay attention to me,” she said, “as best you can. You see, the
man who has been thus far guided in matters of Love, who has beheld
beautiful things in the right order and correctly, is coming now to the
goal of Loving: all of a sudden he will catch sight of something
wonderfully beautiful in its nature; that, Socrates, is the reason for
all his earlier labors: \inouter{211}

“First, it always {\em is} and neither comes to be nor passes away,
neither waxes nor wanes. Second, it is not beautiful this way and ugly
that way, nor beautiful at one time and ugly at another, nor beautiful
in relation to one thing and ugly in relation to another; nor is it
beautiful here but ugly there, as it would be if it were beautiful for
some people and ugly for others. Nor will the beautiful appear to him in
the guise of a face or hands or anything else that belongs to the body.
It will not appear to him as one idea or one kind of knowledge. It is
not anywhere in another thing, as in \inouter{b} an animal, or in earth, or
in heaven, or in anything else, but itself by itself with itself, it is
always one in form; and all the other beautiful things share in that, in
such a way that when those others come to be or pass away, this does not
become the least bit smaller or greater nor suffer any change. So when
someone rises by these stages, through loving boys correctly, and begins
to see this beauty, he has almost grasped his goal. This is what it
\inouter{c} is to go aright, or be led by another, into the mystery of Love:
one goes always upwards for the sake of this Beauty, starting out from
beautiful things and using them like rising stairs: from one body to two
and from two to all beautiful bodies, then from beautiful bodies to
beautiful customs, and from customs to learning beautiful things, and
from these lessons he
arrives\endnote{Reading {\em teleutēsēi}
at c7.} in the end at
this lesson, which is learning of this very Beauty, so that in the end
he comes to know just what it is to be beautiful. \inouter{d}

“And there in life, Socrates, my friend,” said the woman from Mantinea,
“there if anywhere should a person live his life, beholding that Beauty.
If you once see that, it won't occur to you to measure beauty by gold or
clothing or beautiful boys and youths---who, if you see them now, strike
you out of your senses, and make you, you and many others, eager to be
with the boys you love and look at them forever, if there were any way
to do that, forgetting food and drink, everything but looking at them
and \inouter{e} being with them. But how would it be, in our view,” she
said, “if someone got to see the Beautiful itself, absolute, pure,
unmixed, not polluted by human flesh or colors or any other great
nonsense of mortality, but if he \inouter{212} could see the divine
Beauty itself in its one form? Do you think it would be a poor life for
a human being to look there and to behold it by that which he ought, and
to be with it? Or haven't you remembered,” she said, “that in that life
alone, when he looks at Beauty in the only way that Beauty can be
seen---only then will it become possible for him to give birth not to
images of virtue (because he's in touch with no images), but to true
virtue (because he is in touch with the true Beauty). The love of the
gods belongs to anyone who has given birth to true virtue and nourished
it, \inouter{b} and if any human being could become immortal, it would be
he.”

This, Phaedrus and the rest of you, was what Diotima told me. I was
persuaded. And once persuaded, I try to persuade others too that human
nature can find no better workmate for acquiring this than Love. That's
why I say that every man must honor Love, why I honor the rites of Love
myself and practice them with special diligence, and why I commend them
to others. Now and always I praise the power and courage of Love so far
\inouter{c} as I am able. Consider this speech, then, Phaedrus, if you wish,
a speech in praise of Love. Or if not, call it whatever and however you
please to call it.

\blank[line]

Socrates' speech finished to loud applause. Meanwhile, Aristophanes was
trying to make himself heard over their cheers in order to make a
response to something Socrates had said about his own
speech.\endnote{Cf. 205d--e.} Then, all of
a sudden, there was even more noise. A large drunken party had arrived
at the courtyard door and they were rattling it loudly, accompanied by
the shrieks of some flute-girl they had brought along. Agathon at that
point called to his slaves:

\inouter{d} “Go see who it is. If it's people we know, invite them in. If
not, tell them the party's over, and we're about to turn in.”

A moment later they heard Alcibiades shouting in the courtyard, very
drunk and very loud. He wanted to know where Agathon was, he demanded to
see Agathon at once. Actually, he was half-carried into the \inouter{e}
house by the flute-girl and by some other companions of his, but, at the
door, he managed to stand by himself, crowned with a beautiful wreath of
violets and ivy and ribbons in his hair.

“Good evening, gentlemen. I'm plastered,” he announced. “May I join your
party? Or should I crown Agathon with this wreath---which is all I came
to do, anyway---and make myself scarce? I really couldn't make it
yesterday,” he continued, “but nothing could stop me tonight! See, I'm
wearing the garland myself. I want this crown to come directly from my
head to the head that belongs, I don't mind saying, to the cleverest and
best looking man in town. Ah, you laugh; you think I'm drunk! Fine, go
\inouter{213} ahead---I know I'm right anyway. Well, what do you say?
May I join you on these terms? Will you have a drink with me or not?”

Naturally they all made a big fuss. They implored him to join them, they
begged him to take a seat, and Agathon called him to his side. So
Alcibiades, again with the help of his friends, approached Agathon. At
the same time, he kept trying to take his ribbons off so that he could
crown Agathon with them, but all he succeeded in doing was to push them
further down his head until they finally slipped over his eyes. What
with the ivy and all, he didn't see Socrates, who had made room for him
on the couch as soon as he saw him. So Alcibiades sat down between
Socrates \inouter{b} and Agathon and, as soon as he did so, he put his arms
around Agathon, kissed him, and placed the ribbons on his head.

Agathon asked his slaves to take Alcibiades' sandals off. “We can all
three fit on my couch,” he said.

“What a good idea!” Alcibiades replied. “But wait a moment! Who's the
third?”

As he said this, he turned around, and it was only then that he saw
Socrates. No sooner had he seen him than he leaped up and cried:

“Good lord, what's going on here? It's Socrates! You've trapped me
\inouter{c} again! You always do this to me---all of a sudden you'll turn up
out of nowhere where I least expect you! Well, what do you want now? Why
did you choose this particular couch? Why aren't you with Aristophanes
or anyone else we could tease you about? But no, you figured out a way
to find a place next to the most handsome man in the room!”

“I beg you, Agathon,” Socrates said, “protect me from this man! You
\inouter{d} can't imagine what it's like to be in love with him: from the
very first moment he realized how I felt about him, he hasn't allowed me
to say two words to anybody else---what am I saying, I can't so much as
look at an attractive man but he flies into a fit of jealous rage. He
yells; he threatens; he can hardly keep from slapping me around! Please,
try to keep him under control. Could you perhaps make him forgive me?
And if you can't, if he gets violent, will you defend me? The fierceness
of his passion terrifies me!”

“I shall never forgive you!” Alcibiades cried. “I promise you, you'll
pay \inouter{e} for this! But for the moment,” he said, turning to Agathon,
“give me some of these ribbons. I'd better make a wreath for him as
well---look at that magnificent head! Otherwise, I know, he'll make a
scene. He'll be grumbling that, though I crowned you for your first
victory, I didn't honor him even though he has never lost an argument in
his life.”

So Alcibiades took the ribbons, arranged them on Socrates' head, and lay
back on the couch. Immediately, however, he started up again:

“Friends, you look sober to me; we can't have that! Let's have a drink!
Remember our agreement? We need a master of ceremonies; who should it
be? \ldots{} Well, at least till you are all too drunk to care, I elect
\ldots{} myself! Who else? Agathon, I want the largest cup around
\ldots{} No! Wait! You! \inouter{214} Bring me that cooling jar over
there!”

He'd seen the cooling jar, and he realized it could hold more than two
quarts of wine. He had the slaves fill it to the brim, drained it, and
ordered them to fill it up again for Socrates.

“Not that the trick will have any effect on {\em him},” he told the
group. “Socrates will drink whatever you put in front of him, but no one
yet has seen him drunk.”

The slave filled the jar and, while Socrates was drinking, Eryximachus
said to Alcibiades:

\inouter{b} “This is certainly most improper. We cannot simply pour the wine
down our throats in silence: we must have some conversation, or at least
a song. What we are doing now is hardly civilized.”

What Alcibiades said to him was this:

“O Eryximachus, best possible son to the best possible, the most
temperate father: Hi!”

“Greetings to you, too,” Eryximachus replied. “Now what do you suggest
we do?”

“Whatever you say. Ours to obey you, ‘For a medical mind is worth a
million others'.\endnote{{\em Iliad} xi.514.}
Please prescribe what you think fit.”

\inouter{c} “Listen to me,” Eryximachus said. “Earlier this evening we
decided to use this occasion to offer a series of encomia of Love. We
all took our turn---in good order, from left to right---and gave our
speeches, each according to his ability. You are the only one not to
have spoken yet, though, if I may say so, you have certainly drunk your
share. It's only proper, therefore, that you take your turn now. After
you have spoken, you can decide on a topic for Socrates on your right;
he can then do the same for the man to his right, and we can go around
the table once again.”

“Well said, O Eryximachus,” Alcibiades replied. “But do you really think
it's fair to put my drunken ramblings next to your sober orations? And
\inouter{d} anyway, my dear fellow, I hope you didn't believe a single word
Socrates said: the truth is just the opposite! He's the one who will
most surely beat me up if I dare praise anyone else in his
presence---even a god!”

“Hold your tongue!” Socrates said.

“By god, don't you dare deny it!” Alcibiades shouted. “I would
never---{\em never}---praise anyone else with you around.”

\inouter{e} “Well, why not just do that, if you want?” Eryximachus
suggested. “Why don't you offer an encomium to Socrates?”

“What do you mean?” asked Alcibiades. “Do you really think so,
Eryximachus? Should I unleash myself upon him? Should I give him his
punishment in front of all of you?”

“Now, wait a minute,” Socrates said. “What do you have in mind? Are you
going to praise me only in order to mock me? Is that it?”

“I'll only tell the truth---please, let me!”

“I would certainly like to hear the truth from you. By all means, go
ahead,” Socrates replied.

“Nothing can stop me now,” said Alcibiades. “But here's what you can do:
if I say anything that's not true, you can just interrupt, if you want,
and correct me; at worst, there'll be mistakes in my speech, not lies.
But \inouter{215} you can't hold it against me if I don't get
everything in the right order---I'll say things as they come to mind. It
is no easy task for one in my condition to give a smooth and orderly
account of your bizarreness!”

\blank[line]

I'll try to praise Socrates, my friends, but I'll have to use an image.
And though he may think I'm trying to make fun of him, I assure you my
image is no joke: it aims at the truth. Look at him! Isn't he just like
a statue \inouter{b} of Silenus? You know the kind of statue I mean; you'll
find them in any shop in town. It's a Silenus sitting, his
flute\endnote{This is the conventional
translation of the word, but the {\em aulos} was in fact a reed
instrument and not a flute. It was held by the ancients to be the
instrument that most strongly arouses the emotions.} or his pipes in
his hands, and it's hollow. It's split right down the middle, and inside
it's full of tiny statues of the gods. Now look at him again! Isn't he
also just like the satyr
Marsyas?\endnote{Satyrs had the sexual
appetites and manners of wild beasts and were usually portrayed with
large erections. Sometimes they had horses' tails or ears, sometimes the
traits of goats. Marsyas, in myth, dared to compete in music with Apollo
and was skinned alive for his impudence.}

Nobody, not even you, Socrates, can deny that you {\em look} like them.
But the resemblance goes beyond appearance, as you're about to hear.

You are impudent, contemptuous, and vile! No? If you won't admit it,
I'll bring witnesses. And you're quite a fluteplayer, aren't you? In
fact, you're much more marvelous than Marsyas, who needed instruments to
\inouter{c} cast his spells on people. And so does anyone who plays his
tunes today---for even the tunes
Olympus\endnote{Olympus was a legendary
musician who was said to be loved by Marsyas ({\em Minos} 318b5) and to
have made music that moved its listeners out of their senses.} played are
Marsyas' work, since Olympus learned everything from him. Whether they
are played by the greatest flautist or the meanest flute-girl, his
melodies have in themselves the power to possess and so reveal those
people who are ready for the god and his mysteries. That's because his
melodies are themselves divine. The only difference between you and
Marsyas is that you need no instruments; you do exactly what he does,
but with words alone. You know, people hardly \inouter{d} ever take a
speaker seriously, even if he's the greatest orator; but let
anyone---man, woman, or child---listen to you or even to a poor account
of what you say---and we are all transported, completely possessed.

If I were to describe for you what an extraordinary effect his words
have always had on me (I can feel it this moment even as I'm speaking),
\inouter{e} you might actually suspect that I'm drunk! Still, I swear to
you, the moment he starts to speak, I am beside myself: my heart starts
leaping in my chest, the tears come streaming down my face, even the
frenzied Corybantes\endnote{Legendary worshippers of
Cybele, who brought about their own derangement through music and dance.}
seem sane compared to me---and, let me tell you, I am not alone. I have
heard Pericles and many other great orators, and I have admired their
speeches. But nothing like this ever happened to me: they never upset me
so deeply that my very own soul started protesting that my
life---{\em my} life!---was no better than the most miserable slave's.
And yet that is exactly how \inouter{216} this Marsyas here at my side
makes me feel all the time: he makes it seem that my life isn't worth
living! You can't say that isn't true, Socrates. I know very well that
you could make me feel that way this very moment if I gave you half a
chance. He always traps me, you see, and he makes me admit that my
political career is a waste of time, while all that matters is just what
I most neglect: my personal shortcomings, which cry out for the closest
attention. So I refuse to listen to him; I stop my ears and tear \inouter{b}
myself away from him, for, like the Sirens, he could make me stay by his
side till I die.

Socrates is the only man in the world who has made me feel shame---ah,
you didn't think I had it in me, did you? Yes, he makes me feel ashamed:
I know perfectly well that I can't prove he's wrong when he tells me
what I should do; yet, the moment I leave his side, I go back to my old
ways: I cave in to my desire to please the crowd. My whole life has
become one constant effort to escape from him and keep away, but when I
see him, I \inouter{c} feel deeply ashamed, because I'm doing nothing about
my way of life, though I have already agreed with him that I should.
Sometimes, believe me, I think I would be happier if he were dead. And
yet I know that if he dies I'll be even more miserable. I can't live
with him, and I can't live without him! What {\em can} I do about him?

That's the effect of this satyr's music---on me and many others. But
that's the least of it. He's like these creatures in all sorts of other
ways; his powers are really extraordinary. Let me tell you about them,
because, \inouter{d} you can be sure of it, none of you really understands
him. But, now I've started, I'm going to show you what he really is.

To begin with, he's crazy about beautiful boys; he constantly follows
them around in a perpetual daze. Also, he likes to say he's ignorant and
knows nothing. Isn't this just like Silenus? Of course it is! And all
this is just on the surface, like the outsides of those statues of
Silenus. I wonder, my fellow drinkers, if you have any idea what a sober
and temperate man he proves to be once you have looked inside. Believe
me, it couldn't matter less to him whether a boy is beautiful. You can't
imagine how little he \inouter{e} cares whether a person is beautiful, or
rich, or famous in any other way that most people admire. He considers
all these possessions beneath contempt, and that's exactly how he
considers all of us as well. In public, I tell you, his whole life is
one big game---a game of irony. I don't know if any of you have seen him
when he's really serious. But I once caught him when he was open like
Silenus' statues, and I had a glimpse of the figures he keeps hidden
within: they were so godlike---so bright and beautiful, \inouter{217}
so utterly amazing---that I no longer had a choice---I just had to do
whatever he told me.

What I thought at the time was that what he really wanted was {\em me},
and that seemed to me the luckiest coincidence: all I had to do was to
let him have his way with me, and he would teach me everything he
knew---believe me, I had a lot of confidence in my looks. Naturally, up
to that time we'd never been alone together; one of my attendants had
always been present. But with this in mind, I sent the attendant away,
and met \inouter{b} Socrates alone. (You see, in this company I must tell
the whole truth: so pay attention. And, Socrates, if I say anything
untrue, I want you to correct me.)

So there I was, my friends, alone with him at last. My idea, naturally,
was that he'd take advantage of the opportunity to tell me whatever it
is that lovers say when they find themselves alone; I relished the
moment. But no such luck! Nothing of the sort occurred. Socrates had his
usual sort of conversation with me, and at the end of the day he went
off. \inouter{c}

My next idea was to invite him to the gymnasium with me. We took
exercise together, and I was sure that this would lead to something. He
took exercise and wrestled with me many times when no one else was
present. What can I tell you? I got nowhere. When I realized that my
ploy had failed, I decided on a frontal attack. I refused to retreat
from a battle I myself had begun, and I needed to know just where
matters stood. So what I did was to invite him to dinner, as if {\em I}
were his lover and he my young prey! To tell the truth, it took him
quite a while to accept my \inouter{d} invitation, but one day he finally
arrived. That first time he left right after dinner: I was too shy to
try to stop him. But on my next attempt, I started some discussion just
as we were finishing our meal and kept him talking late into the night.
When he said he should be going, I used the lateness of the hour as an
excuse and managed to persuade him to spend the night at my house. He
had had his meal on the couch next to mine, so he just made himself
comfortable and lay down on it. No one else was there. \inouter{e}

Now you must admit that my story so far has been perfectly decent; I
could have told it in any company. But you'd never have heard me tell
the rest of it, as you're about to do, if it weren't that, as the saying
goes, ‘there's truth in wine when the slaves have left'---and when
they're present, too. Also, would it be fair to Socrates for me to
praise him and yet to fail to reveal one of his proudest
accomplishments? And, furthermore, you know what people say about
snakebite---that you'll only talk about it with your fellow victims:
only they will understand the pain and forgive you\inouter{218} for
all the things it made you do. Well, something much more painful than a
snake has bitten me in my most sensitive part---I mean my heart, or my
soul, or whatever you want to call it, which has been struck and bitten
by philosophy, whose grip on young and eager souls is much more vicious
than a viper's and makes them do the most amazing things. Now, \inouter{b}
all you people here, Phaedrus, Agathon, Eryximachus, Pausanias,
Aristodemus, Aristophanes---I need not mention Socrates himself---and
all the rest, have all shared in the madness, the Bacchic frenzy of
philosophy. And that's why you will hear the rest of my story; you will
understand and forgive both what I did then and what I say now. As for
the house slaves and for anyone else who is not an initiate, my story's
not for you: block your ears!

\inouter{c} To get back to the story. The lights were out; the slaves had
left; the time was right, I thought, to come to the point and tell him
freely what I had in mind. So I shook him and whispered:

“Socrates, are you asleep?”

“No, no, not at all,” he replied.

“You know what I've been thinking?”

“Well, no, not really.”

“I think,” I said, “you're the only worthy lover I have ever had---and
yet, look how shy you are with me! Well, here's how I look at it. It
would \inouter{d} be really stupid not to give you anything you want: you
can have me, my belongings, anything my friends might have. Nothing is
more important to me than becoming the best man I can be, and no one can
help me more than you to reach that aim. With a man like you, in fact,
I'd be much more ashamed of what wise people would say if I did
{\em not} take you as my lover, than I would of what all the others, in
their foolishness, would say if I did.”

He heard me out, and then he said in that absolutely inimitable ironic
manner of his:

\inouter{e} “Dear Alcibiades, if you are right in what you say about me, you
are already more accomplished than you think. If I really have in me the
power to make you a better man, then you can see in me a beauty that is
really beyond description and makes your own remarkable good looks pale
in comparison. But, then, is this a fair exchange that you propose? You
seem to me to want more than your proper share: you offer me the merest
appearance of beauty, and in return you want the thing itself, ‘gold
\inouter{219} in exchange for
bronze.'\endnote{{\em Iliad} vi.232--36
tells the famous story of the exchange by Glaucus of golden armor for
bronze.}

“Still, my dear boy, you should think twice, because you could be wrong,
and I may be of no use to you. The mind's sight becomes sharp only when
the body's eyes go past their prime---and you are still a good long time
away from that.”

When I heard this I replied:

“I really have nothing more to say. I've told you exactly what I think.
Now it's your turn to consider what you think best for you and me.”

\inouter{b} “You're right about that,” he answered. “In the future, let's
consider things together. We'll always do what seems the best to the two
of us.”

His words made me think that my own had finally hit their mark, that he
was smitten by my arrows. I didn't give him a chance to say another
word. I stood up immediately and placed my mantle over the light cloak
which, though it was the middle of winter, was his only clothing. I
slipped underneath the cloak and put my arms around this man---this
utterly \inouter{c} unnatural, this truly extraordinary man---and spent the
whole night next to him. Socrates, you can't deny a word of it. But in
spite of all my efforts, this hopelessly arrogant, this unbelievably
insolent man---he turned me down! He spurned my beauty, of which I was
so proud, members of the jury---for this is really what you are: you're
here to sit in judgment of Socrates' amazing arrogance and pride. Be
sure of it, I swear to you by all the gods and goddesses together, my
night with Socrates went no \inouter{d} further than if I had spent it with
my own father or older brother!

How do you think I felt after that? Of course, I was deeply humiliated,
but also I couldn't help admiring his natural character, his moderation,
his fortitude---here was a man whose strength and wisdom went beyond my
wildest dreams! How could I bring myself to hate him? I couldn't bear to
lose his friendship. But how could I possibly win him over? I knew
\inouter{e} very well that money meant much less to him than enemy weapons
ever meant to Ajax,\endnote{Ajax, a hero of the Greek
army at Troy, carried an enormous shield and so was virtually
invulnerable to enemy weapons.}
and the only trap by means of which I had thought I might capture him
had already proved a dismal failure. I had no idea what to do, no
purpose in life; ah, no one else has ever known the real meaning of
slavery!

All this had already occurred when Athens invaded
Potidaea,\endnote{Potidaea, a city in
Thrace allied to Athens, was induced by Corinth to revolt in 432 B.C.
The city was besieged by the Athenians and eventually defeated in a
bloody local war, 432--430 B.C.} where we
served together and shared the same mess. Now, first, he took the
hardships of the campaign much better than I ever did---much better, in
fact, than anyone in the whole army. When we were cut off from our
supplies, as often happens in the field, no one else stood up to hunger
as \inouter{220} well as he did. And yet he was the one man who could
really enjoy a feast; and though he didn't much want to drink, when he
had to, he could drink the best of us under the table. Still, and most
amazingly, no one ever saw him drunk (as we'll straightaway put to the
test).

Add to this his amazing resistance to the cold---and, let me tell you,
the \inouter{b} winter there is something awful. Once, I remember, it was
frightfully cold; no one so much as stuck his nose outside. If we
absolutely had to leave our tent, we wrapped ourselves in anything we
could lay our hands on and tied extra pieces of felt or sheepskin over
our boots. Well, Socrates went out in that weather wearing nothing but
this same old light cloak, and even in bare feet he made better progress
on the ice than the other \inouter{c} soldiers did in their boots. You
should have seen the looks they gave him; they thought he was only doing
it to spite them!

So much for that! But you should hear what else he did during that same
campaign,

\blank[line]

\quad{\em The exploit our strong-hearted hero dared to
do.}\endnote{{\em Odyssey} iv.242,
271.}

\blank[line]

One day, at dawn, he started thinking about some problem or other; he
just stood outside, trying to figure it out. He couldn't resolve it, but
he wouldn't give up. He simply stood there, glued to the same spot. By
midday, many soldiers had seen him, and, quite mystified, they told
everyone that Socrates had been standing there all day, thinking about
something. He was still there when evening came, and after dinner some
Ionians \inouter{d} moved their bedding outside, where it was cooler and
more comfortable (all this took place in the summer), but mainly in
order to watch if Socrates was going to stay out there all night. And so
he did; he stood on the very same spot until dawn! He only left next
morning, when the sun came out, and he made his prayers to the new day.

And if you would like to know what he was like in battle---this is a
tribute he really deserves. You know that I was decorated for bravery
during \inouter{e} that campaign: well, during that very battle, Socrates
single-handedly saved my life! He absolutely did! He just refused to
leave me behind when I was wounded, and he rescued not only me but my
armor as well. For my part, Socrates, I told them right then that the
decoration really belonged to you, and you can blame me neither for
doing so then nor for saying so now. But the generals, who seemed much
more concerned with my social position, insisted on giving the
decoration to me, and, I must say, you were more eager than the generals
themselves for me to have it.

\inouter{221} You should also have seen him at our horrible retreat
from Delium.\endnote{At Delium, a town on the
Boeotian coastline just north of Attica, a major Athenian expeditionary
force was routed by a Boeotian army in 424 B.C. For another description
of Socrates' action during the retreat, see {\em Laches} 181b.} I was
there with the cavalry, while Socrates was a foot soldier. The army had
already dispersed in all directions, and Socrates was retreating
together with Laches. I happened to see them just by chance, and the
moment I did I started shouting encouragements to them, telling them I
was never going to leave their side, and so on. That day I had a better
opportunity \inouter{b} to watch Socrates than I ever had at Potidaea, for,
being on horseback, I wasn't in very great danger. Well, it was easy to
see that he was remarkably more collected than Laches. But when I looked
again I couldn't get your words, Aristophanes, out of my mind: in the
midst of battle he was making his way exactly as he does around
town,

\blank[line]

\quad{\em \ldots{} with swagg'ring gait and roving
eye.}\endnote{Cf. Aristophanes,
{\em Clouds} 362.}

\blank[line]

He was observing everything quite calmly, looking out for friendly
troops and keeping an eye on the enemy. Even from a great distance it
was obvious that this was a very brave man, who would put up a terrific
fight if anyone approached him. This is what saved both of them. For, as
a rule, you try to put as much distance as you can between yourself and
such men in battle; you go after the others, those who run away
helter-skelter. \inouter{c}

You could say many other marvelous things in praise of Socrates. Perhaps
he shares some of his specific accomplishments with others. But, as a
whole, he is unique; he is like no one else in the past and no one in
the present---this is by far the most amazing thing about him. For we
might be able to form an idea of what Achilles was like by comparing him
to Brasidas or some other great warrior, or we might compare Pericles
with Nestor or Antenor or one of the other great
orators.\endnote{Brasidas, among the most
effective Spartan generals during the Peloponnesian War, was mortally
wounded while defeating the Athenians at Amphipolis in 422 B.C. Antenor
(for the Trojans) and Nestor (for the Greeks) were legendary wise
counsellors during the Trojan War.} There is a
parallel \inouter{d} for everyone---everyone else, that is. But this man
here is so bizarre, his ways and his ideas are so unusual, that, search
as you might, you'll never find anyone else, alive or dead, who's even
remotely like him. The best you can do is not to compare him to anything
human, but to liken him, as I do, to Silenus and the satyrs, and the
same goes for his ideas and arguments.

Come to think of it, I should have mentioned this much earlier: even his
ideas and arguments are just like those hollow statues of Silenus. If
\inouter{e} you were to listen to his arguments, at first they'd strike you
as totally ridiculous; they're clothed in words as coarse as the hides
worn by the most vulgar satyrs. He's always going on about pack asses,
or blacksmiths, or cobblers, or tanners; he's always making the same
tired old points in the same tired old words. If you are foolish, or
simply unfamiliar with him, you'd find it impossible not to laugh at his
arguments. But if you \inouter{222} see them when they open up like
the statues, if you go behind their surface, you'll realize that no
other arguments make any sense. They're truly worthy of a god, bursting
with figures of virtue inside. They're of great---no, of the
greatest---importance for anyone who wants to become a truly good man.

Well, this is my praise of Socrates, though I haven't spared him my
\inouter{b} reproach, either; I told you how horribly he treated me---and
not only me but also Charmides, Euthydemus, and many others. He has
deceived us all: he presents himself as your lover, and, before you know
it, you're in love with him yourself! I warn you, Agathon, don't let him
fool you! Remember our torments; be on your guard: don't wait, like the
fool in the \inouter{c} proverb, to learn your lesson from your own
misfortune.\endnote{Cf. {\em Iliad} xvii.32.}

\blank[line]

Alcibiades' frankness provoked a lot of laughter, especially since it
was obvious that he was still in love with Socrates, who immediately
said to him:

“You're perfectly sober after all, Alcibiades. Otherwise you could never
have concealed your motive so gracefully: how casually you let it drop,
almost like an afterthought, at the very end of your speech! As if the
real \inouter{d} point of all this has not been simply to make trouble
between Agathon and me! You think that I should be in love with you and
no one else, while you, and no one else, should be in love with
Agathon---well, we were {\em not} deceived; we've seen through your
little satyr play. Agathon, my friend, don't let him get away with it:
let no one come between us!”

Agathon said to Socrates:

\inouter{e} “I'm beginning to think you're right; isn't it proof of that
that he literally came between us here on the couch? Why would he do
this if he weren't set on separating us? But he won't get away with it;
I'm coming right over to lie down next to you.”

“Wonderful,” Socrates said. “Come here, on my other side.”

“My god!” cried Alcibiades. “How I suffer in his hands! He kicks me when
I'm down; he never lets me go. Come, don't be selfish, Socrates; at
least, let's compromise: let Agathon lie down between us.”

“Why, that's impossible,” Socrates said. “You have already delivered
your praise of me, and now it's my turn to praise whoever's on my right.
But if Agathon were next to you, he'd have to praise me all over again
\inouter{223} instead of having me speak in his honor, as I very much
want to do in any case. Don't be jealous; let me praise the boy.”

“Oh, marvelous,” Agathon cried. “Alcibiades, nothing can make me stay
next to you now. I'm moving no matter what. I simply {\em must} hear
what Socrates has to say about me.”

“There we go again,” said Alcibiades. “It's the same old story: when
Socrates is around, nobody else can get close to a good-looking man.
Look \inouter{b} how smoothly and plausibly he found a reason for Agathon to
lie down next to him!”

And then, all of a sudden, while Agathon was changing places, a large
drunken group, finding the gates open because someone was just leaving,
walked into the room and joined the party. There was noise everywhere,
and everyone was made to start drinking again in no particular order.

At that point, Aristodemus said, Eryximachus, Phaedrus, and some \inouter{c}
others among the original guests made their excuses and left. He himself
fell asleep and slept for a long time (it was winter, and the nights
were quite long). He woke up just as dawn was about to break; the
roosters were crowing already. He saw that the others had either left or
were asleep on their couches and that only Agathon, Aristophanes, and
Socrates were still awake, drinking out of a large cup which they were
passing around \inouter{d} from left to right. Socrates was talking to them.
Aristodemus couldn't remember exactly what they were saying---he'd
missed the first part of their discussion, and he was half-asleep
anyway---but the main point was that Socrates was trying to prove to
them that authors should be able to write both comedy and tragedy: the
skillful tragic dramatist should also be a comic poet. He was about to
clinch his argument, though, to tell the truth, sleepy as they were,
they were hardly able to follow his reasoning. In fact, Aristophanes
fell asleep in the middle of the discussion, and very soon thereafter,
as day was breaking, Agathon also drifted off.

But after getting them off to sleep, Socrates got up and left, and
Aristodemus followed him, as always. He said that Socrates went directly
to the Lyceum, washed up, spent the rest of the day just as he always
did, and only then, as evening was falling, went home to rest.
\stopchapter
