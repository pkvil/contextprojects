\chapter[Phaedrus]{Phaedrus}

Translated by A. Nehamas and P. Woodruff.

Phaedrus {\em is commonly paired on the one hand with} Gorgias {\em and
on the other} {\em with} Symposium{\em ---with the former in sharing its
principal theme, the nature} {\em and limitations of rhetoric, with the
latter in containing speeches devoted to the} {\em nature and value of
erotic love. Here the two interests combine in manifold} {\em ways.
Socrates, a city dweller little experienced in the pleasures of the
country,} {\em walks out from Athens along the river Ilisus, alone with
his friend Phaedrus,} {\em an impassioned admirer of oratory, for a
private conversation: in Plato most of} {\em his conversations take
place in a larger company, and no other in the private} {\em beauty of a
rural retreat. There he is inspired to employ his knowledge of
philosophy} {\em in crafting two speeches on the subject of erotic love,
to show how paltry} {\em is the best effort on the same subject of the
best orator in Athens, Lysias, who} {\em knows no philosophy. In the
second half of the dialogue he explains to Phaedrus} {\em exactly how
philosophical understanding of the truth about any matter}
{\em discoursed upon, and about the varieties of human soul and their
rhetorical susceptibilities,} {\em is an indispensable basis for a
rhetorically accomplished speech---} {\em such as he himself delivered
in the first part of the dialogue. By rights, Phaedrus'} {\em passionate
admiration for oratory ought therefore to be transformed into} {\em an
even more passionate love of philosophical knowledge, fine oratory's
essential} {\em prerequisite. Socrates' own speeches about erotic love
and his dialectical presentation} {\em of rhetoric's subservience to
philosophy are both aimed at persuading} {\em Phaedrus to this
transformation.}

{\em In his great second speech Socrates draws upon the psychological
theory of} {\em the} Republic {\em and the metaphysics of resplendent
Forms common to that dialogue} {\em and several others (notably} Phaedo
{\em and} Symposium{\em ) to inspire in Phaedrus} {\em a love for
philosophy. By contrast, the philosophy drawn upon in the second,}
{\em dialectical, half of the dialogue is linked closely to the much
more austere,} {\em logically oriented investigations via the ‘method of
divisions' that we find in} Sophist, Statesman, {\em and}
Philebus{\em ---where the grasp of any important philosophical}
{\em idea (any Form) proceeds by patient, detailed mapping of its
relations} {\em to other concepts and to its own subvarieties, not
through an awe-inspiring vision} {\em of a self-confined, single
brilliant entity. One of Socrates' central claims in} {\em the second
part of the dialogue is that a rhetorical composition, of which his
second} {\em speech is a paragon, must construct in words mere
resemblances of the} {\em real truth, ones selected to appeal to the
specific type of ‘soul' that its hearers} {\em possess, so as to draw
them on toward knowledge of the truth---or else to disguise} {\em it! A
rhetorical composition does not actually convey the truth; the truth}
{\em is known only through philosophical study---of the sort whose
results are presented} {\em in the second half of the dialogue. So
Socrates himself warns us that the} {\em ‘philosophical theories'
embodied in his speech are resemblances only, motivated} {\em in fact by
his desire to win Phaedrus away from an indiscriminate love of rhetoric}
{\em to a controlled but elevated love of philosophical study.}

Phaedrus {\em is one of Plato's most admired literary masterpieces. Yet
toward} {\em its end Socrates criticizes severely those who take their
own writing seriously---} {\em any writing, not just orators' speeches.
Writings cannot contain or constitute} {\em knowledge of any important
matter. Knowledge can only be lodged in a} {\em mind, and its essential
feature there is an endless capacity to express, interpret,} {\em and
reinterpret itself suitably, in response to every challenge---something
a} {\em written text once let go by its author plainly lacks:} it
{\em can only keep on repeating} {\em the same words to whoever picks it
up. But does not a Platonic dialogue,} {\em in engaging its reader in a
creative, multilayered intellectual encounter, have a} {\em similar
capacity for ever-deeper reading, for the discovery of underlying
meaning} {\em beyond the simple presentation of its surface ideas?
Knowledge is only in} {\em souls, but, despite the} Phaedrus' {\em own
critique of writing, reading such a dialogue} {\em may be a good way of
working to attain it.}

{\em J.M.C.}\par
\blank[line]

\saysocrates Phaedrus, my friend! Where have you been? And where are you 
going?

\sayphaedrus I was with Lysias, the son of Cephalus,\endnote{Cephalus is prominent in the opening section of Plato\symbol[rightquote]s {\em Republic}, which is set in his home in Piraeus, the port of Athens. His sons Lysias, Polemarchus, and Euthydemus were known for their democratic sympathies.} Socrates, and I am going for a walk outside the city walls because I was with him for a long time, sitting there the whole morning. You see, I'm keeping in
mind the advice of our mutual friend Acumenus,\endnote{Acumenus was a doctor and a relative of the doctor Eryximachus who speaks in the {\em Symposium.}} who says it's more refreshing to walk along country roads than city streets.

\saysocrates He is quite right, too, my friend. So Lysias, I take it, is in
the city?

\sayphaedrus Yes, at the house of Epicrates, which used to belong to
Morychus,\endnote{Morychus is mentioned for
his luxurious ways in a number of Aristophanes\symbol[rightquote] plays.} 
near the temple of the Olympian Zeus.

\saysocrates What were you doing there? Oh, I know: Lysias must have been
entertaining you with a feast of eloquence.

\sayphaedrus You'll hear about it, if you are free to come along and
listen.

\saysocrates What? Don't you think I would consider it “more important than
the most pressing engagement,” as Pindar says, to hear how you and
Lysias spent your time?\endnote{Pindar, {\em Isthmian} I.2,
adapted by Plato.}

\sayphaedrus Lead the way, then.

\saysocrates If only you will tell me.

\sayphaedrus In fact, Socrates, you're just the right person to hear the
speech that occupied us, since, in a roundabout way, it was about love.
It is aimed at seducing a beautiful boy, but the speaker is not in love
with him---this is actually what is so clever and elegant about it:
Lysias argues that it is better to give your favors to someone who does
not love you than to someone who does.

\saysocrates What a wonderful man! I wish he would write that you should 
give your favors to a poor rather than to a rich man, to an
older rather than to a younger one---that is, to someone like me and
most other people: then his speeches would be really sophisticated, and
they'd contribute to the public good besides! In any case, I am so eager
to hear it that I would follow you even if you were walking all the way
to Megara, as Herodicus recommends, to touch the wall and come back
again.\endnote{Herodicus was a medical expert whose regimen Socrates 
criticizes in {\em Republic}.}

\sayphaedrus What on earth do you mean, Socrates? Do you think that a 
mere dilettante like me could recite from memory in a
manner worthy of him a speech that Lysias, the best of our writers, took
such time and trouble to compose? Far from it---though actually I would
rather be able to do that than come into a large fortune!

\saysocrates Oh, Phaedrus, if I don't know my Phaedrus I must be forgetting
who I am myself---and neither is the case. I know very well that he did
not hear Lysias' speech only once: he asked him to repeat it over and
over again, and Lysias was eager to oblige. But not even that was enough
for him. In the end, he took the book himself and pored over the
parts he liked best. He sat reading all morning long, and when he got
tired, he went for a walk, having learned---I am quite sure---the whole
speech by heart, unless it was extraordinarily long. So he started for
the country, where he could practice reciting it. And running into a man
who is sick with passion for hearing speeches, seeing him---just seeing
him---he was filled with delight: he had found a partner for his
frenzied dance, and he urged him to lead the way. But when that
lover of speeches asked him to recite it, he played coy and pretended
that he did not want to. In the end, of course, he was going to recite
it even if he had to force an unwilling audience to listen. So, please,
Phaedrus, beg him to do it right now. He'll do it soon enough anyway.

\sayphaedrus Well, I'd better try to recite it as best I can: you'll
obviously not leave me in peace until I do so one way or another.

\saysocrates You are absolutely right.

\sayphaedrus That's what I'll do, then. But, Socrates, it really is true
that I did not memorize the speech word for word; instead, I
will give a careful summary of its general sense, listing all the ways
he said the lover differs from the non-lover, in the proper order.

\saysocrates Only if you first show me what you are holding in your left
hand under your cloak, my friend. I strongly suspect you have the speech
itself. And if I'm right, you can be sure that, though I love you
dearly, I'll never, as long as Lysias himself is present, allow you to
practice your own speechmaking on me. Come on, then, show me.

\sayphaedrus Enough, enough. You've dashed my hopes of using you as my
training partner, Socrates. All right, where do you want to sit while we
read?

\saysocrates Let's leave the path here and walk along the Ilisus; then we 
can sit quietly wherever we find the right spot.

\sayphaedrus How lucky, then, that I am barefoot today---you, of course,
are always so. The easiest thing to do is to walk right in the stream;
this way, we'll also get our feet wet, which is very pleasant,
especially at this hour and season.

\saysocrates Lead the way, then, and find us a place to sit.

\sayphaedrus Do you see that very tall plane tree?

\saysocrates Of course.

\sayphaedrus It's shady, with a light breeze; we can sit or, if we prefer,
lie down on the grass there.

\saysocrates Lead on, then.

\sayphaedrus Tell me, Socrates, isn't it from somewhere near this stretch
of the Ilisus that people say Boreas carried Orithuia
away?\endnote{According to legend,
Orithuia, daughter of the Athenian king Erechtheus, was abducted by
Boreas while she was playing with Nymphs along the banks of the Ilisus
River. Boreas personifies the north wind.}

\saysocrates So they say.

\sayphaedrus Couldn't this be the very spot? The stream is lovely, pure and
clear: just right for girls to be playing nearby.

\saysocrates No, it is two or three hundred yards farther downstream, 
where one crosses to get to the district of Agra. I think there
is even an altar to Boreas there.

\sayphaedrus I hadn't noticed it. But tell me, Socrates, in the name of
Zeus, do you really believe that that legend is true?

\saysocrates Actually, it would not be out of place for me to reject it, as
our intellectuals do. I could then tell a clever story: I could claim
that a gust of the North Wind blew her over the rocks where she was
playing with Pharmaceia; and once she was killed that way people said
she had been carried off by Boreas---or was it, perhaps, from the
Areopagus? The story is also told that she was carried away from
there instead. Now, Phaedrus, such explanations are amusing enough, but
they are a job for a man I cannot envy at all. He'd have to be far too
ingenious and work too hard---mainly because after that he will have to
go on and give a rational account of the form of the Hippocentaurs, and
then of the Chimera; and a whole flood of Gorgons and Pegasuses
and other monsters, in large numbers and absurd forms, will overwhelm
him. Anyone who does not believe in them, who wants to explain them away
and make them plausible by means of some sort of rough ingenuity, will
need a great deal of time.

But I have no time for such things; and the reason, my friend, is this.
I am still unable, as the Delphic inscription orders, to
know myself; and it really seems to me ridiculous to look into other
things before I have understood that. This is why I do not concern
myself with them. I accept what is generally believed, and, as I was
just saying, I look not into them but into my own self: Am I a beast
more complicated and savage than Typhon,\endnote{Typhon is a 
fabulous multiform beast with a hundred heads resembling many different 
animal species.} or am I a tamer, simpler animal with a share in a 
divine and gentle nature? But look, my 
friend---while we were talking, haven't we reached the tree you were
taking us to?

\sayphaedrus That's the one.

\saysocrates By Hera, it really is a beautiful resting place. The plane
tree is tall and very broad; the chaste-tree, high as it is, is
wonderfully shady, and since it is in full bloom, the whole place is
filled with its fragrance. From under the plane tree the loveliest
spring runs with very cool water---our feet can testify to that. The
place appears to be dedicated to Achelous and some of the Nymphs, if we
can judge from the statues and votive 
offerings.\endnote{Achelous is a river god. The
Nymphs are benevolent female deities associated with natural phenomena
such as streams, woods, and mountains.} Feel the
freshness of the air; how pretty and pleasant it is; how it echoes with
the summery, sweet song of the cicadas' chorus! The most exquisite thing
of all, of course, is the grassy slope: it rises so gently that you can
rest your head perfectly when you lie down on it. You've really been the
most marvelous guide, my dear Phaedrus.

\sayphaedrus And you, my remarkable friend, appear to be totally out of 
place. Really, just as you say, you seem to need a guide, not to
be one of the locals. Not only do you never travel abroad---as far as I
can tell, you never even set foot beyond the city walls.

\saysocrates Forgive me, my friend. I am devoted to learning; landscapes
and trees have nothing to teach me---only the people in the city can do
that. But you, I think, have found a potion to charm me into leaving.
For just as people lead hungry animals forward by shaking
branches of fruit before them, you can lead me all over Attica or
anywhere else you like simply by waving in front of me the leaves of a
book containing a speech. But now, having gotten as far as this place
this time around, I intend to lie down; so choose whatever position you
think will be most comfortable for you, and read on.

\sayphaedrus Listen, then:

“You understand my situation: I've told you how good it would be for us,
in my opinion, if this worked out. In any case, I don't think I should 
lose the chance to get what I am asking for, merely
because I don't happen to be in love with you.

A man in love will wish he had not done you any favors once his desire
dies down, but the time will never come for a man who's not in love to
change his mind. That is because the favors he does for you are not
forced but voluntary; and he does the best that he possibly can for you,
just as he would for his own business.

Besides, a lover keeps his eye on the balance sheet---where his
interests have suffered from love, and where he has done well; and when
he adds up all the trouble he has taken, he thinks he's long since given
the boy he loved a fair return. A non-lover, on the other hand,
can't complain about love's making him neglect his own business; he
can't keep a tab on the trouble he's been through, or blame you for the
quarrels he's had with his relatives. Take away all those headaches and
there's nothing left for him to do but put his heart into whatever he
thinks will give pleasure.

Besides, suppose a lover does deserve to be honored because, as they 
say, he is the best friend his loved one will ever have, and he
stands ready to please his boy with all those words and deeds that are
so annoying to everyone else. It's easy to see (if he is telling the
truth) that the next time he falls in love he will care more for his new
love than for the old one, and it's clear he'll treat the old one
shabbily whenever that will please the new one.

And anyway, what sense does it make to throw away something like that
on a person who has fallen into such a miserable condition that those 
who have suffered it don't even try to defend themselves against
it? A lover will admit that he's more sick than sound in the head. He's
well aware that he is not thinking straight; but he'll say he can't get
himself under control. So when he does start thinking straight, why
would he stand by decisions he had made when he was sick?

Another point: if you were to choose the best of those who are in love
with you, you'd have a pretty small group to pick from; but you'll have
a large group if you don't care whether he loves you or not and just
pick the one who suits you best; and in that larger pool you'll have a
much better hope of finding someone who deserves your friendship.

Now suppose you're afraid of conventional standards and the stigma that
will come to you if people find out about this. Well, it stands to
reason that a lover---thinking that everyone else will admire him for
his success as much as he admires himself---will fly
into words and proudly declare to all and sundry that his labors were
not in vain. Someone who does not love you, on the other hand, can
control himself and will choose to do what is best, rather than seek the
glory that comes from popular reputation.

Besides, it's inevitable that a lover will be found out: many people
will see that he devotes his life to following the boy he loves. The
result is that whenever people see you talking with him they'll
think you are spending time together just before or just after giving
way to desire. But they won't even begin to find fault with people for
spending time together if they are not lovers; they know one has to talk
to someone, either out of friendship or to obtain some other pleasure.

Another point: have you been alarmed by the thought that it is hard for
friendships to last? Or that when people break up, it's ordinarily just 
as awful for one side as it is for the other, but when you've
given up what is most important to you already, then your loss is
greater than his? If so, it would make more sense for you to be afraid
of lovers. For a lover is easily annoyed, and whatever happens, he'll
think it was designed to hurt him. That is why a lover prevents the boy
he loves from spending time with other people. He's afraid that wealthy
men will outshine him with their money, while men of education will turn
out to have the advantage of greater intelligence. And he watches like a
hawk everyone who may have any other advantage over him! Once
he's persuaded you to turn those people away, he'll have you completely
isolated from friends; and if you show more sense than he does in
looking after your own interests, you'll come to quarrel with him.

But if a man really does not love you, if it is only because of his
excellence that he got what he asked for, then he won't be jealous of
the people who spend time with you. Quite the contrary! He'll hate
anyone who does not want to be with you; he'll think they look down on
him while those who spend time with you do him good; so you
should expect friendship, rather than enmity, to result from this
affair.

Another point: lovers generally start to desire your body before they
know your character or have any experience of your other traits, with
the result that even they can't tell whether they'll still want to be
friends with you after their desire has passed.
Non-lovers, on the other hand, are friends with you even before they
achieve their goal, and you've no reason to expect that benefits
received will ever detract from their friendship for you. No, those
things will stand as reminders of more to come.

Another point: you can expect to become a better person if you are won
over by me, rather than by a lover. A lover will praise what you say and
what you do far beyond what is best, partly because he is afraid of 
being disliked, and partly because desire has impaired his
judgment. Here is how love draws conclusions: When a lover suffers a
reverse that would cause no pain to anyone else, love makes him think
he's accursed! And when he has a stroke of luck that's not worth a
moment's pleasure, love compels him to sing its praises. The result is,
you should feel sorry for lovers, not admire them.

If my argument wins you over, I will, first of all, give you my time
with no thought of immediate pleasure; I will plan instead for the
benefits that are to come, since I am master of myself and have
not been overwhelmed by love. Small problems will not make me very
hostile, and big ones will make me only gradually, and only a little,
angry. I will forgive you for unintentional errors and do my best to
keep you from going wrong intentionally. All this, you see, is the proof
of a friendship that will last a long time.

Have you been thinking that there can be no strong friendship in the
absence of erotic love? Then you ought to remember that we would not 
care so much about our children if that were so, or about our
fathers and mothers. And we wouldn't have had any trustworthy friends,
since those relationships did not come from such a desire but from doing
quite different things.

Besides, if it were true that we ought to give the biggest favor to
those who need it most, then we should all be helping out the very
poorest people, not the best ones, because people we've saved from the
worst troubles will give us the most thanks. For instance, the right
people to invite to a dinner party would be beggars and people
who need to sate their hunger, because they're the ones who'll be fond
of us, follow us, knock on our
doors,\endnote{This is classic behavior in
ancient Greek literature of a lovesick man pursuing his prey.} take the most
pleasure with the deepest gratitude, and pray for our success. No, it's
proper, I suppose, to grant your favors to those who are best able to
return them, not to those in the direst need---that is, not to those who
merely desire the thing, but to those who really deserve
it---not to people who will take pleasure in the bloom of your youth,
but to those who will share their goods with you when you are older; not
to people who achieve their goal and then boast about it in public, but
to those who will keep a modest silence with everyone; not to people
whose devotion is short-lived, but to those who will be steady friends
their whole lives; not to the people who look for an excuse to quarrel
as soon as their desire has passed, but to those who will prove 
their worth when the bloom of your youth has faded. Now, remember what I
said and keep this in mind: friends often criticize a lover for bad
behavior; but no one close to a non-lover ever thinks that desire has
led him into bad judgment about his interests.

And now I suppose you'll ask me whether I'm urging you to give your
favors to everyone who is not in love with you. No. As I see it, a lover
would not ask you to give in to all your lovers either. You would not,
in that case, earn as much gratitude from each recipient, and
you would not be able to keep one affair secret from the others in the
same way. But this sort of thing is not supposed to cause any harm, and
really should work to the benefit of both sides.

Well, I think this speech is long enough. If you are still longing for
more, if you think I have passed over something, just ask.”

How does the speech strike you, Socrates? Don't you think it's simply
superb, especially in its choice of words?

\saysocrates It's a miracle, my friend; I'm in ecstasy. And it's all your 
doing, Phaedrus: I was looking at you while you were reading and
it seemed to me the speech had made you radiant with delight; and since
I believe you understand these matters better than I do, I followed your
lead, and following you I shared your Bacchic frenzy.

\sayphaedrus Come, Socrates, do you think you should joke about this?

\saysocrates Do you really think I am joking, that I am not serious?

\sayphaedrus You are not at all serious, Socrates. But now tell me
the truth, in the name of Zeus, god of friendship: Do you think that any
other Greek could say anything more impressive or more complete on this
same subject?

\saysocrates What? Must we praise the speech even on the ground that its
author has said what the situation demanded, and not instead simply on
the ground that he has spoken in a clear and concise manner, with a
precise turn of phrase? If we must, I will have to go along for your
sake, since---surely because I am so ignorant---that
passed me by. I paid attention only to the speech's style. As to the
other part, I wouldn't even think that Lysias himself could be satisfied
with it. For it seemed to me, Phaedrus---unless, of course, you
disagree---that he said the same things two or even three times, as if
he really didn't have much to say about the subject, almost as if he
just weren't very interested in it. In fact, he seemed to me to be
showing off, trying to demonstrate that he could say the same thing in
two different ways, and say it just as well both times.

\sayphaedrus You are absolutely wrong, Socrates. That is in fact
the best thing about the speech: He has omitted nothing worth mentioning
about the subject, so that no one will ever be able to add anything of
value to complete what he has already said himself.

\saysocrates You go too far: I can't agree with you about that. If, as a
favor to you, I accept your view, I will stand refuted by all the wise
men and women of old who have spoken or written about this subject.

\sayphaedrus Who are these people? And where have you heard anything 
better than this?

\saysocrates I can't tell you offhand, but I'm sure I've heard better
somewhere; perhaps it was the lovely Sappho or the wise Anacreon or even
some writer of prose. So, what's my evidence? The fact, my dear friend,
that my breast is full and I feel I can make a different speech, even
better than Lysias'. Now I am well aware that none of these ideas can
have come from me---I know my own ignorance. The only other possibility,
I think, is that I was filled, like an empty jar, by the words
of other people streaming in through my ears, though I'm so stupid that
I've even forgotten where and from whom I heard them.

\sayphaedrus But, my dear friend, you couldn't have said a better thing!
Don't bother telling me when and from whom you've heard this, even if I
ask you---instead, do exactly what you said: You've just promised to
make another speech making more points, and better ones, without
repeating a word from my book. And I promise you that, like the Nine
Archons, I shall set up in return a life-sized golden statue at
Delphi, not only of myself but also of
you.\endnote{The archons were
magistrates chosen by lot in classical Athens. On taking office they
swore an oath to set up a golden statue if they violated the laws.}

\saysocrates You're a real friend, Phaedrus, good as gold, to think I'm
claiming that Lysias failed in absolutely every respect and that I can
make a speech that is different on every point from his. I am sure that
that couldn't happen even to the worst possible author. In our own case,
for example, do you think that anyone could argue that one should favor
the non-lover rather than the lover without praising the former for
keeping his wits about him or condemning the latter for
losing his---points that are essential to make---and still have
something left to say? I believe we must allow these points, and concede
them to the speaker. In their case, we cannot praise their novelty but
only their skillful arrangement; but we can praise both the arrangement
and the novelty of the nonessential points that are harder to think up.

\sayphaedrus I agree with you; I think that's reasonable. This, then, is
what I shall do. I will allow you to presuppose that the lover is less
sane than the non-lover---and if you are able to add anything of
value to complete what we already have in hand, you will stand in
hammered gold beside the offering of the Cypselids in
Olympia.\endnote{The Cypselids were rulers
of Corinth in the seventh century B.C.; an ornate chest in which
Cypselus was said to have been hidden as an infant was on display at
Olympia, perhaps along with other offerings of theirs.}

\saysocrates Oh, Phaedrus, I was only criticizing your beloved in order to
tease you---did you take me seriously? Do you think I'd really try to
match the product of his wisdom with a fancier speech?

\sayphaedrus Well, as far as that goes, my friend, you've fallen into your
own trap. You have no choice but to give your speech as best you can: 
otherwise you will force us into trading vulgar jibes the way
they do in comedy. Don't make me say what you said: “Socrates, if I
don't know my Socrates, I must be forgetting who I am myself,” or “He
wanted to speak, but he was being coy.” Get it into your head that we
shall not leave here until you recite what you claimed to have “in your
breast.” We are alone, in a deserted place, and I am younger and
stronger. From all this, “take my
meaning”\endnote{A line of Pindar's.} and don't
make me force you to speak when you can do so willingly.

\saysocrates But, my dear Phaedrus, I'll be ridiculous---a mere dilettante,
improvising on the same topics as a seasoned professional!

\sayphaedrus Do you understand the situation? Stop playing hard to get! I
know what I can say to make you give your speech.

\saysocrates Then please don't say it!

\sayphaedrus Oh, yes, I will. And what I say will be an oath. I swear to
you---by which god, I wonder? How about this very plane tree?---I swear 
in all truth that, if you don't make your speech right next to
this tree here, I shall never, never again recite another speech for
you---I shall never utter another word about speeches to you!

\saysocrates My oh my, what a horrible man you are! You've really found the
way to force a lover of speeches to do just as you say!

\sayphaedrus So why are you still twisting and turning like that?

\saysocrates I'll stop---now that you've taken this oath. How could I
possibly give up such treats?

\sayphaedrus Speak, then.

\saysocrates Do you know what I'll do?

\sayphaedrus What?

\saysocrates I'll cover my head while I'm speaking. In that way, as I'm
going through the speech as fast as I can, I won't get embarrassed by
having to look at you and lose the thread of my argument.

\sayphaedrus Just give your speech! You can do anything else you like.

\saysocrates Come to me, O you clear-voiced Muses, whether you are called
so because of the quality of your song or from the musical people of
Liguria,\endnote{Socrates here suggests a
farfetched etymology for a common epithet of the Muses, as the
“clear-voiced” ones, on the basis of its resemblance to the Greek name
for the Ligurians, who lived in what is now known as the French Riviera.} “come, take up my burden” in telling the tale that this fine fellow forces upon me
so that his companion may now seem to him even more clever than
he did before:

% TO HERE

There once was a boy, a youth rather, and he was very beautiful, and had
very many lovers. One of them was wily and had persuaded him that he was
not in love, though he loved the lad no less than the others. And once
in pressing his suit to him, he tried to persuade him that he ought to
give his favors to a man who did not love him rather than to one who
did. And this is what he said:

“If you wish to reach a good decision on any topic, my boy, there is 
only one way to begin: You must know what the decision is about,
or else you are bound to miss your target altogether. Ordinary people
cannot see that they do not know the true nature of a particular
subject, so they proceed as if they did; and because they do not work
out an agreement at the start of the inquiry, they wind up as you would
expect---in conflict with themselves and each other. Now you and I had
better not let this happen to us, since we criticize it in others.
Because you and I are about to discuss whether a boy should make friends
with a man who loves him rather than with one who does not, we
should agree on defining what love is and what effects it has. Then we
can look back and refer to that as we try to find out whether to expect
benefit or harm from love. Now, as everyone plainly knows, love is some
kind of desire; but we also know that even men who are not in love have
a desire for what is beautiful. So how shall we distinguish between a
man who is in love and one who is not? We must realize that each of us
is ruled by two principles which we follow wherever they lead: one is
our inborn desire for pleasures, the other is our acquired judgment that
pursues what is best. Sometimes these two are in agreement; but
there are times when they quarrel inside us, and then sometimes one of
them gains control, sometimes the other. Now when judgment is in control
and leads us by reasoning toward what is best, that sort of self-control
is called ‘being in your right mind'; but when desire 
takes command in us and drags us without reasoning toward pleasure, then
its command is known as
‘outrageousness'.\endnote{I.e., {\em hubris}, which
ranges from arrogance to the sort of crimes to which arrogance gives
rise, sexual assault in particular.} 
Now outrageousness has as many names as the forms it can take, and these are
quite diverse.\endnote{Reading {\em polumeles kai
polueides}, lit. “multilimbed and multiformed”.}
Whichever form stands out in a particular case gives its name to the
person who has it---and that is not a pretty name to be called, not
worth earning at all. If it is desire for food that overpowers a
person's reasoning about what is best and suppresses his other desires,
it is called gluttony and it gives him the name of a glutton,
while if it is desire for drink that plays the tyrant and leads the man
in that direction, we all know what name we'll call him then! And now it
should be clear how to describe someone appropriately in the other
cases: call the man by that name---sister to these others---that derives
from the sister of these desires that controls him at the time. As for
the desire that has led us to say all this, it should be obvious
already, but I suppose things said are always better understood than
things unsaid: The unreasoning desire that overpowers a person's
considered impulse to do right and is driven to take pleasure in beauty, 
its force reinforced by its kindred desires for beauty in human
bodies---this desire, all-conquering in its forceful drive, takes its
name from the word for force ({\em rhōmē}) and is called {\em erōs}.”

There, Phaedrus my friend, don't you think, as I do, that I'm in the
grip of something divine?

\sayphaedrus This is certainly an unusual flow of words for you, Socrates.

\saysocrates Then be quiet and listen. There's something really divine
about this place, so don't be surprised if I'm quite taken by the
Nymphs' madness as I go on with the speech. I'm on the edge of
speaking in dithyrambs\endnote{A dithyramb was a choral
poem originally connected with the worship of Dionysus. In classical
times it became associated with an artificial style dominated by music.}
as it is.

\sayphaedrus Very true!

\saysocrates Yes, and you're the cause of it. But hear me out; the attack
may yet be prevented. That, however, is up to the god; what we must do
is face the boy again in the speech:

“All right then, my brave friend, now we have a definition for the
subject of our decision; now we have said what it really is; so let us
keep that in view as we complete our discussion. What benefit or
harm is likely to come from the lover or the non-lover to the boy who
gives him favors? It is surely necessary that a man who is ruled by
desire and is a slave to pleasure will turn his boy into whatever is
most pleasing to himself. Now a sick man takes pleasure in anything that
does not resist him, but sees anyone who is equal or
superior to him as an enemy. That is why a lover will not willingly put
up with a boyfriend who is his equal or superior, but is always working
to make the boy he loves weaker and inferior to himself. Now, the
ignorant man is inferior to the wise one, the coward to the brave, the
ineffective speaker to the trained orator, the slow-witted to the quick.
By necessity, a lover will be delighted to find all these mental defects
and more, whether acquired or innate in his boy; and if he does not, he
will have to supply them or else lose the pleasure of the moment. 
The necessary consequence is that he will be jealous and keep
the boy away from the good company of anyone who would make a better man
of him; and that will cause him a great deal of harm, especially if he
keeps him away from what would most improve his mind---and that is, in
fact, divine philosophy, from which it is necessary for a lover to keep
his boy a great distance away, out of fear the boy will eventually come
to look down on him. He will have to invent other ways, too, of keeping
the boy in total ignorance and so in total dependence on himself. That
way the boy will give his lover the most pleasure, though the
harm to himself will be severe. So it will not be of any use to your
intellectual development to have as your mentor and companion a man who
is in love.

“Now let's turn to your physical development. If a man is bound by
necessity to chase pleasure at the expense of the good, what sort of
shape will he want you to be in? How will he train you, if he is in
charge? You will see that what he wants is someone who is soft, not
muscular, and not trained in full sunlight but in dappled
shade---someone who has never worked out like a man, never touched hard,
sweaty exercise. Instead, he goes for a boy who has known only a
soft unmanly style of life, who makes himself pretty with cosmetics
because he has no natural color at all. There is no point in going on
with this description: it is perfectly obvious what other sorts of
behavior follow from this. We can take up our next topic after drawing
all this to a head: the sort of body a lover wants in his boy is one
that will give confidence to the enemy in a war or other great crisis
while causing alarm to friends and even to his lovers. Enough of that;
the point is obvious.

“Our next topic is the benefit or harm to your possessions that
will come from a lover's care and company. Everyone knows the answer,
especially a lover: His first wish will be for a boy who has lost his
dearest, kindliest and godliest possessions---his mother and father and
other close relatives. He would be happy to see the boy deprived of
them, since he would expect them either to block him
from the sweet pleasure of the boy's company or to criticize him
severely for taking it. What is more, a lover would think any money or
other wealth the boy owns would only make him harder to snare and, once
snared, harder to handle. It follows by absolute necessity that wealth
in a boyfriend will cause his lover to envy him, while his poverty will
be a delight. Furthermore, he will wish for the boy to stay wifeless,
childless, and homeless for as long as possible, since that's how long
he desires to go on plucking his sweet fruit.

“There are other troubles in life, of course, but some divinity has
mixed most of them with a dash of immediate pleasure. A flatterer, for
example, may be an awful beast and a dreadful nuisance, but
nature makes flattery rather pleasant by mixing in a little culture with
its words. So it is with a mistress---for all the harm we accuse her of
causing---and with many other creatures of that character, and their
callings: at least they are delightful company for a day. But besides
being harmful to his boyfriend, a lover is simply disgusting to
spend the day with. ‘Youth delights youth,' as the old proverb
runs---because, I suppose, friendship grows from similarity, as boys of
the same age go after the same pleasures. But you can even have too much
of people your own age. Besides, as they say, it is miserable for anyone
to be forced into anything by necessity---and this (to say nothing of
the age difference) is most true for a boy with his lover. The older man
clings to the younger day and night, never willing to leave him, driven 
by necessity and goaded on by the sting that gives him pleasure
every time he sees, hears, touches, or perceives his boy in any way at
all, so that he follows him around like a servant, with pleasure.

“As for the boy, however, what comfort or pleasure will the lover give
to him during all the time they spend together? Won't it be disgusting
in the extreme to see the face of that older man who's lost his looks?
And everything that goes with that face---why, it is a misery even to
hear them mentioned, let alone actually handle them, as you would
constantly be forced to do! To be watched and guarded
suspiciously all the time, with everyone! To hear praise of yourself
that is out of place and excessive! And then to be falsely
accused---which is unbearable when the man is sober and not only
unbearable but positively shameful when he is drunk and lays into you
with a pack of wild barefaced insults!

“While he is still in love he is harmful and disgusting, but after his
love fades he breaks his trust with you for the future, in spite of all
the promises he has made with all those oaths and entreaties which just
barely kept you in a relationship that was troublesome
at the time, in hope of future benefits. So, then, by the time he should
pay up, he has made a change and installed a new ruling government in
himself: right-minded reason in place of the madness of love. The boy
does not even realize that his lover is a different man. He insists on
his reward for past favors and reminds him of what they had done and
said before---as if he were still talking to the same man! The lover,
however, is so ashamed that he does not dare tell the boy how much he
has changed or that there is no way, now that he is in his right mind
and under control again, that he can stand by the promises he had sworn
to uphold when he was under that old mindless regime. He is
afraid that if he acted as he had before he would turn out the same and
revert to his old self. So now he is a refugee, fleeing from those old
promises on which he must default by necessity; he, the former lover,
has to switch roles and flee, since the coin has fallen the other way,
while the boy must chase after him, angry and cursing. All along he has
been completely unaware that he should never have given his
favors to a man who was in love---and who therefore had by necessity
lost his mind. He should much rather have done it for a man who was not
in love and had his wits about him. Otherwise it follows necessarily
that he'd be giving himself to a man who is deceitful, irritable,
jealous, disgusting, harmful to his property, harmful to his physical
fitness, and absolutely devastating to the cultivation of his soul,
which truly is, and will always be, the most valuable thing to gods and
men.

“These are the points you should bear in mind, my boy. You should know
that the friendship of a lover arises without any good will at all. 
No, like food, its purpose is to sate hunger. ‘Do wolves love
lambs? That's how lovers befriend a boy!'\thinspace”

That's it, Phaedrus. You won't hear another word from me, and you'll
have to accept this as the end of the speech.

\sayphaedrus But I thought you were right in the middle---I thought you
were about to speak at the same length about the non-lover, to list his
good points and argue that it's better to give one's favors to him. So
why are you stopping now, Socrates?

\saysocrates Didn't you notice, my friend, that even though I am
criticizing the lover, I have passed beyond lyric into epic
poetry?\endnote{The overheated choral
poems known as dithyrambs were written in lyric meters. The
meter of the last line of Socrates' speech, however, was epic, and it is
the tradition in epic poetry to glorify a hero, not to attack him.} 
What do you suppose will happen to me if I begin to praise his opposite? Don't you
realize that the Nymphs to whom you so cleverly exposed me will take
complete possession of me? So I say instead, in a word, that every
shortcoming for which we blamed the lover has its contrary advantage,
and the non-lover possesses it. Why make a long speech of it? That's
enough about them both. This way my story will meet the
end it deserves, and I will cross the river and leave before you make me
do something even worse.

\sayphaedrus Not yet, Socrates, not until this heat is over. Don't you see
that it is almost exactly noon, “straight-up” as they say? Let's wait
and discuss the speeches, and go as soon as it turns cooler.

\saysocrates You're really superhuman when it comes to speeches, Phaedrus;
you're truly amazing. I'm sure you've brought into being more of
the speeches that have been given during your lifetime than anyone else,
whether you composed them yourself or in one way or another forced
others to make them; with the single exception of Simmias the Theban,
you are far ahead of the
rest.\endnote{Simmias, a companion of
Socrates, was evidently a lover of discussion (cf. {\em Phaedo}).} 
Even as we speak, I think, you're managing to cause me to produce yet 
another one.

\sayphaedrus Oh, how wonderful! But what do you mean? What speech?

\saysocrates My friend, just as I was about to cross the river, the
familiar divine sign came to me which, whenever it occurs, holds
me back from something I am about to do. I thought I heard a voice
coming from this very spot, forbidding me to leave until I made
atonement for some offense against the gods. In effect, you see, I am a
seer, and though I am not particularly good at it, still---like people
who are just barely able to read and write---I am good enough for my own
purposes. I recognize my offense clearly now. In fact, the soul too, my
friend, is itself a sort of seer; that's why, almost from the beginning
of my speech, I was disturbed by a very uneasy feeling, as
Ibycus puts it, that “for offending the gods I am honored by
men.”\endnote{Ibycus was a sixth-century
poet, most famous for his passionate love poetry.} But now I
understand exactly what my offense has been.

\sayphaedrus Tell me, what is it?

\saysocrates Phaedrus, that speech you carried with you here---it was
horrible, as horrible as the speech you made me give.

\sayphaedrus How could that be?

\saysocrates It was foolish, and close to being impious. What could be more
horrible than that?

\sayphaedrus Nothing---if, of course, what you say is right.

\saysocrates Well, then? Don't you believe that Love is the son of
Aphrodite? Isn't he one of the gods?

\sayphaedrus This is certainly what people say.

\saysocrates Well, Lysias certainly doesn't and neither does your speech,
which you charmed me through your potion into delivering myself. But if
Love is a god or something divine---which he is---he can't be bad in 
any way; and yet our speeches just now spoke of him as if he
were. That is their offense against Love. And they've compounded it with
their utter foolishness in parading their dangerous falsehoods and
preening themselves over perhaps deceiving a few silly
people and coming to be admired by them.

And so, my friend, I must purify myself. Now for those whose offense
lies in telling false stories about matters divine, there is an ancient
rite of purification---Homer did not know it, but Stesichorus did. When
he lost his sight for speaking ill of Helen, he did not, like Homer,
remain in the dark about the reason why. On the contrary, true follower
of the Muses that he was, he understood it and immediately composed
these lines:\crlf
\crlf

{\em There's no truth to that story:}

{\em You never sailed that lovely ship,}

{\em You never reached the tower of
Troy.} \crlf
\crlf

And as soon as he completed the poem we call the Palinode, he
immediately regained his sight. Now I will prove to be wiser than Homer
and Stesichorus to this small extent: I will try to offer my Palinode to
Love before I am punished for speaking ill of him---with my head bare,
no longer covered in shame.

\sayphaedrus No words could be sweeter to my ears, Socrates.

\saysocrates You see, my dear Phaedrus, you understand how
shameless the speeches were, my own as well as the one in your book.
Suppose a noble and gentle man, who was (or had once been) in love with
a boy of similar character, were to hear us say that lovers start
serious quarrels for trivial reasons and that, jealous of their beloved,
they do him harm---don't you think that man would think we had been
brought up among the most vulgar of sailors, totally ignorant of
love among the freeborn? Wouldn't he most certainly refuse to
acknowledge the flaws we attributed to Love?

\sayphaedrus Most probably, Socrates.

\saysocrates Well, that man makes me feel ashamed, and as I'm also afraid
of Love himself, I want to wash out the bitterness of what we've heard
with a more tasteful speech. And my advice to Lysias, too, is to write
as soon as possible a speech urging one to give similar favors to a
lover rather than to a non-lover.

\sayphaedrus You can be sure he will. For once you have spoken in praise 
of the lover, I will most definitely make Lysias write a speech
on the same topic.

\saysocrates I do believe you will, so long as you are who you are.

\sayphaedrus Speak on, then, in full confidence.

\saysocrates Where, then, is the boy to whom I was speaking? Let him hear
this speech, too. Otherwise he may be too quick to give his favors to
the non-lover.

\sayphaedrus He is here, always right by your side, whenever you want him.

\saysocrates You'll have to understand, beautiful boy, that
the previous speech was by Phaedrus, Pythocles' son, from Myrrhinus,
while the one I am about to deliver is by Stesichorus, Euphemus' son,
from Himera.\endnote{Etymologically:
“Stesichorus son of Good Speaker, from the Land of Desire.” Myrrhinus
was one of the demes of ancient Athens.} And here
is how the speech should go:

“‘There's no truth to that story'---that when a lover is available you
should give your favors to a man who doesn't love you instead, because
he is in control of himself while the lover has lost his head. That
would have been fine to say if madness were bad, pure and simple; but in
fact the best things we have come from madness, when it is given as a
gift of the god.

“The prophetess of Delphi and the priestesses at Dodona are out
of their minds when they perform that fine work of theirs for all of
Greece, either for an individual person or for a whole city, but they
accomplish little or nothing when they are in control of themselves. We
will not mention the Sybil or the others who foretell many things by
means of god-inspired prophetic trances and give sound guidance to many
people---that would take too much time for a point that's obvious to
everyone. But here's some evidence worth adding to our case: The people
who designed our language in the old days never thought of madness as
something to be ashamed of or worthy of blame; otherwise they would not
have used the word ‘manic' for the finest experts of all---the ones who
tell the future---thereby weaving insanity into prophecy. They
thought it was wonderful when it came as a gift of the god, and that's
why they gave its name to prophecy; but nowadays people don't know the
fine points, so they stick in a ‘t' and call it ‘{\em mantic}.'
Similarly, the clear-headed study of the future, which uses birds and
other signs, was originally called {\em oionoïstic}, since it uses
reasoning to bring intelligence ({\em nous}) and learning
({\em historia}) into human thought; but now modern speakers call it
{\em oiōnistic}, putting on airs with their long ‘{\em ō}'. To
the extent, then, that prophecy, {\em mantic}, is more perfect and more
admirable than sign-based prediction, {\em oiōnistic}, in both name and
achievement, madness ({\em mania}) from a god is finer than self-control
of human origin, according to the testimony of the ancient language
givers.

“Next, madness can provide relief from the greatest plagues of trouble
that beset certain families because of their guilt for ancient crimes:
it turns up among those who need a way out; it gives prophecies and
takes refuge in prayers to the gods and in worship, discovering
mystic rites and purifications that bring the man it
touches through to
safety for this and all time to come. So it is that the right sort of
madness finds relief from present hardships for a man it has possessed.

“Third comes the kind of madness that is possession by the Muses, 
which takes a tender virgin soul and awakens it to a
Bacchic frenzy of songs and poetry that glorifies the achievements of
the past and teaches them to future generations. If anyone comes to the
gates of poetry and expects to become an adequate poet by acquiring
expert knowledge of the subject without the Muses' madness, he will
fail, and his self-controlled verses will be eclipsed by the poetry of
men who have been driven out of their minds.

“There you have some of the fine achievements---and I could tell you 
even more---that are due to god-sent madness. We must not have
any fear on this particular point, then, and we must not let anyone
disturb us or frighten us with the claim that you should prefer a friend
who is in control of himself to one who is disturbed. Besides proving
that point, if he is to win his case, our opponent must show that love
is not sent by the gods as a benefit to a lover and his boy. And we, for
our part, must prove the opposite, that this sort of madness is given us
by the gods to ensure our greatest good fortune. It will be a proof that
convinces the wise if not the clever.

“Now we must first understand the truth about the nature of the soul,
divine or human, by examining what it does and what is done to it. Here
begins the proof:

“Every soul\endnote{Alternatively, “All soul.”} is
immortal. That is because whatever is always in motion is immortal,
while what moves, and is moved by, something else stops living when it
stops moving. So it is only what moves itself that never desists from
motion, since it does not leave off being itself. In fact, this
self-mover is also the source and spring of motion in everything else
that moves; and a source has no beginning. That is because
anything that has a beginning comes from some source, but there is no
source for this, since a source that got its start from something else
would no longer be the source. And since it cannot have a beginning,
then necessarily it cannot be destroyed. That is because if a source
were destroyed it could never get started again from anything else and
nothing else could get started from it---that is, if everything gets
started from a source. This then is why a self-mover is a source of
motion. And {\em that} is incapable of being destroyed or
starting up; otherwise all heaven and everything that has been started
up would collapse,
come to a stop, and never have cause to start moving again. But since we
have found that a self-mover is immortal, we should have no qualms about
declaring that this is the very essence and principle of a soul, for
every bodily object that is moved from outside has no soul, while a body
whose motion comes from within, from itself, does have a soul, that
being the nature of a soul; and if this is so---that whatever moves
itself is essentially a soul---then it follows necessarily that soul
should have neither birth nor death.

“That, then, is enough about the soul's immortality. Now
here is what we must say about its structure. To describe what the soul
actually is would require a very long account, altogether a task for a
god in every way; but to say what it is like is humanly possible and
takes less time. So let us do the second in our speech. Let us then
liken the soul to the natural union of a team of winged horses and their
charioteer. The gods have horses and charioteers that are themselves all
good and come from good stock besides, while everyone else has a
mixture. To begin with, our driver is in charge of a pair of horses;
second, one of his horses is beautiful and good and from stock of the
same sort, while the other is the opposite and has the opposite sort of
bloodline. This means that chariot-driving in our case is inevitably a
painfully difficult business.

“And now I should try to tell you why living things are said to include
both mortal and immortal beings. All soul looks after all that lacks a
soul, and patrols all of heaven, taking different shapes at
different times. So long as its wings are in perfect condition it flies
high, and the entire universe is its dominion; but a soul that sheds its
wings wanders until it lights on something solid, where it settles and
takes on an earthly body, which then, owing to the power of this soul,
seems to move itself. The whole combination of soul and body is called a
living thing, or animal, and has the designation ‘mortal' as well. Such
a combination cannot be immortal, not on any reasonable account. In fact
it is pure fiction, based neither on observation nor on adequate
reasoning, that a god is an immortal living thing which has a body and a
soul, and that these are bound together by nature for all time---but of
course we must let this be as it may please the gods, and speak
accordingly.

“Let us turn to what causes the shedding of the wings, what makes them
fall away from a soul. It is something of this sort: By their nature
wings have the power to lift up heavy things and raise them aloft where
the gods all dwell, and so, more than anything that pertains to the
body, they are akin to the divine, which has beauty, wisdom, goodness,
and everything of that sort. These nourish the soul's wings,
which grow best in their presence; but foulness and ugliness make the
wings shrink and disappear.

“Now Zeus, the great commander in heaven, drives his winged chariot
first in the procession, looking after everything and putting all things
in order. Following him is an army of gods and spirits arranged in
eleven sections. Hestia is the only one who remains at
the home of the gods; all the rest of the twelve are lined up in
formation, each god in command of the unit to which he is assigned.
Inside heaven are many wonderful places from which to look and many
aisles which the blessed gods take up and back, each seeing to his own
work, while anyone who is able and wishes to do so follows along, since
jealousy has no place in the gods' chorus. When they go to feast at the
banquet they have a steep climb to the high tier at the rim of
heaven; on this slope the gods' chariots move easily, since they are
balanced and well under control, but the other chariots barely make it.
The heaviness of the bad horse drags its charioteer toward the earth and
weighs him down if he has failed to train it well, and this causes the
most extreme toil and struggle that a soul will face. But when the souls
we call immortals reach the top, they move outward and take their stand
on the high ridge of heaven, where its circular motion carries 
them around as they stand while they gaze upon what is outside heaven.

“The place beyond heaven---none of our earthly poets has ever sung or
ever will sing its praises enough! Still, this is the way it is---risky
as it may be, you see, I must attempt to speak the truth, especially
since the truth is my subject. What is in this place is without color
and without shape and without solidity, a being that really is what it
is, the subject of all true knowledge, visible only to intelligence, the
soul's steersman. Now a god's mind is nourished by intelligence
and pure knowledge, as is the mind of any soul that is concerned to take
in what is appropriate to it, and so it is delighted at last to be
seeing what is real and watching what is true, feeding on all this and
feeling wonderful, until the circular motion brings it around to where
it started. On the way around it has a view of Justice as it is; it has
a view of Self-control; it has a view of Knowledge---not the knowledge
that is close to change, that becomes different as it knows the
different things which we consider real down here. No, it is the
knowledge of what really is what it is. And when the soul has
seen all the things that are as they are and feasted on them, it sinks
back inside heaven and goes home. On its arrival, the charioteer stables
the horses by the manger, throws in ambrosia, and gives them nectar to
drink besides.

“Now that is the life of the gods. As for the other
souls, one that follows a god most closely, making itself most like that
god, raises the head of its charioteer up to the place outside and is
carried around in the circular motion with the others. Although
distracted by the horses, this soul does have a view of Reality, just
barely. Another soul rises at one time and falls at another, and because
its horses pull it violently in different directions, it sees some real
things and misses others. The remaining souls are all eagerly straining
to keep up, but are unable to rise; they are carried around below the
surface, trampling and striking one another as each tries to get
ahead of the others. The result is terribly noisy, very sweaty, and
disorderly. Many souls are crippled by the incompetence of the drivers,
and many wings break much of their plumage. After so much trouble, they
all leave without having seen reality, uninitiated, and when they have
gone they will depend on what they think is nourishment---their own
opinions.

“The reason there is so much eagerness to see the plain where truth 
stands is that this pasture has the grass that is the right food
for the best part of the soul, and it is the nature of the wings that
lift up the soul to be nourished by it. Besides, the law of Destiny is
this: If any soul becomes a companion to a god and catches sight of any
true thing, it will be unharmed until the next circuit; and if it is
able to do this every time, it will always be safe. If, on the other
hand, it does not see anything true because it could not keep up, and by
some accident takes on a burden of forgetfulness and wrongdoing, then it
is weighed down, sheds its wings and falls to earth. At that
point, according to the law, the soul is not born into a wild animal in
its first incarnation; but a soul that has seen the most will be planted
in the seed of a man who will become a lover of
wisdom\endnote{I.e., a philosopher.} or of beauty,
or who will be cultivated in the arts and prone to erotic love. The
second sort of soul will be put into someone who will be a lawful king
or warlike commander; the third, a statesman, a manager of a household,
or a financier; the fourth will be a trainer who loves exercise or a
doctor who cures the body; the fifth will lead the life of a
prophet or priest of the mysteries. To the sixth the life of a poet or
some other representational artist is properly assigned; to the seventh
the life of a manual laborer or farmer; to the eighth the career of a
sophist or demagogue, and to the ninth a tyrant.

“Of all these, any who have led their lives with justice will change to
a better fate, and any who have led theirs with injustice, to a worse
one. In fact, no soul returns to the place from which it came for ten
thousand years, since its wings will not grow before
then, except for the soul of a man who practices philosophy without
guile or who loves boys philosophically. If, after the third cycle of
one thousand years, the last-mentioned souls have chosen such a life
three times in a row, they grow their wings back, and they depart in the
three-thousandth year. As for the rest, once their first life is over,
they come to judgment; and, once judged, some are condemned to go to
places of punishment beneath the earth and pay the full penalty for
their injustice, while the others are lifted up by justice to a place in
heaven where they live in the manner the life they led in human 
form has earned them. In the thousandth year both groups arrive at a
choice and allotment of second lives, and each soul chooses the life it
wants. From there, a human soul can enter a wild animal, and a soul that
was once human can move from an animal to a human being again. But a
soul that never saw the truth cannot take a human shape, since a human
being must understand speech in terms of general forms, proceeding to 
bring many perceptions together into a reasoned
unity. That process is
the recollection of the things our soul saw when it was traveling with
god, when it disregarded the things we now call real and lifted up its
head to what is truly real instead.

“For just this reason it is fair that only a philosopher's mind grows
wings, since its memory always keeps it as close as possible to those
realities by being close to which the gods are divine. A man who uses
reminders of these things correctly is always at the highest, most
perfect level of initiation, and he is the only one who is perfect as
perfect can be. He stands outside human concerns and draws close to the
divine; ordinary people think he is disturbed and rebuke him for
this, unaware that he is possessed by god. Now this takes me to the
whole point of my discussion of the fourth kind of madness---that which
someone shows when he sees the beauty we have down here and is reminded
of true beauty; then he takes wing and flutters in his eagerness to rise
up, but is unable to do so; and he gazes aloft, like a bird, paying no
attention to what is down below---and that is what brings on him the
charge that he has gone mad. This is the best and noblest of all
the forms that possession by god can take for anyone who has it or is
connected to it, and when someone who loves beautiful boys is touched by
this madness he is called a lover. As I said, nature requires that the
soul of every human being has seen reality; otherwise, no soul could
have entered this sort of living thing. But not every 
soul is easily reminded of the reality there by what it finds here---not
souls that got only a brief glance at the reality there, not souls who
had such bad luck when they fell down here that they were twisted by bad
company into lives of injustice so that they forgot the sacred objects
they had seen before. Only a few remain whose memory is good enough; and
they are startled when they see an image of what they saw up there. Then
they are beside themselves, and their experience is beyond their
comprehension because they cannot fully grasp what it is that they are
seeing.

“Justice and self-control do not shine out through their images down
here, and neither do the other objects of the soul's admiration; the
senses are so murky that only a few people are able to make out, with
difficulty, the original of the likenesses they encounter here. But
beauty was radiant to see at that time when the souls, along with the
glorious chorus (we\endnote{I.e., we philosophers.}
were with Zeus, while others followed other gods), saw that blessed and
spectacular vision and were ushered into the mystery that we may rightly 
call the most blessed of all. And we who celebrated it were
wholly perfect and free of all the troubles that awaited us in time to
come, and we gazed in rapture at sacred revealed objects that were
perfect, and simple, and unshakeable and blissful. That was the ultimate
vision, and we saw it in pure light because we were pure ourselves, not
buried in this thing we are carrying around now, which we call a body,
locked in it like an oyster in its shell.

“Well, all that was for love of a memory that made me stretch out my 
speech in longing for the past. Now beauty, as I said, was
radiant among the other objects; and now that we have come down here we
grasp it sparkling through the clearest of our senses. Vision, of
course, is the sharpest of our bodily senses, although it does not see
wisdom. It would awaken a terribly powerful love if an image of wisdom
came through our sight as clearly as beauty does, and the same goes for
the other objects of inspired love. But now beauty alone has
this privilege, to be the most clearly visible and the most loved. Of
course a man who was initiated long ago or who has become defiled is not
to be moved abruptly from here to a vision of Beauty itself when he sees
what we call beauty here; so instead of gazing at the latter reverently,
he surrenders to pleasure and sets out in the manner of a four-footed
beast, eager to make babies; and, wallowing in vice, he 
goes after unnatural pleasure too, without a trace of fear or shame. A
recent initiate, however, one who has seen much in heaven---when he sees
a godlike face or bodily form that has captured Beauty well, first he
shudders and a fear comes over him like those he felt at the earlier
time; then he gazes at him with the reverence due a god, and if he
weren't afraid people would think him completely mad, he'd even
sacrifice to his boy as if he were the image of a god. Once he
has looked at him, his chill gives way to sweating and a high fever,
because the stream of beauty that pours into him through his eyes warms
him up and waters the growth of his wings. Meanwhile, the heat warms him
and melts the places where the wings once grew, places that were long
ago closed off with hard scabs to keep the sprouts from coming back; but
as nourishment flows in, the feather shafts swell and rush to grow from
their roots beneath every part of the soul (long ago, you see, the
entire soul had wings). Now the whole soul seethes and throbs in
this condition. Like a child whose teeth are just starting to grow in,
and its gums are all aching and itching---that is exactly how the soul
feels when it begins to grow wings. It swells up and aches and tingles
as it grows them. But when it looks upon the beauty of the boy and takes
in the stream of particles flowing into it from his beauty (that is why
this is called
‘desire'\endnote{“Desire” is {\em himeros}:
the derivation is from {\em merē} (“particles”), {\em ienai} (“go”) and
{\em rhein} (“flow”).}), when it is
watered and warmed by this, then all its pain subsides and is replaced
by joy. When, however, it is separated from the boy and runs
dry, then the openings of the passages in which the feathers grow are
dried shut and keep the wings from sprouting. Then the stump of each
feather is blocked in its desire and it throbs like a pulsing artery
while the feather pricks at its passageway, with the result that the
whole soul is stung all around, and the pain simply drives it wild---but
then, when it remembers the boy in his beauty, it recovers its joy. From
the outlandish mix of these two feelings---pain and joy---comes anguish
and helpless raving: in its madness the lover's soul cannot 
sleep at night or stay put by day; it rushes, yearning, wherever it
expects to see the person who has that beauty. When it does see him, it
opens the sluice-gates of desire and sets free the parts that were
blocked up before. And now that the pain and the goading have stopped,
it can catch its breath and once more suck in, for the moment, this
sweetest of all pleasures. This it is not at all willing to give up, and
no one is more important to it than the beautiful boy.
It forgets mother and brothers and friends entirely and doesn't care at
all if it loses its wealth through neglect. And as for proper and
decorous behavior, in which it used to take pride, the soul despises the
whole business. Why, it is even willing to sleep like a slave, anywhere,
as near to the object of its longing as it is allowed to get! That is
because in addition to its reverence for one who has such beauty, the 
soul has discovered that the boy is the only doctor for all that
terrible pain.

“This is the experience we humans call love, you beautiful boy (I mean
the one to whom I am making this
speech). You are so
young that what the gods call it is likely to strike you as funny. Some
of the successors of Homer, I believe, report two lines from the less
well known poems, of which the second is quite indecent and does not
scan very well. They praise love this way:\crlf
\crlf

{\em Yes, mortals call him powerful winged ‘Love';}

{\em But because of his need to thrust out the wings,} {\em the gods
call him
‘Shove.'}\endnote{The lines are probably
Plato's invention, as the language is not consistently Homeric. The pun
in the original is on {\em erōs} and {\em pterōs} (“the winged one”).}\crlf
\crlf

You may believe this or not as you like. But, seriously, the cause of
love is as I have said, and this is how lovers really feel.

“If the man who is taken by love used to be an attendant on Zeus, he
will be able to bear the burden of this feathered force with dignity.
But if it is one of Ares' troops who has fallen prisoner of love---if
that is the god with whom he took the circuit---then if he has the
slightest suspicion that the boy he loves has done him wrong, he turns
murderous, and he is ready to make a sacrifice of himself as well as the
boy.

“So it is with each of the gods: everyone spends his life
honoring the god in whose chorus he danced, and emulates that god in
every way he can, so long as he remains undefiled and in his first life
down here. And that is how he behaves with everyone at every turn, not
just with those he loves. Everyone chooses his love after his own
fashion from among those who are beautiful, and then treats the
boy like his very own god, building him up and adorning him as an image
to honor and worship. Those who followed Zeus, for example, choose
someone to love who is a Zeus himself in the nobility of his soul. So
they make sure he has a talent for philosophy and the guidance of
others, and once they have found him and are in love with him they do
everything to develop that talent. If any lovers have not yet embarked
on this practice, then they start to learn, using any source they can
and also making progress on their own. They are well equipped to track
down their god's true nature with their own resources
because of their driving need to gaze at the god, and as they are in
touch with the god by memory they are inspired by him and adopt his
customs and practices, so far as a human being can share a god's life.
For all of this they know they have the boy to thank, and so they love
him all the more; and if they draw their inspiration from Zeus, then,
like the Bacchants,\endnote{Bacchants were worshippers
of Dionysus who gained miraculous abilities when possessed by the
madness of their god.}
they pour it into the soul of the one they love in order to help
him take on as much of their own god's qualities as possible. Hera's
followers look for a kingly character, and once they have found him they
do all the same things for him. And so it is for followers of Apollo or
any other god: They take their god's path and seek for their own a boy
whose nature is like the god's; and when they have got him they emulate
the god, convincing the boy they love and training him to follow their
god's pattern and way of life, so far as is possible in each case. They
show no envy, no mean-spirited lack of generosity, toward the boy, but
make every possible effort to draw him into being totally like
themselves and the god to whom they are devoted. This, then, is any true
lover's heart's desire: if he follows that desire in the manner I
described, this friend who has been driven mad by love will secure a
consummation for the
one he has befriended that is as beautiful and blissful as I said---if,
of course, he captures him. Here, then, is how the captive is caught:

“Remember how we divided each soul in three at the beginning of our 
story---two parts in the form of horses and the third in that of
a charioteer? Let us continue with that. One of the horses, we said, is
good, the other not; but we did not go into the details of the goodness
of the good horse or the badness of the bad. Let us do that now. The
horse that is on the right, or nobler, side is upright in frame and well
jointed, with a high neck and a regal nose; his coat is white, his eyes
are black, and he is a lover of honor with modesty and self-control;
companion to true glory, he needs no whip, and is guided by verbal
commands alone. The other horse is a crooked great jumble of
limbs with a short bull-neck, a pug nose, black skin, and bloodshot
white eyes; companion to wild boasts and indecency, he is shaggy around
the ears---deaf as a post---and just barely yields to horsewhip and goad
combined. Now when the charioteer looks in the eye of love, his entire
soul is suffused with a sense of warmth and starts to fill with tingles
and the goading of desire. As for the horses, the one who is obedient to
the charioteer is still controlled, then as always, by its sense 
of shame, and so prevents itself from jumping on the
boy. The other one, however, no longer responds to the whip or the goad
of the charioteer; it leaps violently forward and does everything to
aggravate its yokemate and its charioteer, trying to make them go up to
the boy and suggest to him the pleasures of sex. At first the other two
resist, angry in their belief that they are being made to do
things that are dreadfully wrong. At last, however, when they see no end
to their trouble, they are led forward, reluctantly agreeing to do as
they have been told. So they are close to him now, and they are struck
by the boy's face as if by a bolt of lightning. When the charioteer sees
that face, his memory is carried back to the real nature of Beauty, and
he sees it again where it stands on the sacred pedestal next to
Self-control. At the sight he is frightened, falls over backwards
awestruck, and at the same time has to pull the reins back so fiercely
that both horses are set on their haunches, one falling back
voluntarily with no resistance, but the other insolent and quite
unwilling. They pull back a little further; and while one horse drenches
the whole soul with sweat out of shame and awe, the other---once it has
recovered from the pain caused by the bit and its fall---bursts into a
torrent of insults as soon as it has caught its breath, accusing its
charioteer and yokemate of all sorts of cowardice and unmanliness for
abandoning their position and their agreement. Now once more it
tries to make its unwilling partners advance, and gives in grudgingly
only when they beg it to wait till later. Then, when the promised time
arrives, and they are pretending to have forgotten, it reminds them; it
struggles, it neighs, it pulls them forward and forces them to approach
the boy again with the same proposition; and as soon as they are near,
it drops its head, straightens its tail, bites the bit, and pulls
without any shame at all. The charioteer is now struck with the same 
feelings as before, only worse, and he's falling back as he
would from a starting gate; and he violently yanks the bit back out of
the teeth of the insolent horse, only harder this time, so that he
bloodies its foul-speaking tongue and jaws, sets its legs and haunches
firmly on the ground, and ‘gives it over to
pain.' When the bad
horse has suffered this same thing time after time, it stops being so
insolent; now it is humble enough to follow the charioteer's warnings,
and when it sees the beautiful boy it dies of fright, with the result
that now at last the lover's soul follows its boy in reverence and awe.

“And because he is served with all the attentions due a
god by a lover who is not pretending otherwise but is truly in the
throes of love, and because he is by nature disposed to be a friend of
the man who is serving him (even if he has already been set against love
by school friends or others who say that it is shameful to associate
with a lover, and initially rejects the lover in consequence), as time
goes forward he is brought by his ripening age and a sense of
what must be to a point where he lets the man spend time with him. It is
a decree of fate, you see, that bad is never friends with bad, while
good cannot fail to be friends with good. Now that he allows his lover
to talk and spend time with him, and the man's good will is close at
hand, the boy is amazed by it as he realizes that all the friendship he
has from his other friends and relatives put together is nothing
compared to that of this friend who is inspired by a god.

“After the lover has spent some time doing this, staying near the boy
(and even touching him during sports and on other occasions), then the
spring that feeds the stream Zeus named ‘Desire' when he was in
love with Ganymede begins to flow mightily in the lover and is partly
absorbed by him, and when he is filled it overflows and runs away
outside him. Think how a breeze or an echo bounces back from a smooth
solid object to its source; that is how the stream of beauty goes back
to the beautiful boy and sets him aflutter. It enters through his eyes,
which are its natural route to the soul; there it waters the
passages for the wings, starts the wings growing, and fills the soul of
the loved one with love in return. Then the boy is in love, but has no
idea what he loves. He does not understand, and cannot explain, what has
happened to him. It is as if he had caught an eye disease from someone
else, but could not identify the cause; he does not realize that he is
seeing himself in the lover as in a mirror. So when the lover is near,
the boy's pain is relieved just as the lover's is, and when they are
apart he yearns as much as he is yearned for, because he has a
mirror image of love in him---‘backlove'---though he neither speaks nor
thinks of it as love, but as friendship. Still, his desire is nearly the
same as the lover's, though weaker: he wants to see, touch, kiss, and
lie down with him; and of course, as you might expect, he acts on these
desires soon after they occur.

“When they are in bed, the lover's undisciplined horse has a word to 
say to the charioteer---that after all its sufferings it
is entitled to a little fun. Meanwhile, the boy's bad horse has nothing
to say, but swelling with desire, confused, it hugs the lover and kisses
him in delight at his great good will. And whenever they are lying
together it is completely unable, for its own part, to deny the lover
any favor he might beg to have. Its yokemate, however, along with its
charioteer, resists such requests with modesty and reason. Now if the
victory goes to the better elements in both their minds, which lead them
to follow the assigned regimen of philosophy, their life here below is
one of bliss and shared understanding. They are modest and fully
in control of themselves now that they have enslaved the part that
brought trouble into the soul and set free the part that gave it virtue.
After death, when they have grown wings and become weightless, they have
won the first of three rounds in these, the true Olympic Contests. There
is no greater good than this that either human self-control or divine
madness can offer a man. If, on the other hand, they adopt a lower way
of living, with ambition in place of philosophy, then pretty soon when 
they are careless because they have been drinking or for some
other reason, the pair's undisciplined horses will catch their souls off
guard and together bring them to commit that act which ordinary people
would take to be the happiest choice of all; and when they have
consummated it once, they go on doing this for the rest of their lives,
but sparingly, since they have not approved of what they are doing with
their whole minds. So these two also live in mutual friendship (though
weaker than that of the philosophical pair), both while they are in love
and after they have passed beyond it, because they realize they
have exchanged such firm vows that it would be forbidden for them ever
to break them and become enemies. In death they are wingless when they
leave the body, but their wings are bursting to sprout, so the prize
they have won from the madness of love is considerable, because those
who have begun the sacred journey in lower heaven may not by law be sent
into darkness for the journey under the earth; their lives are bright
and happy as they travel together, and thanks to their love they
will grow wings together when the time comes.

“These are the rewards you will have from a lover's friendship, my boy,
and they are as great as divine gifts should be. A non-lover's
companionship, on the other hand, is diluted by human self-control; all
it pays are cheap, human dividends, and though the slavish attitude it
engenders in a friend's soul is widely praised as virtue, it tosses the
soul around for nine thousand years on the earth and
leads it, mindless, beneath it.

“So now, dear Love, this is the best and most beautiful
palinode we could
offer as payment for our debt, especially in view of the rather poetical
choice of words Phaedrus made me
use. Forgive us our
earlier speeches in return for this one; be kind and gracious toward my
expertise at love, which is your own gift to me: do not, out of anger,
take it away or disable it; and grant that I may be held in higher
esteem than ever by those who are beautiful. If Phaedrus and I
said anything that shocked you in our earlier speech, blame it on
Lysias, who was its father, and put a stop to his making speeches of
this sort; convert him to philosophy like his brother Polemarchus so
that his lover here may no longer play both sides as he does now, but
simply devote his life to Love through philosophical discussions.”

\sayphaedrus I join you in your prayer, Socrates. If this is really
best for us, may it come to pass. As to your speech, I admired it from
the moment you began: You managed it much better than your first one.
I'm afraid that Lysias' effort to match it is bound to fall flat, if of
course he even dares to try to offer a speech of his own. In fact, my
marvelous friend, a politician I know was only recently taking Lysias to
task for just that reason: All through his invective, he kept calling
him a “speech writer.” So perhaps his pride will keep him from writing
this speech for us.

\saysocrates Ah, what a foolish thing to say, young man. How wrong
you are about your friend: he can't be intimidated so easily! But
perhaps you thought the man who was taking him to task meant what he
said as a reproach?

\sayphaedrus He certainly seemed to, Socrates. In any case, you are surely
aware yourself that the most powerful and renowned politicians are
ashamed to compose speeches or leave any writings behind; they are
afraid that in later times they may come to be known as “sophists.”

\saysocrates Phaedrus, you don't understand the expression “Pleasant 
Bend”---it originally referred to the long bend of the
Nile.\endnote{Apparently this was a
familiar example of something named by language that means the
opposite---though called “pleasant” it was really a long, nasty bend.} And, besides the
bend, you also don't understand that the most ambitious politicians love
speechwriting and long for their writings to survive. In fact, when they
write one of their speeches, they are so pleased when people praise it
that they add at the beginning a list of its admirers everywhere.

\sayphaedrus What do you mean? I don't understand.

\saysocrates Don't you know that the first thing
politicians put in their
writings is the names
of their admirers?

\sayphaedrus How so?

\saysocrates “Resolved,” the author often begins, “by the Council” or “by
the People” or by both, and “So-and-so
said”\endnote{This is the standard form
for decisions, including legislation, made by the assembly of Athens,
though it is not the standard beginning for even the most political of
speeches.}---meaning
himself, the writer, with great solemnity and self-importance. Only then
does he go on with what he has to say, showing off his wisdom to his
admirers, often composing a very long document. Do you think there's any
difference between that and a written speech?

\sayphaedrus No, I don't.

\saysocrates Well, then, if it remains on the books, he is delighted and
leaves the stage a poet. But if it is struck down, if he fails as a
speech writer and isn't considered worthy of having his work written
down, he goes into deep mourning, and his friends along with him.

\sayphaedrus He certainly does.

\saysocrates Clearly, then, they don't feel contempt for speechwriting; on
the contrary, they are in awe of it.

\sayphaedrus Quite so.

\saysocrates There's this too. What of an orator or a king who acquires 
enough power to match Lycurgus, Solon, or Darius as a
lawgiver\endnote{Lycurgus was the legendary
lawgiver of Sparta. Solon reformed the constitution of Athens in the
early sixth century B.C. and was revered by both democrats and their
opponents. Darius was king of Persia (521--486 B.C.). None of these was
famous as a speech writer.} and acquires
immortal fame as a speech writer in his city? Doesn't he think that he
is equal to the gods while he is still alive? And don't those who live
in later times believe just the same about him when they behold his
writings?

\sayphaedrus Very much so.

\saysocrates Do you really believe then that any one of these people,
whoever he is and however much he hates Lysias, would reproach him for
being a writer?

\sayphaedrus It certainly isn't likely in view of what you said, for he
would probably be reproaching his own ambition as well.

\saysocrates This, then, is quite clear: Writing speeches is not in itself
a shameful thing.

\sayphaedrus How could it be?

\saysocrates It's not speaking or writing well that's shameful; what's
really shameful is to engage in either of them shamefully or badly.

\sayphaedrus That is clear.

\saysocrates So what distinguishes good from bad writing? Do we need to ask
this question of Lysias or anyone else who ever did or will write
anything---whether a public or a private document, poetic verse or plain
prose?

\sayphaedrus You ask if we need to? Why else should one live, I say, if 
not for pleasures of this sort? Certainly not for those you
cannot feel unless you are first in pain, like most of the pleasures of
the body, and which for this reason we call the pleasures of slaves.

\saysocrates It seems we clearly have the time. Besides, I think that the
cicadas, who are singing and carrying on conversations with one another 
in the heat of the day above our heads, are also
watching us. And if they saw the two of us avoiding conversation at
midday like most people, diverted by their song and, sluggish of mind,
nodding off, they would have every right to laugh at us, convinced that
a pair of slaves had come to their resting place to sleep like sheep
gathering around the spring in the afternoon. But if they see us in
conversation, steadfastly navigating around them as if they were
the Sirens, they will be very pleased and immediately give us the gift
from the gods they are able to give to mortals.

\sayphaedrus What is this gift? I don't think I have heard of it.

\saysocrates Everyone who loves the Muses should have heard of this. The
story goes that the cicadas used to be human beings who lived before the
birth of the Muses. When the Muses were born and song was created for
the first time, some of the people of that time were so overwhelmed 
with the pleasure of singing that they forgot to eat or drink;
so they died without even realizing it. It is from them that the race of
the cicadas came into being; and, as a gift from the Muses, they have no
need of nourishment once they are born. Instead, they immediately burst
into song, without food or drink, until it is time for them to die.
After they die, they go to the Muses and tell each one of them which
mortals have honored her. To Terpsichore they report those who
have honored her by their devotion to the dance and thus make them
dearer to her. To Erato, they report those who honored her by dedicating
themselves to the affairs of love, and so too with the other Muses,
according to the activity that honors each. And to Calliope, the oldest
among them, and Urania, the next after her, who preside over the heavens
and all discourse, human and divine, and sing with the sweetest voice,
they report those who honor their special kind of music by leading a
philosophical life.

There are many reasons, then, why we should talk and not waste our
afternoon in sleep.

\sayphaedrus By all means, let's talk.

\saysocrates Well, then, we ought to examine the topic we proposed
just now: When is a speech well written and delivered, and when is it
not?

\sayphaedrus Plainly.

\saysocrates Won't someone who is to speak well and nobly have to have in
mind the truth about the subject he is going to discuss?

\sayphaedrus What I have actually heard about this, Socrates, my friend, 
is that it is not necessary for the intending orator to
learn what is really just, but only what will seem just to the crowd who
will act as judges. Nor again what is really good or noble, but only
what will seem so. For that is what persuasion proceeds from, not truth.

\saysocrates Anything that wise men say, Phaedrus, “is not lightly to be
cast aside”;\endnote{{\em Iliad} ii.361} we must
consider whether it might be right. And what you just said, in
particular, must not be dismissed.

\sayphaedrus You're right.

\saysocrates Let's look at it this way, then.

\sayphaedrus How?

\saysocrates Suppose I were trying to convince you that you should
fight your enemies on horseback, and neither one of us knew what a horse
is, but I happened to know this much about you, that Phaedrus believes a
horse is the tame animal with the longest ears---

\sayphaedrus But that would be ridiculous, Socrates.

\saysocrates Not quite yet, actually. But if I were seriously trying to
convince you, having composed a speech in praise of the donkey in which
I called it a horse and claimed that having such an animal is of immense
value both at home and in military service, that it is good for fighting
and for carrying your baggage and that it is useful for much else
besides---

\sayphaedrus Well, that would be totally ridiculous.

\saysocrates Well, which is better? To be ridiculous and a friend? Or
clever and an enemy?

\sayphaedrus The former.

\saysocrates And so, when a rhetorician who does not know good from bad
addresses a city which knows no better and attempts to sway it, not
praising a miserable donkey as if it were a horse, but bad as if it were
good, and, having studied what the people believe, persuades them to do
something bad instead of good---with that as its seed, what sort of crop 
do you think rhetoric can harvest?

\sayphaedrus A crop of really poor quality.

\saysocrates But could it be, my friend, that we have mocked the art of
speaking more rudely than it deserves? For it might perhaps reply, “What
bizarre nonsense! Look, I am not forcing anyone to learn how to make
speeches without knowing the truth; on the contrary, my advice, for what
it is worth, is to take me up only after mastering the truth. But I do
make this boast: even someone who knows the truth couldn't produce
conviction on the basis of a systematic art without me.”

\sayphaedrus Well, is that a fair reply?

\saysocrates Yes, it is---if, that is, the arguments now advancing upon
rhetoric testify that it is an art. For it seems to me as if I hear
certain arguments approaching and protesting that that is a lie and that
rhetoric is not an art but an artless
practice.\endnote{For a criticism of
rhetoric as not an art, see {\em Gorgias} 462b--c.} As the
Spartan said, there is no genuine art of speaking without a grasp of
truth, and there never will be.

\sayphaedrus We need to hear these arguments, Socrates. Come, produce
them, and examine them: What is their point? How do they
make it?

\saysocrates Come to us, then, noble creatures; convince Phaedrus, him of
the beautiful
offspring,\endnote{Cf. 242a--b;
{\em Symposium} 209b--e.} that unless
he pursues philosophy properly he will never be able to make a proper
speech on any subject either. And let Phaedrus be the one to answer.

\sayphaedrus Let them put their questions.

\saysocrates Well, then, isn't the rhetorical art, taken as a whole, a way
of directing the soul by means of speech, not only in the lawcourts and
on other public occasions but also in private? Isn't it one and the same
art whether its subject is great or small, and no more to be held in
esteem---if it is followed correctly---when its questions are
serious than when they are trivial? Or what have you heard about all
this?

\sayphaedrus Well, certainly not what {\em you} have! Artful speaking and
writing is found mainly in the lawcourts; also perhaps in the Assembly.
That's all I've heard.

\saysocrates Well, have you only heard of the rhetorical treatises of
Nestor and Odysseus---those they wrote in their spare time in Troy?
Haven't you also heard of the works of
Palamedes?\endnote{Nestor and Odysseus are
Homeric heroes known for their speaking ability. Palamedes, who does not
figure in Homer, was proverbial for his cunning.}

\sayphaedrus No, by Zeus, I haven't even heard of Nestor's---unless
by Nestor you mean Gorgias, and by Odysseus, Thrasymachus or
Theodorus.\endnote{Gorgias of Leontini was
the most famous teacher of rhetoric to visit Athens. About Thrasymachus
of Chalcedon (cf. 267c) we know little beyond what we can infer from his
appearance in Book 1 of the {\em Republic}. On Theodorus of Byzantium
(not to be confused with the geometer who appears in the
{\em Theaetetus}) see 266e and Aristotle {\em Rhetoric} 3.13.5.}

\saysocrates Perhaps. But let's leave these people aside. Answer this
question yourself: What do adversaries do in the lawcourts? Don't they
speak on opposite sides? What else can we call what they do?

\sayphaedrus That's it, exactly.

\saysocrates About what is just and what is unjust?

\sayphaedrus Yes.

\saysocrates And won't whoever does this artfully make the same
thing appear to the same people sometimes just and sometimes, when he
prefers, unjust?

\sayphaedrus Of course.

\saysocrates And when he addresses the Assembly, he will make the city
approve a policy at one time as a good one, and reject it---the very
same policy---as just the opposite at another.

\sayphaedrus Right.

\saysocrates Now, don't we know that the Eleatic Palamedes is such an
artful speaker that his listeners will perceive the same things to be
both similar and dissimilar, both one and many, both at rest and also in
motion?\endnote{The Eleatic Palamedes is
presumably Zeno of Elea, the author of the famous paradoxes about
motion.}

\sayphaedrus Most certainly.

\saysocrates We can therefore find the practice of speaking on opposite
sides not only in the lawcourts and in the Assembly. Rather, it
seems that one single art---if, of course, it is an art in the first
place---governs all speaking. By means of it one can make out as similar
anything that can be so assimilated, to everything to which it can be
made similar, and expose anyone who tries to hide the fact that that is
what he is doing.

\sayphaedrus What do you mean by that?

\saysocrates I think it will become clear if we look at it this way. Where
is deception most likely to occur---regarding things that differ much or
things that differ little from one another?

\sayphaedrus Regarding those that differ little.

\saysocrates At any rate, you are more likely to escape detection, as you
shift from one thing to its opposite, if you proceed in small steps
rather than in large ones.

\sayphaedrus Without a doubt.

\saysocrates Therefore, if you are to deceive someone else and to avoid
deception yourself, you must know precisely the respects in which things
are similar and dissimilar to one another.

\sayphaedrus Yes, you must.

\saysocrates And is it really possible for someone who doesn't know what
each thing truly is to detect a similarity---whether large or
small---between something he doesn't know and anything else?

\sayphaedrus That is impossible.

\saysocrates Clearly, therefore, the state of being deceived and holding
beliefs contrary to what is the case comes upon people by reason of
certain similarities.

\sayphaedrus That is how it happens.

\saysocrates Could someone, then, who doesn't know what each thing is ever
have the art to lead others little by little through similarities away
from what is the case on each occasion to its opposite? Or could he
escape this being done to himself?

\sayphaedrus Never.

\saysocrates Therefore, my friend, the art of a speaker who doesn't know 
the truth and chases opinions instead is likely to be a
ridiculous thing---not an art at all!

\sayphaedrus So it seems.

\saysocrates So, shall we look for instances of what we called the artful
and the artless in the speech of Lysias you carried here and in our own
speeches?

\sayphaedrus That's the best thing to do---because, as it is, we are
talking quite abstractly, without enough examples.

\saysocrates In fact, by some chance the two speeches do, as it seems,
contain an example of the way in which someone who knows the truth 
can toy with his audience and mislead them. For my part,
Phaedrus, I hold the local gods responsible for this---also, perhaps,
the messengers of the Muses who are singing over our heads may have
inspired me with this gift: certainly {\em I} don't possess any art of
speaking.

\sayphaedrus Fine, fine. But explain what you mean.

\saysocrates Come, then---read me the beginning of Lysias' speech.

\sayphaedrus “You understand my situation: I've told you how good it 
would be for us, in my opinion, if we could work this out. In
any case, I don't think I should lose the chance to get what I am asking
for, merely because I don't happen to be in love with you. A man in love
will wish he had not done you any favors---”

\saysocrates Stop. Our task is to say how he fails and writes artlessly.
Right?

\sayphaedrus Yes.

\saysocrates Now isn't this much absolutely clear: We are in accord with
one another about some of the things we discourse about and in discord
about others?

\sayphaedrus I think I understand what you are saying; but, please, can you
make it a little clearer?

\saysocrates When someone utters the word “iron” or “silver,” don't we all
think of the same thing?

\sayphaedrus Certainly.

\saysocrates But what happens when we say “just” or “good”? Doesn't each
one of us go in a different direction? Don't we differ with one another
and even with ourselves?

\sayphaedrus We certainly do.

\saysocrates Therefore, we agree about the former and disagree
about the latter.

\sayphaedrus Right.

\saysocrates Now in which of these two cases are we more easily deceived?
And when does rhetoric have greater power?

\sayphaedrus Clearly, when we wander in different directions.

\saysocrates It follows that whoever wants to acquire the art of rhetoric
must first make a systematic division and grasp the particular character
of each of these two kinds of thing, both the kind where most people
wander in different directions and the kind where they do not.

\sayphaedrus What a splendid thing, Socrates, he will have
understood if he grasps {\em that!}

\saysocrates Second, I think, he must not be mistaken about his subject; he
must have a sharp eye for the class to which whatever he is about to
discuss belongs.

\sayphaedrus Of course.

\saysocrates Well, now, what shall we say about love? Does it belong to the
class where people differ or to that where they don't?

\sayphaedrus Oh, surely the class where they differ. Otherwise, do you
think you could have spoken of it as you did a few minutes ago, first
saying that it is harmful both to lover and beloved and then immediately
afterward that it is the greatest good?

\saysocrates Very well put. But now tell me this---I can't remember
at all because I was completely possessed by the gods: Did I define love
at the beginning of my speech?

\sayphaedrus Oh, absolutely, by Zeus, you most certainly did.

\saysocrates Alas, how much more artful with speeches the Nymphs, daughters
of Achelous, and Pan, son of Hermes, are, according to what you say,
than Lysias, son of Cephalus! Or am I wrong? Did Lysias too, at 
the start of his love-speech, compel us to assume that love is the
single thing that he himself wanted it to be? Did he then complete his
speech by arranging everything in relation to that? Will you read its
opening once again?

\sayphaedrus If you like. But what you are looking for is not there.

\saysocrates Read it, so that I can hear it in his own words.

\sayphaedrus “You understand my situation: I've told you how good it would
be for us, in my opinion, if we could work this out. In any case, I
don't think I should lose the chance to get what I am asking for, merely
because I don't happen to be in love with you. A man in
love will wish he had not done you any favors, once his desire dies
down---”

\saysocrates He certainly seems a long way from doing what we wanted. He
doesn't even start from the beginning but from the end, making his
speech swim upstream on its back. His first words are what a lover would
say to his boy as he was concluding his speech. Am I wrong, Phaedrus,
dear heart?

\sayphaedrus Well, Socrates, that was the end for which he gave the speech!

\saysocrates And what about the rest? Don't the parts of the speech appear
to have been thrown together at random? Is it evident that the second
point had to be made second for some compelling reason? Is that so for
any of the parts? I at least---of course I know nothing about such
matters---thought the author said just whatever came to mind next,
though not without a certain noble willfulness. But you, do you know any
principle of speech-composition compelling him to place these things one
after another in this order?

\sayphaedrus It's very generous of you to think that I can understand his reasons so clearly.

\saysocrates But surely you will admit at least this much: Every speech
must be put together like a living creature, with a body of its own; it
must be neither without head nor without legs; and it must have a middle
and extremities that are fitting both to one another and to the whole
work.

\sayphaedrus How could it be otherwise?

\saysocrates But look at your friend's speech: Is it like that or is it
otherwise? Actually, you'll find that it's just like the epigram people
say is inscribed on the tomb of Midas the Phrygian.

\sayphaedrus What epigram is that? And what's the matter with it?

\saysocrates It goes like this:\crlf
\crlf

{\em A maid of bronze am I, on Midas' tomb I lie}

{\em As long as water flows, and trees grow tall}

{\em Shielding the grave where many come to cry}

{\em That Midas rests here I say to one and all.}\crlf
\crlf

I'm sure you notice that it makes no difference at all which of its
verses comes first, and which last.

\sayphaedrus You are making fun of our speech, Socrates.

\saysocrates Well, then, if that upsets you, let's leave that speech
aside---even though I think it has plenty of very useful examples,
provided one tries to emulate them as little as possible---and turn to
the others. I think it is important for students of speechmaking to pay
attention to one of their features.

\sayphaedrus What do you mean?

\saysocrates They were in a way opposite to one another. One claimed that
one should give one's favors to the lover; the other, to the non-lover.

\sayphaedrus Most manfully, too.

\saysocrates I thought you were going to say “madly,” which would have been
the truth, and is also just what I was looking for: We did say, didn't
we, that love is a kind of madness?

\sayphaedrus Yes.

\saysocrates And that there are two kinds of madness, one produced by human
illness, the other by a divinely inspired release from normally accepted
behavior?

\sayphaedrus Certainly.

\saysocrates We also distinguished four parts within the divine kind and
connected them to four gods. Having attributed the inspiration of the
prophet to Apollo, of the mystic to Dionysus, of the poet to the Muses,
and the fourth part of madness to Aphrodite and to Love, we said that
the madness of love is the best. We used a certain sort of image to
describe love's passion; perhaps it had a measure of truth in it, though
it may also have led us astray. And having whipped up a not altogether
implausible speech, we sang playfully, but also appropriately
and respectfully, a story-like hymn to my master and yours,
Phaedrus---to Love, who watches over beautiful boys.

\sayphaedrus And I listened to it with the greatest pleasure.

\saysocrates Let's take up this point about it right away: How was the
speech able to proceed from censure to praise?

\sayphaedrus What exactly do you mean by that?

\saysocrates Well, everything else in it really does appear to me to have
been spoken in play. But part of it was given with Fortune's guidance, 
and there were in it two kinds of things the nature of which it
would be quite wonderful to grasp by means of a systematic art.

\sayphaedrus Which things?

\saysocrates The first consists in seeing together things that are
scattered about everywhere and collecting them into one kind, so that by
defining each thing we can make clear the subject of any instruction we
wish to give. Just so with our discussion of love: Whether its
definition was or was not correct, at least it allowed the speech to
proceed clearly and consistently with itself.

\sayphaedrus And what is the other thing you are talking about, Socrates?

\saysocrates This, in turn, is to be able to cut up each kind
according to its species along its natural joints, and to try not to
splinter any part, as a bad butcher might do. In just this way, our two
speeches placed all mental derangements into one common
kind. Then, just as each single body has parts that naturally come in
pairs of the same name (one of them being called the right-hand and the
other the left-hand one), so the speeches, having considered unsoundness
of mind to be by nature one single kind within us, proceeded to cut it
up---the first speech cut its left-hand part, and continued to cut until
it discovered among these parts a sort of love that can be called
“left-handed,” which it correctly denounced; the second speech, in turn,
led us to the right-hand part of madness; discovered a love that shares
its name with the other but is actually divine; set it out 
before us, and praised it as the cause of our greatest goods.

\sayphaedrus You are absolutely right.

\saysocrates Well, Phaedrus, I am myself a lover of these divisions and
collections, so that I may be able to think and to speak; and if I
believe that someone else is capable of discerning a single thing that
is also by nature capable of encompassing
many, I follow
“straight behind, in his tracks, as if he were a
god.”\endnote{{\em Odyssey} ii.406.} God knows
whether this is the right name for those who can do this correctly or
not, but so far I have always called them “dialecticians.” But
tell me what I must call them now that we have learned all this from
Lysias and you. Or is it just that art of speaking that Thrasymachus and
the rest of them use, which has made them masters of speechmaking and
capable of producing others like them---anyhow those who are willing to
bring them gifts and to treat them as if they were kings?

\sayphaedrus They may behave like kings, but they certainly lack the
knowledge you're talking about. No, it seems to me that you are right in
calling the sort of thing you mentioned dialectic; but, it seems to me,
rhetoric still eludes us.

\saysocrates What are you saying? Could there be anything valuable which 
is independent of the methods I mentioned and is still grasped
by art? If there is, you and I must certainly honor it, and we must say
what part of rhetoric it is that has been left out.

\sayphaedrus Well, there's quite a lot, Socrates: everything, at any rate,
written up in the books on the art of speaking.

\saysocrates You were quite right to remind me. First, I believe, there is
the Preamble with which a speech must begin. This is what you mean,
isn't it---the fine points of the art?

\sayphaedrus Yes.

\saysocrates Second come the Statement of Facts and the Evidence of
Witnesses concerning it; third, Indirect Evidence; fourth, Claims to
Plausibility. And I believe at least that excellent Byzantine
word-wizard adds Confirmation and Supplementary Confirmation.

\sayphaedrus You mean the worthy Theodorus?

\saysocrates Quite. And he also adds Refutation and Supplementary
Refutation, to be used both in prosecution and in
defense. Nor must we forget the most excellent Evenus of
Paros,\endnote{Evenus of Paros was active
as a sophist toward the end of the fifth century B.C. Only a few tiny
fragments of his work survive.} who was the
first to discover Covert Implication and Indirect Praise and who---some
say---has even arranged Indirect Censures in verse as an aid to memory:
a wise man indeed! And
Tisias\endnote{Tisias of Syracuse, with
Corax, is credited with the founding of the Sicilian school of rhetoric,
represented by Gorgias and Polus.} 
and Gorgias?
How can we leave them out when it is they who realized that what is
likely must be held in higher honor than what is true; they who, by the
power of their language, make small things appear great and great things
small; they who express modern ideas in ancient garb, and
ancient ones in modern dress; they who have discovered how to argue both
concisely and at infinite length about any subject? Actually, when I
told Prodicus\high{\goto{51}[phaedrus.htmlux5cux23phaedrfn_51]} this
last, he laughed and said that only he had discovered the art of proper
speeches: What we need are speeches that are neither long nor short but
of the right length.

\sayphaedrus Brilliantly done, Prodicus!

\saysocrates And what about
Hippias?\high{\goto{52}[phaedrus.htmlux5cux23phaedrfn_52]} How can we
omit him? I am sure our friend from Elis would cast his vote with
Prodicus.

\sayphaedrus Certainly.

\saysocrates And what shall we say of the whole gallery of terms
Polus\high{\goto{53}[phaedrus.htmlux5cux23phaedrfn_53]} set
up---speaking with Reduplication, Speaking in Maxims, Speaking in
Images---and of the terms Licymnius gave him as a present to help him
explain Good Diction?\high{\goto{54}[phaedrus.htmlux5cux23phaedrfn_54]}

\sayphaedrus But didn't Protagoras actually use similar
terms?\high{\goto{55}[phaedrus.htmlux5cux23phaedrfn_55]}

\saysocrates Yes, Correct Diction, my boy, and other wonderful things. As
to the art of making speeches bewailing the evils of poverty and old
age, the prize, in my judgment, goes to the mighty
Chalcedonian.\high{\goto{56}[phaedrus.htmlux5cux23phaedrfn_56]} He it is
also who knows best how to inflame a crowd and, once they are
inflamed, how to hush them again with his words' magic spell, as he says
himself. And let's not forget that he is as good at producing slander as
he is at refuting it, whatever its source may be.

As to the way of ending a speech, everyone seems to be in agreement,
though some call it Recapitulation and others by some other name.

\sayphaedrus You mean, summarizing everything at the end and reminding the
audience of what they've heard?

\saysocrates That's what I mean. And if you have anything else to add about
the art of speaking---

\sayphaedrus Only minor points, not worth making.

\saysocrates Well, let's leave minor points aside. Let's
hold what we do have closer to the light so that we can see precisely
the power of the art these things produce.

\sayphaedrus A very great power, Socrates, especially in front of a crowd.

\saysocrates Quite right. But now, my friend, look closely: Do you think,
as I do, that its fabric is a little threadbare?

\sayphaedrus Can you show me?

\saysocrates All right, tell me this. Suppose someone came to your friend
Eryximachus or his father Acumenus and said: “I know treatments to raise
or lower (whichever I prefer) the temperature of people's bodies; if I
decide to, I can make them vomit or make their bowels move, and
all sorts of things. On the basis of this knowledge, I claim to be a
physician; and I claim to be able to make others physicians as well by
imparting it to them.” What do you think they would say when they heard
that?

\sayphaedrus What could they say? They would ask him if he also knew to
whom he should apply such treatments, when, and to what extent.

\saysocrates What if he replied, “I have no idea. My claim is that whoever
learns from me will manage to do what you ask on his own“?

\sayphaedrus I think they'd say the man's mad if he thinks he's a doctor
just because he read a book or happened to come across a few potions; he
knows nothing of the art.

\saysocrates And suppose someone approached Sophocles and Euripides and
claimed to know how to compose the longest passages on trivial topics
and the briefest ones on topics of great importance, that he could make
them pitiful if he wanted, or again, by contrast, terrifying and
menacing, and so on. Suppose further that he believed that by
teaching this he was imparting the knowledge of composing tragedies---

\sayphaedrus Oh, I am sure they too would laugh at anyone who thought a
tragedy was anything other than the proper arrangement of these things:
They have to fit with one another and with the whole work.

\saysocrates But I am sure they wouldn't reproach him rudely. They would
react more like a musician confronted by a man who thought he had
mastered harmony because he was able to produce the highest and lowest 
notes on his strings. The musician would not say fiercely, “You
stupid man, you are out of your mind!” As befits his calling, he would
speak more gently: “My friend, though that too is necessary for
understanding harmony, someone who has gotten as far as you have may
still know absolutely nothing about the subject. What you know is what
it's necessary to learn before you study harmony, but not harmony
itself.”

\sayphaedrus That's certainly right.

\saysocrates So Sophocles would also tell the man who was showing off 
to them that he knew the preliminaries of tragedy, but
not the art of tragedy itself. And Acumenus would say his man knew the
preliminaries of medicine, but not medicine itself.

\sayphaedrus Absolutely.

\saysocrates And what if the “honey-tongued Adrastus” (or perhaps
Pericles)\high{\goto{57}[phaedrus.htmlux5cux23phaedrfn_57]} were to hear
of all the marvelous techniques we just discussed---Speaking Concisely
and Speaking in Images and all the rest we listed and proposed
to examine under the light? Would he be angry or rude, as you and I
were, with those who write of those techniques and teach them as if they
are rhetoric itself, and say something coarse to them? Wouldn't
he---being wiser than we are---reproach us as well and say, “Phaedrus
and Socrates, you should not be angry with these people---you should be
sorry for them. The reason they cannot define rhetoric is that they are
ignorant of dialectic. It is their ignorance that makes them think they
have discovered what rhetoric is when they have mastered only what it is necessary to learn as preliminaries. So they teach these
preliminaries and imagine their pupils have received a full course in
rhetoric, thinking the task of using each of them persuasively and
putting them together into a whole speech is a minor matter, to be
worked out by the pupils from their own resources“?

\sayphaedrus Really, Socrates, the art these men present as rhetoric in
their courses and handbooks is no more than what you say. In my
judgment, at least, your point is well taken. But how, from what
source, could one acquire the art of the true rhetorician, the really
persuasive speaker?

\saysocrates Well, Phaedrus, becoming good enough to be an accomplished
competitor is probably---perhaps necessarily---like everything else. If
you have a natural ability for rhetoric, you will become a famous
rhetorician, provided you supplement your ability with knowledge and
practice. To the extent that you lack any one of them, to that extent
you will be less than perfect. But, insofar as there is an art of
rhetoric, I don't believe the right method for acquiring it is to be
found in the direction Lysias and Thrasymachus have followed.

\sayphaedrus Where can we find it then?

\saysocrates My dear friend, maybe we can see now why Pericles was
in all likelihood the greatest rhetorician of all.

\sayphaedrus How is that?

\saysocrates All the great arts require endless talk and
ethereal speculation about nature: This seems to be what gives them
their lofty point of view and universal applicability. That's just what
Pericles mastered---besides having natural ability. He came across
Anaxagoras, who was just that sort of man, got his full dose of ethereal
speculation, and understood the nature of mind and
mindlessness\high{\goto{58}[phaedrus.htmlux5cux23phaedrfn_58]}---just
the subject on which Anaxagoras had the most to say. From this, I think,
he drew for the art of rhetoric what was useful to it.

\sayphaedrus What do you mean by that?

\saysocrates Well, isn't the method of medicine in a way the same
as the method of rhetoric?

\sayphaedrus How so?

\saysocrates In both cases we need to determine the nature of
something---of the body in medicine, of the soul in rhetoric. Otherwise,
all we'll have will be an empirical and artless practice. We won't be
able to supply, on the basis of an art, a body with the medicines and
diet that will make it healthy and strong, or a soul with the reasons
and customary rules for conduct that will impart to it the convictions
and virtues we want.

\sayphaedrus That is most likely, Socrates.

\saysocrates Do you think, then, that it is possible to reach a serious
understanding of the nature of the soul without understanding
the nature of the world as a whole?

\sayphaedrus Well, if we're to listen to Hippocrates, Asclepius'
descendant,\high{\goto{59}[phaedrus.htmlux5cux23phaedrfn_59]} we won't
even understand the body if we don't follow that method.

\saysocrates He speaks well, my friend. Still, Hippocrates aside, we must
consider whether argument supports that view.

\sayphaedrus I agree.

\saysocrates Consider, then, what both Hippocrates and true argument say
about nature. Isn't this the way to think systematically about the
nature of anything? First, we must consider whether the object
regarding which we intend to become experts and capable of transmitting
our expertise is simple or complex. Then, if it is simple, we must
investigate its power: What things does it have what natural power of
acting upon? By what things does it have what natural disposition to be
acted upon? If, on the other hand, it takes many forms, we must
enumerate them all and, as we did in the simple case, investigate how
each is naturally able to act upon what and how it has a natural
disposition to be acted upon by what.

\sayphaedrus It seems so, Socrates.

\saysocrates Proceeding by any other method would be like walking with 
the blind. Conversely, whoever studies anything on the basis of
an art must never be compared to the blind or the deaf. On the contrary,
it is clear that someone who teaches another to make speeches as an art
will demonstrate precisely the essential nature of that to which
speeches are to be applied. And that, surely, is the soul.

\sayphaedrus Of course.

\saysocrates This is therefore the object toward which the speaker's whole effort is directed, since it is in the soul that he
attempts to produce conviction. Isn't that so?

\sayphaedrus Yes.

\saysocrates Clearly, therefore, Thrasymachus and anyone else who teaches
the art of rhetoric seriously will, first, describe the soul with
absolute precision and enable us to understand what it is: whether it is
one and homogeneous by nature or takes many forms, like the shape of
bodies, since, as we said, that's what it is to demonstrate the nature
of something.

\sayphaedrus Absolutely.

\saysocrates Second, he will explain how, in virtue of its nature, it acts
and is acted upon by certain things.

\sayphaedrus Of course.

\saysocrates Third, he will classify the kinds of speech and of
soul there are, as well as the various ways in which they are affected,
and explain what causes each. He will then coordinate each kind of soul
with the kind of speech appropriate to it. And he will give instructions
concerning the reasons why one kind of soul is necessarily convinced by
one kind of speech while another necessarily remains unconvinced.

\sayphaedrus This, I think, would certainly be the best way.

\saysocrates In fact, my friend, no speech will ever be a product of art,
whether it is a model or one actually given, if it is delivered or
written in any other way---on this or on any other subject. But
those who now write {\em Arts of Rhetoric}---we were just discussing
them---are cunning people: they hide the fact that they know very well
everything about the soul. Well, then, until they begin to speak and
write in this way, we mustn't allow ourselves to be convinced that they
write on the basis of the art.

\sayphaedrus What way is that?

\saysocrates It's very difficult to speak the actual words, but as to how
one should write in order to be as artful as possible---that I am
willing to tell you.

\sayphaedrus Please do.

\saysocrates Since the nature of speech is in fact to direct the
soul, whoever intends to be a rhetorician must know how many kinds of
soul there are. Their number is so-and-so many; each is of such-and-such
a sort; hence some people have such-and-such a character and others have
such-and-such. Those distinctions established, there are, in turn,
so-and-so many kinds of speech, each of such-and-such a sort. People of
such-and-such a character are easy to persuade by speeches of
such-and-such a sort in connection with such-and-such an issue for this
particular reason, while people of such-and-such another sort are
difficult to persuade for those particular reasons.

The orator must learn all this well, then put his theory into practice
and develop the ability to discern each kind clearly as it
occurs in the actions of real life. Otherwise he won't be any better off
than he was when he was still listening to those discussions in school.
He will now not only be able to say what kind of person is convinced by
what kind of speech; on meeting someone he will be able
to discern what he is like and make clear to himself that the person
actually standing in front of him is of just this particular sort of
character he had learned about in school---to that he must now apply
speeches of such-and-such a kind in this particular way in order to
secure conviction about such-and-such an issue. When he has learned all
this---when, in addition, he has grasped the right occasions for
speaking and for holding back; and when he has also understood when the
time is right for Speaking Concisely or Appealing to Pity or
Exaggeration or for any other of the kinds of speech he has learned and
when it is not---then, and only then, will he have finally mastered the
art well and completely. But if his speaking, his teaching, or his
writing lacks any one of these elements and he still claims to
be speaking with art, you'll be better off if you don't believe him.

“Well, Socrates and Phaedrus,” the author of this discourse might say,
“do you agree? Could we accept an art of speaking presented in any other
terms?”

\sayphaedrus That would be impossible, Socrates. Still, it's evidently
rather a major undertaking.

\saysocrates You're right. And that's why we must turn all our arguments
every which way and try to find some easier and shorter route to the
art: we don't want to follow a long rough path for no good
reason when we can choose a short smooth one instead.

Now, try to remember if you've heard anything helpful from Lysias or
anybody else. Speak up.

\sayphaedrus It's not for lack of trying, but nothing comes to mind right
now.

\saysocrates Well, then, shall I tell you something I've heard people say
who care about this topic?

\sayphaedrus Of course.

\saysocrates We do claim, after all, Phaedrus, that it is fair to give the
wolf's side of the story as well.

\sayphaedrus That's just what you should do.

\saysocrates Well, these people say that there is no need to be so solemn
about all this and stretch it out to such lengths. For the fact is, as
we said ourselves at the beginning of this
discussion,\high{\goto{60}[phaedrus.htmlux5cux23phaedrfn_60]} that one
who intends to be an able rhetorician has no need to know the truth
about the things that are just or good or yet about the people who are
such either by nature or upbringing. No one in a lawcourt, you see,
cares at all about the truth of such matters. They only care about what
is convincing. This is called “the likely,” and that is what a
man who intends to speak according to art should concentrate on.
Sometimes, in fact, whether you are prosecuting or defending a case, you
must not even say what actually happened, if it was not likely to have
happened---you must say something that is likely instead. Whatever you
say, you should pursue what is likely and leave the truth aside: the
whole art consists in cleaving to that throughout your
speech.

\sayphaedrus That's an excellent presentation of what people say who
profess to be expert in speeches, Socrates. I recall that we raised this
issue briefly earlier on, but it seems to be their single most important
point.

\saysocrates No doubt you've churned through Tisias' book quite carefully.
Then let Tisias tell us this also: By “the likely” does he mean anything
but what is accepted by the crowd?

\sayphaedrus What else?

\saysocrates And it's likely it was when he discovered this clever and
artful technique that Tisias wrote that if a weak but spunky man is
taken to court because he beat up a strong but cowardly one and stole
his cloak or something else, neither one should tell the truth. The
coward must say that the spunky man didn't beat him up all by himself,
while the latter must rebut this by saying that only the two of
them were there, and fall back on that well-worn plea, “How could a man
like me attack a man like him?” The strong man, naturally, will not
admit his cowardice, but will try to invent some other lie, and may thus
give his opponent the chance to refute him. And in other cases, speaking
as the art dictates will take similar forms. Isn't that so, Phaedrus?

\sayphaedrus Of course.

\saysocrates Phew! Tisias---or whoever else it was and whatever name he
pleases to use for
himself\high{\goto{61}[phaedrus.htmlux5cux23phaedrfn_61]}---seems\high{\goto{62}[phaedrus.htmlux5cux23phaedrfn_62]}
to have discovered an art which he has disguised very well! But now, my
friend, shall we or shall we not say to him---

\sayphaedrus What?

\saysocrates This: “Tisias, some time ago, before you came into the
picture, we were saying that people get the idea of what is likely
through its similarity to the truth. And we just explained that in every
case the person who knows the truth knows best how to determine
similarities. So, if you have something new to say about the art of
speaking, we shall listen. But if you don't, we shall remain convinced
by the explanations we gave just before: No one will ever possess the
art of speaking, to the extent that any human being can, unless
he acquires the ability to enumerate the sorts of characters to be found
in any audience, to divide everything according to its kinds, and to
grasp each single thing firmly by means of one form. And no one can
acquire these abilities without great effort---a laborious effort a
sensible man will make not in order to speak and act among human beings,
but so as to be able to speak and act in a way that pleases the gods as
much as possible. Wiser people than ourselves, Tisias, say that a
reasonable man must put his mind to being pleasant not to his fellow 
slaves (though this may happen as a side effect) but to
his masters, who are wholly good. So, if the way round is long, don't be
astonished: we must make this detour for the sake of things that are
very important, not for what you have in mind. Still, as our argument
asserts, if that is what you want, you'll get it best as a result of
pursuing our own goal.

\sayphaedrus What you've said is wonderful, Socrates---if only it could be
done!

\saysocrates Yet surely whatever one must go through on the way to
an honorable goal is itself honorable.

\sayphaedrus Certainly.

\saysocrates Well, then, that's enough about artfulness and artlessness in
connection with speaking.

\sayphaedrus Quite.

\saysocrates What's left, then, is aptness and ineptness in connection with
writing: What feature makes writing good, and what inept? Right?

\sayphaedrus Yes.

\saysocrates Well, do you know how best to please god when you either use
words or discuss them in general?

\sayphaedrus Not at all. Do you?

\saysocrates I can tell you what I've heard the ancients said, though they 
alone know the truth. However, if we could discover that
ourselves, would we still care about the speculations of other people?

\sayphaedrus That's a silly question. Still, tell me what you say you've
heard.

\saysocrates Well, this is what I've heard. Among the ancient gods of
Naucratis\high{\goto{63}[phaedrus.htmlux5cux23phaedrfn_63]} in Egypt
there was one to whom the bird called the ibis is sacred. The name of
that divinity was
Theuth,\high{\goto{64}[phaedrus.htmlux5cux23phaedrfn_64]} and it was he
who first discovered number and calculation, geometry and astronomy, as
well as the games of checkers and dice, and, above all else,
writing.

Now the king of all Egypt at that time was
Thamus,\high{\goto{65}[phaedrus.htmlux5cux23phaedrfn_65]} who lived in
the great city in the upper region that the Greeks call Egyptian Thebes;
Thamus they call
Ammon.\high{\goto{66}[phaedrus.htmlux5cux23phaedrfn_66]} Theuth came to
exhibit his arts to him and urged him to disseminate them to all the
Egyptians. Thamus asked him about the usefulness of each art, and while
Theuth was explaining it, Thamus praised him for whatever he
thought was right in his explanations and criticized him for whatever he
thought was wrong.

The story goes that Thamus said much to Theuth, both for and against
each art, which it would take too long to repeat. But when they came to
writing, Theuth said: “O King, here is something that, once learned,
will make the Egyptians wiser and will improve their memory; I have
discovered a potion for memory and for wisdom.” Thamus, however,
replied: “O most expert Theuth, one man can give birth to the elements
of an art, but only another can judge how they can benefit or harm those
who will use them. And now, since you are the father of
writing, your affection for it has made you describe its effects as the
opposite of what they really are. In fact, it will introduce
forgetfulness into the soul of those who learn it: they will not
practice using their memory because they will put their trust in
writing, which is external and depends on signs that belong to others,
instead of trying to remember from the inside, completely on their own.
You have not discovered a potion for remembering, but for reminding; you
provide your students with the appearance of wisdom, not with its
reality. Your invention will enable them to hear many things without
being properly taught, and they will imagine that they have come
to know much while for the most part they will know nothing. And they
will be difficult to get along with, since they will merely appear to be
wise instead of really being so.”

\sayphaedrus Socrates, you're very good at making up stories from Egypt or
wherever else you want!

\saysocrates But, my friend, the priests of the temple of Zeus at Dodona
say that the first prophecies were the words of an oak. Everyone who
lived at that time, not being as wise as you young ones are today, found
it rewarding enough in their simplicity to listen to an oak or even a
stone, so long as it was telling the truth, while it seems to
make a difference to you, Phaedrus, who is speaking and where he comes
from. Why, though, don't you just consider whether what he says is right
or wrong?

\sayphaedrus I deserved that, Socrates. And I agree that the Theban king
was correct about writing.

\saysocrates Well, then, those who think they can leave written
instructions for an art, as well as those who accept them, thinking that
writing can yield results that are clear or certain, must be quite naive
and truly ignorant of Ammon's prophetic judgment: otherwise, how could
they possibly think that words that have been written down can
do more than remind those who already know what the writing is about?

\sayphaedrus Quite right.

\saysocrates You know, Phaedrus, writing shares a strange feature with
painting. The offsprings of painting stand there as if they are alive,
but if anyone asks them anything, they remain most solemnly silent. The
same is true of written words. You'd think they were speaking as if they
had some understanding, but if you question anything that has been said
because you want to learn more, it continues to signify just that very
same thing forever. When it has once been written down, every
discourse roams about everywhere, reaching indiscriminately those with
understanding no less than those who have no business with it, and it
doesn't know to whom it should speak and to whom it should not. And when
it is faulted and attacked unfairly, it always needs its father's
support; alone, it can neither defend itself nor come to its own
support.

\sayphaedrus You are absolutely right about that, too.

\saysocrates Now tell me, can we discern another kind of
discourse, a legitimate brother of this one? Can we say how it comes
about, and how it is by nature better and more capable?

\sayphaedrus Which one is that? How do you think it comes about?

\saysocrates It is a discourse that is written down, with knowledge, in the
soul of the listener; it can defend itself, and it knows for whom it
should speak and for whom it should remain silent.

\sayphaedrus You mean the living, breathing discourse of the man who knows,
of which the written one can be fairly called an image.

\saysocrates Absolutely right. And tell me this. Would a sensible farmer, 
who cared about his seeds and wanted them to yield fruit, plant
them in all seriousness in the gardens of Adonis in the middle of the
summer and enjoy watching them bear fruit within seven days? Or would he
do this as an amusement and in honor of the holiday, if he did it at
all?\high{\goto{67}[phaedrus.htmlux5cux23phaedrfn_67]} Wouldn't he use
his knowledge of farming to plant the seeds he cared for when it was
appropriate and be content if they bore fruit seven months later?

\sayphaedrus That's how he would handle those he was serious about, 
Socrates, quite differently from the others, as you say.

\saysocrates Now what about the man who knows what is just, noble, and
good? Shall we say that he is less sensible with his seeds than the
farmer is with his?

\sayphaedrus Certainly not.

\saysocrates Therefore, he won't be serious about writing them in ink,
sowing them, through a pen, with words that are as incapable of speaking
in their own defense as they are of teaching the truth adequately.

\sayphaedrus That wouldn't be likely.

\saysocrates Certainly not. When he writes, it's likely he will sow gardens 
of letters for the sake of amusing himself, storing up reminders
for himself “when he reaches forgetful old age” and for everyone who
wants to follow in his footsteps, and will enjoy seeing them sweetly
blooming. And when others turn to different amusements, watering
themselves with drinking parties and everything else that goes along
with them, he will rather spend his time amusing himself with the things
I have just described.

\sayphaedrus Socrates, you are contrasting a vulgar amusement with the 
very noblest---with the amusement of a man who can while away
his time telling stories of justice and the other matters you mentioned.

\saysocrates That's just how it is, Phaedrus. But it is much nobler to be
serious about these matters, and use the art of dialectic. The
dialectician chooses a proper soul and plants and sows within it
discourse accompanied by knowledge---discourse capable of helping itself
as well as the man who planted it, which is not barren but produces a
seed from which more discourse grows in the character of
others. Such discourse makes the seed forever immortal and renders the
man who has it as happy as any human being can be.

\sayphaedrus What you describe is really much nobler still.

\saysocrates And now that we have agreed about this, Phaedrus, we are
finally able to decide the issue.

\sayphaedrus What issue is that?

\saysocrates The issue which brought us to this point in the first place:
We wanted to examine the attack made on Lysias on account of his writing speeches, and to ask which speeches are written artfully and
which not. Now, I think that we have answered that question clearly
enough.

\sayphaedrus So it seemed; but remind me again how we did it.

\saysocrates First, you must know the truth concerning everything you are
speaking or writing about; you must learn how to define each thing in
itself; and, having defined it, you must know how to divide it into
kinds until you reach something indivisible. Second, you must understand
the nature of the soul, along the same lines; you must determine which
kind of speech is appropriate to each kind of soul, prepare and
arrange your speech accordingly, and offer a complex and elaborate
speech to a complex soul and a simple speech to a simple one. Then, and
only then, will you be able to use speech artfully, to the extent that
its nature allows it to be used that way, either in order to teach or in
order to persuade. This is the whole point of the argument we have been
making.

\sayphaedrus Absolutely. That is exactly how it seemed to us.

\saysocrates Now how about whether it's noble or shameful to give
or write a speech---when it could be fairly said to be grounds for
reproach, and when not? Didn't what we said just a little while ago make
it clear---

\sayphaedrus What was that?

\saysocrates That if Lysias or anybody else ever did or ever does
write---privately or for the public, in the course of proposing some
law---a political document which he believes to embody clear knowledge
of lasting importance, then this writer deserves reproach, whether
anyone says so or not. For to be unaware of the difference between a
dream-image and the reality of what is just and unjust, good and
bad, must truly be grounds for reproach even if the crowd praises it
with one voice.

\sayphaedrus It certainly must be.

\saysocrates On the other hand, take a man who thinks that a written
discourse on any subject can only be a great amusement, that no
discourse worth serious attention has ever been written in verse or
prose, and that those that are recited in public without questioning and
explanation, in the manner of the rhapsodes, are given
only in order to produce conviction. He believes that at their very best
these can only serve as reminders to those who already know. And he also
thinks that only what is said for the sake of understanding and
learning, what is truly written in the soul concerning what is just,
noble, and good can be clear, perfect, and worth serious attention: Such
discourses should be called his own legitimate children, first the
discourse he may have discovered already within himself and then
its sons and brothers who may have grown naturally in other souls
insofar as these are worthy; to the rest, he turns his back. Such a man,
Phaedrus, would be just what you and I both would pray to become.

\sayphaedrus I wish and pray for things to be just as you say.

\saysocrates Well, then: our playful amusement regarding discourse is
complete. Now you go and tell Lysias that we came to the spring which is
sacred to the Nymphs and heard words charging us to deliver a message
to Lysias and anyone else who composes speeches, as well as to
Homer and anyone else who has composed poetry either spoken or sung, and
third, to Solon and anyone else who writes political documents that he
calls laws: If any one of you has composed these things with a knowledge
of the truth, if you can defend your writing when you are challenged,
and if you can yourself make the argument that your writing is of little
worth, then you must be called by a name derived not from these writings
but rather from those things that you are seriously pursuing.

\sayphaedrus What name, then, would you give such a man?

\saysocrates To call him wise, Phaedrus, seems to me too much, and proper
only for a god. To call him wisdom's lover---a philosopher---or
something similar would fit him better and be more seemly.

\sayphaedrus That would be quite appropriate.

\saysocrates On the other hand, if a man has nothing more valuable than
what he has composed or written, spending long hours twisting it around,
pasting parts together and taking them apart---wouldn't you be right to
call him a poet or a speech writer or an author of laws?

\sayphaedrus Of course.

\saysocrates Tell that, then, to your friend.

\sayphaedrus And what about you? What shall you do? We must surely not
forget your own friend.

\saysocrates Whom do you mean?

\sayphaedrus The beautiful
Isocrates.\high{\goto{68}[phaedrus.htmlux5cux23phaedrfn_68]} What are
you going to tell him, Socrates? What shall we say he is?

\saysocrates Isocrates is still young, Phaedrus. But I want to tell you
what I foresee for him.

\sayphaedrus What is that?

\saysocrates It seems to me that by his nature he can outdo anything that
Lysias has accomplished in his speeches; and he also has a nobler
character. So I wouldn't be at all surprised if, as he gets older and
continues writing speeches of the sort he is composing now, he makes
everyone who has ever attempted to compose a speech seem like a child in
comparison. Even more so if such work no longer satisfies him and a
higher, divine impulse leads him to more important things. For nature,
my friend, has placed the love of wisdom in his mind.

That is the message I will carry to my beloved, Isocrates, from the gods
of this place; and you have your own message for your Lysias.

\sayphaedrus So it shall be. But let's be off, since the heat has died down
a bit.

\saysocrates Shouldn't we offer a prayer to the gods here before we leave?

\sayphaedrus Of course.

\saysocrates O dear Pan and all the other gods of this place, grant that I
may be beautiful inside. Let all my external possessions be in friendly 
harmony with what is within. May I consider the wise man rich.
As for gold, let me have as much as a moderate man
could bear and carry with him.

Do we need anything else, Phaedrus? I believe my prayer is enough for
me.

\sayphaedrus Make it a prayer for me as well. Friends have everything in
common.

\saysocrates Let's be off.\crlf
\crlf


