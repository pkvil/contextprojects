\environment suif-environment

\starttext

%\setbox0\hbox{abcdefghijklmnopqrstuvwxyz}
%\the\wd0

\mychapter{}{WHY CATS PAINT: FELINE AESTHETICS IN THEORY AND PRACTICE}

\dropcapS{ome test text} in ArnoPro 11/12, using smalltext, subheading, and display
design sizes. 
While theories on the aesthetics of nonprimate
signing are hardly new, we are aware that the great popularity
of the domestic cat as a pet means any attempt to describe their
marks as art carries with it certain dangers. The growing commercial
value of cat art has, for example, led not only to some misguided
breeding programs, but also to a few cases, thankfully rare, where 
cats have been trained to create art for reward.

It is our belief that we must suppress our desire to see cats
confirming our perceptions and values through their art, and rather than
attempting to determine the direction of their aesthetic development on our terms,
we must allow those few cats who paint to develop their own special potential.
Only in this way can we be certain they will be able to communicate their
unique, undiluted view of the world and perhaps provide us with the
clues we need to ensure the survival and future wellbeing of all species – provided, of course, we can trust them to tell us the truth.

We shall never know the origin of the primal feline aesthetic gesture, but it seems that
wherever domestic cats are well looked after and have little need to define
their territories, their marking behavior tends, in some rare instances, to become
what Desmond Morris calls a self-rewarding activity. They normally occur in animals which have their
survival problems under control and have a surplus of nervous activity that seems to require
an outlet.

Because cats show a distinct preference for the works of Van Gogh, usually attributed
to their being able to relate to the swirling furlike nature of the brushstrokes, Williams chose
four posters of this artist’s work and set them low on the wall in the cats’ living space.
By measuring the amount of time each cat spent looking at the different pictures over a six-week period, he was able to identify where cats had preferred sitting positions in the room, which he called “points of harmonic resonance”. He noticed that cats spent a large amount of time purring in these places prior to painting and has theorized that some kind of force field, detectable only to cats, may
trigger the feline aesthetic response.

\stoptext
