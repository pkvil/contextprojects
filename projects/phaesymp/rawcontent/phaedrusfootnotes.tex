\goto{14}[phaedrus.htmlux5cux23phaedrsup_14]. I.e., {\em hubris}, which
ranges from arrogance to the sort of crimes to which arrogance gives
rise, sexual assault in particular.

\goto{15}[phaedrus.htmlux5cux23phaedrsup_15]. Reading {\em polumeles kai
polueides} at a3 (lit., “multilimbed and multiformed”).

\goto{16}[phaedrus.htmlux5cux23phaedrsup_16]. A dithyramb was a choral
poem originally connected with the worship of Dionysus. In classical
times it became associated with an artificial style dominated by music.

\goto{17}[phaedrus.htmlux5cux23phaedrsup_17]. The overheated choral
poems known as dithyrambs (see 238d) were written in lyric meters. The
meter of the last line of Socrates' speech, however, was epic, and it is
the tradition in epic poetry to glorify a hero, not to attack him.

\goto{18}[phaedrus.htmlux5cux23phaedrsup_18]. Simmias, a companion of
Socrates, was evidently a lover of discussion (cf. {\em Phaedo} 85c).

\goto{19}[phaedrus.htmlux5cux23phaedrsup_19]. Ibycus was a sixth-century
poet, most famous for his passionate love poetry.

\goto{20}[phaedrus.htmlux5cux23phaedrsup_20]. Frg. 18 (Edmonds).

\goto{21}[phaedrus.htmlux5cux23phaedrsup_21]. Etymologically:
“Stesichorus son of Good Speaker, from the Land of Desire.” Myrrhinus
was one of the demes of ancient Athens.

\goto{22}[phaedrus.htmlux5cux23phaedrsup_22]. Retaining {\em heautēs} at
e3.

\goto{23}[phaedrus.htmlux5cux23phaedrsup_23]. Alternatively, “All soul.”

\goto{24}[phaedrus.htmlux5cux23phaedrsup_24]. Reading {\em pasan te
genesin} at e1.

\goto{25}[phaedrus.htmlux5cux23phaedrsup_25]. I.e., a philosopher.

\goto{26}[phaedrus.htmlux5cux23phaedrsup_26]. Accepting the emendation
{\em iont'} at b7.

\goto{27}[phaedrus.htmlux5cux23phaedrsup_27]. I.e., we philosophers; cf.
252e.

\goto{28}[phaedrus.htmlux5cux23phaedrsup_28]. “Desire” is {\em himeros}:
the derivation is from {\em merē} (“particles”), {\em ienai} (“go”) and
{\em rhein} (“flow”).

\goto{29}[phaedrus.htmlux5cux23phaedrsup_29]. Cf. 237b, 238d, 243e.

\goto{30}[phaedrus.htmlux5cux23phaedrsup_30]. The lines are probably
Plato's invention, as the language is not consistently Homeric. The pun
in the original is on {\em erōs} and {\em pterōs} (“the winged one”).

\goto{31}[phaedrus.htmlux5cux23phaedrsup_31]. Bacchants were worshippers
of Dionysus who gained miraculous abilities when possessed by the
madness of their god.

\goto{32}[phaedrus.htmlux5cux23phaedrsup_32]. Reading {\em teleutē} at
c3.

\goto{33}[phaedrus.htmlux5cux23phaedrsup_33]. Cf. {\em Iliad} v.397 and
{\em Odyssey} xvii.567.

\goto{34}[phaedrus.htmlux5cux23phaedrsup_34]. Cf. 243b.

\goto{35}[phaedrus.htmlux5cux23phaedrsup_35]. Cf. 234c, 238c.

\goto{36}[phaedrus.htmlux5cux23phaedrsup_36]. Apparently this was a
familiar example of something named by language that means the
opposite---though called “pleasant” it was really a long, nasty bend.

\goto{37}[phaedrus.htmlux5cux23phaedrsup_37]. Reading {\em suggramatos}
at a1.

\goto{38}[phaedrus.htmlux5cux23phaedrsup_38]. This is the standard form
for decisions, including legislation, made by the assembly of Athens,
though it is not the standard beginning for even the most political of
speeches.

\goto{39}[phaedrus.htmlux5cux23phaedrsup_39]. Lycurgus was the legendary
lawgiver of Sparta. Solon reformed the constitution of Athens in the
early sixth century B.C. and was revered by both democrats and their
opponents. Darius was king of Persia (521--486 B.C.). None of these was
famous as a speech writer.

\goto{40}[phaedrus.htmlux5cux23phaedrsup_40]. {\em Iliad} ii.361.

\goto{41}[phaedrus.htmlux5cux23phaedrsup_41]. For a criticism of
rhetoric as not an art, see {\em Gorgias} 462b--c.

\goto{42}[phaedrus.htmlux5cux23phaedrsup_42]. Cf. 242a--b;
{\em Symposium} 209b--e.

\goto{43}[phaedrus.htmlux5cux23phaedrsup_43]. Nestor and Odysseus are
Homeric heroes known for their speaking ability. Palamedes, who does not
figure in Homer, was proverbial for his cunning.

\goto{44}[phaedrus.htmlux5cux23phaedrsup_44]. Gorgias of Leontini was
the most famous teacher of rhetoric to visit Athens. About Thrasymachus
of Chalcedon (cf. 267c) we know little beyond what we can infer from his
appearance in Book 1 of the {\em Republic}. On Theodorus of Byzantium
(not to be confused with the geometer who appears in the
{\em Theaetetus}) see 266e and Aristotle {\em Rhetoric} 3.13.5.

\goto{45}[phaedrus.htmlux5cux23phaedrsup_45]. The Eleatic Palamedes is
presumably Zeno of Elea, the author of the famous paradoxes about
motion.

\goto{46}[phaedrus.htmlux5cux23phaedrsup_46]. Reading {\em pephukos} at
b6.

\goto{47}[phaedrus.htmlux5cux23phaedrsup_47]. {\em Odyssey} ii.406.

\goto{48}[phaedrus.htmlux5cux23phaedrsup_48]. Cf. 261c.

\goto{49}[phaedrus.htmlux5cux23phaedrsup_49]. Evenus of Paros was active
as a sophist toward the end of the fifth century B.C. Only a few tiny
fragments of his work survive.

\goto{50}[phaedrus.htmlux5cux23phaedrsup_50]. Tisias of Syracuse, with
Corax, is credited with the founding of the Sicilian school of rhetoric,
represented by Gorgias and Polus.

\goto{51}[phaedrus.htmlux5cux23phaedrsup_51]. Prodicus of Ceos, who
lived from about 470 till after 400 B.C., is frequently mentioned by
Plato in connection with his ability to make fine verbal distinctions.

\goto{52}[phaedrus.htmlux5cux23phaedrsup_52]. Hippias of Elis was born
in the mid-fifth century and traveled widely teaching a variety of
subjects, including mathematics, astronomy, harmony, mnemonics, ethics,
and history as well as public speaking.

\goto{53}[phaedrus.htmlux5cux23phaedrsup_53]. Polus was a pupil of
Gorgias; Plato represents him in the {\em Gorgias}, esp. at 448c and
471a--c. He was said to have composed an {\em Art of Rhetoric}
({\em Gorgias}, 462b).

\goto{54}[phaedrus.htmlux5cux23phaedrsup_54]. Licymnius of Chios was a
dithyrambic poet and teacher of rhetoric.

\goto{55}[phaedrus.htmlux5cux23phaedrsup_55]. Protagoras of Abdera,
whose life spanned most of the fifth century B.C., was the most famous
of the early sophists. We have a vivid portrayal of him in Plato's
{\em Protagoras} and an intriguing reconstruction of his epistemology in
the {\em Theaetetus}.

\goto{56}[phaedrus.htmlux5cux23phaedrsup_56]. Literally, “the might of
the Chalcedonian”: a Homeric figure referring to Thrasymachus, who came
from Chalcedon. Cf. 261c.

\goto{57}[phaedrus.htmlux5cux23phaedrsup_57]. Pericles, who dominated
Athens from the 450s until his death in 429 B.C., was famous as the most
successful orator-politician of his time. The quotation is from the
early Spartan poet Tyrtaeus, fragment 12.8 (Edmonds). Adrastus is a
legendary warrior hero of Argos, one of the main characters in
Euripides' {\em Suppliants}.

\goto{58}[phaedrus.htmlux5cux23phaedrsup_58]. Reading {\em anoias} at
a5.

\goto{59}[phaedrus.htmlux5cux23phaedrsup_59]. Hippocrates, a
contemporary of Socrates, is the famous doctor whose name is given to
the Hippocratic Oath. None of the written works that have come down to
us under his name express the view attributed to him in what follows.
All doctors were said to be descendants of Asclepius, hero and god of
healing.

\goto{60}[phaedrus.htmlux5cux23phaedrsup_60]. At 259e ff.

\goto{61}[phaedrus.htmlux5cux23phaedrsup_61]. Socrates may be referring
to Corax, whose name is also the Greek word for “crow.”

\goto{62}[phaedrus.htmlux5cux23phaedrsup_62]. Literally, “is likely.”

\goto{63}[phaedrus.htmlux5cux23phaedrsup_63]. Naucratis was a Greek
trading colony in Egypt. The story that follows is probably an invention
of Plato's (see 275b3) in which he reworks elements from Egyptian and
Greek mythology.

\goto{64}[phaedrus.htmlux5cux23phaedrsup_64]. Theuth (or Thoth) is the
Egyptian god of writing, measuring, and calculation. The Greeks
identified Thoth with Hermes, perhaps because of his role in weighing
the soul. Thoth figures in a related story about the alphabet at
{\em Philebus} 18b.

\goto{65}[phaedrus.htmlux5cux23phaedrsup_65]. As king of the Egyptian
gods, Ammon (Thamus) was identified by Egyptians with the sun god Ra and
by the Greeks with Zeus.

\goto{66}[phaedrus.htmlux5cux23phaedrsup_66]. Accepting the emendation
of {\em Thamoun} at d4.

\goto{67}[phaedrus.htmlux5cux23phaedrsup_67]. Gardens of Adonis were
pots or window boxes used for forcing plants during the festival of
Adonis.

\goto{68}[phaedrus.htmlux5cux23phaedrsup_68]. Isocrates (436--338 B.C.)
was an Athenian teacher and orator whose school was more famous in its
day than Plato's Academy.
