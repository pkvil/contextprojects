\goto{1}[symposium.htmlux5cux23sympossup_1]. Agathon's name could be
translated “Goodman.” The proverb is, “Good men go uninvited to an
inferior man's feast” (Eupolis fr. 289 Kock).

\goto{2}[symposium.htmlux5cux23sympossup_2]. Menelaus calls on Agamemnon
at {\em Iliad} ii.408. Menelaus is called a limp spearman at
xvii.587--88.

\goto{3}[symposium.htmlux5cux23sympossup_3]. An allusion to {\em Iliad}
x.224, “When two go together, one has an idea before the other.”

\goto{4}[symposium.htmlux5cux23sympossup_4]. Dionysus was the god of
wine and drunkenness.

\goto{5}[symposium.htmlux5cux23sympossup_5]. {\em Theogony} 116--120,
118 omitted.

\goto{6}[symposium.htmlux5cux23sympossup_6]. Acusilaus was an
early-fifth-century writer of genealogies.

\goto{7}[symposium.htmlux5cux23sympossup_7]. Parmenides, B 13
Diels-Kranz.

\goto{8}[symposium.htmlux5cux23sympossup_8]. Accepting the deletion of
{\em ē} in e5.

\goto{9}[symposium.htmlux5cux23sympossup_9]. Cf. {\em Iliad} x.482,
xv.262; {\em Odyssey} ix.381.

\goto{10}[symposium.htmlux5cux23sympossup_10]. Alcestis was the
self-sacrificing wife of Admetus, whom Apollo gave a chance to live if
anyone would go to Hades in his place.

\goto{11}[symposium.htmlux5cux23sympossup_11]. Orpheus was a musician of
legendary powers, who charmed his way into the underworld in search of
his dead wife, Eurydice.

\goto{12}[symposium.htmlux5cux23sympossup_12]. In his play, {\em The
Myrmidons}. In Homer there is no hint of sexual attachment between
Achilles and Patroclus.

\goto{13}[symposium.htmlux5cux23sympossup_13]. Harmodius and Aristogiton
attempted to overthrow the tyrant Hippias in 514 B.C. Although their
attempt failed, the tyranny fell three years later, and the lovers were
celebrated as tyrannicides.

\goto{14}[symposium.htmlux5cux23sympossup_14]. {\em Iliad} ii.71.

\goto{15}[symposium.htmlux5cux23sympossup_15]. Heraclitus of Ephesus, a
philosopher of the early fifth century, was known for his enigmatic
sayings. This one is quoted elsewhere in a slightly different form, frg.
B 51 Diels-Kranz.

\goto{16}[symposium.htmlux5cux23sympossup_16]. {\em Iliad} v.385,
{\em Odyssey} xi.305 ff.

\goto{17}[symposium.htmlux5cux23sympossup_17]. Cf. {\em Odyssey}
viii.266 ff.

\goto{18}[symposium.htmlux5cux23sympossup_18]. Arcadia included the city
of Mantinea, which opposed Sparta, and was rewarded by having its
population divided and dispersed in 385 B.C. Aristophanes seems to be
referring anachronistically to those events; such anachronisms are not
uncommon in Plato.

\goto{19}[symposium.htmlux5cux23sympossup_19]. Contrast 178b.

\goto{20}[symposium.htmlux5cux23sympossup_20]. {\em Iliad} xix.92--93.
“Mischief” translates {\em Atē}.

\goto{21}[symposium.htmlux5cux23sympossup_21]. “Moral character”:
{\em aretē}, i.e., virtue.

\goto{22}[symposium.htmlux5cux23sympossup_22]. A proverbial expression
attributed by Aristotle ({\em Rhetoric} 1406a17--23) to the
fourth-century liberal thinker and rhetorician Alcidamas.

\goto{23}[symposium.htmlux5cux23sympossup_23]. {\em Sōphrosunē}. The
word can be translated also as “temperance” and, most literally,
“sound-mindedness.” (Plato and Aristotle generally contrast
{\em sōphrosunē} as a virtue with self-control: the person with
{\em sōphrosunē} is naturally well-tempered in every way and so does not
need to control himself, or hold himself back.)

\goto{24}[symposium.htmlux5cux23sympossup_24]. From Sophocles, fragment
234b Dindorf: “Even Ares cannot withstand Necessity.” Ares is the god of
war.

\goto{25}[symposium.htmlux5cux23sympossup_25]. See {\em Odyssey}
viii.266--366. Aphrodite's husband Hephaestus made a snare that caught
Ares in bed with Aphrodite.

\goto{26}[symposium.htmlux5cux23sympossup_26]. “Wisdom” translates
{\em sophia}, which Agathon treats as roughly equivalent to {\em technē}
(professional skill); he refers mainly to the ability to produce things.
Accordingly “wisdom” translates {\em sophia} in the first instance;
afterwards in this passage it is “skill” or “art.”

\goto{27}[symposium.htmlux5cux23sympossup_27]. At 186b.

\goto{28}[symposium.htmlux5cux23sympossup_28]. Euripides,
{\em Stheneboea} (frg. 666 Nauck).

\goto{29}[symposium.htmlux5cux23sympossup_29]. After these two lines of
poetry, Agathon continues with an extremely poetical prose peroration.

\goto{30}[symposium.htmlux5cux23sympossup_30]. Accepting the emendation
{\em aganos} at d5.

\goto{31}[symposium.htmlux5cux23sympossup_31]. “Gorgian head” is a pun
on “Gorgon's head.” In his peroration Agathon had spoken in the style of
Gorgias, and this style was considered to be irresistibly powerful. The
sight of a Gorgon's head would turn a man to stone.

\goto{32}[symposium.htmlux5cux23sympossup_32]. The allusion is to
Euripides, {\em Hippolytus} 612.

\goto{33}[symposium.htmlux5cux23sympossup_33]. Cf. 197b.

\goto{34}[symposium.htmlux5cux23sympossup_34]. 197b3--5.

\goto{35}[symposium.htmlux5cux23sympossup_35]. The Greek is ambiguous
between “Love loves beautiful things” and “Love is one of the beautiful
things.” Agathon had asserted the former (197b5, 201a5), and this will
be a premise in Diotima's argument, but he asserted the latter as well
(195a7), and this is what Diotima proceeds to refute.

\goto{36}[symposium.htmlux5cux23sympossup_36]. {\em Poros} means “way,”
“resource.” His mother's name, {\em Mētis}, means “cunning.” {\em Penia}
means “poverty.”

\goto{37}[symposium.htmlux5cux23sympossup_37]. I.e., a philosopher.

\goto{38}[symposium.htmlux5cux23sympossup_38]. {\em Eudaimonia}: no
English word catches the full range of this term, which is used for the
whole of well-being and the good, flourishing life.

\goto{39}[symposium.htmlux5cux23sympossup_39]. “Poetry” translates
{\em poiēsis}, lit. ‘making', which can be used for any kind of
production or creation. However, the word {\em poiētēs}, lit. ‘maker',
was used mainly for poets---writers of metrical verses that were
actually set to music.

\goto{40}[symposium.htmlux5cux23sympossup_40]. Accepting the emendation
{\em toutou} in b1.

\goto{41}[symposium.htmlux5cux23sympossup_41]. The preposition is
ambiguous between “within” and “in the presence of.” Diotima may mean
that the lover causes the newborn (which may be an idea) to come to be
within a beautiful person; or she may mean that he is stimulated to give
birth to it in the presence of a beautiful person.

\goto{42}[symposium.htmlux5cux23sympossup_42]. Moira is known mainly as
a Fate, but she was also a birth goddess ({\em Iliad} xxiv.209), and was
identified with the birth-goddess Eilithuia (Pindar, {\em Olympian Odes}
vi.42, {\em Nemean} {\em Odes} vii.1).

\goto{43}[symposium.htmlux5cux23sympossup_43]. Codrus was the legendary
last king of Athens. He gave his life to satisfy a prophecy that
promised victory to Athens and salvation from the invading Dorians if
their king was killed by the enemy.

\goto{44}[symposium.htmlux5cux23sympossup_44]. Lycurgus was supposed to
have been the founder of the oligarchic laws and stern customs of
Sparta.

\goto{45}[symposium.htmlux5cux23sympossup_45]. The leader: Love.

\goto{46}[symposium.htmlux5cux23sympossup_46]. I.e., philosophy.

\goto{47}[symposium.htmlux5cux23sympossup_47]. Reading {\em teleutēsēi}
at c7.

\goto{48}[symposium.htmlux5cux23sympossup_48]. Cf. 205d--e.

\goto{49}[symposium.htmlux5cux23sympossup_49]. {\em Iliad} xi.514.

\goto{50}[symposium.htmlux5cux23sympossup_50]. This is the conventional
translation of the word, but the {\em aulos} was in fact a reed
instrument and not a flute. It was held by the ancients to be the
instrument that most strongly arouses the emotions.

\goto{51}[symposium.htmlux5cux23sympossup_51]. Satyrs had the sexual
appetites and manners of wild beasts and were usually portrayed with
large erections. Sometimes they had horses' tails or ears, sometimes the
traits of goats. Marsyas, in myth, dared to compete in music with Apollo
and was skinned alive for his impudence.

\goto{52}[symposium.htmlux5cux23sympossup_52]. Olympus was a legendary
musician who was said to be loved by Marsyas ({\em Minos} 318b5) and to
have made music that moved its listeners out of their senses.

\goto{53}[symposium.htmlux5cux23sympossup_53]. Legendary worshippers of
Cybele, who brought about their own derangement through music and dance.

\goto{54}[symposium.htmlux5cux23sympossup_54]. {\em Iliad} vi.232--36
tells the famous story of the exchange by Glaucus of golden armor for
bronze.

\goto{55}[symposium.htmlux5cux23sympossup_55]. Ajax, a hero of the Greek
army at Troy, carried an enormous shield and so was virtually
invulnerable to enemy weapons.

\goto{56}[symposium.htmlux5cux23sympossup_56]. Potidaea, a city in
Thrace allied to Athens, was induced by Corinth to revolt in 432 B.C.
The city was besieged by the Athenians and eventually defeated in a
bloody local war, 432--430 B.C.

\goto{57}[symposium.htmlux5cux23sympossup_57]. {\em Odyssey} iv.242,
271.

\goto{58}[symposium.htmlux5cux23sympossup_58]. At Delium, a town on the
Boeotian coastline just north of Attica, a major Athenian expeditionary
force was routed by a Boeotian army in 424 B.C. For another description
of Socrates' action during the retreat, see {\em Laches} 181b.

\goto{59}[symposium.htmlux5cux23sympossup_59]. Cf. Aristophanes,
{\em Clouds} 362.

\goto{60}[symposium.htmlux5cux23sympossup_60]. Brasidas, among the most
effective Spartan generals during the Peloponnesian War, was mortally
wounded while defeating the Athenians at Amphipolis in 422 B.C. Antenor
(for the Trojans) and Nestor (for the Greeks) were legendary wise
counsellors during the Trojan War.

\goto{61}[symposium.htmlux5cux23sympossup_61]. Cf. {\em Iliad} xvii.32.
