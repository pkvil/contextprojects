\switchtobodyfont[etbb,10pt]
\chapter[etbb]{ETbb}

ETbb stands for Edward Tufte’s Bembo. As the name suggests, the inspiration comes from a design first used by the Venetian printer Aldus Manutius in 1496 for a book written by Pietro Bembo. It’s a serene and versatile typeface of genuine Renaissance structure. The modern typeface named Bembo was made in 1929 by Monotype, and it has since been one of the most popular typefaces for book production.

When the printing industry went digital, Edward Tufte noticed that digital printing rendered fonts thinner compared to lead type, as it didn’t account for ink squeeze. Unhappy with the results, he commissioned a typeface for his own books called ET-Bembo, which eventually was released under a free software license in 2015. This typeface was enhanced in 2019 under the name XETbook, and finally expanded in 2020 as ETbb.

ETbb adds a full set of figure styles, small caps in all styles, superior letters and figures, inferior figures, and a new capital Sharp S with small caps version. Compared to Cardo, another free Bembo-like typeface, it is significantly darker.


