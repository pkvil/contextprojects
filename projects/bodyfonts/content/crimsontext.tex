\switchtobodyfont[crimsontext,10pt]
\chapter{Crimson Text}

\section{Background}
Sebastian Kosch created Crimson Text due to a lack of open source old-style typefaces. It’s not modeled after a specific historical font, and is often very similar to Minion. It was meant to be complemented by a Crimson Display typeface for headings, but that never came to be. Later the developer made a new main branch to address some issues. This branch, called simply Crimson, was where the continued development happened. There has not been much development of Crimson Text or Crimson for many years, and no more work is expected.

Much later, fonthaus modified Crimson Text to better suit screen use, simplifying outlines and increasing x-height, and called it Crimson Pro. Recently, Michael Sharpe forked Crimson and added all that he thought was missing, like small caps, math support, and many new glyphs to support many languages. The amount of less tested new glyphs introduced inspired the name Cochineal (“Like Crimson, but with more bugs”). The Crimson main branch can also be found on Google Fonts as the latin alphabet of the Amiri typeface by Khaled Hosny. 

Out of these fonts, I personally recommend Cochineal since it’s the most complete and is actively maintaine.

\section{Features}

Alphabet width: 10,11,12pt, ideal line length.

\section{Variants}
Crimson Text -- Crimson -- Cochineal.

\section{Where to get them}
