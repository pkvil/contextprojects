\switchtobodyfont[baskervillef,10pt]
\chapter[baskervillef]{BaskervilleF}

Baskerville is the epitome of Neoclassicism and eighteenth-century rationalism in type. During a time when William Caslon was the dominant figure in printing in England, John Baskerville released a typeface according to his ideals. The glyphs were very regular with few embellishments, high contrast (difference between thick and thin strokes), almost vertical axis of stress, and sharply cut serifs. This required ultra-smooth pressed paper and high quality ink. 

Libre Baskerville is designed as a web font and therefore optimized for computer screen use. With increased x-height, short ascenders, low contrast, and thick serifs, it moved away from the original typeface idea. BaskervilleF is based on this font, while undoing the transformations required for a web font.

John Baskerville used his wealth to create high-end books according to his ideals in the latter half of the 18th century in England. This required him to make his own printing machinery, ultra-smooth pressed paper and high quality ink. During a time when Caslon was the dominant figure in printing, Baskerville released his typeface which relied more on simplicity of form and strict attention to detail. The glyphs were very regular with few embellishments, had high contrast curved strokes that were close to circular, almost vertical axis of modulation, and sharply cut serifs. His rivals spread stories that this typeface, due to excessive sharpness and contrast, was a danger to the eyes. Baskerville eventually abandoned the printing business, and a foundry owned by Joseph Fry cut an imitation of Baskerville that didn't require the highest level of equipment, paper, and ink. This is the version that was eventually revived by ATF as Fry's Baskerville. In 2012, Pablo Impallari created a web font, Libre Baskerville, where he increased the x-height, shortened the ascenders, lowered contrast and thickened the serifs. In short, to work within the restrictions of computer screens, they went the opposite direction of Baskervilles ideals. BaskervilleF is based on this font, while undoing the transformation to a web font.

small caps and text figures,
often omitted, are essential to the spirit of the original

You can download the fonts here:  
