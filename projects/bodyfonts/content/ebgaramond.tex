\switchtobodyfont[duffnerebgaramond,10pt]
\chapter[ebgaramond]{EB Garamond}

Many typefaces in the Garamond style have been devoloped, and they vary considerably in appearance. Some named Garamond are actually not based on the designs by Claude Garamont, while other typefaces not named Garamond are closely based on the original specimens.

There are currently two main versions of EB Garamond. The outlines of the glyphs are identical, but they both come with different strengths and weaknesses.

Georg Duffner created EB Garamond with an aim to provide a true Garamond, with many alternate glyphs, ligatures, and substitutions available for historical accuracy. The original plan was to make a range of optical sizes from 6pt to 40pt, but development seems to have stopped after making the 8pt and 12pt designs. It does not come with a bold weight, since that would be an anachronism. 

Octavio Pardo forked EB Garamond in the 12pt optical size, adding medium, semibold, bold, and extra bold weights, while removing ornamental initials and opentype features rarely used in modern texts. This is the version found in Google fonts.


