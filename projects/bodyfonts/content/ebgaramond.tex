\switchtobodyfont[duffnerebgaramond,10pt]
\chapter{EB Garamond}

\section{Background}
When many people think about a typical book font, they think of Garamond. However, there are many different typefaces called Garamond. To varying degrees they draw inspiration from fonts used by an italian printer called Claude Garamont at the beginning of the 17th century. 

Georg Duffner created EB Garamond with the intention of making a historically faithful digitalization of an original specimen sheet. Because of this, he did not make a bold variant, since bold fonts were not yet invented at the time. He focused instead on copying fonts for different sizes (8pt and 12pt), the ornamental drop caps, and historical ligatures. 

When Google wanted to add this font to their collection, they hired Octavio Pardo to create the bold style and to focus on the modern needs of a typeface. Therefore we now have two typefaces, both called EB Garamond. Duffner’s original and Pardo’s Google version. 

Two sources with different goals. Egenolff-Berner specimen. Ambitious plan. Octavio Pardo for Google. Suitable for.


\section{Features}
Duffner version plenty of historical opentype features, optical sizes. Pardo version bold.

Alphabet width: 10,11,12pt, ideal line length.

\section{Variants}

\section{Where to get them}


