\switchtobodyfont[cochineal,10pt]
\chapter[cochineal]{Cochineal}

{\it Cochineal} is a fork of {\it Crimson,} which aimed to be a classic old-style serif without being based on a specific historical typeface. It’s generally similar to {\it Minion}, though with smaller x-height and less plain in detail.

To make a long story short, the development of the original typeface, {\it Crimson Text}, stopped in favor of a new main branch named {\it Crimson} in 2012. Unfortunately, there has been no new work on either branch for many years.  {\it Cochineal}, a fork of {\it Crimson} first published in 2016, added many new glyphs and is still actively maintained. {\it Crimson Pro} is a complete redesign of the original typeface, commissioned by Google in 2018.

To get an average of 66 characters per line, use a line lenth of 98 mm for a 10 pt body font, 110 mm for an 11 pt body font, and 118 mm for a 12 pt body font.

{\it Cochineal} provides fonts in regular, italic, bold, and bold italic with a full array of features like small caps, old-style and lining figures, a swash Q, and more. There is support for Roman, Greek, and Cyrillic alphabets.

You can download the fonts here:  

https://www.ctan.org/tex-archive/fonts/cochineal/opentype


%10pt: 117.355pt - 

%11pt: 129.18pt - 

%12pt: 140.83pt - 

%\newbox\alphabetbox
%\setbox\alphabetbox=\hbox{abcdefghijklmnopqrstuvwxyz}
%abcdefghijklmnopqrstuvwxyz = \the\dimexpr(\wd\alphabetbox)\par
