\environment assignments-environment

\setuplanguage[en][lefthyphenmin=3,righthyphenmin=3]

\starttext

\nameanddate{Megan Andrews}{January}{29}{2022}

%\chapter[ownnumber=3,title={Anatomy I: Basics}]
\vskip1.3em
\startalignment[middle]
\dontleavehmode
\raise0.4em\hbox{\rotate[rotation=180]{\scale[xscale=1260]{\symbol[zapfswash]}}}%
\hskip0.6em%
{\tfa Topic 4}%
\hskip0.6em
\raise0.4em\hbox{\rotate[rotation=180]{\scale[xscale=1260]{\mirror{\symbol[zapfswash]}}}}
\vskip-\lineheight
\vskip1ex
{\tfa Anatomy II: Spine}
\vskip0.5ex
\dontleavehmode
\scale[xscale=1977]{\mirror{\symbol[zapfswash]}}
\kern2.5pt
\scale[xscale=1997]{\symbol[zapfswash]}
\stopalignment
\vskip2ex

\question[ownnumber={4.1}, title={How many individual vertebrae are in the spinal column? How many vertebrae are in  each curve?}] 

There are 24 individual vertebrae in the spinal column. The curves and their respective number of vertebrae include:
\startitemize[symbol=mybullet1]
\item Cervical Vertebrae - 7
\item Thoracic Vertebrae - 12
\item Lumbar Vertebrae - 5
\item Sacral Vertebrae - 4-5 fused together
\item Coccygeal Vertebrae - 3-5 fused together
\stopitemize

\question[ownnumber={4.2},title={Which of the spinal curves are present  when a baby is born? At what point of development do the others appear?}]

After birth, a baby takes the shape of a comma, so there is a kyphodic curve/convexity in both the thoracic and sacral vertebrae. When the baby begins to lift it's head, the lordadic cervical curve develops. When the baby begins to walk, the lordadic lumbar curve develops.

\question[ownnumber={4.3},title={The cervical spine has the most freedom of movement in all directions. The thoracic and the lumbar areas have limited freedom in some directions. For each of these areas, list the actions and the structures that limit freedom of movement (Example: Thoracic spine: Flexion is limited by the ribs).}]

\startitemize[symbol=mybullet1]
\item Cervical Spine 
    \startitemize[symbol=mybullet2]    
    \item Rotation (turning one's head completely around) is limited by inelastic ligaments in the spine
    \item Lateral flexion may be limited by tension in the trapezius muscles and/or ligaments in the spine
    \stopitemize
\stopitemize
\page[yes]
\startitemize[symbol=mybullet1]
\item Thoracic Spine
    \startitemize[symbol=mybullet2]    
    \item Flexion and lateral flexion are limited by the ribs and ligaments on the posterior of the spine, and ligaments between the transverse processes respectively
    \item Extension  is limited by compression of the spinous processes and ligaments on the anterior of the spine
    \stopitemize
\item Lumbar Spine
    \startitemize[symbol=mybullet2]
    \item Rotation is limited by the shape and fit of the facet joints in this part of the spine
    \stopitemize
\item Sacral Spine
    \startitemize[symbol=mybullet2]
    \item Rotation, lateral flexion, flexion, and extension are limited by the fusion of the sacral vertebrae
    \stopitemize
\item Coccygeal Spine
    \startitemize[symbol=mybullet2]
    \item Rotation, lateral flexion, flexion, and extension are limited by the fusion of the coccygeal vertebrae
    \stopitemize
\stopitemize

\stoptext

