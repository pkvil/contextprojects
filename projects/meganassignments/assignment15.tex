\environment assignments-environment

\starttext

\nameanddate{Megan Andrews}{March}{26}{2022}

\vskip1.3em
\startalignment[middle]
\dontleavehmode
\raise0.35em\hbox{\rotate[rotation=180]{\scale[xscale=1500]{\symbol[zapfswash]}}}%
\hskip0.6em%
Topic 15%
\hskip0.6em
\raise0.35em\hbox{\rotate[rotation=180]{\scale[xscale=1500]{\mirror{\symbol[zapfswash]}}}}
\vskip-\lineheight
\vskip1ex
{Pranayama Techniques I}
\vskip0.5ex
\dontleavehmode
\scale[xscale=2130]{\mirror{\symbol[zapfswash]}}
\kern2.9pt
\scale[xscale=2130]{\symbol[zapfswash]}
\stopalignment
\vskip2.1ex

\questionnonumber{Describe how to teach Ujjayi breath in your own words to a beginner} 

\startalignment[right]
\startitemize[symbol=mybullet1]
\item Find your way to a comfortable seat. Allow the sit bones to sink down into the ground, with the pelvis stacked evenly over top. Let this foundation support the spine effortlessly.
\item Take your time to arrive in a place of effortlessness, of ease. Nothing in the body should feel strained. Tweak your position as needed. You're welcome to close the eyes or soften your gaze downwards to a point on the ground.
\item Bring awareness to the natural breath. Can it flow smoothly in through the mouth and down the length of the spine? Is there any tension in the body obstructing its passage?
\item Now place the palm in front of the mouth, and exhale into your hand, as if you were trying to fog a piece of glass before cleaning it – eyeglasses for instance.
\item *Demonstrate*
\item To create this sort of breath, you might notice a slight constriction in the back of your throat. Stay with this sensation, and when you feel used to it, continue it with your mouth closed. Observe the sound it makes inside the throat. It reminds me of a wind passing through trees, or the ebb and flow of an ocean tide.
\item Now apply this whispery constriction to both your inhale and exhale. Notice and embrace the effect it has on the pace and length of your breath. 
\stopitemize
\stopalignment

\stoptext

