\environment assignments-environment

\starttext

\nameanddate{Megan Andrews}{May}{30}{2022}

\vskip1.3em
\startalignment[middle]
\dontleavehmode
\raise0.35em\hbox{\rotate[rotation=180]{\scale[xscale=1500]{\symbol[zapfswash]}}}%
\hskip0.6em%
Script 3%
\hskip0.6em
\raise0.35em\hbox{\rotate[rotation=180]{\scale[xscale=1500]{\mirror{\symbol[zapfswash]}}}}
\vskip-\lineheight
\vskip1ex
{Urdva Danurasana}
\vskip0.5ex
\dontleavehmode
\scale[xscale=2130]{\mirror{\symbol[zapfswash]}}
\kern2.9pt
\scale[xscale=2130]{\symbol[zapfswash]}
\stopalignment
\vskip2.1ex

%\item \color[inhale]{Inhale}: Rise up, arms stretched upwards, and welcome to warrior one\par

\blank[line]
\noindent {\sc Transition from Laying Down}
\startitemize[symbol=mybullet1]
\item Come to lay down on your back
\item Engage full complete breath, breathing in to all directions of the ribcage from the belly up to the chest
\item Use this breath create more space in the thoracic cavity
\stopitemize
\blank[line]
\noindent {\sc Transition from Setu Bandha Sarvangasana}
\startitemize[symbol=mybullet1]
\item Return to your regular breath
\item Bend your knees, placing your feet underneath your knees, close to your seat
\item Notice that the feet are spaced about hip-width apart
\item Place the arms by your side, palms faced down and pressing in to the ground
\item Tuck in your elbows and shoulder blades towards the midline of the body
\item Press your feet in to the ground to lift the thighs and hips towards the sky
\stopitemize
\blank[line]
\noindent {\sc Urdva Danurasana}
\startitemize[symbol=mybullet1]
\item Extend arms over head, fingers directed towards the heels
\item Tuck in the shoulder blades
\item \color[inhale]{Inhale}: As you inhale, press through the hands and feet, to lift the chest and heart center upwards
\item Attention to the neck: Keep it in line with the rest of the spine or allow the head to hang loose
\stopitemize
\blank[line]
\noindent {\sc Transition Out}
\startitemize[symbol=mybullet1]
\item \color[exhale]{Exhale}: On the next exhale, slowly bend the elbows to lower yourself down while you begin to tuck in your chin
\item Starting with the back of your head, lower your spine down in a wave: from the back of the head to the sacrum.
\stopitemize
\blank[2*line]
\noindent {\sc Modification}
\startitemize[symbol=mybullet1]
\item Option to stay in bridge or supported bridge pose. 
\stopitemize

\stoptext

