\environment assignments-environment

\starttext

%
% Assignment 1
%

\nameanddate{Megan Andrews}{February}{5}{2022}

%\chapter[ownnumber=1,title={What is ISHTA Yoga?}]
\vskip1.3em
\startalignment[middle]
\dontleavehmode
\raise0.35em\hbox{\rotate[rotation=180]{\scale[xscale=1260]{\symbol[zapfswash]}}}%
\hskip0.6em%
Topic 1%
\hskip0.6em
\raise0.35em\hbox{\rotate[rotation=180]{\scale[xscale=1260]{\mirror{\symbol[zapfswash]}}}}
\vskip-\lineheight
\vskip1ex
What is {\kerncharacters[0.08]ISHTA} Yoga?
\vskip0.5ex
\dontleavehmode
\scale[xscale=1885]{\mirror{\symbol[zapfswash]}}
\kern2.5pt
\scale[xscale=1885]{\symbol[zapfswash]}
\stopalignment
\vskip2.1ex

\questionnonumber{What does yoga mean to me?} 

From my perspective, yoga is ultimately a form of {\em homecoming}. I interpret homecoming as a return to one’s innermost spiritual nature, while yoga is the vehicle that has the capacity to take me there. 

Lower levels of cortisol, increased mobility, strengthened muscles, and improved quality of sleep are all measurable and tangible benefits of yoga that I have personally experienced and enjoyed. And while these physical benefits are signficant, as I’ve furthered my journey in yoga, I have found that there is the less tangible benefit of spiritual connection that has proven even more valuable. Part of this spiritual connection is the realization that we are intrinsically worthy and equal, as well as parts of a greater whole and interconnectedness. This has been profoundly comforting to me.

From childhood and onward, a person may have accumulated any number of labels, traumas, categorizations, diagnoses, or performance evaluations – either self-imposed or imposed by others. All of these labels can feel like defining characteristics upon which our self-worth relies. To my mind, practicing yoga offers a reprieve from this judgmental narrative. Comments like, “I should have done…” or “I dislike … about myself” are replaced with “I am enough” and “I am worthy.” That is to say, my identity isn’t solely comprised of my perceived achievements or failures, and I am not limited to a single categorical box. Rather, my value is unconditional and part of something much more vast and infinite. So I try to use yoga as an “undressing” of layers that compound a lot of stress, and alternatively, invite in self-acceptance and embrace my innate value as a living being.

I believe yoga offers this to me through the mindful harmonization of breath and posture. The repetition of movement and refocusing of the breath clears away neuroticism and cognitive distortion  with something more primal.  My {\em self} is no longer entangled with labels, past, trauma, or measurement. The major quality that remains through a good yoga practice is the present – and that is where {\em home} is.

%
% Assignment 3
%
\page[yes]

\nameanddate{Megan Andrews}{February}{5}{2022}

%\chapter[ownnumber=3,title={Anatomy I: Basics}]
\vskip1.3em
\startalignment[middle]
\dontleavehmode
\raise0.35em\hbox{\rotate[rotation=180]{\scale[xscale=1260]{\symbol[zapfswash]}}}%
\hskip0.6em%
Topic 3%
\hskip0.6em
\raise0.35em\hbox{\rotate[rotation=180]{\scale[xscale=1260]{\mirror{\symbol[zapfswash]}}}}
\vskip-\lineheight
\vskip1ex
Anatomy I: Basics
\vskip0.5ex
\dontleavehmode
\scale[xscale=1885]{\mirror{\symbol[zapfswash]}}
\kern2.5pt
\scale[xscale=1885]{\symbol[zapfswash]}
\stopalignment
\vskip2.1ex

\question[ownnumber={3.1}, title={Briefly describe each of these tissues and its function: Muscle, Ligament, Tendon, Cartilage.}] 

{\sc Muscle}
\startnarrower[left]
\noindent A muscle is tissue consisting of bundles of parallel fibers wrapped in fascia.

\blank[halfline]
\noindent Properties:
\startalignment[right]
\startitemize[symbol=mybullet1]
\item Excitable – Reacts to stimuli via innervation
\item Contractible – May contract upon stimulation in 3 different ways
    \startitemize[symbol=mybullet2]
    \item Concentrically - Muscle shortens
    \item Eccentrically - Muscle lengthens
    \item Isometrically  - Muscle neither shortens or lengthens during contraction
    \stopitemize
\item Extensible - May be stretched
\item Elastic - Able to recoil
\stopitemize
\stopalignment

\blank[halfline]
\noindent Roles:
\startitemize[symbol=mybullet1]
\item Agonist - Prime mover muscle
\item Synergist - Supporting muscle to help the agonist in movement
\item Antagonist - Opposes prime mover to slow and monitor movement
\item Protects joint
\item Stabilizer - A muscle that pairs with an opposing muscle to keep a bone in place
\stopitemize
\stopnarrower
\blank[halfline]

\noindent{\sc Ligament}
\startnarrower[left]
\noindent A ligament is a dense piece of connective tissue that connects one bone to another bone.
\blank[halfline]
\startitemize[mybullet1]
\item Protects joints
\item Avascular - no blood supply
\item Barely elastic \symbol[zapfarrow]\ difficult to heal
\stopitemize
\stopnarrower
\blank[halfline]
\page[yes]

\noindent{\sc Tendon}
\startnarrower[left]
\noindent A tendon is a tough, fibrous continuation of muscle fascia that attaches muscle to bone.
\blank[halfline]
\startitemize[mybullet1]
\item Stabilizes joint
\item Transmits and supports muscle force
\item Woven into the periosteum/outer shell of the bone
\stopitemize
\stopnarrower
\blank[halfline]

\noindent{\sc Cartilage}
\startnarrower[left]
\noindent Cartilage is supplementary tissue to muscle providing strength, rigidity, and a little elasticity.
\blank[halfline]
\startitemize[mybullet1]
\item Fibrocartilage
    \startitemize[symbol=mybullet2]
    \item Cushioning, pillowy 
    \item Shock-absorbing
    \item Avascular - no blood supply
    \stopitemize
\item Hyaline Cartilage
    \startitemize[symbol=mybullet2]
    \item Smooth, glassy
    \item Reinforcing coat where bones rub against one another
    \item Avascular - no blood supply
    \stopitemize
\stopitemize
\stopnarrower
\vskip0.5\lineheight

\question[ownnumber={3.2},title={The elbow is a hinge joint and the hip is a ball-and-socket joint. (A) What are the movements of a hinge joint? Name the plane. (B) What are the movements of a ball-and-socket joint? In how many planes does a ball-and-socket joint move and what are the actions in each of the planes?}]

(A) The movements of a hinge joint are flexion and extension, and the plane is the Sagittal Plane.
\blank[line]
\noindent (B) The movements of a ball-and-socket joint and their corresponding planes are:
\startitemize[n]
\item Rotation (internal and external) in the Transverse Plane
\item Flexion & Extension in the Sagittal Plane
\item Adduction & Abduction in Coronal Plane
\stopitemize
\startitemize[symbol=mybullet1]
\item Circumduction involves all movements and planes in combination, except rotation in the Transverse plane
\stopitemize

\question[ownnumber={3.3},title={What is the job of an agonist? What is the job of a synergist?}]

An agonist is the prime mover muscle that does the bulk of the work in a movement. The synergist is a supporting muscle that assists the agonist in doing the movement or compensates if the agonist is injured/dysfunctional.

%
% Assignment 4
%
\page[yes]

\nameanddate{Megan Andrews}{February}{5}{2022}

%\chapter[ownnumber=3,title={Anatomy I: Basics}]
\vskip1.3em
\startalignment[middle]
\dontleavehmode
\raise0.35em\hbox{\rotate[rotation=180]{\scale[xscale=1260]{\symbol[zapfswash]}}}%
\hskip0.6em%
Topic 4%
\hskip0.6em
\raise0.35em\hbox{\rotate[rotation=180]{\scale[xscale=1260]{\mirror{\symbol[zapfswash]}}}}
\vskip-\lineheight
\vskip1ex
Anatomy II: Spine
\vskip0.5ex
\dontleavehmode
\scale[xscale=1885]{\mirror{\symbol[zapfswash]}}
\kern2.5pt
\scale[xscale=1885]{\symbol[zapfswash]}
\stopalignment
\vskip2.1ex

\question[ownnumber={4.1}, title={How many individual vertebrae are in the spinal column? How many vertebrae are in  each curve?}] 

There are 24 individual vertebrae in the spinal column. The curves and their respective number of vertebrae include:
\startitemize[symbol=mybullet1]
\item Cervical Vertebrae - 7
\item Thoracic Vertebrae - 12
\item Lumbar Vertebrae - 5
\item Sacral Vertebrae - 4-5 fused together
\item Coccygeal Vertebrae - 3-5 fused together
\stopitemize

\question[ownnumber={4.2},title={Which of the spinal curves are present  when a baby is born? At what point of development do the others appear?}]

After birth, a baby takes the shape of a comma, so there is a kyphodic curve/convexity in both the thoracic and sacral vertebrae. When the baby begins to lift it's head, the lordadic cervical curve develops. When the baby begins to walk, the lordadic lumbar curve develops.

\question[ownnumber={4.3},title={The cervical spine has the most freedom of movement in all directions. The thoracic and the lumbar areas have limited freedom in some directions. For each of these areas, list the actions and the structures that limit freedom of movement (Example: Thoracic spine: Flexion is limited by the ribs).}]

\startitemize[symbol=mybullet1]
\item Cervical Spine 
    \startitemize[symbol=mybullet2]    
    \item Rotation (turning one's head completely around) is limited by inelastic ligaments in the spine
    \item Lateral flexion may be limited by tension in the trapezius muscles and/or ligaments in the spine
    \stopitemize
\stopitemize
\page[yes]
\startitemize[symbol=mybullet1]
\item Thoracic Spine
    \startitemize[symbol=mybullet2]    
    \item Flexion and lateral flexion are limited by the ribs and ligaments on the posterior of the spine, and ligaments between the transverse processes respectively
    \item Extension  is limited by compression of the spinous processes and ligaments on the anterior of the spine
    \stopitemize
\item Lumbar Spine
    \startitemize[symbol=mybullet2]
    \item Rotation is limited by the shape and fit of the facet joints in this part of the spine
    \stopitemize
\item Sacral Spine
    \startitemize[symbol=mybullet2]
    \item Rotation, lateral flexion, flexion, and extension are limited by the fusion of the sacral vertebrae
    \stopitemize
\item Coccygeal Spine
    \startitemize[symbol=mybullet2]
    \item Rotation, lateral flexion, flexion, and extension are limited by the fusion of the coccygeal vertebrae
    \stopitemize
\stopitemize

%
% Assignment 7
%
\page[yes]

\nameanddate{Megan Andrews}{March}{26}{2022}

\vskip1.3em
\startalignment[middle]
\dontleavehmode
\raise0.35em\hbox{\rotate[rotation=180]{\scale[xscale=1500]{\symbol[zapfswash]}}}%
\hskip0.6em%
Topic 7%
\hskip0.6em
\raise0.35em\hbox{\rotate[rotation=180]{\scale[xscale=1500]{\mirror{\symbol[zapfswash]}}}}
\vskip-\lineheight
\vskip1ex
Anatomy III: Hips and Pelvis
\vskip0.5ex
\dontleavehmode
\scale[xscale=2130]{\mirror{\symbol[zapfswash]}}
\kern2.9pt
\scale[xscale=2130]{\symbol[zapfswash]}
\stopalignment
\vskip2.1ex

\question[ownnumber={7.1}, title={What muscles are the major hip extensors? Name two asanas that stretch these muscles and two that strengthen them.}] 

Hip Extensor Muscles: 
\startalignment[right]
\startitemize[symbol=mybullet1]
\item Hamstrings
    \startitemize[symbol=mybullet2]
    \item Stretch: Uttanasana
    \item Strengthen: Chair, 3 legged dog
    \stopitemize
\item Gluteus Maximus
    \startitemize[symbol=mybullet2]
    \item Stretch: Pigeon
    \item Strengthen: Half Bridge
    \stopitemize
\stopitemize
\stopalignment

\question[ownnumber={7.2},title={Which muscle is the prime mover in hip flexion?}]

Hip flexion prime mover muscle:
\startitemize[symbol=mybullet1]
    \item Psoas
\stopitemize

\question[ownnumber={7.3},title={How do the hip adductors help when inversions and in arm balances?}]

The hip adductors draw the inner thighs towards one another towards the midline of the body. This helps with centering and balance in inversions.

%
% Assignment 8
%
\page[yes]

\nameanddate{Megan Andrews}{March}{26}{2022}

\vskip1.3em
\startalignment[middle]
\dontleavehmode
\raise0.35em\hbox{\rotate[rotation=180]{\scale[xscale=1500]{\symbol[zapfswash]}}}%
\hskip0.6em%
Topic 8%
\hskip0.6em
\raise0.35em\hbox{\rotate[rotation=180]{\scale[xscale=1500]{\mirror{\symbol[zapfswash]}}}}
\vskip-\lineheight
\vskip1ex
Anatomy IV: Shoulder Girdle
\vskip0.5ex
\dontleavehmode
\scale[xscale=2130]{\mirror{\symbol[zapfswash]}}
\kern2.9pt
\scale[xscale=2130]{\symbol[zapfswash]}
\stopalignment
\vskip2.1ex

\question[ownnumber={8.1}, title={What is the major fuction of the rotator cuff muscles?}] 

They stabilize the humeral head and keep it in the scapula socket.

\question[ownnumber={8.2},title={When you lift your arm in abduction or flexion beyond 45 degrees, what action must the scapula do?}]

Upward rotation.

\question[ownnumber={8.3},title={Which 2 muscles stabilize the scapula on the back of the rib cage in Plank Poses and Chaturanga? In which direction does each muscle pull on the scapula?}]

Stabilizing Muscles: 
\startitemize[symbol=mybullet1]
    \item Serratus Anterior: Protracts the scapula
    \item Pectoralis Minor: Protracts and depresses the scapula
\stopitemize

%
% Assignment 9
%
\page[yes]

\nameanddate{Megan Andrews}{May}{30}{2022}

\vskip1.3em
\startalignment[middle]
\dontleavehmode
\raise0.35em\hbox{\rotate[rotation=180]{\scale[xscale=1500]{\symbol[zapfswash]}}}%
\hskip0.6em%
Topic 9%
\hskip0.6em
\raise0.35em\hbox{\rotate[rotation=180]{\scale[xscale=1500]{\mirror{\symbol[zapfswash]}}}}
\vskip-\lineheight
\vskip1ex
Anatomy V: Nervous system\par
A\&P of the Breath
\vskip0.5ex
\dontleavehmode
\scale[xscale=2130]{\mirror{\symbol[zapfswash]}}
\kern2.9pt
\scale[xscale=2130]{\symbol[zapfswash]}
\stopalignment
\vskip2.1ex

\question[ownnumber={9.1}, title={The involuntary Nervous System is commonly called the Autonomic Nervous System. There are two parts to this system. (A) What is each part called? (B) What happens when each part is active?}] 

(A) Sympathetic System and the Parasympathetic System
\blank[line]

\noindent(B)\par
\noindent Sympathetic system: 
\startalignment[right]
\startitemize[symbol=mybullet1]
\item Fight or flight system that is active when we are under stress
\item Blood pressure and heart rate increase
\item Blood directed to muscles of arms and legs
\item Dilated pupils
\item Skin is cool, wet, sweaty
\stopitemize
\stopalignment
\blank[line]
\noindent Parasympathetic System:
\startalignment[right]
\startitemize[symbol=mybullet1]
\item Maintains and conserves body during periods of low stress
\item Responsible for breathing, digestion, elimination
\item Blood directed to organs
\item Pupils contract
\item Warm skin
\stopitemize
\stopalignment

\question[ownnumber={9.2},title={Briefly describe what the diaphragm does during inhalation and exhalation.}]

During inhalation, the diaphragm contracts and flattens while it moves downwards and presses against the abdominal organs. Therefore, there is more space inside the thoracic cavity. Duing exhalation, the diaphragm relaxes into its mushroom/parachute shape and rises upwards.


%
% Assignment 11
%
\page[yes]

\nameanddate{Megan Andrews}{May}{30}{2022}

\vskip1.3em
\startalignment[middle]
\dontleavehmode
\raise0.35em\hbox{\rotate[rotation=180]{\scale[xscale=1500]{\symbol[zapfswash]}}}%
\hskip0.6em%
Topic 11%
\hskip0.6em
\raise0.35em\hbox{\rotate[rotation=180]{\scale[xscale=1500]{\mirror{\symbol[zapfswash]}}}}
\vskip-\lineheight
\vskip1ex
Anatomy VI: \par
Circulations \& Inversions
\vskip0.5ex
\dontleavehmode
\scale[xscale=2130]{\mirror{\symbol[zapfswash]}}
\kern2.9pt
\scale[xscale=2130]{\symbol[zapfswash]}
\stopalignment
\vskip2.1ex

\question[ownnumber={11.1}, title={Briefly explain how gravity affects blood pressure.}] 

\vskip-0.5\baselineskip\relax

Blood is assisted back to the heart from the lower extremities, and this causes blood pressure to be the greatest in the head and the lowest in the feet.

\question[ownnumber={11.2}, title={Briefly describe how inverted postures might serve to lower one's blood pressure, and how they might serve to elevate one's blood pressure.}] 

\vskip-0.5\baselineskip
On the one hand, blood pressure will immediately increase in the head and upper body. However, on balance, inversions will decrease overall blood pressure through reflex hypo\-tension.

\question[ownnumber={11.3}, title={Some students should not do inversions. List 5 contraindications for performing inversions.}]

\vskip-0.5\baselineskip\relax

\startalignment[right]
\startitemize[n]
\item Glaucoma
\item Detached retina
\item History of stroke
\item High or low blood pressure
\item Recent dental or facial surgery
\stopitemize
\stopalignment

\question[ownnumber={11.4}, title={List 2 Positive effects and 2 negative effects of performing inversions in a regular class (physiological or psychological).
}]

\vskip-0.5\baselineskip

Positive Effects:
\startalignment[right]
\startitemize[symbol=mybullet1]
\item Teaches students to move in to their fears
\item Can decrease overall blood pressure through reflex hypotension
\stopitemize
\stopalignment
\noindent Negative Effects:
\startalignment[right]
\startitemize[symbol=mybullet1]
\item Potential risk to cervical vertebral discs and nerve damage
\item Blood pooling on the head, no valves to assist blood flow back to the heart
\stopitemize
\stopalignment

%
% Assignment 14
%
\page[yes]

\nameanddate{Megan Andrews}{March}{26}{2022}

\vskip1.3em
\startalignment[middle]
\dontleavehmode
\raise0.35em\hbox{\rotate[rotation=180]{\scale[xscale=1500]{\symbol[zapfswash]}}}%
\hskip0.6em%
Topic 14%
\hskip0.6em
\raise0.35em\hbox{\rotate[rotation=180]{\scale[xscale=1500]{\mirror{\symbol[zapfswash]}}}}
\vskip-\lineheight
\vskip1ex
{What is Energy:\par Shakti,
Prana and Pranayama}
\vskip0.5ex
\dontleavehmode
\scale[xscale=2130]{\mirror{\symbol[zapfswash]}}
\kern2.9pt
\scale[xscale=2130]{\symbol[zapfswash]}
\stopalignment
\vskip2.1ex

\questionnonumber{Describe the anatomy and physiology of a Full Complete Breath. Describe how you would teach it to a beginner, in your own words. Why would you teach it that way?} 

{\sc Anatomy & Physiology}
\startnarrower[left]
\noindent Full complete breath is a deep and deliberate way of breathing that works to draw the breath into all dimensions and directions of the torso. It slows and lengthens the inhalations and exhalations in a mindful way while one's attention turns inwards (pratyahara). It is effective in breathing softness into points of tension in the body, and is an effective salve for stress and anxious dispositions. 

In order, the breath moves into the chest, swelling down into the ribs, and finally inflating the lower abdomen/belly. With this, the “parachute-like” diaphragm lowers. 

Upon exhalation, the breath leaves the lower abdomen - where the navel draws closer to the spine), deflates in the ribs and, lastly, leaves the chest. The diaphragm floats back upwards in the thoracic cavity.
\stopnarrower
\blank[line]
\noindent{\sc Teaching Full Complete Breath}
\startnarrower[left]
\startalignment[right]
\startitemize[symbol=mybullet1]
\item Start with laying on your back. Bend the knees and place the feet around mat-width distance apart, allowing the knees to fall in against one another.Try lifting the head and drawing the chin down slightly to lengthen the spine. Place one hand over the chest, and the other on top of your stomach. 
\item As you breathe in through the nose, feel the chest rise, the ribs expand in all directions, and finally - the belly rise.
\item As you breathe out, feel the stomach sink back down towards the spine, the ribs deflate, and the chest lower back down.
\item Try slowing down your inhalations and exhalations to savour all the sensations and movements in your abdomen. As you're breathing in, feel the hands move further apart from each other. On the exhale, feel the hands move closer to one another.
\item Harness your breath, this life force (prana) for maximum expansion and softness. Feel empowered to make more room in your body for more of this life force.
\item Notice:
    \startitemize[symbol=mybullet2] 
        \item Can you breathe fully into the back of the ribs, feeling them slightly press into the ground beneath you?
        \item If the mind wanders, try to bring the focus back to the tangible aspects of the breath:  the coolness of the air in your nostrils, the rise and fall of your abdomen beneath your fingertips.
    \stopitemize
\stopitemize
\stopalignment
\stopnarrower

%
% Assignment 15
%
\page[yes]

\nameanddate{Megan Andrews}{March}{26}{2022}

\vskip1.3em
\startalignment[middle]
\dontleavehmode
\raise0.35em\hbox{\rotate[rotation=180]{\scale[xscale=1500]{\symbol[zapfswash]}}}%
\hskip0.6em%
Topic 15%
\hskip0.6em
\raise0.35em\hbox{\rotate[rotation=180]{\scale[xscale=1500]{\mirror{\symbol[zapfswash]}}}}
\vskip-\lineheight
\vskip1ex
{Pranayama Techniques I}
\vskip0.5ex
\dontleavehmode
\scale[xscale=2130]{\mirror{\symbol[zapfswash]}}
\kern2.9pt
\scale[xscale=2130]{\symbol[zapfswash]}
\stopalignment
\vskip2.1ex

\questionnonumber{Describe how to teach Ujjayi breath in your own words to a beginner} 

\startalignment[right]
\startitemize[symbol=mybullet1]
\item Find your way to a comfortable seat. Allow the sit bones to sink down into the ground, with the pelvis stacked evenly over top. Let this foundation support the spine effortlessly.
\item Take your time to arrive in a place of effortlessness, of ease. Nothing in the body should feel strained. Tweak your position as needed. You're welcome to close the eyes or soften your gaze downwards to a point on the ground.
\item Bring awareness to the natural breath. Can it flow smoothly in through the mouth and down the length of the spine? Is there any tension in the body obstructing its passage?
\item Now place the palm in front of the mouth, and exhale into your hand, as if you were trying to fog a piece of glass before cleaning it – eyeglasses for instance.
\item *Demonstrate*
\item To create this sort of breath, you might notice a slight constriction in the back of your throat. Stay with this sensation, and when you feel used to it, continue it with your mouth closed. Observe the sound it makes inside the throat. It reminds me of a wind passing through trees, or the ebb and flow of an ocean tide.
\item Now apply this whispery constriction to both your inhale and exhale. Notice and embrace the effect it has on the pace and length of your breath. 
\stopitemize
\stopalignment

%
% Assignment 19
%
\page[yes]

\nameanddate{Megan Andrews}{May}{30}{2022}

\vskip1.3em
\startalignment[middle]
\dontleavehmode
\raise0.35em\hbox{\rotate[rotation=180]{\scale[xscale=1500]{\symbol[zapfswash]}}}%
\hskip0.6em%
Topic 19%
\hskip0.6em
\raise0.35em\hbox{\rotate[rotation=180]{\scale[xscale=1500]{\mirror{\symbol[zapfswash]}}}}
\vskip-\lineheight
\vskip1ex
{Prenatal Yoga}
\vskip0.5ex
\dontleavehmode
\scale[xscale=2130]{\mirror{\symbol[zapfswash]}}
\kern2.9pt
\scale[xscale=2130]{\symbol[zapfswash]}
\stopalignment
\vskip2.1ex

\questionnonumber{List 5 contraindicated poses during pregnancy. List 5 poses to replace them, which at least accomplish some component of the contraindicated pose.} 

Jumping back to plank — Stepping back to plank
\blank[line]
\noindent Wheel — Bridge
\blank[line]
\noindent Bow — Seated side leans to open breathing space
\blank[line]
\noindent Kapalabhati — Citali
\blank[line]
\noindent Marichasana — Seated Angular Twists

%
% Script 1
%
\page[yes]

\nameanddate{Megan Andrews}{February}{5}{2022}

\vskip1.3em
\startalignment[middle]
\dontleavehmode
\raise0.35em\hbox{\rotate[rotation=180]{\scale[xscale=1450]{\symbol[zapfswash]}}}%
\hskip0.6em%
Script 1%
\hskip0.6em
\raise0.35em\hbox{\rotate[rotation=180]{\scale[xscale=1450]{\mirror{\symbol[zapfswash]}}}}
\vskip-\lineheight
\vskip1ex
Virabhadrasana I / Warrior I
\vskip0.5ex
\dontleavehmode
\scale[xscale=2100]{\mirror{\symbol[zapfswash]}}
\kern2.5pt
\scale[xscale=2100]{\symbol[zapfswash]}
\stopalignment
\vskip2.1ex

\noindent {\sc Transition from Downward Dog}
\startitemize[symbol=mybullet1]
\item \color[inhale]{Inhale}: Look between your hands and send the right foot gently between the palms.\par
\noindent *\color[exhale]{Exhale}*
\item Swivel the back foot to about 45 degrees, keeping the outer edge of the foot firmly planted 
\item \color[inhale]{Inhale}: Rise up, arms stretched upwards, and welcome to warrior one\par
\noindent *\color[exhale]{Exhale}* 
\item As you arrive in Virabhadrasana, start to notice in which ways your body could use more support and ease. 
    \startitemize[symbol=mybullet2]
    \item Maybe allow your hips more space by moving the feet wider apart/toward the edges of the mat
    \item Play with the position of the arms
        \startitemize[symbol=mybullet3]
        \item They don't have to be beside the ears, they can be slightly in front
        \stopitemize
    \stopitemize
\item Stay with your breath
\item What feedback are you getting from your feet? Allow the stability from your feet to spread up through the legs, so that each leg supports your weight equally with the torso well balanced in the center
\item Imagine the pelvis is a bowl filled with liquid that you want to keep contained
    \startitemize[symbol=mybullet2]
    \item It doesn't tilt too forward or backward
    \item Can you square your pelvis with the front of your mat?
    \stopitemize
\item Draw the navel to the spine and start to lengthen up to the crown of your head
\item Breathe in through the crown, down the spine, through the feet and into the ground below
\stopitemize
\blank[line]
\noindent {\sc Transition Out}
\startitemize[symbol=mybullet1]
\item *\color[inhale]{Inhale}*
\item \color[exhale]{Exhale}: Frame the front foot with your hands
\item \color[inhale]{Inhale}: Step back to downward dog
\item \color[exhale]{Exhale}: Release all the air out of your body and settle into downward dog for 5 breaths
\stopitemize
\blank[line]
\noindent {\sc Repeat Other Side}
\blank[line]
\noindent {\sc Modification}
\startitemize[symbol=mybullet1]
\item If you feel discomfort, you're welcome to come to a high lunge instead
    \startitemize[symbol=mybullet2]
    \item Just swivel the back foot forward and lift the heel
    \stopitemize
\stopitemize

%
% Script 2
%
\page[yes]

\nameanddate{Megan Andrews}{March}{26}{2022}

\vskip1.3em
\startalignment[middle]
\dontleavehmode
\raise0.35em\hbox{\rotate[rotation=180]{\scale[xscale=1450]{\symbol[zapfswash]}}}%
\hskip0.6em%
Script 2%
\hskip0.6em
\raise0.35em\hbox{\rotate[rotation=180]{\scale[xscale=1450]{\mirror{\symbol[zapfswash]}}}}
\vskip-\lineheight
\vskip1ex
Trikonasana
\vskip0.5ex
\dontleavehmode
\scale[xscale=2100]{\mirror{\symbol[zapfswash]}}
\kern2.5pt
\scale[xscale=2100]{\symbol[zapfswash]}
\stopalignment
\vskip2.1ex

\blank[line]
\noindent(Preparation: Keep one block close to the front of your mat.)
\blank[line]
\noindent {\sc Transition from Warrior 2 (right leg forward)}
\startitemize[symbol=mybullet1]
    \item Straighten the right leg and shorten your stance, but try to keep that bouncy microbend in the front knee. Turn the left foot inward slightly.
\item Take a moment to notice that all four corners of the feet are rooted firmly in the ground. Let this be the foundation from which you draw all your stability and balance.
\item \color[inhale]{Inhale}: lengthen the spine up through the crown of the head
\item \color[exhale]{Exhale}: Let the right arm lead the torso forward over the right leg, as if you're reaching for something. Allow the left hip to slide back and draw the torso over the right leg. 
\item Drop the right arm, placing your hand on the block nearby.  Keep the integrity of the spine and the waist equally long on both sides. Use the height of the block that allows you to maintain this balance.
\item Allow the left arm to float up towards the sky, reaching the fingertips upwards to continue the line of your lower arm.
\item \color[inhale]{Inhale}: Roll the chest open slightly towards the sky, perhaps allowing the gaze to follow if it doesn't strain the neck.
\item \color[exhale]{Exhale}: Find expansion in the chest for the breath to flow in and out, and take any adjustments as needed.
\stopitemize
\blank[line]
\noindent {\sc Transition Out}
\startitemize[symbol=mybullet1]
\item \color[inhale]{Inhale}: Bend the front knee generously and press the feet in to the ground
\item \color[exhale]{Exhale}: Activate the lower abdomen and start to raise the torso back up 
\stopitemize
\blank[line]
\noindent {\sc Repeat Other Side}


%
% Script 3
%
\page[yes]

\nameanddate{Megan Andrews}{May}{30}{2022}

\vskip1.3em
\startalignment[middle]
\dontleavehmode
\raise0.35em\hbox{\rotate[rotation=180]{\scale[xscale=1500]{\symbol[zapfswash]}}}%
\hskip0.6em%
Script 3%
\hskip0.6em
\raise0.35em\hbox{\rotate[rotation=180]{\scale[xscale=1500]{\mirror{\symbol[zapfswash]}}}}
\vskip-\lineheight
\vskip1ex
{Urdva Danurasana}
\vskip0.5ex
\dontleavehmode
\scale[xscale=2130]{\mirror{\symbol[zapfswash]}}
\kern2.9pt
\scale[xscale=2130]{\symbol[zapfswash]}
\stopalignment
\vskip2.1ex

%\item \color[inhale]{Inhale}: Rise up, arms stretched upwards, and welcome to warrior one\par

\blank[line]
\noindent {\sc Transition from Laying Down}
\startitemize[symbol=mybullet1]
\item Come to lay down on your back
\item Engage full complete breath, breathing in to all directions of the ribcage from the belly up to the chest
\item Use this breath create more space in the thoracic cavity
\stopitemize
\blank[line]
\noindent {\sc Transition from Setu Bandha Sarvangasana}
\startitemize[symbol=mybullet1]
\item Return to your regular breath
\item Bend your knees, placing your feet underneath your knees, close to your seat
\item Notice that the feet are spaced about hip-width apart
\item Place the arms by your side, palms faced down and pressing in to the ground
\item Tuck in your elbows and shoulder blades towards the midline of the body
\item Press your feet in to the ground to lift the thighs and hips towards the sky
\stopitemize
\blank[line]
\noindent {\sc Urdva Danurasana}
\startitemize[symbol=mybullet1]
\item Extend arms over head, fingers directed towards the heels
\item Tuck in the shoulder blades
\item \color[inhale]{Inhale}: As you inhale, press through the hands and feet, to lift the chest and heart center upwards
\item Attention to the neck: Keep it in line with the rest of the spine or allow the head to hang loose
\stopitemize
\blank[line]
\noindent {\sc Transition Out}
\startitemize[symbol=mybullet1]
\item \color[exhale]{Exhale}: On the next exhale, slowly bend the elbows to lower yourself down while you begin to tuck in your chin
\item Starting with the back of your head, lower your spine down in a wave: from the back of the head to the sacrum.
\stopitemize
\blank[2*line]
\noindent {\sc Modification}
\startitemize[symbol=mybullet1]
\item Option to stay in bridge or supported bridge pose. 
\stopitemize

%
% Script 4
%
\page[yes]

\nameanddate{Megan Andrews}{May}{30}{2022}

\startalignment[middle]
\dontleavehmode
\raise0.35em\hbox{\rotate[rotation=180]{\scale[xscale=1500]{\symbol[zapfswash]}}}%
\hskip0.6em%
Script 4%
\hskip0.6em
\raise0.35em\hbox{\rotate[rotation=180]{\scale[xscale=1500]{\mirror{\symbol[zapfswash]}}}}
\vskip-\lineheight
\vskip1ex
{Sirsasana}
\vskip0.5ex
\dontleavehmode
\scale[xscale=2130]{\mirror{\symbol[zapfswash]}}
\kern2.9pt
\scale[xscale=2130]{\symbol[zapfswash]}
\stopalignment
\vskip2.1ex

\noindent {\sc Preparation}
\startitemize[symbol=mybullet1]
\item Have the mat placed in front of a wall before you enter Adho Mukha Svanasana/\-Downward Dog
\stopitemize
\blank[line]
\noindent {\sc From Adho Mukha Svanasana to Catur Svanasana}
\startitemize[symbol=mybullet1]
\item Lower down onto your forearms, keeping the elbows beneath the shoulders
\item Interlace the fingers
\item \color[inhale]{Inhale}: Allow your forearms to bear more of your body weight as you tip-toe the feet closer to the shoulders. Eventually stack your hips above your shoulders. 
\item \color[exhale]{Exhale}: Lower the crown of your head on to the mat between the forearms.
\item Pause for a couple breaths in Dolphin Pose  
\item \color[inhale]{Inhale}: Press through the shoulders and fold the knees in towards your abdomen
\item Feel that the weight pours into your forearms more than your head
\item \color[exhale]{Exhale}: Uncurl the legs and extend them up towards the ceiling, pointing your toes (plantar flexion)
\item Maintain your breath and picture the spine mimicking that of a Tadasana spine.
\item Have your focal point directly in front of you to assist with balance
\stopitemize
\blank[line]
\noindent {\sc Transition Out}
\startitemize[symbol=mybullet1]
\item \color[exhale]{Exhale}: With strong engagement of the core and pelvic floor, start to bend at the hips and curl the knees back in towards the abdomen
\item Keep using the support of your forearms
\item Allow the toes to touch back to your mat
\item \color[inhale]{Inhale}: Release the head
\item \color[exhale]{Exhale}: Send the seat back towards your heels for a yummy Child's Pose
\stopitemize
\blank[line]
\noindent {\sc Modifications}
\startitemize[symbol=mybullet1]
\item Stay in Dolphin Pose/Catur Svanasana to experiment with gradually introducing more weight and load into both the forearms and shoulders.
    \startitemize[symbol=mybullet2]
    \item Option to lift one leg at a time in Dolphin Pose
    \item Practice routinely for upper body strengthening
    \stopitemize
\stopitemize

%
% Script 5
%
\page[yes]

\nameanddate{Megan Andrews}{May}{30}{2022}

\startalignment[middle]
\dontleavehmode
\raise0.35em\hbox{\rotate[rotation=180]{\scale[xscale=1500]{\symbol[zapfswash]}}}%
\hskip0.6em%
Script 5%
\hskip0.6em
\raise0.35em\hbox{\rotate[rotation=180]{\scale[xscale=1500]{\mirror{\symbol[zapfswash]}}}}
\vskip-\lineheight
\vskip1ex
{Janu Sirsasana}
\vskip0.5ex
\dontleavehmode
\scale[xscale=2130]{\mirror{\symbol[zapfswash]}}
\kern2.9pt
\scale[xscale=2130]{\symbol[zapfswash]}
\stopalignment
\vskip2.1ex

\noindent {\sc Transition From Dandasana}
\startitemize[symbol=mybullet1]
\item Come to sit with your legs straight in front of you with flexed feet (dorsal flexion)
\item Keep your legs active with a microbend in the knee
\item Breathe length in the spine, allowing the crown to reach toward the ceiling
\item Bend in the left knee, such that the sole of the foot presses against the upper thigh of the opposite leg
\item \color[inhale]{Inhale}
\item \color[exhale]{Exhale}: Rotate the torso to “face” the extended leg
\item \color[inhale]{Inhale}: Breathe space in to the midline of the body
\item \color[exhale]{Exhale}: Hinge from the hips to slowly fold over the extended leg
\item Finding the “edge:”
    \startitemize[symbol=mybullet2]
        \item Go as far as your body can without compromising important components of the posture:
        \startitemize[symbol=mybullet3]
        \item Keep the spine naturally curved, as opposed to rounded
        \item Do not collapse the chest, keep your collarbones wide 
        \item Strive to move “forward” as well as downwards
        \item IF the belly reaches the thigh, you may allow for a rounding of the spine
        \stopitemize    
    \stopitemize
\stopitemize
\blank[line]
\noindent {\sc Coming out of the Pose}
\startitemize[symbol=mybullet1]
\item \color[inhale]{Inhale}: Engage the core and lift the torso back upwards
\item \color[exhale]{Exhale}
\item Straighten both legs and give them a bit of a jiggle
\stopitemize
\blank[line]
\noindent {\sc (*Repeat other Side*)}
\blank[line]
\noindent {\sc Modifications}
\startitemize[symbol=mybullet1]
\item Elevate your seat on a blanket 
\item Use blocks for your hands to rest on 
\stopitemize

\stoptext
