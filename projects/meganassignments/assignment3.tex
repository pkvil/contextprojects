\environment assignments-environment

\starttext

\nameanddate{Megan Andrews}{January}{29}{2022}

%\chapter[ownnumber=3,title={Anatomy I: Basics}]
\vskip1.3em
\startalignment[middle]
\dontleavehmode
\raise0.35em\hbox{\rotate[rotation=180]{\scale[xscale=1260]{\symbol[zapfswash]}}}%
\hskip0.6em%
Topic 3%
\hskip0.6em
\raise0.35em\hbox{\rotate[rotation=180]{\scale[xscale=1260]{\mirror{\symbol[zapfswash]}}}}
\vskip-\lineheight
\vskip1ex
Anatomy I: Basics
\vskip0.5ex
\dontleavehmode
\scale[xscale=1885]{\mirror{\symbol[zapfswash]}}
\kern3pt
\scale[xscale=1885]{\symbol[zapfswash]}
\stopalignment
\vskip2.1ex

\question[ownnumber=3.1, title=Briefly describe each of these tissues and its function: Muscle, Ligament, Tendon and Cartilage] 

{\sc A muscle} is tissue consisting of bundles of parallel fibers wrapped in fascia.
\blank[line]

\start
\setupalign[nothyphenated]
\startcolumns[n=2,separator=rule,balance=no]
\startalignment[middle]
Properties
\stopalignment
\startalignment[right]
\startitemize[symbol=mybullet1]
\item Excitable – Reacts to stimuli via innervation
\item Contractible – May contract upon stimulation in 3 different ways
    \startitemize[symbol=mybullet2]
    \item Concentrically – Muscle shortens
    \item Eccentrically – Muscle lengthens
    \item Isometrically  – Muscle neither shortens or lengthens during contraction
    \stopitemize
\item Extensible - May be stretched
\item Elastic - Able to recoil
\stopitemize
\stopalignment
\column
\startalignment[middle]
Roles
\stopalignment
\startitemize[symbol=mybullet1]
\item Agonist – Prime mover muscle
\item Synergist – Supporting muscle to help the agonist in movement
\item Antagonist – Opposes prime mover to slow and monitor movement
\item Protects joint
\item Stabilizer – A muscle that pairs with an opposing muscle to keep a bone in place
\stopitemize
\stopcolumns

\stop

\start
\setupalign[nothyphenated]

\noindent Properties:

\startalignment[right]
\startitemize[symbol=mybullet1]
\item Excitable – Reacts to stimuli via innervation
\item Contractible – May contract upon stimulation in 3 different ways
    \startitemize[symbol=mybullet2]
    \item Concentrically – Muscle shortens
    \item Eccentrically – Muscle lengthens
    \item Isometrically  – Muscle neither shortens or lengthens during contraction
    \stopitemize
\item Extensible - May be stretched
\item Elastic - Able to recoil
\stopitemize
\stopalignment

\noindent Roles:

\startitemize[symbol=mybullet1]
\item Agonist – Prime mover muscle
\item Synergist – Supporting muscle to help the agonist in movement
\item Antagonist – Opposes prime mover to slow and monitor movement
\item Protects joint
\item Stabilizer – A muscle that pairs with an opposing muscle to keep a bone in place
\stopitemize


\stop



%LIGAMENT: 
A {\sc ligament} is a dense piece of connective tissue that connects one bone to another bone.
Protects joints
Avascular - no blood supply
Barely elastic → difficult to heal

%TENDON: 
A {\sc tendon} is a tough, fibrous continuation of muscle fascia that attaches muscle to bone.
Stabilizes joint
Transmits and supports muscle force
Woven into the periosteum/outer shell of the bone


CARTILAGE: Cartilage is supplementary tissue to muscle providing strength, rigidity, and a little elasticity.

Fibrocartilage
Hyaline Cartilage
Cushioning, pillowy 
Shock-absorbing
Avascular - no blood supply
Smooth, glassy
Reinforcing coat where bones rub against one another
Avascular - no blood supply



\question{The elbow is a hinge joint and the hip is a ball-and-socket joint.
What are the movements of a hinge joint? Name the plane.}

The movements of a hinge joint are flexion and extension, and the plane is the Sagittal Plane.

\question{What are the movements of a ball-and-socket joint? In how many planes does a ball-and-socket joint move and what are the actions in each of the planes?}

The movements of a ball-and-socket joint and their corresponding planes are:
1. Rotation (internal and external) in the Transverse Plane
2. Flexion & Extension in the Sagittal Plane
3. Adduction & Abduction in Coronal Plane
* Circumduction involves all movements and planes in combination

\question{What is the job of an agonist? What is the job of a synergist?}

An agonist is the prime mover muscle that does the bulk of the work in a movement. The synergist is a supporting muscle that assists the agonist in doing the movement or compensates if the agonist is injured/dysfunctional.

\stoptext

