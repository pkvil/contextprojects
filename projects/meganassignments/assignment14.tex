\environment assignments-environment

\starttext

\nameanddate{Megan Andrews}{March}{26}{2022}

\vskip1.3em
\startalignment[middle]
\dontleavehmode
\raise0.35em\hbox{\rotate[rotation=180]{\scale[xscale=1500]{\symbol[zapfswash]}}}%
\hskip0.6em%
Topic 14%
\hskip0.6em
\raise0.35em\hbox{\rotate[rotation=180]{\scale[xscale=1500]{\mirror{\symbol[zapfswash]}}}}
\vskip-\lineheight
\vskip1ex
{What is Energy:\\ Shakti,
Prana and Pranayama}
\vskip0.5ex
\dontleavehmode
\scale[xscale=2130]{\mirror{\symbol[zapfswash]}}
\kern2.9pt
\scale[xscale=2130]{\symbol[zapfswash]}
\stopalignment
\vskip2.1ex

\questionnonumber{Describe the anatomy and physiology of a Full Complete Breath. Describe how you would teach it to a beginner, in your own words. Why would you teach it that way?} 

{\sc Anatomy & Physiology}
\startnarrower[left]
\noindent Full complete breath is a deep and deliberate way of breathing that works to draw the breath into all dimensions and directions of the torso. It slows and lengthens the inhalations and exhalations in a mindful way while one's attention turns inwards (pratyahara). It is effective in breathing softness into points of tension in the body, and is an effective salve for stress and anxious dispositions. 

In order, the breath moves into the chest, swelling down into the ribs, and finally inflating the lower abdomen/belly. With this, the “parachute-like” diaphragm lowers. 

Upon exhalation, the breath leaves the lower abdomen - where the navel draws closer to the spine), deflates in the ribs and, lastly, leaves the chest. The diaphragm floats back upwards in the thoracic cavity.
\stopnarrower
\blank[line]
\noindent{\sc Teaching Full Complete Breath}
\startnarrower[left]
\startalignment[right]
\startitemize[symbol=mybullet1]
\item Start with laying on your back. Bend the knees and place the feet around mat-width distance apart, allowing the knees to fall in against one another.Try lifting the head and drawing the chin down slightly to lengthen the spine. Place one hand over the chest, and the other on top of your stomach. 
\item As you breathe in through the nose, feel the chest rise, the ribs expand in all directions, and finally - the belly rise.
\item As you breathe out, feel the stomach sink back down towards the spine, the ribs deflate, and the chest lower back down.
\item Try slowing down your inhalations and exhalations to savour all the sensations and movements in your abdomen. As you're breathing in, feel the hands move further apart from each other. On the exhale, feel the hands move closer to one another.
\item Harness your breath, this life force (prana) for maximum expansion and softness. Feel empowered to make more room in your body for more of this life force.
\item Notice:
    \startitemize[symbol=mybullet2] 
        \item Can you breathe fully into the back of the ribs, feeling them slightly press into the ground beneath you?
        \item If the mind wanders, try to bring the focus back to the tangible aspects of the breath:  the coolness of the air in your nostrils, the rise and fall of your abdomen beneath your fingertips.
    \stopitemize
\stopitemize
\stopalignment

\stoptext

