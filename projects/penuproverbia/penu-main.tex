\input penu-macros

\define[1]\rubrik{%
    \blank[line]%
    \centerline{%
        \bigfraktur% 
        \color[red]{#1}%
    }%
    \blank[line]%
}%

\starttext

\rubrik{Aa}

{\bigfraktur Aas lockar Örnen vth.}
{\smallroman Id est,}
{\smallfraktur ther Maat wanckar, tijt sanckas och Äthare.}
wanckar --- SAOB VANKA, förekomma, finnas
sanckas --- SAOB SAMKA, samla sig, församlas

\rubrik{Ab}

{\bigfraktur Abbothen bär Thärning, så haar Muncken godt speela.}
{\smallroman Id est,}
{\smallfraktur aff Herrars onda exempel, blijr Tienaren offta förförd.}
\blank[line]
{\bigfraktur Abborren haar goda dagar, han dricker när han will.}
{\smallroman Id est,}
{\smallfraktur Den som haar Korszet i handen, han signar sigh sielfwan först. Som man elliest wille säya:
Den som råder om Kiök och Kiällare i Hwset, han Äter och Dricker när han will.}

\rubrik{Ac}

{\bigfraktur Achta tigh för Hunden, Skuggen bijtz intet.\par ABCDEFGHIJKLMNOPQRSTUVWXYZÅÄÖ}
{\smallroman Id est,}
{\smallfraktur Hoot dräper ingen. Man säger förthenskull: Dhen som döör aff hoot, han bötes igen medh Morgonkrydder.}
bötes (pres. pass. ind. sg.): botas --- SAOB BÖTA 2), bota, hela, kurera

\stoptext
Id est. 
[⋯]
Der Åthelen är, tijt försambla sigh och Örnarna.
[⋯]
wanckar → SAOB VANKA, förekomma, finnas
sanckas → SAOB SAMKA, samla sig, församlas
⇒ Swenske ordsedher (1604) #1179:
Ther som aset är tit församblar sigh örnenar.
⇒ Swenske ordsedher (1604) #492:
Huarest aset är ther är korpen.
huarest: där → SAOB VAREST I. (ålderdomligt) såsom adv. 2) (numera mindre brukligt) såsom rel. adv.: där

#2 Abbothen bär Thärning, så haar Muncken godt speela.
i. e. Aff Herrars onda exempel, blijr Tienaren offta förförd.
[⋯]

#3 Abborren haar goda dagar, han dricker när han will.
i. e. Den som haar Korszet i handen, han signar sigh sielfwan först.
[⋯]
Som man elliest wille säya:
Den som råder om Kiök och Kiällare i Hwset, han Äter och Dricker när han will.

                                              — 2 —
#1 Achta tigh för en förlijkt Fiende.
i. e. Troo ingen twungen Wänskap. Ty gambla Såår blöda gärna.
[⋯]

#2 Achte hwar sitt, låte annars fara.
i. e. Bekymbre sig ingen om annars wärck.
[⋯]
annars (arkais.; gen. sg. till fsv. annar): annans → SAOB ANNAN
wärck: verksamhet → SAOB VERK 1) handling, gärning; verksamhet, arbete

#3 Achta tigh för Hunden, Skuggen bijtz intet.
i. e. Hoot dräper ingen. Man säger förthenskull:
Dhen som döör aff hoot, han bötes igen medh Morgonkrydder.
[⋯]
bötes (pres. pass. ind. sg.): botas → SAOB BÖTA 2), bota, hela, kurera

                                              — 3 —
#1  Acht kommer intet allom i Pungen.
i. e. Alla Tanckar komma intet i Säcken.
Item: All Anslagh lyckas intet.
allom (arkais.; dat. pl. till all): alla → SAOB ALL
[…] kommer intet allom […]: alla får inte […]
pungen → SAOB PUNG, portmonnä, börs
anslagh → SAOB ANSLAG 12) a) plan, avsikt, uppsåt, rådslut, rådslag, beslut
⇒ Swenske ordsedher (1604) #68:
Acht kommer ey allom i pungh.

#2  Adel vthan Dygd, är Lychta vthan Liws.
i. e. Som man wille säya: Hwad är Adelskap vthan Dygd och Ähra?
Derföre säger man och: Virtus nobilitat. Dygd gör Adel. (”Dygd förädlar.”)
[⋯]
liws → SAOB LJUS

#3  Adeligh och Ährligh.
i. e. All Adeligh Gärning bör hafwa Ähran i föllie.
[⋯]

#4 Aff Barn blijr ock gammalt Folck.
i. e. Som man wille säya: Oxen haar och warit Kalff.
[⋯]

                                              — 4 —
#1 Aff Eelden blijr man bränder, och aff Skiökian skämder.
i. e. Then som fååsz wijd Tiäran han blifwer ther aff besmittad.
[⋯]
fååsz wijd → SAOB FÅ, ha att göra med (någon / något), röra vid (något)

#2 Aff Flugan en Elephant.
i. e. Giöra stoort aff lijtet. Tyda alt til thet wärsta.
[⋯]

#3 Affbedia, är bästa boot.
i. e. När man vthi något förseende bedzföre, så synes nog wara bött.
Erranti medicina confessio. haar Cicero sagt. (”För den som begår ett fel är en bekännelse botmedlet.”)
[⋯]
affbedia → SAOB AFBEDJA, bedja om tillgift / nåd / förskoning
bedzföre → SAOB FÖR- | -BEDJA II. bedja (någon) om ursäkt, om förlåtelse
bedzföre: beds om ursäkt
bött (perf. part.): bötat → SAOB BÖTA 4) a) gottgöra, sona, försona (synd, brott och dyl.)

#4 Aff elackt Läder görs slemma Skoor.
i. e. Ondt göra godt aff thet som intet doger.
[⋯]
elackt: dåligt → SAOB ELAK 1) (†) dålig med avseende på beskaffenhet (kvalitet)
slemma → SAOB SLEM, dålig, underhaltig, undermålig
ondt (adv.): svårt → SAOB OND A. 1) svår, besvärlig
doger: duger → SAOB DUGA (doga)

#5 Aff gambla Oxen lärer then vnga draga.
i. e. Fadrens dygd gör Sonen lijk.
[⋯]
Grymtar så  Grijs som gammalt Swijn.
[⋯]
⇒ Swenske ordsedher (1604) #3:
Aff thet stoora nöthet lärer vngnöt dragha.

#6 Aff HErrar kan man både wärma och bränna sigh.
i. e. Med höga Herrar måste man vmgå såsom medh Eelden;
Är man för långt ifrå, så fryys man,
i. e. Niuter them intet til goda;
Gåår man them för näär, så bränner man sigh,
i. e. Råkar i miszhagh och onåder.
[⋯]
vmgå medh (tr.): umgås med → SAOB UMGÅS / UMGÅ 2) a) (†) mer / mindre regelbundet träffa / vara tillsammans med, ha såsom sällskap
ifrå → SAOB FRÅN
niuter til goda → SAOB NJUTA något till godo, ha fördel / nytta / glädje / hjälp av något

                                              — 5 —
#1 Aff skoda wäxer Kärleek.
i. e. Den man aldrigh sågh, kan man intet blifwa käär åth.
[⋯]

#2 Aff Kläderna kännes Kahren.
i. e. Såsom mycken lättferdigheet står i Kläderna;
Altså dömes offta om eens humeur, effter som han skickar sigh i Kläderna.
[⋯]

#3 Aff lijten Gnista, en stoor Eeld.
i. e. Aff ringa orsak komma offta stoora trättor och Krijgh.
[⋯]
trättor → SAOB TRÄTA, gräl
⇒ Swenske ordsedher (1604) #1:
Aff een lithen gnista kommer en stoor eldh.
Låle (1300-talet) #350: Aff lidhen gnijsth wordher offthe stoor ildh
                                         ex minima magnus scintilla nascitur ignis
Låle (1300-talet) #1051: Offthe ger lidhen gnisth angherligh ildh
                                           Sepe dat attrocem scintilla minuscula torem
YFSv (ca. 1450) #924: opta gør litin gnista anghir fwl eldh
                                      Sepe dat atrocem sintilla minuscula torrem

#4 Aff lijtet Trää, huggs små Spåner.
i. e. Aff lijtet förrådh kan lijtet spenderas.
trää → SAOB TRÄD

                                              — 6 —
#1 Aff Nödh giörs offta en Dygd.
i. e. När en Menniskia råkar i någon stoor Fahra, och blifwer theröfwer försagd,
fattar man ther öfwer en disperat resolution, som wäl lyckas, och blifwer sedan
räknat för Dygd och Mandom.
[⋯]

#2 Aff sidsta Bäkaren kommer första Kinpusten.
i. e. Den som will wara then sijdsta i Laget, han får stundom thet första slaget.
[⋯]
bäkaren ändrat från bäkanen enl. Grubbs rättelser s. 916 (saknar paginering)
bäkaren: bägaren → SAOB BÄGARE
kinpusten: örfilen → SAOB KIND- | -PUST, slag på kinden, örfil
⇒ Swenske ordsedher (1604) #6:
Aff siste begare, kommer först kinpusten.
begare: bägaren → SAOB BÄGARE
först: första → SAOB FÖRSTE
Låle (1300-talet) #451: Aff then sijsthe bæghere kommer gernæ then førsthe pwsth
                                         Haustus a fine surgit origo mine
Låle (1300-talet) #452: Thet sijsthe begheræ faar then førsth kijndhest
                                         Haste flet victu primo ciphus vltimus ictu
YFSv (ca. 1450) #394: æptærsta bikarin giwir vth førsta huggit
                                       haustus a fine surgit origo mine

#3 Aff ruggotta Fohlar blijr och goda Hästar.
i. e. Man får intet skoda Hunden effter Håret.
Ty aff ringa Personer i vngdommen, blifwer medh tijden myndige Män.
Therföre säger man och: Oreena Barn kastar man intet bort.
[⋯]
ruggotta: lurviga → SAOB RUGGOT, raggig; lurvig; som har borstig / tovig hårbeklädnad
⇒ Swenske ordsedher (1604) #10:
Aff ruggosta folarne blifwe beste hestarne.
ruggosta (superl. till ruggot): lurvigaste → SAOB RUGGOT, raggig; lurvig; som har borstig / tovig hårbeklädnad

#4 Aff lijtet Kläde blifwer en trång Brook.
i. e. Ringa förråd gör stumpadt måål.
brook → SAOB BROK, byxa / byxor
måål → SAOB MÅL, ett mål mat, måltid

                                              — 7 —
#1 Aff skadan blijr man wijs, men icke rijk.
i. e. Som man elliest säger:
Brändt Barn flyyr Eelden. ☞ 58 #4
[⋯]
⇒ Swenske ordsedher (1604) #2:
Aff skadan bliffuer man wijs, och icke rijk.

#2 Affsijdes i Roo, är bäst at Boo.
i. e. Den som medh lämpa kan sittia affsijdes ifrån stoort bekymber,
och låta sigh nöya medh en ringa lägenheet, han måår vndertijdhen bäst.
[⋯]
lämpa → SAOB LÄMPA, med lämpa, med lätthet, utan besvär / svårighet
vndertijden (adv.) → SAOB TID, tidvis; stundom; ibland; emellanåt

#3 Affsijdes boor och Folck.
i. e. På ringa Orter finnes offta kloka Hufwud.


\stoptext
