\environment nio-environment

\starttext
\startfrontmatter
    \input content/titlepage
    \input content/tableofcontent
\stopfrontmatter

\startbodymatter

\chapter[middagsgästen]{Middagsgästen}

\dropcapM{argareta blickade uttråkat} ut igenom fönstret. Därute virvlade novemberlöven omkring. Vattenpölarna tycktes mörka och dystra. Inne i lägenheten kändes det också dystert. För några dagar sedan hade barnen hjälpt henne att flytta in. ”Så skönt med en liten lägenhet” hade de sagt. Nu skulle Margareta slippa allt jobb med trädgård och den tunga städningen som den stora direktörsvillan på Gulsparvsvägen hade inneburit. 

Margareta suckade. Tittade sig omkring i sitt nya lilla hem. Så ensamt det var. De finaste möblerna hade hon velat ta med sig. Visst, det blev fint med trångt. Barnen hade våndats över det stora köksbordet som Margareta prompt ville ha in i köket. Resultatet blev att köksbordet fick ställas ända intill väggen, hoptryckt mot den höga rokoko-spegeln som Margareta och hennes make Georg fått i bröllopspresent en gång för 50 år sedan. Man kunde sitta högst en person vid köksbordet, möjligen två om man pressade sig intill köksbänken men det var verkligen inte att rekommendera. Det var en liten 2:a som barnen hittat till henne i det nybyggda seniorboendet för de som var i åldern 60 plus. 

Margareta hade ändå försonats med tanken på en flytt när barnen berättade att hon skulle få så många nya grannar och att barnen skulle få närmare till att besöka henne. Nu visade det sig att de flesta i huset var långt mycket mer äldre än 60 plus. Många hade svårt att ta sig ut och hemtjänsten kom och besökte flera gånger om dagen. Inte var det några friskusar precis men nu skulle ju Margaretas barn kunna komma över lite oftare när de bodde så nära. Det skulle bli så bra. Dagarna gick och barnen lyste med sin frånvaro. Visst ringde de då och då, klagade över att det var så mycket på jobbet och att de knappt hann med att laga middag hemma på vardagarna. De hade inte tid att besöka Margareta när hon frågade.

En onsdagsmorgon i slutet av november blickade Margareta åter ut genom fönstret. Löven dansade i virvlar, kanske något livligare än sist och vattenpölarna blänkte i silver istället för den vanliga, dystra uppsynen. Ikväll skulle hon bjuda hem barnen på kalkonmiddag! Rödkål och äppelmos hade hon hemma. Efter frukosten gick hon sin vanliga promenad men svängde inom saluhallen för att köpa en liten kalkon, brysselkål och sötpotatis. Väl hemma skalade hon potatis, förberedde kalkonen och penslade den med smör och soja. Några skrumpna äpplen fick följa med in i ugnen bredvid den brunblaskiga fågeln. Nu tog det några timmar och sedan skulle det bli fest! Margareta var glad i hågen när hon slog numret till sin dotter Karin. Innan samtalet var slut hade det glada bytts till besvikelse. Karin skulle ut med jobbet ikväll och kunde inte komma. Margareta försökte ringa till sin son Patrik men han var tydligen utomlands på tjänsteresa. Nu var hon ensam med sina middagsplaner. Så trist. Hon torkade bort en tår ur ögonvrån. Nog ville de komma allt barnen men de hade inte tid med henne. De hade  i alla fall låtit lite ledsna på rösten, som om de hade fått dåligt samvete. 

Den kvällen satt Margareta själv vid sitt köksbord och åt en bit kalkon med alla tillbehör. Hon såg en sorgsen bild av sitt ansikte i spegeln. Hade hon verkligen så många rynkor? Aptiten var nu som bortblåst och Margareta dukade av och stoppade in kalkonen och alla tillbehör i kylskåpet igen. Imorgon bitti kunde hon ringa till Karin igen och se om hon ville hjälpa henne att äta upp resterna. Imorgon blir en ny dag tänkte hon innan hon somnade den kvällen.

Morgonen därpå hade ett tunt lager nysnö bäddat in träd och buskar i ett fluffigt, vitt täcke. Margareta var på gott humör, åt sin frukost framför Nyhetsmorgon på tv och såg framemot att äta av de goda kalkonresterna tillsammans med dottern på kvällen. Efter disken satte hon sig i köket och slog numret till Karin. Margareta stålsatte sig, hoppades på det bästa men förväntade sig det värsta  –  att Karin inte kunde komma ikväll heller. Men även om det skulle bli det senare skulle hon inte bli besviken. Hon skulle göra det bästa av det i så fall! 

När Karin återigen förklarat att hon inte kunde komma gav sig Margareta ut på stan. Hon skulle minsann köpa sig något nytt, klä upp sig ikväll och äta en fin kalkonmiddag, njuta av stunden och piffa till sig lite. På kvällen hade hon dukat fint, tänt ljus på köksbordet och omsorgsfullt vikt servetten av linne. I badrummet tog hon fram sin sminkväska som under flera månaders tid legat orörd. Mascaran piggade upp, lite läppstift och rouge, ja nu kände hon sig i alla fall snygg! I köket lade hon upp resterna vackert på tallriken, hällde upp ett glas rött och tittade förstrött ut över det vintriga landskapet. Plötsligt fick hon en underlig känsla av att vara betraktad. I ögonvrån skymtade hon en gestalt, en kvinna i köket?  Kunde det vara Karin i alla fall som kommit? Nej, det var inte Karin.  Kvinnan som smugit sig in obemärkt i Margaretas lägenhet var i samma ålder som hon själv. Kunde det vara en av de osynliga grannarna som tillslut vågat sig på att bli lite social?  Margareta vände sig om för att välkomna gästen. Hon log och gästen log tillbaka, hon sa inte så mycket men verkade glad över att få träffa Margareta. Margareta lade inte så stor vikt vid att hon inte hört att det ringt på dörrklockan och att gästen själv öppnat och gått in i lägenheten. Det gjorde väl inget, nu var hon ju inte ensam! Gästen hade minsann ansträngt sig för att klä sig fint och besöka Margareta. Nu bjöd Margareta henne till bords och slog upp ett till glas vin och serverade en tallrik med fint upplagda kalkonrester. 

Det var en trevlig middag och gästen, som presenterat sig som Margareta, delade inte bara namn med Margareta, hon hade dessutom liknande klädsmak och samma färg på läppstiftet. Middagsgästen var inte stor i maten, lite vin gick väl ner men för det mesta verkade hon bara smutta litegrann. Margareta tänkte att det inte gjorde något, hon var bara så glad att hon hade sällskap till middagen denna kväll. Hon satte på lite kaffe efter maten för det ville även Margareta ha. Efter kaffe och kaka var de båda nöjda och mätta. Margareta dukade av och tog disken. Middagsgästen verkade ha tackat för sig och begett sig hemåt. Hon bodde här i huset hade hon sagt när Margareta frågat. Kalkonresterna hade nästan gått åt och Margareta gladde sig åt sin nya vän. Kanske, kanske skulle hon dyka upp imorgon kväll igen? Margareta hoppades och log. 

Nästa dag begav sig Margareta ut för att handla lite i den lokala matvarubutiken på hörnan. Lite lax och potatis kunde vara gott tänkte hon. Liksom föregående kväll gjorde hon sig fin framför badrumsspegeln. Lite smink piggade upp. Hon gick ut i köket och förberedde middagen och liksom föregående kväll märkte hon att någon betraktade henne. I ögonvrån stod middagsgästen Margareta där igen, glad och lite uppklädd. Margareta tänkte att det var konstigt att hon åter inte hört någon dörrklocka men tänkte att hörseln kanske hade börjat bli sämre på äldre dar. Tillsammans åt de två nyfunna vännerna en fantastisk middag med lax och potatis, pratade om svunna tider och hur trevligt det var med sällskap vid middagsbordet. Det blev en hel del mat över, Margaretas vän var verkligen inte stor i maten. Resterna åt Margareta till lunch framför tv:n dagen efter, då de visade Hem till gården. 

Gästen fortsatte att dyka upp till middagarna varje kväll under december månad förutom den kvällen det var Bingolotto på tv. Margareta satt då framför sin tv med en varm macka och tänkte att hennes nyfunna vän nog gjorde detsamma hemma hos sig. Middagsgästen Margareta hade sagt att hon skulle bjuda hem Margareta till sig någon gång och Margareta hade sett framemot detta. Nu lät inbjudan vänta på sig men Margareta tänkte att gästen kanske inte hade det lika väl ställt som hon själv och Margareta tyckte inte att det gjorde något att de bara var hos henne när det var middag om kvällarna.

Julen närmade sig och barnen Karin och Patrik hörde äntligen av sig och ville komma på lite julmat en kväll. Margareta ursäktade sig då och sa: ”Tyvärr, jag har en ny vän som brukar komma på middag i mitt kök nästan varje kväll och i jul vill jag bjuda henne på något extra gott”. Karin och Peter hade blivit besvikna men inte alltför besvikna. De hade mer undrat över hur hon kunde bjuda hem någon på middag i det lilla, lilla köket som endast rymde det stora köksbordet och en stol. På julaftonsmorgon dalade stora, vita, fluffiga flingor ner från himlen. Margareta pälsade på sig och skyndade till mataffären för att köpa hem godsaker till sitt och Margaretas julbord. Så fint att slippa sitta ensam på julafton, tänkte hon när hon klev in i den julpyntade butiken. Under tiden hon handlade stod lägenheten tom. I det lilla, lilla, julpyntade köket vilade pepparkaksdegen på bänken. Vid väggen intill fönstret stod en ensam stol vid det stora köksbordet som var intryckt mittemot väggen med den stora, vackra rokoko-spegeln. 


\chapter[fiskmannen]{Fiskmannen}            
 
\dropcapK{oltrastarna utanför lägenhetsfönstret} väckte Anita med sin tropiska sång. Hon brukade fantisera om att de lärt sig dessa kvitter utomlands, vid Victoriasjön i Afrika eller i någon djungel. Hon hade alltid varit intresserad av fåglar, djur och natur. Särskilt fascinerande var det att sitta vid havet och se fiskmåsarna dyka och ta upp en och annan firre ur havet. Det hade varit självklart att det var lärare i biologi som hon ville utbilda sig till. På den lilla högstadieskolan trivdes hon som fisken i vattnet.

Anita steg upp och började göra sig iordning framför badrumsspegeln. Ibland kände hon sig som en gammal nucka med sina 55 år och begynnande gråhårighet. Det var längesedan hon hade varit i ett förhållande. Den senaste mannen hon träffat hade förvisso varit underbar men tyvärr inte att lita på. Han hade varit hal som en ål och slingrat sig den gången Anita hade ställt honom mot väggen. Läppstiftet på kragen kunde han inte förklara och Anita hade gråtande kastat ut honom från lägenheten. Det var sista gången hon såg honom. Han hade skickat ett vykort från en semester vid Döda havet och önskat att de skulle försöka igen. Men Anitas känslor var lika döda som havet han semestrade vid. Nästa man som hon skulle våga släppa in i sitt liv måste vara en trygg, stabil karl, som bara hade ögon för henne. Som var en fena på att uppvakta och hålla vad han lovat. Anita rättade till bluskragen, drog en kam genom håret, log och tog ett djupt andetag – vad skulle högstadieynglen ställa till med idag då?

På kafferasten i personalrummet fick hon sällskap av skolvärdinnan Berit. Berit var lite som en extramamma för Anita och Berit var någon som hon gärna anförtrodde sig till. 

”Du måste ut på banan igen Anita, du är för ung och vacker för att leva ensam.” Berit blinkade med ögat. Hon var alltid så käck den där Berit. Tyckte att Anita skulle ut och dansa, i klänning och nätstrumpbyxor, slänga ut sina krokar lite här och var, men Anita var mer av den lågmälda sorten. 

”Följer Du med mig till fiskaffären efter jobbet i alla fall?” frågade Berit. Jovisst, det skulle hon göra. Tydligen hade det börjat ett nytt butiksbiträde där och hans namn var nu på alla kvinnliga lärares läppar. Det nya manliga biträdet hade väckt stor uppmärksamhet på mycket kort tid. Ryktet om hans stiliga utseende spred sig som ringar på vattnet. Han var tydligen finne, Pekka Salmo hette han. Ganska ny var han i det lilla samhället och han hade fått arbete direkt hos den lokala fiskhandlaren. Pekka Salmo var tydligen en fena på att filéa fisk och dessutom var han utbildad kock. Ja ja, hon skulle i alla fall handla lite strömming till sig själv. Så spännande kunde han väl ändå inte vara den där fiskmannen, tänkte Anita.

Aprilsolen värmde faktiskt på eftermiddagen när Anita och Berit steg ut från Sjöängsskolan. Högstadieeleverna hade mer än bråttom att komma därifrån. I biologiläxa hade de fått 50 sidor att läsa, det handlade om försurningen i de svenska sjöarna. Eleverna var inte särskilt engagerade i frågan, de hade mest suttit och tittat ut genom fönstret och längtat ut, liksom Anita, för att lapa eftermiddagens sista solstrålar i frihet. 

Väl framme i fiskaffären ringlade kön sig lång fram till disken. Gamle Lennart stod där såklart, lite mer lättad i sitt sinne nu när han hade Pekka till hjälp. Dessutom kunde Pekka ge damerna,  ja det var mest damer som handlade där, goda råd om hur fisken skulle tillagas, tipsa om tillbehör och vilket vin som passade till. Anita hajade till när hon upptäckte den uppenbart flirtiga stämningen i kön. Damerna tisslade och tasslade. Innan någon kom fram i kön till disken, piffades det med läppstift och borstar. Blusar rättades till eller knäpptes upp en bit och ögon tindrade. Pekka var snygg, det vattenkammade håret låg i en prydlig bena på huvudet. Dessutom var han trevlig och professionell. Han besvarade kundernas frågor mycket artigt och kunnigt men var inte den som lät sig håvas in av smicker. Hans ögon var snälla och han hade ett leende som fick en att smälta. 

”Strömming, ett kilo tack” sa Anita. Innan hon visste ordet av hade han rensat och filéat den glänsande strömmingen,  förpackat den snyggt och korrekt och sträckt fram plastpåsen till Anita med ett charmerande leende. Anitas ben hade känts som gelé. Någonting glänsande i Pekkas blick fastnade i Anitas tankar den dagen. 

Besöken till fiskaffären blev allt fler under de kommande veckorna. Anita och Pekka började bekanta sig med varandra under småprat och leenden. Han såg bra ut Pekka. Lång, mörk och stilig var han. Han fann det otroligt uppfriskande att Anita kunde så många olika fiskarter. Det hade imponerat på honom att hon var biologilärare och att hon dessutom hade en fil kand i limnologi. Pekka började ge Anita lite rabatt då hon handlade och ibland slängde han i en extra bonus i plastpåsen. Det kunde vara några rökta räkor eller en bit gravad lax, för Anita var det en lika romantisk gest som att ge en bukett blommor. Deras småprat hade successivt bytts till djupare samtal om pH-värden och humushalt i svenska insjöar. En gång hade Pekka frågat Anita om hon hade sällskap. Anita hade rodnat och svarat nej såklart. Efter det hade Pekkas ögon fått en ännu djupare glans och Anita hade nästan drunknat i de blåsvarta ögonen. 

Ytterligare några veckor gick och Pekka hade till slut vågat bjuda ut henne på en date. De hade åkt till Akvariet, tittat på fiskar och rockor hela dagen. Anita fascinerades av Pekkas kunskaper och engagemang, intensivt studerade han fiskarna bakom glaset och hade ibland svårt att slita sig från vissa av dem. Mot slutet av deras date hade Pekka tagit hennes hand och kysst henne. ”Vi simmar ju i samma vatten” hade han sagt. Vad han menade med det hade Anita egentligen ingen aning om. Förundrad över hans blöta kyss gick hon som på små rosa moln resten av dagen. Det faktum att den inte alltför angenäma fiskdoften från butiken satt kvar i Pekkas hår och kläder hade hon för längesedan vant sig vid.

Anitas och Pekkas relation fördjupades och de började äta middag tillsammans på kvällarna hemma hos Anita. Det kändes naturligt att prata om fiske, matlagning och klimathotets påverkan på livet i haven. Då och då hände det att Pekka stannade över natten men aldrig så länge att han var kvar när Anita steg upp och skulle äta frukost. Han skulle oftast hjälpa till i fiskbutiken tidigt och ta emot dagens leverans av nyfångad fisk. Förutom relationen med Anita var Pekka mycket engagerad i matlagning och fiske. Han bjöd på hälleflundra med vitt vin. Sjötunga och röding. Stuvade räkor. Anita var mycket imponerad av alla de fantastiska rätter han bjöd på. Hon fäste inte så stor vikt vid att Pekka inte hade bjudit hem henne till sig. Han hade förklarat att hans lägenhet var vattenskadad och att det inte gick att bjuda hem henne av den anledningen. Han var så finkänslig Pekka, tillgiven och omsorgsfull.

Nu hade paret träffats under ett par månader och känslorna hade blivit bestående. Visst kunde Anita ibland störa sig lite på att han lämnade stora svettpölar efter sig i sängen. Lakanet var oftast plaskblött när Anita vaknade och Pekka redan givit sig iväg till jobbet. Hon brydde sig i alla fall väldigt mycket om honom och hade till och med köpt en återfuktande kräm till honom då hon upptäckt att han hade väldigt fjällande, torr hud på överkroppen.  Fisketurerna var en annan sak som Anita hakade upp sig på. Ibland kunde Pekka vara iväg flera dagar i sträck. ”Ute med grabbarna vid sjön” svarade han bara när hon frågat var han hållit hus. När han kommit tillbaka hade han varit pigg som en mört, bubblande glad och pratig och liksom sprattlat av glädje. Så Anita lät honom hållas. 

En kväll i augusti efter en av Pekkas längre fisketurer hade paret ätit en god middag med skaldjur och avnjutit en flaska vin i skenet av stearinljusen. Pekka hade valt att stanna över natten men han skulle upp tidigt nästa dag och öppna fiskaffären, gubben Lennart hade tagit ledigt. Anita skulle också upp tidigt, det var i början av läsåret och hon såg framemot att hålla i terminens biologilektioner på skolan igen. Anita vaknade lite tidigare än vanligt morgonen därpå. De röda siffrorna på väckarklockan visade att klockan bara var fyra. Usch, det var tidigt. Hon kände efter Pekka vid sidan om sig. Där var det tomt och lakanet var åter blött och fuktigt av svett. Han hade redan gått upp, det var väl ändå lite väl tidigt? Anita passade på att gå till badrummet. På golvet låg flera blöta pölar, mådde han inte bra Pekka? Han verkade svettas mer än vanligt. Hon rumlade sömndrucken vidare in till toaletten. Vid sidan om henne var duschdraperiet fördraget vid badkaret. Anita gläntade försiktigt på det. I badkaret, som var fyllt med vatten till mer än hälften, simmade en stor, mörk varelse. Anita hajade till, å fy! Den bjässen hade hon inte lagt märke till igår! Pekka hade inte heller berättat att han behövde låna hennes badkar till sin fångst från fisketuren. Kanske var det en överraskning? Skulle Pekka bjuda henne på middag ikväll igen? En alldeles perfekt kolja? Eller var det en gädda? Anita var först något bekymrad över att hon inte kunde artbestämma den på en gång, varelsen liknade egentligen varken en gädda eller kolja. Hon visste inte exakt men hon kände på sig att Pekka hade planerat något alldeles speciellt för henne. Därav den fina fisken. Skulle han fria? Komma med presenter? Att han var en karl som gärna bjöd på sig själv och var generös det visste hon ju. Så romantiskt! För att visa sin uppskattning bestämde hon sig för att själv förbereda den vackra fisken, precis som Pekka lärt henne. 

Innan Anita gav sig iväg till skolan den morgonen hade hon resolut tagit upp den sprattlande fisken ur badkaret. Hållit den mot skärbrädan i ett fast grepp i köket och med ett enda hugg med kökskniven dekapiterat den mörka, slemmiga varelsen och sedan lagt den i kylskåpet. Huvudet med de blåsvarta ögonen slängde hon i sopnedkastet på vägen ut. Jodå, nog skulle Pekka bli imponerad av att hon kunde förbereda hans fina fångst och ta hand om den så väl. 

Berit, skolvärdinnan var förvisso glad över att Anita äntligen fått en karl på kroken men en kolja i badkaret? Det var väl lite udda? Inte särskilt romantiskt tyckte hon när Anita berättade om att Pekka nog skulle ställa till med överraskningsmiddag. Berit gladde sig ändå för Anitas skull, för att hon med sina 55 år på nacken och begynnande gråhårighet lyckats håva in en sådan stilig karl som Pekka. Då kunde man nog stå ut med en del egenheter tänkte hon. 

Hela arbetsdagen fantiserade Anita om vad det var för överraskning som Pekka skulle komma med. Hon var glad och upprymd, tankarna och känslorna sprattlade som småspigg inombords. En fin middag på kolja eller vad det nu var, lite vin och tända ljus och Pekka. Pekka med en bukett blommor och kanske, kanske en liten smyckesask?

Väl hemma igen fortsatte Anita att fjälla fisken, filéa den och sedan lade hon den i kylen i väntan på att Pekka skulle komma hem från fiskaffären. Klockan blev 18. Han var lite sen Pekka. Anita hällde upp ett glas vin. Klockan 19 var han fortfarande inte hemma. Så dumt att hon inte kunde ringa honom! I morgonbrådskan hade hon sett att han lämnat sin mobil och sina lägenhetsnycklar hemma. 19.15 tog Anita saken i egna händer, nu skulle det bli fina fisken! Hon tog varsamt fram fisken ur kylen, lite smör i stekpannan och salt och peppar. Den ljuvliga doften av smält smör och nyfångad fisk spred sig i lägenheten. Fisken blev kärleksfullt tillagad. Hon kokade potatis och gjorde en beurre blanc såsom Pekka lärt henne i början när de hade träffats. Nu fick han komma närsomhelst. Hon var redo. Han skulle bli så glad och imponerad av hennes matlagning. Han hade lärt henne allt han visste om fisk, matat henne med kunskap och goda recept. 19.30. Ingen Pekka. 20.00 och två glas vin senare satt Anita fortfarande och tittade förväntansfullt ut genom fönstret. På andra sidan av det lilla samhället stod fiskaffären mörk och tom. Den hade inte varit öppen på hela dagen.


\chapter[ennydag]{En ny dag}

\dropcapD{et var den} 7:e december. Snöflingorna hade sakta börjat dala ner på det lilla sömniga postkontoret i byn. Posttjänstemannen Dag Börjesson tittade förstrött på högen med rek-brev. Hur var det nu igen, vilken säck skulle de ligga i? Varför var det alltid så krångligt med specialposten? Han suckade, tittade över axeln för att försäkra sig om att inte chefen Brandin såg på och tippade därefter lite nonchalant ner rek-högen i den röda julpostsäcken. Så där ja, nu var det löst och Dag kunde ta sig en ny kopp kaffe i personalrummet. Det hade han gjort sig förtjänt av. Logistik och köteori var inte hans favoritämnen. Nej, roligare var det med kvantfysik och våglära. Trots Dags kunskaper i fysik och studier på universitetet hade han endast fått tjänst som posttjänsteman i en liten, sömnig håla där tiden tycktes stå stilla, men det bekom honom inte särskilt mycket. 

Postmästare Gösta Brandin satt vid sitt prydliga skrivbord på kontoret. Han suckade och kikade ut över sitt rike. Den här morgonen, liksom alla andra, hade den anställde Dag Börjesson, återigen varit mycket sen till jobbet. Han var dessutom arrogant, småfet och inte direkt någon glädjespridare på posten. De andra medarbetarna hade ofta klagat på att han dessutom inte verkade sköta sin hygien, var ohyfsad och otrevlig mot kunderna och inte var särskilt noggrann med brevsorteringen. Arbetsskygg hade de sagt. Gösta Brandin som var en mycket godtrogen man, brottades med det faktum att Dag Börjesson nog inte passade in i hans arbetslag och han våndades över att han en dag skulle behöva säga upp honom. Skulle idag vara den dagen? Nervositeten började kännas i hårfästet. Gösta slog bort tanken igen, han skulle fundera lite till. Det var ju ändå han som bestämde över det här postkontoret.

Kerstin och Karin stod vid sorteringsbandet. Lite längre ner stod Dag. Kerstin och Karin tittade menande på varandra och himlade med ögonen. Hur kunde han ha fått anställning här? Dag var en medelålders man, ogift vad de visste. Han hade en utbildning i logistik och hade dessutom studerat teknisk fysik på universitetet. Trots hans fina referenser och examen var det här han hade hamnat. Han verkade mycket ointresserad av arbetet på det lilla postkontoret. Ofta satt han i personalrummet med en stor kopp kaffe, surfandes på sin mobiltelefon, lödandes på något slags kretskort eller bläddrandes i en Allers. Otaliga var de gånger då kunder efter en lång tids väntan, irriterat knackat på luckan då Dag låtsats att inte varken höra eller se att kön ringlade sig lång bredvid postboxarna. Saktmodigt hade han då öppnat luckan och motvilligt hjälpt kunderna med deras ärenden, givit dem ett påklistrat leende och sedan bestämt stängt luckan igen. De enda gångerna Kerstin och Karin hade uppmärksammat någon slags iver hos Dag hade varit just innan stängningsdags eller vid den tiden på månaden då lönen utbetalades.

Dag märkte att medarbetarna vid sorteringsbandet himlade med ögonen och kastade menande blickar mot varandra. Hur det än var så var Dag mycket oberörd över detta. Nu hade han dessutom råkat spilla ut en hel kopp kaffe över sorteringsbandet så att hela sorteringen blev pausad. Kerstin blev rasande och höll på att gå upp i limningen.

”Nu får det väl ändå räcka!”, utropade Kerstin och konfronterade den lågpresterande Dag. ”Nu går jag och Karin till chefen, så här kan Du inte fortsätta!” Dag såg förundrat på Kerstin, hur kunde någon bli så arg för några droppar kaffe? Dessutom fick hon ju också en välbehövlig paus i sitt arbete. Det där med tid verkade de inte förstå, hans arbetskamrater. Hade de vidareutbildat sig som han själv hade de förstått. Tid. Det fanns alltid tid tyckte Dag. Dessutom var tiden väldigt konstant och rättvis. Alla människor äger exakt 24 timmar var, varje dygn. Det var rättvist. Tid fanns det för alla och allt hade sin tid enligt Dag.

Gösta Brandin ryckte till ur sin tupplur när dörren till hans kontor plötsligt rycktes upp. In stormade Kerstin med kollegan Karin i släptåg. I sitt yrvakna tillstånd kunde Brandin se att de två kvinnorna var mycket uppretade, minst sagt. De två postmedarbetarna ställde sig med armarna i kors framför hans skrivbord. 

”Nu har det gått för långt. Vi vill att Du avskedar Dag idag!”, krävde de båda kvinnorna. ”Brandin måste inse fakta, vi har inte tid med Dag. Nu har sorteringsarbetet stannat av helt. En hel kopp kaffe välte han ut över posten och han bad inte ens om ursäkt!”. Det var uppenbart att dessa kvinnor hade fått nog, deras tålamod var slut. Brandin förstod nu att tiden var inne, måttet var rågat och det var dags att ta ett beslut, om än det var i elfte timmen. Ville han att Karin och Kerstin skulle stanna kvar i hans lilla post-kungarike, ville han behålla två kompetenta personer då var han tvungen att göra sig av med Dag. Dag som redan från Dag 1 hade visat sig något ointresserad och haft svårt med det sociala. Dag som alltid kom försent till sitt arbete och alltid lyckades ha en ursäkt för att lämna arbetet före stängning. Sävlig var han, och långsam. Utom när han plockade med sin mobil och sina kretskort som alltid stack ut ur tjänstejackans fickor. Tid verkade han ha gott om, han hade en lustig förmåga att aldrig verka stressad eller bekymrad över någonting. Inte ens när han fick en varning, inte ens när arbetskamraterna hoppade på honom eller sa åt honom att hjälpa till. Nu var tiden inne. Det skulle bli en ny dag i Brandins kungadöme, en ny dag för Posten i det lilla sömniga samhället. Det var dags att vakna upp! Till en ny dag.

Kerstin och Karin klev chockade ut ifrån postmästare Brandins kontor. Var det verkligen sant? Hade Brandin nu äntligen lyssnat på dem och valt att agera? Jo, visst var det så. Han hade förstått att Posten inte hade tid för sådana medarbetare som Dag Börjesson. Dag skulle bytas ut. Han skulle avskedas. Det skulle bli en ny Dag. Kerstin och Karin såg på varandra i samförstånd och log. Ikväll skulle det bli tårta och bubbel hemma. Timmen var slagen för Dag och det var inte en sekund för tidigt. Han skulle ut!   

Dagen på det lilla postkontoret, där tiden verkade stå stilla, började närma sig sitt slut. Skymningen föll och tillsammans med den också vackra, vita snöflingor. Sekundvisaren på klockan i personalrummet verkade gå långsammare och långsammare. Dag tittade på sin mobil. Nej, den stämde. Han lutade sig avslappnad tillbaka och såg tillbaka på sin dag. Den hade varit lugn som alla andra dagar brukade vara. Han hade in i det längsta dragit sig för att bära ut julpostsäcken till lastbilen som stått och väntat. Chauffören hade irriterat påpekat något om att ”här verkade allting gå allt långsammare”, men som vanligt hade Dag bara sett mycket oberörd ut. Nu såg han Kerstin och Karin göra det sista av dagens plockarbete med julkort och annat. Han såg också att postmästare Gösta Brandin satt på sitt kontor och verkade förbereda sig för något stort, det såg ut som om han övade på ett tal. Brandin vankade av och an i sitt lilla kontor, höll upp pekfingret förmanande och stannade till vid en spegel och såg mycket allvarlig ut. Var det så att han tänkte avskeda Dag idag? Jo, tänkte Dag. Nu var det nog dags att gå hem. Innan det var försent. 

Klockan 15.50 gick Dag ut till omklädningsrummet för att ta på sig sina vanliga kläder. Nu fick det vara nog med arbete för en dag. I samma stund tänkte även Gösta Brandin att det fick vara nog med arbete för en Dag. Nu skulle det ske, Dag skulle sluta. Bli uppsagd. Med nervösa harklingar och på darrande ben klev postmästare Gösta Brandin ut från sitt kontor och stegade fram mot personalentrén. Klockan 15.55 gick Dag Börjesson ut från omklädningsrummet och fram till postboxarna som fanns utanför glasluckan där kunderna stått i kö. Nu var det helt tomt,  5 minuter före stängningsdags. I ögonvrån såg Dag hur chefen Brandin vankade av och an i korridoren framför vägguret. Dag tog fram en nyckel ur fickan, öppnade postbox nr 911 och tog fram en stor, tung fjärrkontroll. 15.58 kopplade Dag ihop fjärrkontrollen med ett kretskort som stack upp ur byxfickan. En grön diod tändes, kvartsoscillatorn fungerade idag igen! Dag ställde in frekvensen, kontrollerade transformatorn, anslöt den till ett relä och väntade med ett brett leende. 

Klockan 15.59 hade Gösta Brandin Dag inom synhåll, det var i grevens tid för nu skulle det ske, men vad i all världen var det som Dag hade plockat ut ur postboxen? Något kändes annorlunda, Gösta Brandin gick som i sirap och han tittade förfärat på vägguret i entrén. Höll han på att bli galen? Visarna på klockan hade gått i spinn och rörde sig motsols.  Klockan blev 16.00. Tystnad och sedan ett poff! Det blev svart. Tomt.

Klockan 07.00 den 7:e december dalade snöflingorna åter sakta ner över det lilla sömniga postkontoret i byn där tiden tycktes stå stilla. Hemma i sin säng låg Dag och slumrade till ljudet av väckarklockans radio. Idag var en ny dag men samma dag som igår. En ny dag för Dag.     


\chapter[fastfolket]{Fästfolket}

\dropcapS{tina och Orvar} Gradvall hade under hösten som gått firat 5-årig bröllopsdag. Paret Gradvall hade som många andra par träffats då de båda gått ur realskolan och varit på examensbalen tillsammans med goda vänner. Båda två hade med ens insett att de hade mycket gemensamt, däribland ett särskilt drag av pedanteri och noggrannhet. Stina hade lagt märke till de perfekta pressvecken på Orvars finbyxor och han hade märkt att Stina hade otroligt vita fina kläder med en ren frisk doft som av nytvättade lakan som fått torka ute i friska luften. Det hade med ens blivit kärlek och de två blev fästfolk och förlovade sig snart. Efter avlagd examen hade Orvar fått en anställning på den lokala limfabriken.  Paret hade då flyttat till en liten hyreslägenhet i Limhamn och Stina tog hand om hushållet. 

I bröllopspresent hade paret önskat sig en toppmatad tvättmaskin, ett torkskåp och en mangel, vilket i och för sig var en mycket ambitiös önskan, men Orvars arbetsgivare hade varit mycket generös med uppvaktningen. På lysningen hade det pedantiska paret fått allt det som de önskat sig och lite till. Tiden efter bröllopet hade kantats med diverse utflykter och tillställningar. Smekmånaden var inte alltför vidlyftig eller spännande,  men det unga paret fäste sig vid tanken på att hålla ihop och skapa en vardag tillsammans. Orvar återgick till sitt arbete efter smekmånaden och vardagens verklighet började infinna sig för paret. Stina och Orvar levde ett vanligt Svensson-liv men fann ändå en stor glädje i att de klarade sig ekonomiskt och att de hade det så fint och rent i sin lägenhet. På helgerna brukade Stina duka med den vita linneduken och det bästa porslinet. Allt var så rent och fint som det kunde vara. Hon fullkomligt njöt av allt det rena och vita. 

En sensommardag kom Orvar hem lite senare än vanligt. Han hade blivit uppehållen på jobbet berättade han. Fastnat i en diskussion med chefen på limfabriken och det hade gällt ett större uppdrag som Orvar eventuellt var lämplig för. Stina tyckte sig märka ett påklistrat leende hos Orvar när han berättade, men hon fäste sig inte så mycket vid det. Hon tänkte att detta kanske skulle kunna vara hans stora chans att stiga i graderna, att bli befordrad och få en högre lön. Orvar förklarade vidare att det hela rörde sig om ett nytt slags klister som fabriken skulle börja producera. En helt ny, revolutionerande produkt som kunde hålla samman allt. Precis allt, och att det var nödvändigt i dessa tider av förfall och osämja i världen. Stina lyssnade inte så noga. Hon fantiserade mest om vilka nya möjligheter det kunde innebära, ett hus kanske, med en riktig tvättstuga! 

Det nya projektet sattes igång på fabriken. Orvars chef var mycket nöjd med hans insats. Orvar var ordentligt engagerad och hade satt ihop ett mycket sammansvetsat team för uppgiften. Klistret som hade börjat tillverkas motsvarade verkligen många förväntningar. Vidhäftningsförmågan var fantastisk. Orvar ägnade många timmar på arbetet och kom hem sent om kvällarna. Visst var Stina glad över hans framsteg, karriären hade tagit ny fart och det syntes också i lönekuvertet. Tyvärr syntes det också på Orvars skjorta och byxor. Stina, som var mycket duktig på fläckborttagning, hade svårt att få bort de sega, brunsvarta limfläckarna på Orvars kläder. Oftast var det bara en liten fläck men den satt som berg i tyget. Stina lade ner mycket tid på att få hans kläder rena. Hon tyckte att kombinationen lacknafta och grönsåpa var den som fungerade bäst hittills. 

Månaderna gick och Orvar gjorde nya framsteg på limfabriken. Nu hade formulan till det nya klistret förstärkts ytterligare och försäljningen hade fullkomligt exploderat. Det var som om alla marknader och länder var intresserade av detta klister. Det användes flitigt överallt, till trä, glas och porslin. Även byggmarknaden var intresserad. Förfrågningar kom från olika håll, från politiska partier och regeringar världen över. Stina var fascinerad över efterfrågan. Så fantastiskt kunde det väl inte vara? Fläckarna var ju nu i stort sett omöjliga att tvätta bort. Med stor sorg hade Stina också märkt att Orvar hade börjat förändras. De som hade varit så sammanfogade. Från att ha delat hennes intresse för städning och pedanteri, var han nu hopplöst oberörd av fläckarna på sina kläder. Dessutom hade han börjat få limfläckar på huden. På de tidigare starka, bleka, svagt hårbeprydda underarmarna, satt nu små hårda, brunsvarta fläckar av klister. Det störde Stina. Duschade han ens längre? Brydde han sig om hur han såg ut? Eller ens hur hon såg ut längre? Den fräscha, rena doften av en smekmånad för 5 år sedan hade nu sedan en tid ersatts av en frän lukt som klistrade sig fast i hennes näsborrar. Stina kunde inte göra annat än att tvätta, gno och använda starkare fläckborttagningsmedel på Orvars kläder. Ibland kändes det som om hon höll på att gå upp i limningen. Det som tidigare varit optiskt vitt hade nu antagit en grågul dassig kulör som var allt annat än inspirerande. En grå vardag blev tillslut den dominerande stämningen därhemma. Till och med samtalen med Orvar om kvällarna framför tv:n hade blivit grå och trista. De hade mest handlat om olika tillverkningsprocesser och kemiska formler för det fantastiska men också fruktansvärt effektiva klistret. Stina kunde inte koncentrera sig på allt han sa. Hennes tankar höll inte ihop längre.

Det nya klistret var verkligen revolutionerande. Till och med på nyheterna talades det högljutt om klistret som kunde sätta ihop en bil, laga en raket och till och med sår! Det senare hade Stina svårt att tänka sig. Kleta det brunsvarta, fräna limmet på ett sår, det kunde väl inte bli så bra?  Det fanns även en lite skum subkultur som näst intill verkade dyrka klistrets egenskaper och det gick ett lite udda rykte att limmet också kanske kunde reparera sprickor i politiska partier och mellan länder. Tydligen hade relationen mellan USA och Sovjet förbättrats avsevärt sedan de båda stormakterna börjat köpa in stora mängder lim till sitt försvar. Trams! tänkte Stina. Skulle det kanske också då kunna limma ihop brustna hjärtan, relationer eller ett trasigt äktenskap? Pyttsan!

Orvar gick mer och mer upp i sitt arbete. Stina gick alltmer upp i tvätt och städning. Fläckar av klister fanns nu också på möblerna där Orvar suttit. Den benvita soffan i skinn som de hade köpt som nyförlovade, var nu ett minne blott och liknade alltmer en stor och svullen dalmatiner. Hon suckade och bet ihop. Den senaste tiden hade Orvar varit så ointresserad av henne, av deras hem och framtid ihop. Hon funderade allvarligt på att lämna honom. Tankarna fladdrade långsamt bort igen, som fjärilar till något bättre fläckborttagningsmedel? Till en vit vardag med en doft av rena fräscha lakan. Det här måste ju gå att fixa! tänkte Stina. Hon blev nästan lite förvånad över sin egen nyfunna optimism. 

Dagarna gick och en dag kom Orvar hem med en överraskning: en ny tvättmaskin, denna gång en avancerad industri-tvättmaskin av modellen CX911. Stina tindrade med ögonen, nu kanske det skulle gå lättare att få bort de envisa limfläckarna. Han var nog inte så dum ändå den där Orvar. På sina starka, brunsvartprickiga armar bar han in den stora, glänsande tvättmaskinen i badrummet. Stina tackade med en kram om hans midja, försökte att tänka bort alla brunsvarta limprickar på den en gång så vita skjortan. Deras ögon möttes och något fastklistrad i hans armar förstod hon att klistret faktiskt var så bra som alla hade sagt. Det skulle aldrig skilja dem åt och det hade kanske också lagat deras äktenskap. Kanske rentav hela världen skulle bli en bättre plats tack vare ett klister som tycktes kunna foga samman allt. Allt verkade vara möjligt!


\chapter[duschen]{Duschen}

\dropcapJ{ag skriver det} här som en viktig upplysning, ja, kanske rentav som en varning för att inte andra ska råka drabbas av det som nu verkar ha hänt mig. Att för mycket träning skulle kunna vara skadligt är kanske inte helt orimligt, det var i alla fall så det började för mig. När jag var 40 år fyllda började flera i min bekantskapskrets noja sig för sin ålder och sitt utseende. De klagade på ölmage och uttryckte rädsla för att bli flintskalliga. Det var särskilt Jonte och Mackan som höll på. De fick för sig att göra ”En svensk klassiker” något som kändes för mycket 40-årskris enligt mig för att haka på. Urlakade och slitna men lyckliga, kom de tillbaka efter sommaren, stolta som tuppar över att ha genomfört en bedrift som snarast kan liknas vid denna tidens mandomsprov. Strax därefter skaffade de sig likadana tatueringar, det var en dåligt gestaltad orange tiger i ett blåaktigt gräs. Det såg förfärligt ut men Mackan och Jonte visade stolt upp sina överarmar på badstranden följande sommar. Nu hade de börjat träna på gym och deras sex-pack var inte av samma sort som jag själv stjälpte i mig under fredagkvällen precis. En eftermiddag på stranden hetsade de mig och jag gick motvilligt med på att börja cykelträna med grabbarna. Det var så det började.  

Jag kan inte påstå att jag gillade träningen till en början. Bara utrustningen och kläderna fick mig att känna mig löjlig. Cykelbyxorna satt hårt åt och stoppningen inuti fick mig att associera till en blöja. Kunde man verkligen känna sig, eller se manlig ut i sådana? Cykelsadeln var smal som ett korvbröd så jag förstod snabbt stoppningens funktion i cykelbyxan. Efter cykelpasset var belöningen att hoppa in i duschen, skölja bort smuts, lera och svett i det varma vattnet. Jag var glad och tacksam att jag hade ett fint, nyrenoverat badrum i min lilla 2:a på Kemigatan 11. Granne med mig på nr 9, låg en kemtvätt. Det var ofta där jag lämnade in mina smutsiga träningskläder. Jag hade alltid tyckt att det var hopplöst att få syntetkläder rena och väldoftande i min egen tvättmaskin. Som singel och ungkarl hade jag dessutom blivit väldigt bekväm av mig. Lillemor Larsson, som ägde kemtvätten, var alltid så snäll och hjälpsam. Hon var en liten, kort kvinna men med ett stort hjärta. Jag behövde sällan betala mer än en 50-lapp för min inlämnade säck med smutstvätt. Det var det väl värt tyckte jag. 

Veckorna gick och jag kom faktiskt in i en rutin med cyklandet. Det kändes helt ok men den stora belöningen kom alltid efteråt. Jag fullkomligt älskade att stå länge i min varma dusch och känna det varma vattnet strömma över hela kroppen. Successivt vande jag mig vid lite varmare vatten. Jag kunde stå gott och väl i en timme med vattnet påslaget. Jag kände mig så otroligt ren och fräsch efteråt. Det var roligt att vara ute och cykla med Mackan och Jonte men jag började göra fler och fler cykelturer på egen hand också. En gång när vi tre grabbar skulle ut tillsammans på en längre tur märkte jag att min sadel hade ändrats i höjd. Den var nu en aning för hög för mig och jag fick nästan anstränga mig för att kunna ha hela foten på pedalen. Jag tänkte att Mackan och Jonte hade skojat med mig och därför ställde jag bara in sadeln på nytt efter en höjd som jag tyckte passade. De skulle alltid försöka driva med mig på något sätt sedan jag hade hakat på dem i deras träningsiver.  

Jag cyklade varje dag och när Mackan och Jonte valde att åka iväg och cykla Vätternrundan hade jag mitt eget cykellopp runt Lillasjön som bara ligger ett stenkast från lägenheten där jag bor. Nu var även cykelstyret för högt, det konstiga var att Mackan och Jonte inte varit hemma på ett tag så jag förstod inte riktigt vad som hänt, eller vem som hade fingrat på min fina cykel. Efter den rundan belönade jag mig själv genom att duscha riktigt länge och jag duschade riktigt varmt. Hett. Jag var väldigt trött och nickade nästan till när jag stod där i duschen. Så här i efterhand var det kanske lite dumt att stå där så länge under det skållheta vattnet. När jag såg mig i badrumsspegeln efteråt var huden liksom rodnad och skrynklig över mina axlar och till och med lite i ansiktet. Efter det mådde jag inte så bra, men efter en god natts sömn var jag pigg igen. Jag upptäckte till min stora glädje att den intensiva träningen hade gjort nytta och att jag började gå ner i vikt. Cykelbyxorna kändes för stora, de hängde och slängde runt midjan och låren. Vågen visade på flera kilo mindre än den brukade. Glad i hågen unnade jag mig en shoppingrunda i sportaffären och köpte nya kläder. Det visade sig att jag även behövde en storlek mindre i t-shirts. Jag började komma i finfin form!

Träningen fortsatte. De långa och heta duscharna likaså. Till min stora förvåning upptäckte jag att jag lyckats tänja ut mina cykelskor, det måste varit dålig kvalité. Det var dock inte bara dessa skor som verkade för stora. Mina vanliga morgonskor kändes som skor för en jätte och mina sneakers, som jag haft i ett antal år, var som ett badkar på foten. Tankarna började snurra och jag fick inte riktigt ihop det. Jag var så van vid att lämna in mina kläder på kemtvätten och jag hade aldrig haft något att klaga på inför Lillemor,men nu… Nu verkade det som om kemtvätten hade misslyckats stort och töjt ut alla mina kläder. Snart passade ingenting och jag såg ut som en trashank i alldeles för stora kläder. Det störde mig en del. Något som också hade börjat störa mig var cykeln. Det gick inte längre att sänka sadeln eller styret mer och jag hade stora problem med att komma upp på den. Höll jag på att bli sjuk?  Få problem med rörelseapparaten? Det var också svårare för mig att resa mig från sängen och fåtöljen. 

När cyklingen inte längre verkade fungera för mig, jag nådde inte längre ner till pedalerna, började jag istället att jogga. Det blev lite konstigt att köpa löparskor på barnavdelningen i sportaffären. Joggingbyxor fick jag också köpa där. Jag hade blivit så liten och tunn. Mackan och Jonte hade tidigare sagt något om att:  ”Nu får Du väl inte bli mindre? Träna inte så mycket”. Men jag kunde inte låta bli. Det enda som fick mig på andra tankar var träningen. Träningen, men framför allt den varma, sköna duschen efteråt. Jag kunde väl ändå inte vara sjuk? Var det därför jag gått ner så mycket i vikt och hade svårt att klara trapporna upp till lägenheten? Nej, jag kände mig ju ändå pigg, även om det var knepigt ibland. Jag började få svårt att nå upp till köksskåpen, fick börja använda pallar och böcker att stå på. Sedan gick allting väldigt, väldigt fort. 

Jag började förstå att det var något riktigt konstigt som hände med mig, med min kropp. Allt runtomkring mig hade blivit utom räckhåll. Dörrhandtaget till ytterdörren. Handtaget till kylskåpet. Kranen vid vasken. Jag kunde inte ens nå upp att se mig själv i spegeln längre. Inga kläder passade. En morgon när jag vaknade trodde jag att allt bara var en dröm. Jag låg med fötterna på huvudkudden, men det var inte bara fötterna som låg på den. Hela jag fick plats på min kudde och jag kämpade med hela min lilla kropp för att hala mig ner. Via nattduksbordet och en lampsladd firade jag mig ner på golvet. Nu var jag liten som en mus. Det fanns inte en chans att jag skulle kunna komma ut ur lägenheten i detta skick. Helt naken var jag dessutom. Mina kläder i barnstorlek låg där i sängen. Storlek 110 centilong. Jag grät i stor förtvivlan och jag skrek! Ur min lilla hals kom bara ett litet pip. 

Jag var inte galen och det var ingen dröm. Tv:n stod på i vardagsrummet. Den lokala nyhetsuppläsaren informerade om att Larssons kemtvätt på Kemigatan 9 hade fått en anmärkning. Förbjudna kemikalier hade använts och kemtvätten hade fått böter och förbud att fortsätta driva verksamheten vidare. Kunderna skulle få ersättning för krympta kläder och hela fastigheten skulle undersökas noggrant, tydligen var det något med varmvattenledningarna. Jag blev iskall när jag hörde detta. Var det sant? Kunde mina varma duschar ha krympt mig? Kombinationen med nytvättade kläder från Lillemor Larssons  kemtvätt och hett vatten – tanken gjorde mig illamående. Min laptop stod uppslagen på soffbordet. Jag klättrade upp på soffan på en gammal kofta som jag slängt över armstödet. Vad skulle det bli av mig? Kunde jag kanske kontakta någon via mailen? Be om hjälp? Fanns det ens hjälp? 

Så det är därför jag skriver detta nu, för att varna andra och för att kanske få hjälp att rädda mig själv. Det fortsätter att hända saker i min kropp. Nyss kunde jag trycka ner varje tangent med mina händer men nu måste jag hoppa med hela min lilla tyngd på varje tangent för att kunna göra ett avtryck på skärmen. Min stora skräck är att jag blir så liten att jag kommer att fastna mellan tangenterna och inte kan ta mig losssss  sssss\ \  sssssssssssssss\ \ ssssssssssssssssssssssssssssssssssss 
\ s\ \ \ \ s\ \ \ \ \ s


\chapter[trollbunden]{Trollbunden}

\dropcapD{et hände mig} något när jag var tonåring. Det var i början på höstterminen 1971. Jag  var som trollbunden redan från början. Den nya tjejen i klassen var helt magisk i mina ögon. Mina kamrater kunde inte riktigt förstå vad jag såg hos den bleka, rödhåriga tjejen men något var det i alla fall. Ok, hon var kanske inte vacker direkt, snarare mer enkel, alldaglig med sina fräknar och mellanrummet mellan framtänderna. Mina kamrater kunde i alla fall erkänna att hon hade en mycket fin figur. Jag hade alltid varit förtjust i rödhåriga tjejer, ända sedan lekis. Nu stod det en sådan rakt framför mig. Pigg och glad var hon dessutom, med ett underligt leende och en glans i de gröna ögonen som gjorde mig väldigt nyfiken. Det dröjde inte många dagar förrän Hulda kom fram till mig och frågade var kemisalen låg någonstans.  Som elever på Naturvetenskapsprogrammet fann vi att vi båda var intresserade av kemi, biologi och naturen. Vi hade samma humor och kunde plugga långa stunder tillsammans i skolbiblioteket. 

Hulda var ett naturbarn, det var solklart. Hon kände sig mest obekväm när vi gick på stan eller där det var stora folksamlingar. Hon var enkel och jordnära. Föräldrarna verkade livnära sig på skogsbruk och Hulda kunde allt om skogen. Hon var helt lysande på biologin, kunde allt om fanerogamer, gömfröiga växter och mykologi. Hon älskade fältstudierna och artbestämningen. Jag var väldigt imponerad över alla hennes kunskaper. Det var också väldigt enkelt att umgås med henne. Jag kände att vi hörde ihop på något sätt. Efter några veckor bjöd jag ut henne på bio och man kan väl säga att vi blev ett par efter det. Hon var ofta hemma hos mig, sa att det var för stökigt hemma hos henne själv. Jag hade inte träffat hennes föräldrar och det blev aldrig så att jag följde henne hem efter skolan. Hon följde med mig istället. 

Vi brukade hålla varandra i handen på rasterna och någon gång hade jag kysst henne på munnen. Mest av allt älskade jag att drunkna i hennes mossgröna ögon. Det fanns något spännande där men varje gång jag bara ville titta djupare in i Huldas ögon, vände hon fnittrandes bort huvudet. De röda lockarna dansade som ett draperi framför mig. Föreställningen var över och det brukade alltid dröja några dagar innan jag fick samma möjlighet igen att blicka in i de gröna, förtrollande ögonen. Hemlighetsfull var hon allt litegrann. Mystisk och finurlig. Jag förstod att hon hemmavid tillbringade mycket tid i skogen. Hon var oftast klädd i en rutig flanellskjorta, jeans och grova kängor. I hennes hårlockar eller på skjortkragen kunde det ibland dröja sig kvar små granbarr eller torra löv. Jag fann det hela mycket charmigt. Jag ville bara ägna tid åt Hulda och började försumma mina killkompisar alltmer. Hulda och jag lånade böcker på biblioteket och satt hemma i mitt pojkrum och läste. Hon älskade att läsa om örter och medicinalväxter men hon hade också en fascination för thrillers och kriminalromaner. Hon berättade att där hon hade bott förut, långt uppe i de jämtländska skogarna, hade det varit en manlig lärare som plötsligt försvunnit helt oförklarligt. En skolklass hade varit på utflykt och orienterat, gått tipsrunda med korvgrillning och när bussen skulle köra tillbaka till skolan saknades klassföreståndaren. Man hade letat och ropat utan resultat. Hemvärnet hade ordnat en skallgångskedja och kyrkan var engagerad och vädjade om hjälp från allmänheten. Polisen  hade varit mycket förbryllad. Läraren verkade helt ha försvunnit från jordens yta, spårlöst! Han återfanns aldrig. Jag blev alldeles tagen av Huldas berättelse men också förundrad över den särskilda glans och iver i hennes ögon när hon berättade historien om den försvunna läraren. Kanske var det en utbildning i kriminologi som Hulda skulle påbörja efter gymnasiet? Eller kemist, hon var verkligen skärpt på det området och framför allt intresserad.

Veckorna gick och jag blev mer och mer attraherad av Hulda, min flickvän. Som mina killkompisar tidigare nämnt så hade hon en fantastisk figur. Midjan var smal, bysten fyllig. De tajta jeansen fyllde hon ut perfekt. Sedan var det något med hennes rörelsemönster också. Hon rörde sig mycket graciöst, nästan sensuellt. Jag ville bara se på henne, röra vid henne. Det senare visade sig bli alltmer något av en utmaning. Även om vi hade kyssts så hade vi inte hånglat på riktigt. Jag kände att det nog snart var dags att ta förhållandet till nästa nivå. Vid något tillfälle hemma hos mig försökte jag närma mig henne i tv-soffan. Lade handen på hennes lår och tryckte mig försiktigt mot henne. Hon såg glad ut och vi kysstes. Jag ville knäppa upp en knapp i hennes flanellskjorta men då hoppade hon raskt åt sidan med ett fnitter. Hon såg på mig igen och jag började långsamt sjunka in med min blick i de där vansinnigt vackra ögonen. Det var som att vandra i mjuk, grön mossa och jag nästan drunknade innan hon skrattande vände bort sitt huvud. Ridån av röda, vilda lockar föll ner framför hennes ansikte. Något besviken satt jag kvar och kände mig ratad. Hulda märkte detta men försäkrade mig om att vi nog skulle kunna ta vårt förhållande till nästa nivå. Snart. Det fick mig att känna hopp och ännu mer attraktion till Hulda. 

Några dagar senare satt vi åter i skolans bibliotek. Hulda såg uttråkad ut. Jag föreslog att vi skulle gå på bio eller åka in till stan ock käka hamburgare. Hulda föreslog istället att vi kunde gå hem till henne. Det var nog på tiden att jag fick se var hon bodde. Hennes familj hade skogsbruk och jag gladde mig åt att äntligen få träffa hennes föräldrar. En eftermiddag i mitten av oktober tog vi bussen efter skolan, hoppade av i utkanten av vårt lilla samhälle och började gå mot Kullatorpet. Där i närheten bodde Hulda och hennes familj. Vi vek snart av från den stora vägen och vandrade in på en mindre skogsväg. Hulda höll mig hårt i handen och var på extra gott humör denna dag. Vi gick förbi en liten göl och vandrade förbi några kohagar. Hulda pratade hela tiden och jag gladde mig åt att lyssna på hennes klara, vackra röst. I smyg sneglade jag på hennes bröst och upptäckte att hon inte knäppt flanellskjortan ända upp denna dag. Jag rodnade och kände mig nervös då jag förstod att det var kanske i eftermiddag det skulle ske. Jag skulle få vara nära Hulda och kanske skulle det bli mer än bara en kyss denna gång. Jag var helt uppe i mina egna tankar. Vi gick och gick och jag tänkte inte ens på att jag inte hade en aning om var vi var någonstans. Vi borde gått förbi Kullatorpet för längesedan men jag hade inte lagt märke till det. Vi stannade till vid en stor sten i en liten hage. Hulda öppnade sin ryggsäck och bjöd mig att dricka ur en pet-flaska. 

Här!, sa hon och glittrade med ögonen. Jag frågade vad det var i flaskan men hon tvekade att svara. Då tänkte jag att hon kanske var nervös och hade hällt något stärkande i den porlande, gröna läskedrycken. Eftersom jag själv hade hunnit bli ordentligt nervös, brydde jag mig inte om att hon inte svarade. Jag tog tre rejäla klunkar ur flaskan. Det var sött och gott men hade också en stark smak av granbarr. Hulda drack inget utan lät mig behålla flaskan. Jag tog en klunk till och kände mig plötsligt otippat lugn och glad. Sprudlande av lycka och ja, som i ett rus befann jag mig. Jag förstod att Hulda nog hade spetsat drycken med alkohol men jag brydde mig inte om det. Nu tänkte jag bara på Hulda, hennes ögon, hennes doft, hennes kropp. Hulda sprang iväg en bit ifrån mig. Hon ropade: ”Kom då, vi är snart hemma hos mig!”

Jag halvsprang efter, glad i hågen. Vi kom längre och längre in i skogen. Granarna reste sig höga och täta runt oss. Den mulliga skogsdoften omslöt mig och jag omfamnade skogen med alla mina sinnen. Färgerna, känslan av mjuk mossa under mina fötter, ljuden av fåglar och bristande grenar. Huldas porlande, förtrollande skratt. Skogsluften kändes stark och frisk, den fyllde mina lungor och mitt sinne. Nu hade Hulda stannat framför mig. Hon knäppte upp en knapp till i sin skjorta och jag kunde ana hennes bleka, fräkniga hud därunder. Jag gick fram till henne och höll om henne. Kyssen som hon gav mig var helt magisk. Jag stirrade in i hennes gröna ögon och visste instinktivt att Hulda nu var villig att låta mig komma riktigt nära henne.  Jag fumlade med mina fingrar och knäppte upp alla knappar i hennes rutiga skjorta. Hon hade fångat min blick igen med sin trolska blick, ögonen verkade nu om än mer djupa och det fanns en gnista i dem som jag inte förut sett. Hennes ansiktsuttryck var också annorlunda, hon log med inte på samma sätt som hon gjort förut. Jag kände med mina händer över hennes byst, flyttade dem nedåt och öppnade hennes skjorta. Det som jag upptäckte sedan gjorde mig helt chockad. ”Va!” Jag skrek till av skräck när jag såg och kände att under hennes bleka, fylliga bröst fanns ingenting. Bara ett stort hål i bröstkorgen. Genom hålet såg jag marken och skogen bakom henne och hennes leende var nu mer hånfullt än snällt. Jag måste tuppat av med en gång, ramlat ner på mossan och slagit huvudet i en rot som stack upp bredvid stigen. Jag låg avsvimmad, säkert i flera minuter. 

När jag vaknade hade det börjat skymma. Jag ropade på Hulda men det kom inget ljud ur min strupe. Jag hörde heller inga ljud. Det var som om skogen tömts på alla sina ljud. Jag var förvirrad och ledsen. Vad hade hänt? Lite längre bort vid några granar tyckte jag mig ana en kvinnogestalt. Naken. Hon gick mycket sensuellt och graciöst av och an, det såg ut som om hon spanade efter något. Återigen ropade jag efter Hulda men inget ljud kom ur min strupe. Kvinnogestalten vände sig om och började halvspringa. Den bleka, nakna kroppen syntes nu tydligt avteckna sig mot skogsbacken. Rött, virvlande hår, en rygg med ett stort hål i, bleka, släta skinkor och mellan dessa en lång, blek svans med en hårtofs i slutet. Jag rös till av obehag. Det kan väl ändå inte varit Hulda? Jag kände mig fortfarande mycket förvirrad. Förvirrad, illamående och vansinnigt trött. Jag somnade igen.

När jag vaknade av solens strålar som letade sig ner genom granarna, kände jag mig frusen. Jag var torr i munnen, men ljuden var tillbaka. Jag testade min röst och sade ”Hallå” flera gånger rakt ut i luften. Jo, det hördes. Kanske inte riktigt den målbrottsröst jag var van vid men ändå, det hördes. Skogens ljud var också tillbaka. Det kvittrade och prasslade. Jag stampade mina fötter över några grenar. Det knakade och knastrade rejält. Hur skulle jag hitta hem? Jag skrek efter Hulda. Hennes namn ekade högt mellan de höga granarna. Jag kände mig ömsom arg, ömsom förtvivlad. Framför allt så kände jag mig övergiven och olyckligt kär. Mina kläder hade antagit en annan färg och struktur sedan jag somnat i skogen. Mina jeans var mossgrönbruna och tröjan hade fått en konstig unken lukt. Jag tog mig för munnen – jag hade stort skägg! Hur länge hade jag legat här? Nu ville jag bara ta mig ut ur skogen. Ta mig hem. Efter många timmars irrande hittade jag tillslut ut på en stor väg, den var mycket större än jag mindes vägarna kring vårt lilla samhälle.  Jag började gå åt ett håll i förhoppning om att det var rätt håll. Till slut kom jag fram till busshållplatsen. Då visste jag att det fanns hopp om att jag skulle komma hem igen. Jag satte mig på bänken och väntade. Det fanns ingen busstidtabell i busskuren. Förstrött tittade jag mig omkring. Jag satt där länge, kände mig trött och hungrig. Trött på att vänta tog jag upp en dagstidning som någon lagt ifrån sig på bänken. Det var ju ändå ett tidsfördriv att ha något att läsa under tiden jag väntade. För min inre blick såg jag bara en rödhårig tonårsflicka, blek som snö med ögon som smaragder och jag kände faktiskt ingenting. Jag kände inte någonting förrän jag tog upp tidningen och läste: Blekinge Läns Tidning 10:e oktober 2021.


\chapter[nypajobbet]{Ny på jobbet}

\dropcapC{arl Iber var} ny på jobbet och han älskade det! Slitet med utbildningen till polis hade varit mödan värd och han njöt av varje ögonblick. Den mörkblå uniformen satt som en smäck på hans smala kropp. Den käcka mössan täckte precis hans begynnande flintskallighet och han såg på sig själv beundrande i spegeln. Inte nog med att han hade fått ett nytt jobb, han hade dessutom lyckats få ett hyreskontrakt i första hand i ett hyfsat kvarter i staden. Grannarna på Genvägen 11 verkade toppen och Carl tyckte att lägenheten var ett riktigt kap. Som liten hade Carl alltid drömt om att bli polis, blått var hans favoritfärg och han hade tidigt blivit duktig på att uppmärksamma diverse oegentligheter i bostadsområdet där han bodde. I tonåren hade han slukat alla kriminalromaner som fanns hemma i bokhyllan, på biblioteket och till och med i grannarnas bokhyllor. Det var inte särskilt anmärkningsvärt att hans idol var Leif GW Persson och att favoritprogrammet på tv var Veckans Brott. Carl Iber sög åt sig all kunskap i kriminologi som en svamp. Trots sin kunskapsiver blev betygen på det naturvetenskapliga programmet på gymnasiet inte särskilt imponerande. Matematiken var abstrakt och oförståelig. Fysiken var endast intressant om man skulle beräkna hastighet på en avfyrad gevärskula och kemin handlade alldeles för lite om otillåtna substanser enligt Carl. Idrotten var tuff och ansträngande. Carl tvivlade på att han skulle lyckas komma in på polisutbildningen men hur det än var så hade han gjort det. Visst var det lite knepigt att vissa moment var på distans och mycket byggde på självstudier men så var det ju i dessa pandemitider. Nu var han ny på jobbet, en polis att räkna med i ur och skur, ja till och med regnjackan som ingick i uniformen var mörkblå och fräsig tyckte han. Carl bar sin nya outfit med stolthet och märkte givetvis också hur folk på stan uppskattade hans närvaro. Här kunde han göra nytta! 

De första månaderna som polis flöt på lugnt. Det var en del ärenden med borttappade katter och kaniner, nedskräpning på allmän plats och en och annan bruten trafikregel. Batongen hade han inte behövt använda. Handklovar hade han inte ens fått till sig ännu. Det kom väl med tiden, tänkte Carl. Walkie-talkien verkade dock fungera bra och Carl tvekade aldrig att använda den. Det var bara lite problem ibland att något annat ovidkommande hamnade på samma frekvens men det var ju bara petitesser enligt honom själv. Väl hemma i lägenheten på kvällen åt han en god middag, skötte hushållet med tvätt och städning och fick ibland lite tid och samtal med grannarna. Trevliga var de verkligen. Gästvänliga och inkluderande. Redan första veckan som ny hyresgäst hade han blivit inbjuden till Kenta på 5:e våningen. Kenta hade gröna fingrar, hela lägenheten var full av frodiga, gröna krukväxter. Det märktes att Kenta var ett riktigt proffs, han kunde det där med gödning och växtbelysning, det var solklart. Kenta hade velat bjuda på en öl men Carl tackade nej. Alkohol befattade han sig inte med. För att inte Kenta skulle bli alltför besviken tog han istället emot en liten cigarett när det erbjöds honom. De två grannarna hade haft en riktigt trevlig stund tillsammans och det visade sig att Kenta inte bara var bra på växter, han var också en filosof och en grubblare. De hade haft djupa samtal den kvällen om meningen med livet och om parallella universum. Dock hade Carl inte tålt cigaretten så bra, i början var det en mycket angenäm upplevelse och Carl hade nästan ångrat att han inte provat cigaretter tidigare. Efter några timmar  hade det istället snurrat till ordentligt i huvudet och han kände ett lätt illamående. Han påminde sig själv om att inte röka på tom mage i fortsättningen, tackade för sig och gick hem och lade sig. Väl hemma i sängen hade han känt det som om taket skulle falla ner på honom. Morgonen därpå kändes det mycket bättre, då var allt som vanligt igen. Så trevlig han var den där Kenta, och generös!

Det fanns även andra vänliga själar i trapphuset. En gång hade Carl tappat bort nycklarna till en hyrbil som han hade hyrt över en helg då han ville köra ner till kusten. Då hade Stig i lägenheten mittemot varit så behjälplig. Han kunde tydligen allt om bilar och motorer. Händig var han som sjutton. På tre röda sekunder hade han fått igång hyrbilen igen så att Carl kunde åka och lämna den på macken. Carl fick betala för den borttappade nyckeln såklart men det var ju inget konstigt med det. Stig var en pålitlig karl och väldigt generös. Han handlade med elektronik och hade en gång ringt på hos Carl med hela famnen full av nya, fina bilstereos. Carl hade då ännu inte hunnit få sin första lön, annars hade han alldeles säkert köpt en bilstereo av Stig. 

På andra våningen i hyreshuset bodde en piffig dam som hette Lolita. Hon var alltid så snygg i håret, sminkad och doftade starkt av någon blommig, söt parfym. De blonda lockarna hoppade på hennes axlar när hon skrattade. Carl tyckte att hon var mycket attraktiv men säkerligen något för gammal för honom. Det verkade som om hon var egen företagare och drev någon slags salong hemma i lägenheten. Varje dag strömmade kunder till, det var män i olika åldrar och med olika stilar. Carl tänkte att det var en herrfrisörsalong som Lovisa drev. Det verkade mycket smidigt, öppettiderna var mycket flexibla. Ibland kunde Carl höra sista kunden ge sig av vid tolvtiden på natten. Otroligt vilken bra service hon hade Lolita, så praktiskt för de som var skiftarbetande och inte kunde komma till frisören på vanlig kontorstid. Huruvida Lolita var särskilt duktig som hårfrisörska lät Carl vara osagt. Förvisso såg kunderna mycket nöjda ut när de stegade ut ifrån Lolitas välpyntade lägenhet. Carl kunde dock inte se någon större skillnad på deras frisyrer, möjligen att de var lite mer moderna och rufsiga. Av den anledningen hade Carl vänligt och artigt avböjt när Lolita velat erbjuda honom en tid inne hos sig. Dessutom tog hon bara betalt i kontanter och det hade Carl väldigt sällan några hemma. Lolita hade sett lite besviken ut men försäkrat honom om att han kunde bli erbjuden en tid närhelst han ville. Carl hade tackat såklart och lite generat tagit emot en kindpuss av Lolita. Kanske hade hon ett extra gott öga till yngre män? Såhär i efterhand tänkte Carl att det nog måste varit så det var. Hans lilla hårkalufs var ju inte mycket att hänga i julgranen, hans hårfäste hade sedan lång tid tillbaka krupit högt upp på hjässan och mötte där en stor måne. De hårtestar han hade var absolut inget att skryta med. Carl var förvissad om att det inte gick någon nöd på Lolita, salongen verkade omåttligt populär bland stadens män och han mötte alltid nöjda kunder i trapphuset. 

Carl trivdes med polisarbetet. Nog kunde dagarna bli lite långa ibland men han gladde sig åt att patrullera och synas på stan. Än hade han inte fått åka med någon kollega i tjänstebilen men å andra sidan var det skönt att vara ute i friska luften. Tids nog skulle han få komma med ut på någon utryckning och fram till dess skulle han göra sitt bästa med patrullerandet. Hyresvärden i huset där Carl och hans grannar bodde var mycket nöjd med att ha en polis som hyresgäst. Han betalade alltid hyran i tid, var rök- och spritfri och ställde alltid upp om det behövdes lite extra bevakning nattetid. Carl hade en kväll blivit tillfrågad om han ville övervaka källartrappan på utsidan av huset då hyresvärdens son kom hem med tavlor som han kommit över för en billig peng i Danmark. Tavlorna var tydligen eftertraktade och värda en hel del. Hyresvärden och hans son kunde då tryggt, under polisen Carls beskydd, bära ner tavlorna till ett källarförråd utan att få alltför mycket insyn. Carl kände sig mycket viktig och han gjorde verkligen en god insats tyckte hyresvärden.  Nästa månads hyra behövde Carl inte betala, nej, något skulle han ju ha för besväret tyckte hyresvärden.

Med tiden kom ett paket till Carl på posten. Det var från Polishögskolan. Som ett barn på julafton bar Carl hem det lilla bruna paketet och satte sig förväntansfullt hemma vid köksbordet för att öppna det. Vad kunde det vara? Ett tjänstevapen? Nycklar till en tjänstebil? Med trevande fingrar öppnade Carl paketet. Inuti låg ett par sprillans nya handklovar med 2 st nycklar till. Carl hoppade av förtjusning. Nog skulle han bli en välutrustad polis snart ändå! Nu kunde han stå ut med att vänta lite till på det där tjänstevapnet. Handklovarna skulle komma väl till pass om han stötte på en felparkerare eller en nedskräpare. Det nya materialet som polisen verkade satsa på nu var plast. Handklovarna hade inte samma glans och tyngd som de handklovar han sett andra poliser bära men vad gjorde det. Det här var säkert det allra senaste. Den kvällen åt Carl Iber en god kvällsmåltid med kokt korv och potatismos. I skenet av stearinljuset på köksbordet glittrade en liten tår i ögonvrån. Han hade så mycket att vara tacksam för. Han hade klarat sin polisutbildning, haft turen att få ett hyreskontrakt och han hade verkligen så hederliga, goda grannar. Efter kvällsmaten tänkte han gå upp till Kenta och be om en sådan där cigarett igen, nu hade han ju i alla fall ätit något och kom inte dit på tom mage. Kanske, kanske skulle han också låta Lolita få honom som kund för att visa sin goda vilja och främja den goda grannsämjan. Bakom Carl på den randiga kökstapeten hängde ett inramat diplom i en tjock, guldpläterad ram. I skenet av ljuset skimrade texten. ”Carl Iber” har genomgått godkänd distansutbildning till polisman” Underskriften av rektor Chao Li stod med svart sirligt bläck och under det i stora svarta gemener: ”University of Shanghai”.


\chapter[dennaknasanningen]{Den nakna sanningen}

\dropcapA{dam Lasti var} med sina 75 år inte längre någon ungdom. Visst kände han att åren hade tagit ut sin rätt och att han inte längre var lika vig och rörlig som innan men sinnet kändes fortfarande ungt. Han hade kommit som krigsflykting till Sverige från Finland i tonåren och hamnat i en god fosterfamilj som tagit hand om honom som om han vore deras egen son. Adam hade fått en lärlingsplats hos den lokala skräddaren i Byxelkrok. Därefter kunde han så småningom skaffa sig en egen liten lägenhet, gifte sig och blev senare också delägare till det lilla skrädderiet. Sömnad och tyger var något som Adam förstod sig på. Med en förkärlek för det enkla och naturliga blev hans kreationer något utöver det vanliga, ja vissa skulle nästan kunna kalla hans stil för spartansk och torftig, men den tilltalade en del av den välbärgade befolkningen i alla fall. Själv beskrev han sin stil som bar, avskalad och naturlig. Adam var övertygad om att ”Less is more” och det applicerade han på stora delar av sitt liv, inte minst sin klädstil, att gå barfota var en nödvändighet. Även hemmets inredning var minimalistisk, kal och funktionell. Nu var han pensionär sedan 10 år tillbaka, klädsömnaden hade varit en mycket stor del av hans liv och han mindes mycket väl när han till slut hade börjat tröttna på allt vad kläder och tyger innebar. Det stod honom upp i halsen och det hade blivit mindre och mindre viktigt för honom själv att bära snygga kläder.

Det hade varit skönt att dra sig tillbaka från rampljuset och ställa undan symaskinen för gott. Hänga undan finkostymen och byxorna med pressveck, slå igen garderoben. Adams fru Inga hade också glatt sig åt hans pensionering och de två hade fått några fina år tillsammans innan hon hamnade i någon slags kris och ville skiljas. Hon hade träffat en yngre man, en välbärgad greve från Norrland. Han klädde sig mycket extravagant och strödde pengar omkring sig verkade det som. Ja, ville hon ha det så, tänkte Adam, så fick det väl vara så. Själv var han ju den mer enkla typen, kravatter och broderier var inget för honom. Nej, naturligt ska det va´ hade han sagt och det gamla paret hade skilts åt som vänner med många goda minnen i bagaget. Adam mindes med glädje den gången de provat på att campa för första gången. De var väl i 30-årsåldern och hade åkt till Klädesholmen under några varma sommarveckor. Tältat, badat, solat sig på de nakna klipporna och bara haft det väldigt naturligt och bra. Det var då de hade förstått att campinglivet liksom var deras grej, där de kunde leva enkelt och naturligt, nära naturen och med likasinnade.

För några månader sedan hade han flyttat till fastlandet för att på äldre dar vidga sina vyer. Nu levde han ett lugnt och enkelt liv för sig själv i en lägenhet i Kalmar. Det var en hyreslägenhet på markplan med en egen liten täppa. Där hade han spenderat hela sommaren, solandes på gräsmattan. Några trädgårdsmöbler hade han aldrig skaffat sig, nej naturligt skulle det vara. Vad kunde kännas bättre att ligga på än den mjuka, gröna mattan? Känna doften av gräset och borra ner tårna mellan grässtråna, det var underbart! 

Trots sin goda kondition och något så när vältränade kroppshydda hade Adam Lasti ändå börjat få svårt att ge sig ut och handla, rollatorn var förvisso behändig att packa varor i, men det var mycket mer bekvämt att låta hemtjänsten handla. De var ju ändå alltid ute och cyklade, tänkte han. Då slapp han dessutom skolbarnens dumma kommentarer, påhopp och grannarnas och kundernas arga blickar i affären. Han hade aldrig blivit riktigt accepterad i området där han nu bodde. Det var något han hade svårt att förstå. Han hade inte haft problem med folk förut. Alla semestrar under åren, på campingar runt om i Sverige och Danmark, hade varit fyllda med umgänge och trevliga samtal. Gubbarna på kallbadhuset var alltid så trevliga och roliga. I gymmets bastu efter att han tränat, hade han många och långa samtal med både äldre herrar och lite yngre, medelålders män. Där hade det aldrig varit något problem med att knyta kontakter och småprata om ditt och datt. Det föll sig så naturligt. Nu började Adam fundera på om det varit fel beslut att flytta in till stan och bo på fastlandet. Det verkade inte naturligt för hans nya grannar att ta kontakt eller ens hälsa på honom. Hemtjänsten var också något undvikande då de kom för att hämta inköpslistan och senare lämna matkassen till honom. Han undrade varför. Han som var så enkel och naturlig? Adam var inte den som klädde upp sig inför en fest eller en finare middag ens. Nej, naturligt skulle det vara, det var hans motto. Många av hans och Ingas gamla vänner var av samma uppfattning, ju enklare – desto bättre!

Adam tänkte att sommaren skulle bli en härlig tid och att människor i allmänhet var lite mer glada och öppna då vädret var varmt och soligt. Här i stan verkade det inte vara så. Han började uppleva att promenaderna på stan blev märkligare och märkligare. Folk nästan gick omvägar för att undvika honom där han gick, även om de ibland hälsade tillbaka, var det ingen av hans grannar som stannade för lite småprat. Trots att han såg bra ut och hade ett mycket välvårdat yttre så verkade det som om han väckte anstöt hos de han mötte. Han slutade till sist att försöka skapa kontakt och höll sig mer och mer för sig själv. Hemtjänsten fortsatte att troget göra inköp åt honom och Adam vande sig vid att bara vara hemma i lägenheten. Hösten kom och med den kylan. Ute på stan gick pratet om den udda gubben Lasti på hörnet som de flesta verkade ha en åsikt om.  

Dagarna blev längre och längre. Adam började längta ut igen. Den lilla lägenheten började kännas trång och instängd. Nu hade han inte vågat sig ut sedan augusti och drivorna med höstlöv såg så lockande ut. Han hade alltid tyckt om att pulsa genom löven, slänga sig i lövhögar och känna doften av höst, löv och kyla. Skulle han våga sig ut? Han mindes den varma sommaren och det lilla smultronstället i viken som han hittat. Där låg han och solade på varma, nakna klippor. Han kände sig som ett med naturen och badandes i havet hade faktiskt människor och barn hälsat glatt på honom. Varför var det så annorlunda inne i stan? Och när grannarna gick förbi honom då han var ute i trädgården? Han kunde faktiskt bara minnas en enda gång som en granne stannat vid staketet och tilltalat honom. Det var i mitten av maj, en riktigt fin, varm försommarkväll. Adam hade kånkat ut grillen från förrådet och slängt på några burgare. Röken och grilldoften spred sig i kvarteret. Han hade blivit glatt överraskad när en av de kvinnliga grannarna hälsat på honom med en vinkning där han stod bakom sin klotgrill i sitt grillförkläde. De hade småpratat lite om det fina vädret och hur härligt det var med grilldoften och sommaren. Nästa gång grannen gått förbi och Adam stod i trädgården och vattnade, hälsade hon inte ens. Hennes blick var oåtkomlig och hon gick snabbt och med bestämda steg. Adam hade aldrig förstått varför och det gjorde lite ont i hjärtat när han tänkte på det. 

För att muntra upp sig själv började han fantisera om att ge sig ut för att pulsa i höstlöven, kanske rentav slänga sig i en lövhög – som ett barn och rulla runt? Adam funderade en stund till. Jo, han var ju ung i sinnet och han längtade verkligen ut till hösten. Till de nakna träden, de bara buskagen och alla löv i drivor som så vackert prydde gångstråket utanför köksfönstret. Jo, nog kunde han våga sig ut trots allt. Det var snart skymning. Klockan närmade sig halv tre. Han kunde vara ute i en halvtimme och sedan gå in och värma sig vid kaminen med en kopp kaffe. Det skulle bli mysigt. Enkelt och mysigt. 

Adam Lasti tog på sig en hemstickad mössa på huvudet, det var ju trots allt lite kyligt, stoppade sina barfota fötter i de röda gummistövlarna i hallen och slängde en snabb blick i hallspegeln på vägen ut. Han kände sig som ett barn på nytt. Lövhögen kallade på honom! Med sin rollator traskade han ut genom ytterdörren, han gick med stora kliv och de röda gummistövlarna gnisslade lite lustigt omkring hans fötter. Det var kallt och nästan på gränsen till frost men det var härligt att känna höstkylan nypa i skinnet. De höstfärgade drivorna med löv lockade honom men längre fram, mittemot den nyöppnade jourbutiken, låg en riktigt stor och inbjudande lövhög. Ingen människa syntes till. Adam ställde sin rollator bredvid lövhögen. Pulsade först i kanten på den och sedan - slängde han sig i den! Den var kall och fuktig, doftade höst och han kände ett lyckorus som han inte känt på länge. Det var galet och härligt. Han rullade runt i lövhögen, satte sig upp och slängde upp löven i skyn så att de dalade ner över hans huvud och kropp. Sedan lade han sig raklång ner i lövhögen. Tittade upp mot himlen, vad kunde kännas mer naturligt än det här? Plötsligt stod två  mörka gestalter framför honom. De såg på honom med stor skepsis.

”Vad händer här då?” sa den ene av dem med hög, barsk röst. Adam ryckte till och satte sig upp i lövhögen. Löven hade fastnat i hans mössa och i det välansade skägget. Nu var det roliga slut men han hade i alla fall fått ett härligt bad i lövhögen.

Mittemot, utanför den nyöppnade jourbutiken och i alla fönster på gatan, stod människor och tittade förskräckt ut på den scen som nu utspelade sig framför dem. Två poliser plockade resolut upp gubben Lasti ur en lövhög vid sidan om gångstråket. Han blev erbjuden en filt men kastade genast bort den, fäktade och skrek: ”Naturligt ska det vara!”. Han var endast klädd i en stickad mössa och ett par röda, fina gummistövlar. Varken skymningsljuset eller höstlöven kunde dölja den nakna sanningen. 


\chapter[halet]{Hålet}

\dropcapD{et började med} en liten, liten fläck på det benvita klinkergolvet i parets skafferi. Tomas hade inte stört sig så mycket på den men Fanny önskade få bort den omedelbart. Hon hade själv försökt med såpa, allrengöring och bikarbonat men helt utan resultat. Nu överlät hon istället problemet till sin uppfinningsrika sambo. Tomas och Fanny hade varit ett par i några år, först bott i en liten etta tillsammans, sparat ihop till en slant och till slut lyckats köpa en fin liten torparstuga i Grävamåla, en bit utanför Växjö. Stugan var relativt nyrenoverad. Det fanns en kamin i vardagsrummet och nylagt klinkergolv på nedervåningen. Tomas arbetade som slöjdlärare på den lilla högstadieskolan och Fanny var läkarsekreterare på byns vårdcentral. De hade det bra tillsammans och trivdes med sitt liv. Inga stora utsvävningar var på tapeten utan de gladdes åt sin tillvaro. Jobb och hus med trädgård, vad mer kunde man begära? 

Den lilla fläcken på det benvita klinkergolvet lät sig inte tvättas bort. Fanny glömde snart bort att störa sig på den. Det var ju så mycket annat som var bra med det lilla huset. Hon och Tomas hade det också bra. De hade en del gemensamma intressen såsom pussel och att lösa korsord. Det var fina stunder på kvällarna då de satt och drack te framför brasan i kaminen och lade pussel tillsammans. Tomas var också en mycket omtänksam man. Han var mycket noga med att göra Fanny nöjd. Önskade hon en ny köksbänk eller någon annan förbättring i stugan var han genast igång med att fixa och dona. Det var praktiskt att vara slöjdlärare och snickare när man blev med hus tyckte Tomas. Fanny kunde inte annat än att hålla med om den saken.

Hur bra de än hade det kunde Tomas inte riktigt släppa den lilla, lilla fläcken på klinkergolvet i skafferiet. En dag tog han med sig ett sandpapper hem frän slöjdsalen. Så fort han kommit hem lade han sig på alla fyra i det lilla skafferiet  och började försiktigt slipa på den lilla fläcken. Fanny tittade nyfiket på honom. 

”Vad gör Du gubben?” frågade hon. Tomas svarade att han bara skulle försöka få bort fläcken. Till en början såg det ut som att han skulle lyckas. Fläcken bleknade under sandpapprets repetitiva drag och Tomas började känna sig nöjd och säker på att han skulle lyckas få bort den. Ja, nu såg den faktiskt mindre ut. Efter middagen satte sig paret ner vid brasan igen med det stora pusslet som de påbörjat. När det började skymma fikade de på te och nybakad hålkaka som Fanny bakat. Livet var gott. På kvällen när de gått och lagt sig kysste Tomas Fanny ömt och såg på henne med kärlek i blicken: 

”Det verkar som om fläcken kommer att försvinna”, sa han. Den blev mycket mindre när jag slipade på den med sandpapper.

”Så bra”, sa Fanny och gäspade nöjt. Du är en duktig karl Du!  

Den natten somnade Tomas med ett leende på läpparna. Allt var bra och imorgon skulle han gå på det sista av fläcken med ett lite grövre sandpapper. Som man kände han sig mycket nöjd. Han kunde ordna allt åt Fanny, oavsett om det rörde sig om något större projekt eller om något så litet som en liten defekt på en klinkerplatta.

Nästa morgon när Tomas vaknade hade Fanny redan givit sig iväg till jobbet. Tomas tog en kopp kaffe och innan han skulle cykla till jobbet slängde han ett öga in i skafferiet. Men vad nu? Fläcken som hade bleknat framför honom igår efter gediget sandpappersarbete, lyste nu åter svart och hånfullt mot honom. Inte nog med det, den såg ännu större ut nu än innan han hade börjat sandpappra på den. Det knöt sig lite i magen på Tomas. Dagen träslöjdslektioner segade sig fram. Många var skoleleverna som behövde hjälp med håltagning i sina fågelholkar och skärbrädor. Tomas fann att hans tankar mer eller mindre hade stannat hemma i skafferiet. Förbaskade fläck, tänkte han. När lektionerna var slut för dagen lät han blicken vandra i verktygsskåpet. Lite mer och grövre sandpapper behövde han. En rasp och ett stämjärn fick också följa med hem i väskan. Med nyvunnen optimism och en tilltro på att ingenting är omöjligt cyklade Tomas hemåt.  Fanny fann honom liggandes på alla fyra i skafferiet när hon kom hem.

”Är Du igång nu igen? Jag trodde att fläcken var borta” frågade Fanny nyfiket. Tomas suckade.

”Inte än, jag måste gno en vända till med sandpapperet, sedan ska det nog vara löst”. Tomas lät andfådd. Middagen åt de båda snabbt. Fanny var angelägen om att lägga pussel med Tomas medan Tomas mer inriktade sig på att lägga några till minuter på den envisa fläcken i skafferiet. Fanny satt ensam med pusslet den kvällen. Från skafferiet hördes det slipande ljudet från sandpapper och Tomas som flämtade och suckade. Efter några timmar, alldeles före läggdags, hade det slipande ljudet bytts ut mot ett än mer envist raspande och filande. Tomas suckar hade också ändrat karaktär. Från att ha vittnat om uppgivenhet förut lät det nu mer som om han fått en aha-upplevelse då han bytt från sandpappret till raspen. Mycket sent kröp Tomas i säng bredvid Fanny. Hon hade nästan lyckats somna.

”Fanny, det är något förunderligt och märkligt med fläcken”, sa Tomas uppspelt. ”Den liksom döljer något annat, jag tror att det kanske är något speciellt under den där klinkerplattan”, sa han med stor fascination i rösten. Fanny mumlade något till svar och snart sov de båda sött.
 
Följande dag i slöjdsalen förflöt som vanligt för Tomas. Det snickrades och slipades bland eleverna. Putsades och borrades. Tomas arbetade effektivt, lät den sista lektionen avslutas en halvtimme tidigare till elevernas stora förtjusning. Därefter en flukt i det stora skåpet. Denna dag följde även en slagborrmaskin med ner i Tomas väska. Ivrigt trampade han iväg på cykeln hemåt. Förutom slagborrmaskinen var väskan påfylld med fler raspar och stämjärn. Det skramlade om ekipaget när han cyklade över brunnslocken på gatan.

Middagen slukade han i ett nafs. Det som tidigare varit en liten irriterande fläck på det benvita klinkergolvet var nu istället ett hål på ca 5 cm i diameter. I skafferiet tillbringade han hela kvällen med att fila, raspa och borra. Det dammade och bullrade. Fanny hade nu givit upp om att få med sig Tomas till deras gemensamma pusselstund på kvällen. Tveksamt lät hon honom hållas. Vad det än var han trodde att han skulle hitta var det säkert värt det. När han väl hittade det skulle han mura igen hålet och återställa skafferigolvet till sitt ursprungsläge igen. Det var inget att oroa sig för.

Det gick en vecka. Rektorn hade nu kallat till sig Tomas för ett samtal. Några föräldrar hade synpunkter på att slöjdlektionerna hade avslutats för tidigt och någon klass hade t o m fått besked att deras träslöjd var inställd. Rektorn frågade Tomas om saken. Tomas förklarade att han hade lite bry med sin stuga och det handlade om att ta emot hantverkare och annat. Trots  samtalet med rektorn fortsatte Tomas att ställa in slöjdlektioner och släppa iväg eleverna långt innan lektionen var slut. Han hade fått ett uppdrag, ett viktigt uppdrag hemma, och gick alltmer in i det. Arbetet blev såklart lidande. Fannys sömn började också bli lidande. Nätternas slipande, borrande och hamrande gjorde det nästintill omöjligt att sova. Tomas själv verkade inte låta sig påverkas av sömnbristen. Kafferasten på skolan sov han sig igenom, likaså lunchen. Nu började det bli svårt att få ihop livspusslet. Tiden räckte inte riktigt till för att både undervisa i träslöjd och vara hemma och fixa med hålet. Ja, fixa var kanske inte rätta ordet. Den lilla fläcken, som blivit ett litet hål, hade nu vuxit sig större. 3o cm i diameter och ungefär 40 cm djupt. Tomas var mycket stolt över att han lyckats göra hål i betongen, gröpa ur och gräva, nu var han snart nere vid bärlagret av makadam och grus.  Fanny bara skakade på huvudet. Hon hade börjat se mer och mer sliten ut. Det stora pusslet hade hon fått sitta med helt själv den senaste tiden, 1000 bitar skulle ta ett tag att lägga. Tomas hade tappat intresset för pusslet, han var totalt ointresserad. Inte ens den mysiga kvällsfikan med te och hålkaka kunde motivera honom att komma ut från skafferiet. 

För att undvika fler samtal med rektorn angående missnöjda föräldrar och elever som missade sin träslöjd, valde Tomas att helt sonika säga upp sig från sin tillsvidareanställning på skolan. Fanny höll på att tappa hakan när han meddelade henne sitt beslut. Beslutet skulle innebära ett stort hål i plånboken och i deras gemensamma ekonomi. Hon hade såklart många invändningar mot hans beslut.

”Förstår Du inte?” påpekade han. ”Vi kommer att hitta något fantastiskt i hålet, något som förändrar våra liv till det bättre!” Men Fanny var inte alls övertygad om det. Hon förstod att hålet inte var början på något fantastiskt utan istället slutet på något. Tanken på det gnagde i hennes hjärta och hon blev mer och mer orolig. Vad trodde Tomas egentligen att han skulle hitta i hålet? Det gjorde ont i henne att tänka så. 

Grävandet fortsatte. Tomas hade nu frångått slagborren och kunde börja gräva upp makadam och grus. Han fyllde hink efter hink, ställde dem i hallen. Fanny hade svårt att ta sig in och ut. För att inte låta Tomas iver ta över hennes eget liv ägnade hon sig nu helhjärtat åt pusslet på kvällarna. Hon var nästan färdig. Motivet var en kyrka på en blommig backe. Blå himmel med fantastiska vita moln som naturligtvis var svåra att få ihop. Alla pusselbitar var ju så lika. Framför kyrkan stod ett brudpar. Nätterna blev lite bättre sedan Tomas börjat gräva istället för att använda borren. Hålet blev större och större. Även pusslet växte, bara några bitar kvar…Något som däremot inte växte var Fannys och Tomas relation. Paret verkade prata vid sidan om varandra. De åt och sov på olika tider. Enstaka ord växlades mellan dem i hallen eller vid skafferidörren. Inte ens när Fanny saknade den sista pusselbiten i deras gemensamma pussel hade Tomas tid eller lust att hjälpa henne att leta. Han var som uppslukad av hålet.  

Nu var det stort, 60 cm i diameter och det enda som stack upp ur det var Tomas dammiga kalufs när han reste sig ur hålet för att hiva upp en hink med jord. Hur djupt kunde det vara? 1, 70? Fanny visste inte men det var djupt. Tillräckligt djupt. Hon bad Tomas att sluta gräva men för varje gång hon bad honom blev hans engagemang i gropen ännu större. Han skulle ju bara gräva klart, hitta det där fantastiska som skulle göra allting bra igen och sedan fylla igen allt och återställa det en gång hela, benvita klinkergolvet. Fanny tittade på Tomas med stora ögon och tom blick. 

”Jag har fått nog nu”, sa hon. ”Det räcker nu, jag flyttar ut”. Orden träffade Tomas som en pil rakt i bröstet. Vad menade hon? Han som skulle ställa allt till rätta, göra henne nöjd och lycklig. För att inte låta sin stora besvikelse och sorg ta överhanden fortsatte han genast att gräva i hålet. Han hörde Fanny klä på sig i hallen och för sista gången stänga ytterdörren till deras gemensamma torparstuga. Han skulle bara gräva lite till, lite till…

Sorgen och besvikelsen drabbade honom i efterhand. Då hade han grävt ett bra tag. Flera timmar efter att Fanny lämnat honom. Plötsligt kände han något med den lilla spaden, längst ner i hålet. Något hårt men ändå lite mjukt. Han satte sig på huk i hålet, kände med fingrarna i den leriga jorden. Jo, där var något. Något litet. Vad kunde det vara? Ivrig och darrig skrapade han fram det lilla föremålet och tog upp det i sin hand. Det var smutsigt och lerigt. Han klev upp ur hålet, satte sig vid bordet bredvid det stora 1000-bitars pusslet. Så flitig hon hade varit Fanny. Synd bara på biten som fattades. Tomas blick fastnade på motivet. En vacker bild, kyrkan på den blomstrande backen, ljusblå himmel och små lätta moln. Hur lång tid hade det inte tagit att lägga detta pussel? Hur lång tid hade han lagt på hålet? Hålet som först inte varit något hål utan endast en liten, liten fläck på det annars så perfekta, benvita klinkergolvet. Med ens gick det upp för honom. Allvaret i situationen. Förlusten av Fanny och deras gemensamma liv. Han hade inte bara grävt ett hål i golvet, han hade också grävt ett hål i Fannys hjärta. Kunde han laga det eller inte? Det visste han inte. Kanske allt var förlorat. Med tårarna rinnande nerför kinderna stirrade han på det stora pusslet. Det saknades bara en bit, precis på brudens överkropp, överdelen av den vita klänningen, den som täckte hennes hjärta. Med skälvande händer tog han fram det lilla föremålet som han hittat djupt nere i leran i det djupa hålet. Leran hade torkat lite och han kunde skava av lite med sina fingrar. Under leran och smutsen skymtade en vit nyans. Föremålet var platt, två sidor var det som små inbuktningar i och två sidor hade något som buktade ut. Han gnuggade bort den sista smutsen med fingret.  Mellan tummen och pekfingret höll han den sista pusselbiten som saknades. 

\stopbodymatter
\stoptext

