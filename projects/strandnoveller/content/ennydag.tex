\chapter[ennydag]{En ny dag}

\dropcapD{et var den }7:e december. Snöflingorna hade sakta börjat dala ner på det lilla sömniga postkontoret i byn. Posttjänstemannen Dag Börjesson tittade förstrött på högen med rek-brev. Hur var det nu igen, vilken säck skulle de ligga i? Varför var det alltid så krångligt med specialposten? Han suckade, tittade över axeln för att försäkra sig om att inte chefen Brandin såg på och tippade därefter lite nonchalant ner rek-högen i den röda julpostsäcken. Så, nu var det löst och Dag kunde ta sig en ny kopp kaffe i personalrummet. Det hade han gjort sig förtjänt av. Logistik och köteori var inte hans specialitet. Nej, roligare var det med kvantfysik och våglära. Trots Dags kunskaper i fysik och studier på universitetet hade han hamnat som posttjänsteman i en liten, sömnig håla, men det bekom honom inte särskilt mycket. 

Postmästare Gösta Brandin satt vid sitt prydliga skrivbord på kontoret. Han suckade och kikade ut över sitt rike. Den här morgonen, liksom alla andra, hade den anställde Dag Börjesson återigen varit mycket sen till jobbet. Han var dessutom arrogant, småfet och inte direkt någon glädjespridare på posten. De andra medarbetarna hade ofta klagat på att han dessutom inte verkade sköta sin hygien, var ohyfsad och otrevlig mot kunderna och nästan alltid var sen till jobbet. Arbetsskygg hade de sagt. Gösta Brandin som var en mycket godtrogen man, brottades med det faktum att Dag Börjesson nog inte passade in i hans arbetslag och han våndades över att han en dag skulle behöva säga upp honom. Skulle idag vara den dagen? Nervositeten började kännas i hårfästet. Gösta slog bort tanken igen, han skulle fundera lite till. 

Kerstin och Karin stod vid sorteringsbandet. Lite längre ner stod Dag. Kerstin och Karin tittade menande på varandra och himlade med ögonen. Hur kunde han ha fått anställning här? Dag var en medelålders man, ogift vad de visste. Han hade en utbildning i logistik och hade dessutom studerat teknisk fysik på universitetet. Trots hans fina referenser och examen verkade han mycket ointresserad av arbetet på det lilla postkontoret. Ofta satt han i personalrummet med en stor kopp kaffe, surfandes på sin mobiltelefon, lödandes på något slags kretskort eller bläddrade i en Allers. Otaliga var de gånger då kunder efter en lång tids väntan, irriterat knackat på luckan då Dag låtsats att inte höra eller se att kön ringlade sig lång bredvid postboxarna. Saktmodigt hade han då öppnat luckan och hjälpt kunderna med deras ärenden, givit dem ett påklistrat leende och sedan bestämt stängt luckan igen. De enda gångerna Kerstin och Karin hade uppmärksammat någon slags iver hos Dag hade varit just innan stängningsdags eller vid den tiden på månaden då lönen utbetalades.

Dag märkte att medarbetarna vid sorteringsbandet himlade med ögonen och kastade menande blickar mot varandra. Hur det än var så var Dag mycket obrydd över det hela. Nu hade han dessutom råkat spilla ut en hel kopp kaffe över bandet också så att hela sorteringen blev pausad. 

Kerstin blev rasande och höll på att gå upp i limningen.

– Nu får det väl ändå räcka! sa Kerstin och konfronterade den lågpresterande Dag. Nu går jag och Karin till chefen, såhär kan Du inte fortsätta! 

Dag såg förundrat på Kerstin, hur kunde någon bli så arg för några droppar kaffe? Dessutom fick hon ju också en välbehövlig paus i sitt arbete. Det där med tid verkade de inte förstå, hans arbetskamrater. Hade de vidareutbildat sig som han själv hade de förstått. Tid. Det finns alltid tid tyckte Dag. Dessutom var tiden väldigt rättvis; alla människor har exakt 24 timmar varje dygn. Det är rättvist. Tid fanns det för alla och allt hade sin tid enligt Dag.

Gösta Brandin ryckte till ur sin tupplur när dörren till hans kontor plötsligt rycktes upp. In stormade Kerstin med kollegan Karin i släptåg. I sitt yrvakna tillstånd kunde Brandin se att de var mycket uppretade, minst sagt. De två postarbetande damerna ställde sig med armarna i kors framför hans skrivbord. 

– Nu har det gått för långt. Vi vill att Du avskedar Dag idag, krävde de båda kvinnorna.

– Brandin måste inse fakta, vi har inte tid med Dag. Nu har sorteringsarbetet stannat av helt. En hel kopp kaffe välte han ut över posten och han bad inte ens om ursäkt. Det var uppenbart att dessa kvinnor hade fått nog, deras tålamod var slut.

Brandin förstod nu att tiden var inne, måttet var rågat och det var dags att ta ett beslut, om än det var i elfte timmen. Ville han att Karin och Kerstin skulle stanna kvar i hans lilla post-kungarike, om han ville behålla två kompetenta personer, var han tvungen att göra sig av med Dag. Dag som redan från Dag 1 hade visat sig något ointresserad och haft svårt med det sociala. Dag som alltid kom försent till sitt arbete och alltid lyckades ha en ursäkt för att lämna arbetet före stängning. Sävlig var han, och långsam. Utom när han plockade med sin mobil och sina kretskort som alltid stack ut ur tjänstejackans fickor. Tid verkade han ha gott om, han hade en lustig förmåga att aldrig verka stressad eller bekymrad över någonting. Inte ens när han fick en varning, inte ens när arbetskamraterna hoppade på honom eller sa åt honom att hjälpa till. Nu var tiden inne. Det skulle bli en ny dag i Brandins kungadöme, en ny dag för Posten i det lilla sömniga samhället. Det var dags att vakna upp! Till en ny dag.

Kerstin och Karin klev chockade ut ifrån postmästare Brandins kontor. Var det verkligen sant? Hade Brandin nu äntligen lyssnat på dem och valt att agera? Jo, visst var det så.. Han hade förstått att Posten inte hade tid för sådana medarbetare som Dag Börjesson. Dag skulle bytas ut. Han skulle avskedas. Det skulle bli en ny Dag. Kerstin och Karin såg på varandra i samförstånd och log. Ikväll skulle det bli tårta och bubbel hemma, timmen var slagen för Dag. Han skulle ut!   

Dagen på det lilla postkontoret började närma sig sitt slut. Skymningen föll och tillsammans med den också vackra, vita snöflingor. Sekundvisaren på klockan i personalrummet verkade gå långsammare och långsammare. Dag tittade på sin mobil, nej, den stämde. Han lutade sig tillbaka och såg tillbaka på sin dag. Den hade varit lugn som alla de andra dagar brukade vara. Han hade in i det längsta dragit sig för att bära ut julpostsäcken till lastbilen som stått och väntat. Chauffören hade irriterat påpekat något om att ”här verkade allting gå allt långsammare” men som vanligt hade Dag bara sett mycket obrydd ut. 

Nu såg han Kerstin och Karin göra det sista av dagens plockarbete med julkort och annat. Han såg också att Gösta Brandin satt på sitt kontor och verkade förbereda sig för något stort, det såg ut som om han övade på ett tal. Brandin vankade av och an i sitt lilla kontor, höll upp pekfingret förmanande och stannade till vid en spegel och såg mycket allvarlig ut. Var det så att han tänkte avskeda Dag idag? Jo, tänkte Dag. Nu var det nog dags att gå hem. Innan det var försent. 

Klockan 15.50 Gick Dag ut till omklädningsrummet för att ta på sig sina vanliga kläder. Nu fick det vara nog med arbete för en dag. I samma stund tänkte även Gösta Brandin att det fick vara nog med arbete för en Dag. Nu skulle det ske, Dag skulle sluta. Bli uppsagd. Med nervösa harklingar och på darrande ben klev Gösta Brandin ut från sitt kontor och stegade mot personalentrén. 

Klockan 15.55 gick Dag Börjesson ut från omklädningsrummet och fram till postboxarna som fanns utanför glasluckan där kunderna stod i kö. Nu var där helt tomt. 5 minuter före stängningsdags. I ögonvrån såg Dag chefen Brandin vanka av och an i korridoren framför vägguret. Dag tog fram en nyckel ur fickan, öppnade postbox nr 911 och tog fram en stor, tung fjärrkontroll. 

15.58 kopplade Dag ihop fjärrkontrollen med ett kretskort som stack upp ur byxfickan. En grön diod tändes, kvartsoscillatorn fungerade idag igen! Dag ställde in frekvensen, kontrollerade transformatorn och väntade med ett brett leende. 

15.59 hade Gösta Brandin Dag inom synhåll, det var i grevens tid men nu skulle det ske men vad i all världen var det han hade plockat ut ur postboxen? Något kändes annorlunda och Gösta tittade förfärat på vägguret i entrén, höll han på att bli galen? Visarna på klockan hade gått i spinn och rörde sig motsols.  

16.00 Poff. Det blev svart. Tomt.

Kl 07.00 den 7:e december dalade snöflingorna åter sakta ner över det lilla sömniga postkontoret i byn. Hemma i sin säng låg Dag och slumrade till ljudet av väckarklockans radio. Idag var en ny dag men samma dag som igår. En ny dag för Dag.                
