%Tävlingsversionen av Fiskmannen
\chapter[fiskmannen]{Fiskmannen}

\dropcapD{et var en} vanlig morgon i april. Koltrastarna utanför Anitas lägenhetsfönster hade väckt henne igen med sin tropiska sång. Anita brukade fantisera om att de lärt sig dessa kvitter utomlands, vid Victoriasjön i Afrika eller i någon djungel. Hon hade alltid varit intresserad av fåglar, djur och natur. Särskilt fascinerande var det att sitta vid havet och se fiskmåsarna dyka och ta upp en och annan firre ur havet. Det hade varit självklart att det var lärare i biologi som hon ville utbilda sig till. På den lilla högstadieskolan trivdes hon som fisken i vattnet.

Anita steg upp och började göra sig iordning framför badrumsspegeln. Ibland kände hon sig som en gammal nucka med sina 55 år och begynnande gråhårighet. Det var längesedan hon hade varit i ett förhållande. Den senaste mannen hon träffat hade förvisso varit underbar men tyvärr inte att lita på. Han hade varit hal som en ål och slingrat sig den gången Anita hade ställt honom mot väggen. Läppstiftet på kragen kunde han inte förklara och Anita hade gråtande kastat ut honom från lägenheten. Det var sista gången hon såg honom. Han hade skickat ett vykort från en semester vid Döda havet och önskat att de skulle försöka igen. Men Anitas känslor var lika döda som havet han semestrade vid. Nästa man som hon skulle våga släppa in i sitt liv måste vara en trygg, stabil karl, som bara hade ögon för henne. Som var en fena på att uppvakta och hålla vad han lovat. Anita rättade till bluskragen, drog en kam genom håret, log och tog ett djupt andetag – vad skulle högstadieynglen ställa till med idag då?

På kafferasten i personalrummet fick hon sällskap av skolvärdinnan Berit. Berit var lite som en extramamma för Anita och Berit var någon som hon gärna anförtrodde sig till. 

”Du måste ut på banan igen Anita, du är för ung och vacker för att leva ensam”, Berit blinkade med ögat. Hon var alltid så käck den där Berit. Tyckte att Anita skulle ut och dansa, i klänning och nätstrumpbyxor, slänga ut sina krokar lite här och var, men Anita var mer av den lågmälda sorten. 

”Följer Du med mig till fiskaffären efter jobbet i alla fall?” frågade Berit. Jovisst, det skulle hon göra. Tydligen hade det börjat ett nytt butiksbiträde där och hans namn var nu på alla kvinnliga lärares läppar. Han var tydligen finne, Pekka Salmo hette han. Ganska ny var han i det lilla samhället och han hade fått arbete direkt hos den lokala fiskhandlaren. Pekka Salmo var tydligen en fena på att filéa fisk och dessutom var han utbildad kock. Ja ja, hon skulle i alla fall handla lite strömming till sig själv. Så spännande kunde han väl ändå inte vara den där fiskmannen, tänkte Anita.

Aprilsolen värmde faktiskt på eftermiddagen när Anita och Berit steg ut från Sjöängsskolan. Högstadieeleverna hade mer än bråttom att komma därifrån. I biologiläxa hade de fått 50 sidor att läsa, det handlade om försurningen i de svenska sjöarna. Eleverna var inte särskilt engagerade i frågan, de hade mest suttit och tittat ut genom fönstret och längtat ut, liksom Anita, för att lapa eftermiddagens sista solstrålar i frihet. 

Väl framme i fiskaffären ringlade kön sig lång fram till disken. Gamle Lennart stod där såklart, lite mer lättad i sitt sinne nu när han hade Pekka till hjälp. Dessutom kunde Pekka ge damerna ( ja, det var mest damer som handlade där) goda råd om hur fisken skulle tillagas, tipsa om tillbehör och vilket vin som passade till. Anita hajade till när hon upptäckte den uppenbart flirtiga stämningen i kön. Damerna tisslade och tasslade. Innan någon kom fram i kön till disken, piffades det med läppstift och borstar. Blusar rättades till eller knäpptes upp en bit och ögon tindrade. Pekka var trevlig och professionell. Han besvarade kundernas frågor mycket artigt och kunnigt men var inte den som lät sig håvas in av smicker. Hans ögon var snälla och han hade ett leende som fick en att smälta. 

”Strömming, 1 kilo tack” sa Anita. Innan hon visste ordet av hade han rensat och filéat den glänsande strömmingen,  förpackat den snyggt och korrekt och sträckt fram plastpåsen till Anita med ett charmerande leende. Anitas ben hade känts som gelé. Någonting glänsande i Pekkas blick fastnade i Anitas tankar den dagen. 

Besöken till fiskaffären blev allt fler under de kommande veckorna. Anita och Pekka började bekanta sig med varandra under småprat och leenden. Han såg bra ut Pekka. Lång, mörk och stilig var han.  En gång hade han frågat Anita om hon hade sällskap. Anita hade rodnat och svarat nej såklart. Efter det hade Pekkas ögon fått en ännu djupare glans och Anita hade nästan drunknat i de blåsvarta ögonen. 

Ytterligare några veckor gick och Pekka hade till slut vågat bjuda ut henne på en date. De hade åkt till Akvariet, tittat på fiskar och rockor hela dagen och mot slutet av deras date hade Pekka tagit hennes hand och kysst henne. ”Vi simmar ju i samma vatten” hade han sagt. Vad han menade med det hade Anita egentligen ingen aning om.

Anitas och Pekkas relation utvecklades och de började äta middag på kvällarna hemma hos Anita. Då och då hände det att Pekka stannade över natten men aldrig så länge att han var kvar när Anita steg upp och skulle äta frukost. Han skulle oftast hjälpa till i fiskbutiken tidigt och ta emot dagens leverans av nyfångad fisk. Förutom relationen med Anita var Pekka mycket engagerad i matlagning och fiske. Han bjöd på hälleflundra med vitt vin. Sjötunga och röding. Stuvade räkor. Anita var mycket imponerad av alla de fantastiska rätter han bjöd på. Anita fäste inte så stor vikt vid att Pekka inte hade bjudit hem henne till sig. Han hade förklarat att hans lägenhet var vattenskadad och att det inte gick att bjuda hem henne av den anledningen. Han var så finkänslig Pekka, tillgiven och omsorgsfull.

Nu hade de träffats under ett par månader och känslorna hade blivit bestående. Visst kunde Anita ibland störa sig lite på att han lämnade stora svettpölar efter sig i sängen. Lakanet var oftast plaskblött när Anita vaknade och Pekka redan givit sig iväg till jobbet. Hon brydde sig i alla fall väldigt mycket om honom och hade till och med köpt en återfuktande kräm till honom då hon upptäckt att han hade väldigt fjällande, torr hud på överkroppen.  Fisketurerna var en annan sak som Anita hakade upp sig på. Ibland kunde Pekka vara iväg flera dagar i sträck. ”Ute med grabbarna vid sjön” svarade han bara när hon frågat var han hållit hus. När han kommit tillbaka hade han varit pigg som en mört, glad och pratig och liksom bubblat av glädje. Så Anita lät honom hållas. 

En kväll i augusti efter en av Pekkas längre fisketurer hade paret ätit en god middag med skaldjur och avnjutit en flaska vin i skenet av stearinljusen. Pekka hade valt att stanna över natten men han skulle upp tidigt morgonen därpå och öppna fiskaffären, gubben Lennart hade tagit ledigt. Anita skulle också upp tidigt, det var i början av läsåret och hon såg framemot att hålla i terminens biologilektioner på skolan igen. Anita vaknade lite tidigare än vanligt morgonen därpå. De röda siffrorna på väckarklockan visade att klockan bara var fyra. Usch, det var tidigt. Anita kände efter Pekka vid sidan om sig. Där var det tomt och lakanet var åter blött och fuktigt av svett. Han hade redan gått upp, det var väl ändå lite väl tidigt? Anita passade på att gå till badrummet. På golvet låg flera blöta pölar, mådde han inte bra Pekka? Han verkade svettas mer än vanligt? Anita rumlade sömndrucken vidare in till toaletten. Vid sidan om henne var duschdraperiet fördraget vid badkaret. Anita gläntade försiktigt på det. I badkaret, som var fyllt med vatten till mer än hälften, simmade en stor, mörk varelse. Anita hajade till, å fy! Den bjässen hade hon inte lagt märke till igår! Pekka hade inte heller berättat att han behövde låna hennes badkar till sin fångst från fisketuren. Kanske var det en överraskning? Skulle Pekka bjuda henne på middag ikväll igen? En alldeles perfekt kolja? Eller var det en gädda? Anita var först något bekymrad över att hon inte kunde artbestämma den på en gång, varelsen liknade egentligen varken en gädda eller kolja. Hon visste inte exakt men hon kände på sig att Pekka hade planerat något alldeles speciellt för henne. Därav den fina fisken. Skulle han fria? Komma med presenter? Att han var en karl som gärna bjöd på sig själv och var generös det visste hon ju. Så romantiskt! För att visa sin uppskattning bestämde hon sig för att själv förbereda den vackra fisken, precis som Pekka lärt henne. Innan Anita gav sig iväg till skolan den morgonen hade hon resolut tagit upp den sprattlande fisken ur badkaret. Hållit den mot skärbrädan i ett fast grepp i köket och med ett enda hugg med kökskniven dekapiterat den mörka, slemmiga varelsen och sedan lagt den i kylskåpet. Huvudet med de blåsvarta ögonen slängde hon i sopnedkastet på vägen ut. Jodå, nog skulle Pekka bli imponerad av att hon kunde förbereda hans fina fångst och ta hand om den så väl. 

Berit, skolvärdinnan var förvisso glad över att Anita äntligen fått en karl på kroken men en kolja i badkaret? Det var väl lite udda? Inte särskilt romantiskt tyckte hon när Anita berättade om att Pekka nog skulle ställa till med överraskningsmiddag. Berit gladde sig ändå för Anitas skull, för att hon med sina 55 år på nacken och begynnande gråhårighet lyckats håva in en sådan stilig karl som Pekka. Då kunde man nog stå ut med en del egenheter tänkte hon. 

Hela arbetsdagen fantiserade Anita om vad det var för överraskning som Pekka skulle komma med. Hon var glad och upprymd, tankarna och känslorna sprattlade som småspigg inombords. En fin middag på kolja eller vad det nu var, lite vin och tända ljus och Pekka. Pekka med en bukett blommor och kanske, kanske en liten smyckesask?

Väl hemma igen fortsatte Anita att fjälla fisken, filéa den och sedan lade hon den i kylen i väntan på att Pekka skulle komma hem från fiskaffären. Klockan blev 18. Han var lite sen Pekka. Anita hällde upp ett glas vin. Klockan 19 var han fortfarande inte hemma. Så dumt att hon inte kunde ringa honom! I morgonbrådskan hade hon sett att han lämnat sin mobil och sina lägenhetsnycklar hemma. 19.15 tog Anita saken i egna händer, nu skulle det bli fina fisken! Hon tog varsamt fram fisken ur kylen, lite smör i stekpannan och salt och peppar. Den ljuvliga doften av smält smör och nyfångad fisk spred sig i lägenheten. Fisken blev kärleksfullt tillagad. Hon kokade potatis och gjorde en beurre blanc såsom Pekka lärt henne i början när de hade träffats. Nu fick han komma närsomhelst. Hon var redo. Han skulle bli så glad och imponerad av hennes matlagning. Han hade lärt henne allt han visste om fisk, matat henne med kunskap och goda recept. 19.30. Ingen Pekka. 20.00 och 2 glas vin senare satt Anita fortfarande och tittade förväntansfullt ut genom fönstret. På andra sidan av det lilla samhället stod fiskaffären mörk och tom. Den hade inte varit öppen på hela dagen.
