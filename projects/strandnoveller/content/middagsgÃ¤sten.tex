\chapter[middagsgästen]{Middagsgästen}

Margareta blickade uttråkat ut igenom fönstret. Därute virvlade novemberlöven runt. Vattenpölarna tycktes mörka och dystra. Inne i lägenheten kändes det också dystert. För några dagar sedan hade barnen hjälpt henne att flytta in. ”Så skönt med en liten lägenhet ” hade de sagt. Nu skulle Margareta slippa allt jobb med trädgård och den tunga städningen som den stora direktörsvillan på Gulsparvsvägen hade inneburit.

Margareta suckade. Tittade sig omkring i sitt nya lilla hem. Så ensamt det var. De finaste möblerna hade hon velat ta med sig. Visst, det blev fint med trångt. Barnen hade våndats över köksbordet som Margareta prompt ville ha in i köket. Resultatet blev att köksbordet fick ställas ända intill väggen, hoptryckt mot den höga rokoko-spegeln som Margareta och hennes make Georg fått i bröllopspresent en gång för 50 år sedan. Man kunde sitta högst en person vid köksbordet, möjligen två om man pressade sig intill köksbänken men det var verkligen inte att rekommendera. Det var en liten 2:a som barnen hittat till henne i det nybyggda seniorboendet för 60-plussare. 

Margareta hade ändå försonats med tanken på en flytt när barnen berättade att hon skulle få så många nya grannar och att barnen skulle få närmre till att besöka henne. Nu visade det sig att de flesta i huset var långt mycket mer äldre än 60 plus. Många hade svårt att ta sig ut och hemtjänsten kom och besökte flera gånger om dagen. Inte var det några friskusar precis men nu skulle ju Margaretas barn kunna komma över lite oftare när de bodde så nära. Det skulle bli så bra. Dagarna gick och barnen lyste med sin frånvaro. Visst ringde de då och då, klagade över att det var så mycket på jobbet och att de knappt hann med att laga middag hemma på vardagarna. De hade inte tid att besöka Margareta när hon frågade.

En onsdagsmorgon i slutet av november blickade Margareta åter ut genom fönstret. Löven dansade i virvlar, kanske något livligare än sist och vattenpölarna blänkte i silver istället för deras vanliga, dystra uppsyn. Ikväll skulle hon bjuda hem barnen på kalkonmiddag! Rödkål och äppelmos hade hon hemma. Efter frukosten gick hon sin vanliga promenad men svängde inom saluhallen för att köpa en liten kalkon, brysselkål och sötpotatis. Väl hemma skalade hon potatis, förberedde kalkonen och penslade den med smör och soya. Några skrumpna äpplen fick följa med in i ugnen bredvid den brunblaskiga fågeln. Nu tog det några timmar och sedan skulle det bli fest! Margareta var glad i hågen när hon slog numret till sin dotter Karin. Innan samtalet var slut hade det glada bytts i besvikelse. Karin skulle ut med jobbet ikväll och kunde inte komma. Margareta försökte ringa till sin son Patrik men han var tydligen utomlands på tjänsteresa. Nu var hon ensam med sina middagsplaner. Så trist. Hon torkade bort en tår ur ögonvrån. Nog ville de komma allt barnen, men de hade inte tid med henne. De hade låtit lite ledsna på rösten, som om de hade fått lite dåligt samvete. 

Den kvällen satt Margareta själv vid sitt köksbord och åt en bit kalkon med alla tillbehör. Hon såg en sorgsen bild av sitt ansikte i spegeln, hade hon verkligen så många rynkor? Aptiten var nu som bortblåst och Margareta dukade av och stoppade in kalkonen och alla tillbehör i kylskåpet igen. Imorgon bitti kunde hon ringa till Karin igen och se om hon ville hjälpa henne att äta upp resterna. Imorgon blir en ny dag tänkte hon innan hon somnade den kvällen.

Morgonen därpå hade ett tunt lager nysnö bäddat in träd och buskar i ett fluffigt täcke. Margareta var på gott humör, åt sin frukost framför Nyhetsmorgon på tv och såg framemot att äta av de goda kalkonresterna tillsammans med dottern på kvällen. Efter disken satte hon sig i köket och slog numret till Karin. Margareta stålsatte sig, hoppades på det bästa men förväntade sig det värsta – att Karin inte kunde komma ikväll heller. Men, även om det skulle bli det senare skulle hon inte bli besviken. Hon skulle göra det bästa av det i så fall. 

När Karin återigen förklarat att hon inte kunde komma gav sig Margareta ut på stan. Hon skulle minsann köpa sig något nytt, klä upp sig ikväll och äta en fin kalkonmiddag, njuta av stunden och piffa till sig lite.

På kvällen hade hon dukat fint, tänt ljus på köksbordet och omsorgsfullt vikt servetten av linne. I badrummet tog hon fram sin sminkväska som under flera månaders tid inte använts. Mascara piggade upp, lite läppstift och rouge, ja nu kände hon sig i alla fall snygg!

I köket lade hon upp resterna vackert på tallriken, hällde upp ett glas rött och tittade förstrött ut över det vintriga landskapet. I ögonvrån skymtade hon en gestalt, en kvinna i köket?  Kunde det vara Karin i alla fall som kommit? Nej, det var inte Karin, kvinnan som smugit sig in obemärkt i Margaretas lägenhet var i samma ålder som hon. Kunde det vara en av de osynliga grannarna som tillslut vågat på sig att bli lite social? 

Margareta vände sig om för att välkomna gästen. Hon log och gästen log tillbaka, hon sa inte så mycket men verkade glad över att få träffa Margareta. Margareta lade inte så stor vikt vid att hon inte hört att det ringt på dörrklockan och att gästen själv öppnat och gått in i lägenheten. Nu var hon ju inte ensam! Gästen hade minsann ansträngt sig för att klä sig fin och besöka Margareta. Nu bjöd hon henne till bords och slog upp ett till glas vin och serverade en tallrik med fint upplagda kalkonrester. 

Det var en trevlig middag och gästen, som presenterat sig som Margareta, delade inte bara namn med Margareta, hon hade dessutom liknande klädsmak och samma märke på läppstiftet. Middagsgästen var inte stor i maten, lite vin gick väl ner men för det mesta verkade hon bara smutta litegrann. Margareta tänkte att det inte gjorde något, hon var bara så glad att hon hade sällskap till middagen denna kväll. Hon satte på lite kaffe efter maten för det ville även Margareta ha. 

Efter kaffe och kaka var de båda nöjda och mätta. Margareta dukade av och tog disken. Middagsgästen verkade ha tackat för sig och begett sig hemåt. Hon bodde här i huset hade hon sagt när Margareta frågat. Kalkonresterna hade nästan gått åt och Margareta gladde sig åt sin nya vän. Kanske, kanske skulle hon dyka upp imorgon kväll igen? Margareta hoppades och log. 

Nästa dag begav sig Margareta ut för att handla lite i den lokala matvarubutiken på hörnan. Lite lax och potatis kunde vara gott tänkte hon. Liksom föregående kväll gjorde hon sig fin framför badrumsspegeln, lite smink piggade upp. Hon gick ut i köket och förberedde middagen och liksom föregående kväll märkte hon att någon såg på henne. I ögonvrån stod middagsgästen Margareta igen, glad och lite uppklädd. Margareta tänkte att det var konstigt att hon åter inte hört någon dörrklocka men tänkte att hörseln kanske hade börjat bli sämre på äldre dar. Men hennes nyfunna vän Margareta verkade i alla fall höra henne och Margareta hörde också tydligt vad hon sa. Tillsammans åt de en fantastisk middag med lax och potatis, pratade om svunna tider och hur trevligt det var med sällskap vid middagsbordet. Det blev lax och potatis över, Margaretas vän var verkligen inte stor i maten. Resterna brukade Margareta äta till lunch framför tv:n då de visade Hem till gården. 

Gästen fortsatte att dyka upp till middagarna varje kväll under december månad, men inte den kvällen det var Bingolotto på tv. Margareta satt då framför tv:n med en varm macka och tänkte att hennes nyfunna vän nog gjorde detsamma hemma hos sig. Middagsgästen Margareta hade sagt att hon skulle bjuda hem Margareta till sig någon gång och Margareta hade sett framemot detta. Nu lät inbjudan vänta på sig men Margareta tänkte att gästen kanske inte hade det lika väl ställt som hon själv och Margareta tyckte inte att det gjorde något att de bara var hos henne när det var middag på kvällarna.

Julen närmade sig och barnen Karin och Patrik hörde tillslut av sig och ville komma på lite julmat en kväll. Margareta ursäktade sig då och sa att ” Tyvärr, jag har en ny vän som brukar komma på middag i mitt kök nästan varje kväll och i jul vill jag bjuda henne på något extra gott”. Karin och Peter hade blivit besvikna men inte alltför besvikna. De hade mer undrat över hur hon kunde bjuda hem någon på middag i det lilla, lilla köket som endast rymde köksbordet och en stol. 

På julaftonsmorgon dalade stora, vita, fluffiga flingor ner från himlen. Margareta pälsade på sig och skyndade till mataffären för att köpa hem godsaker till sitt och Margaretas julbord. Så fint att slippa sitta ensam på julafton, tänkte hon när hon klev in i den julpyntade butiken. Under tiden hon handlade stod lägenheten tom. I det lilla, lilla, julpyntade köket vilade pepparkaksdegen på bänken. Vid väggen intill fönstret stod en ensam stol vid det stora köksbordet som var intryckt mittemot väggen med den stora, vackra rokoko-spegeln. 
